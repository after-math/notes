\special{dvipdfmx:config z 0}
% ————文档类基本设置————
\documentclass[10pt,a4paper,openany]{ctexbook}
% openany → 章节可以从奇偶页任意开始(不强制新页右侧)

% ————基础宏包————
\usepackage{xcolor}
\usepackage{geometry}
\geometry{left=3cm,right=3cm,top=2.5cm,bottom=2.5cm}
\usepackage{indentfirst}
\setlength{\parindent}{2em}

% ====== 字体配置 ======
\setmainfont{Times New Roman}
\setsansfont{Arial}
\setmonofont{Courier New}
\setCJKmainfont{Songti SC}
\setCJKsansfont{PingFang SC}
\setCJKmonofont{Kaiti SC}

% ————数学公式支持————
\usepackage{mathtools,amssymb,bm,amsthm}

% ————插图与表格————
\usepackage{tabularx}        % 自动调整列宽并支持换行
\usepackage{graphicx}
\usepackage{booktabs,siunitx,multirow,threeparttable,diagbox}
\sisetup{
	round-mode=places,
	round-precision=3,
	table-number-alignment=center,
	detect-weight=true,
}

% ————标题与引用————
\usepackage{caption}
\captionsetup{font=small,labelfont=bf}

\usepackage{hyperref}
\hypersetup{
	colorlinks=true,
	linkcolor=blue,
	citecolor=teal,
	urlcolor=magenta,
	pdfauthor={嘉明},
	pdftitle={抽象代数}
}

% ————页眉页脚————
\usepackage{fancyhdr}
\pagestyle{fancy}
\fancyhf{}
\fancyhead[CE]{\leftmark} % 偶数页章标题
\fancyhead[CO]{\rightmark} % 奇数页节标题
\fancyfoot[C]{\thepage}
\renewcommand{\headrulewidth}{0.4pt}

% ————参考文献————
\usepackage[backend=biber,style=gb7714-2015]{biblatex}
% \addbibresource{ref.bib}

% ————代码环境————
\usepackage{listings}
\lstset{
	basicstyle=\ttfamily\small,
	numbers=none,
	keywordstyle=\color{blue},
	commentstyle=\color{green!50!black},
	stringstyle=\color{red},
	breaklines=true,
	frame=single
}

% ————引用————
\usepackage{cleveref}

% ————tcolorbox(定理/定义等用)————
\usepackage{tcolorbox}

% ————枚举环境————
\usepackage{enumitem}

% ————目录美化————
\usepackage{titlesec}  % 控制标题(章、节等)样式的宏包
\usepackage{titletoc}  % 控制目录(table of contents)样式的宏包

\definecolor{myred}{RGB}{0,0,0}  % 定义一个颜色,名称为 myred(这里是黑色)

% 一级目录(章)
\titlecontents{chapter}            % 定义目录中“章”级别的样式
[0pt]                              % 设置章标题在目录中的左缩进(这里为 0,左对齐)
{\bfseries\color{myred}}           % 设置章标题的字体样式:加粗并使用定义的颜色
{\thecontentslabel \quad}          % 定义编号(如“1”或“第1章”)与标题间的格式,用 \quad 加空格
{}                                 % 定义无编号章节(如 \chapter*{前言})的格式,这里留空表示不特殊处理
{\titlerule*[0.6pc]{.}\contentspage} % 定义右侧点线与页码格式:点线间距 0.6pc,页码右对齐显示

% 二级目录(节)
\titlecontents{section}
[1.5em]{\color{myred}}
{\thecontentslabel \quad}
{}
{\titlerule*[0.6pc]{.}\contentspage}

% 三级目录(小节)
% ---------------------------- 子节 (subsection) 目录格式设置 
\titlecontents{subsection}
[3.5em]                              % 左缩进
{\color{myred}}                      % 整体样式
{}                                   % ← 这里留空,就不会显示编号
{}                                   % 无编号条目前缀
{\titlerule*[0.6pc]{.}\contentspage} % 右端点线与页码



% ————章节标题美化————
\titleformat{\chapter}[block]
{\Huge\bfseries\color{black}}
{第\,\thechapter\,章}
{1em}{}

% 修改目录大小写
\ctexset{
	chapter = {
		name = {第,章},         % “第” 和 “章” 仍保留
		number = \arabic{chapter} % 使用阿拉伯数字而非中文数字
	}
}
\setcounter{tocdepth}{3}  % 显示到3级别


\usepackage{tikz}
\usepackage{tikz-cd}
\usepackage{mdframed}

% ————标题信息————
\title{抽象代数}
\author{嘉明}
\date{\today}

% ————正文开始————
\begin{document}
	\begin{titlepage}
		\centering
		
		% 上方留白
		\vspace*{2cm}
		
		% 校徽(上方正中央)
		\includegraphics[width=0.3\textwidth]{校徽.png}
		
		% 校徽与标题间距
		\vspace{2cm}
		
		% 标题(隶书)
		{\zihao{0}\bfseries\CJKfamily{kai}
			抽象代数部分习题解答\par}
		
		% 底部自动撑开
		\vfill
	\end{titlepage}
	
%========================
% 前言(可直接粘贴到你的书稿里)
%========================

\chapter*{前言}
\addcontentsline{toc}{chapter}{前言}

本笔记整理于本人本学期学习《抽象代数》期间。由于教材配套习题在市面上缺少可供参考的完整解答,许多题目只能依靠课堂讨论与个人推导逐步完成。为便于复习与交流,我将老师布置的作业题进行系统整理,并尽力写成**步骤尽可能细、推理尽可能清**的解答,希望即使是初学者也能循着思路看懂每一步的来龙去脉。

受时间与精力所限,当前文档仅完成了部分内容,难免疏漏与不足。之所以仍选择将其发布到资料群中,是希望它能成为一个可持续完善的“起点”:愿后续使用这份资料的同学——尤其是新入学的学弟学妹——能够在此基础上补全未竟之处、订正可能的错误,并将更好的版本继续传递下去。若本笔记能在大家学习抽象代数的道路上提供哪怕一点帮助,便已足矣。

\begin{flushright}
	嘉明\\
	QQ:3393955216
	\today
\end{flushright}



	\tableofcontents
	\mainmatter                % 正文部分(从第1章开始编号)

	\chapter{群}
	% -------------------- 1.1 半群与群 --------------------
	\section{半群与群}
		\clearpage
	\subsection*{课后习题答案}
	\addcontentsline{toc}{subsection}{\textcolor{red}{课后习题答案}}
	\begin{enumerate}[label=\textcolor{blue}{\textbf{\large\arabic*.}}]
		\item 2.
		\textbf{题目.}\;
		在 $\mathbb Z\times \mathbb Z$ 中定义乘法
		\[
		(x_1,x_2)(y_1,y_2)=(x_1y_1+2x_2y_2,\;x_1y_2+x_2y_1).
		\]
		证明:$\mathbb Z\times \mathbb Z$ 关于此乘法构成一个交换幺半群。
		
		\medskip
		\textbf{证明.}
		
		\medskip
		\textbf{(1) 封闭性.}
		取任意 $(x_1,x_2),(y_1,y_2)\in \mathbb Z\times \mathbb Z$,
		由于整数在加法与乘法下封闭,
		\[
		x_1y_1+2x_2y_2\in\mathbb Z,\qquad x_1y_2+x_2y_1\in\mathbb Z,
		\]
		故 $(x_1,x_2)(y_1,y_2)\in\mathbb Z\times\mathbb Z$。
		
		\medskip
		\textbf{(2) 结合律.}
		取任意 $(x_1,x_2),(y_1,y_2),(z_1,z_2)\in\mathbb Z\times\mathbb Z$,
		先计算
		\[
		(x_1,x_2)(y_1,y_2)=(a_1,a_2),
		\]
		其中
		\[
		a_1=x_1y_1+2x_2y_2,\qquad a_2=x_1y_2+x_2y_1.
		\]
		于是
		\[
		\begin{aligned}
			\bigl((x_1,x_2)(y_1,y_2)\bigr)(z_1,z_2)
			&=(a_1z_1+2a_2z_2,\;a_1z_2+a_2z_1)\\
			&=(x_1y_1z_1+2x_2y_2z_1+2x_1y_2z_2+2x_2y_1z_2,\\
			&\quad x_1y_1z_2+2x_2y_2z_2+x_1y_2z_1+x_2y_1z_1).
		\end{aligned}
		\]
		另一方面,
		\[
		(y_1,y_2)(z_1,z_2)=(b_1,b_2),
		\]
		其中
		\[
		b_1=y_1z_1+2y_2z_2,\qquad b_2=y_1z_2+y_2z_1.
		\]
		于是
		\[
		\begin{aligned}
			(x_1,x_2)\bigl((y_1,y_2)(z_1,z_2)\bigr)
			&=(x_1b_1+2x_2b_2,\;x_1b_2+x_2b_1)\\
			&=(x_1y_1z_1+2x_1y_2z_2+2x_2y_1z_2+2x_2y_2z_1,\\
			&\quad x_1y_1z_2+x_1y_2z_1+x_2y_1z_1+2x_2y_2z_2).
		\end{aligned}
		\]
		逐项比较可得两者完全相同,因此乘法满足结合律。
		
		\medskip
		\textbf{(3) 幺元存在.}
		取 $(1,0)\in\mathbb Z\times\mathbb Z$,对任意 $(x_1,x_2)\in\mathbb Z\times\mathbb Z$,
		\[
		(x_1,x_2)(1,0)=(x_1\cdot1+2x_2\cdot0,\;x_1\cdot0+x_2\cdot1)=(x_1,x_2),
		\]
		\[
		(1,0)(x_1,x_2)=(1\cdot x_1+2\cdot0\cdot x_2,\;1\cdot x_2+0\cdot x_1)=(x_1,x_2),
		\]
		故 $(1,0)$ 为幺元。
		
		\medskip
		\textbf{(4) 交换律.}
		对任意 $(x_1,x_2),(y_1,y_2)\in\mathbb Z\times\mathbb Z$,
		\[
		(x_1,x_2)(y_1,y_2)=(x_1y_1+2x_2y_2,\;x_1y_2+x_2y_1)
		=(y_1,x_2)(x_1,x_2),
		\]
		因为整数乘法满足交换律,
		故该乘法是交换的。
		
		\medskip
		综上,$\mathbb Z\times\mathbb Z$ 在上述乘法下构成一个交换幺半群。
		\item 10.
		\textbf{题目.}\;
		证明集合
		\[
		G=\left\{\,e^{\frac{2k\sqrt{-1}\pi}{n}}
		=\cos\frac{2k\pi}{n}+\sqrt{-1}\sin\frac{2k\pi}{n}
		\;\middle|\; k=0,1,\dots,n-1 \right\}
		\]
		在复数乘法下构成一个 $n$ 个元素的乘法群。
		
		\medskip
		\textbf{答案:}
		
		\textbf{证明.}
		
		\medskip
		\textbf{(1) 封闭性.}
		任取
		\[
		e^{\frac{2k\sqrt{-1}\pi}{n}},\quad e^{\frac{2\ell\sqrt{-1}\pi}{n}}\in G,
		\]
		则
		\[
		e^{\frac{2k\sqrt{-1}\pi}{n}}\cdot e^{\frac{2\ell\sqrt{-1}\pi}{n}}
		= e^{\frac{2(k+\ell)\sqrt{-1}\pi}{n}}.
		\]
		若 $k+\ell<n$,则该元素仍属于 $G$;  
		若 $k+\ell\ge n$,则
		\[
		e^{\frac{2(k+\ell)\sqrt{-1}\pi}{n}}
		= e^{\frac{2(k+\ell-n)\sqrt{-1}\pi}{n}}\cdot e^{2\sqrt{-1}\pi}
		= e^{\frac{2(k+\ell-n)\sqrt{-1}\pi}{n}},
		\]
		而 $0\le k+\ell-n\le n-1$,仍属于 $G$。
		故 $G$ 对乘法封闭。
		
		\medskip
		\textbf{(2) 结合律.}
		$G$ 中的运算为复数乘法,而复数乘法本身满足结合律,
		因此对任意 $a,b,c\in G$,
		\[
		(ab)c=a(bc).
		\]
		
		\medskip
		\textbf{(3) 幺元存在.}
		当 $k=0$ 时,
		\[
		e^{\frac{2\cdot0\cdot\sqrt{-1}\pi}{n}}=e^{0}=1,
		\]
		显然 $1$ 是复数乘法的幺元,且 $1\in G$。
		
		\medskip
		\textbf{(4) 逆元存在.}
		对任意
		\[
		e^{\frac{2k\sqrt{-1}\pi}{n}}\in G,
		\]
		其逆元为
		\[
		\left(e^{\frac{2k\sqrt{-1}\pi}{n}}\right)^{-1}
		= e^{-\frac{2k\sqrt{-1}\pi}{n}}
		= e^{\frac{2(n-k)\sqrt{-1}\pi}{n}},
		\]
		其中 $n-k\in\{0,1,\dots,n-1\}$,故逆元仍属于 $G$。
		
		\medskip
		\textbf{(5) 元素个数.}
		若
		\[
		e^{\frac{2k\sqrt{-1}\pi}{n}}=e^{\frac{2\ell\sqrt{-1}\pi}{n}},
		\]
		则
		\[
		e^{\frac{2(k-\ell)\sqrt{-1}\pi}{n}}=1
		\;\Longrightarrow\;
		\frac{2(k-\ell)\pi}{n}\in 2\pi\mathbb Z
		\;\Longrightarrow\;
		k\equiv \ell\pmod n.
		\]
		在 $k,\ell\in\{0,1,\dots,n-1\}$ 下只能有 $k=\ell$,
		故 $G$ 中元素互不相同,共有 $n$ 个元素。
		
		\medskip
		综上,$G$ 在复数乘法下构成一个 $n$ 元的乘法群。
		\item 11.
		\textbf{题目.}\;
		设 $n\in\mathbb N$,$\mathbb Z_n$ 表示集合 $\{\bar0,\bar1,\dots,\overline{n-1}\}$(模 $n$ 的同余类)。
		在 $\mathbb Z_n$ 中定义运算:
		\[
		\bar a\cdot \bar b=\bar c,\ \text{其中 }c\equiv ab\pmod n;\qquad
		\bar a+\bar b=\bar d,\ \text{其中 }d\equiv a+b\pmod n.
		\]
		\begin{enumerate}
			\item[(1)] 证明:$\{\mathbb Z_n;\cdot\}$ 是交换幺半群,$\{\mathbb Z_n;+\}$ 是交换群;
			\item[(2)] 构造 $\{\mathbb Z_4;\cdot\}$ 的半群表;
			\item[(3)] 设 $\mathbb Z_n^{*}=\{\bar a\in\mathbb Z_n\mid (a,n)=1\}$,证明:$\{\mathbb Z_n^{*};\cdot\}$ 是群;
			\item[(4)] 设 $p$ 是素数,利用群的思想证明 Wilson 定理:$(p-1)!\equiv -1\pmod p$。
		\end{enumerate}
		
		\textbf{解答:}
		
		\textbf{(1) $\{\mathbb Z_n;\cdot\}$ 为交换幺半群,$\{\mathbb Z_n;+\}$ 为交换群.}
		
		\textbf{良定义(两种运算都要先做).}
		若 $\bar a=\bar a'$、$\bar b=\bar b'$(即 $a\equiv a'\pmod n,\ b\equiv b'\pmod n$),
		则
		\[
		ab\equiv a'b'\pmod n,\qquad a+b\equiv a'+b'\pmod n,
		\]
		从而 $\overline{ab}=\overline{a'b'}$ 且 $\overline{a+b}=\overline{a'+b'}$。
		故 $\cdot$ 与 $+$ 在 $\mathbb Z_n$ 上良定义。
		
		\textbf{(a) 乘法:交换幺半群.}
		\[
		(\bar a\cdot \bar b)\cdot \bar c=\overline{ab}\cdot \bar c=\overline{(ab)c}=\overline{a(bc)}
		=\bar a\cdot \overline{bc}=\bar a\cdot(\bar b\cdot \bar c),
		\]
		故结合律成立。
		又因整数乘法交换,
		\[
		\bar a\cdot \bar b=\overline{ab}=\overline{ba}=\bar b\cdot \bar a,
		\]
		故交换律成立。
		并且
		\[
		\bar1\cdot \bar a=\overline{1\cdot a}=\bar a=\overline{a\cdot 1}=\bar a\cdot \bar1,
		\]
		故 $\bar1$ 是单位元。
		因此 $\{\mathbb Z_n;\cdot\}$ 是交换幺半群(从而当然也是交换半群)。
		
		\textbf{(b) 加法:交换群.}
		结合律同理:
		\[
		(\bar a+\bar b)+\bar c=\overline{a+b}+\bar c=\overline{(a+b)+c}=\overline{a+(b+c)}
		=\bar a+\overline{b+c}=\bar a+(\bar b+\bar c).
		\]
		交换律:
		\[
		\bar a+\bar b=\overline{a+b}=\overline{b+a}=\bar b+\bar a.
		\]
		单位元为 $\bar0$,因为 $\bar a+\bar0=\overline{a+0}=\bar a$。
		任取 $\bar a\in\mathbb Z_n$,令
		\[
		-\bar a:=\overline{n-a}\quad (\text{若 }a=0\text{ 则 }n-a=n\equiv 0),
		\]
		则
		\[
		\bar a+(-\bar a)=\overline{a+(n-a)}=\overline{n}=\bar0.
		\]
		故每个元素都有加法逆元。
		因此 $\{\mathbb Z_n;+\}$ 是交换群。
		
		\textbf{(2) $\{\mathbb Z_4;\cdot\}$ 的半群表.}
		
		元素为 $\bar0,\bar1,\bar2,\bar3$,乘法按模 $4$ 计算:
		\[
		\begin{array}{c|cccc}
			\cdot & \bar0 & \bar1 & \bar2 & \bar3\\\hline
			\bar0 & \bar0 & \bar0 & \bar0 & \bar0\\
			\bar1 & \bar0 & \bar1 & \bar2 & \bar3\\
			\bar2 & \bar0 & \bar2 & \bar0 & \bar2\\
			\bar3 & \bar0 & \bar3 & \bar2 & \bar1
		\end{array}
		\]
		
		\textbf{(3) $\{\mathbb Z_n^{*};\cdot\}$ 是群.}
		
		先说明 $\mathbb Z_n^{*}\neq\varnothing$,因为 $\bar1\in\mathbb Z_n^{*}$ 且 $(1,n)=1$。
		
		\textbf{封闭性.}
		若 $\bar a,\bar b\in\mathbb Z_n^{*}$,即 $(a,n)=1$ 且 $(b,n)=1$。
		若 $d\mid (ab)$ 且 $d\mid n$,则 $d$ 的任一素因子也同时整除 $ab$ 与 $n$。
		但它若整除 $a$ 就与 $(a,n)=1$ 矛盾;若整除 $b$ 就与 $(b,n)=1$ 矛盾。
		故只能 $d=1$,从而 $(ab,n)=1$,即 $\overline{ab}\in\mathbb Z_n^{*}$。
		
		\textbf{结合律与交换律.}
		由 (1) 中 $\mathbb Z_n$ 乘法的结合律与交换律直接继承到其子集 $\mathbb Z_n^{*}$。
		
		\textbf{单位元.}
		$\bar1\in\mathbb Z_n^{*}$ 且对任意 $\bar a\in\mathbb Z_n^{*}$ 有 $\bar1\cdot \bar a=\bar a$。
		
		\textbf{逆元存在.}
		由 $(a,n)=1$,存在整数 $x,y$ 使得(裴蜀等式)
		\[
		ax+ny=1.
		\]
		两边模 $n$ 取同余类得 $\overline{ax}=\bar1$,即
		\[
		\bar a\cdot \bar x=\bar1.
		\]
		且由 $ax\equiv 1\pmod n$ 可知 $(x,n)=1$(若素数 $q\mid x$ 且 $q\mid n$,则 $q\mid ax$ 与 $q\mid n$ 推出 $q\mid 1$ 矛盾),
		故 $\bar x\in\mathbb Z_n^{*}$,它就是 $\bar a$ 的逆元。
		
		综上,$\{\mathbb Z_n^{*};\cdot\}$ 是群(且为交换群)。
		
		\textbf{(4) Wilson 定理:若 $p$ 为素数,则 $(p-1)!\equiv -1\pmod p$.}
		
		当 $p=2$ 时,$(p-1)!=1\equiv -1\pmod 2$,成立。
		以下设 $p$ 为奇素数。
		
		由 (3) 知 $\{\mathbb Z_p^{*};\cdot\}$ 是交换群,且
		\[
		\mathbb Z_p^{*}=\{\bar1,\bar2,\dots,\overline{p-1}\},\qquad |\mathbb Z_p^{*}|=p-1.
		\]
		在群中考虑所有元素的乘积:
		\[
		P:=\bar1\cdot \bar2\cdots \overline{p-1}\;=\;\overline{(p-1)!}\in\mathbb Z_p^{*}.
		\]
		
		对任意 $a\in\{1,2,\dots,p-1\}$,元素 $\bar a$ 在群中有唯一逆元 $\bar a^{-1}$。
		若 $\bar a\neq \bar a^{-1}$,则可把 $\bar a$ 与 $\bar a^{-1}$ 配对,它们的乘积为 $\bar1$。
		
		\textbf{关键:哪些元素满足 $\bar a=\bar a^{-1}$?}
		这等价于 $\bar a^2=\bar1$,即
		\[
		a^2\equiv 1\pmod p \quad\Longleftrightarrow\quad (a-1)(a+1)\equiv 0\pmod p.
		\]
		由于 $p$ 为素数,故 $p\mid(a-1)$ 或 $p\mid(a+1)$,即
		\[
		a\equiv 1\pmod p \quad\text{或}\quad a\equiv -1\pmod p.
		\]
		在 $\{1,2,\dots,p-1\}$ 中,这两解对应 $a=1$ 与 $a=p-1$。
		因此群中只有 $\bar1$ 与 $\overline{p-1}=\overline{-1}$ 是自逆元。
		
		于是,在乘积 $P$ 中,除去这两个自逆元,其余元素都与不同的逆元成对出现,每一对贡献 $\bar1$。
		故
		\[
		P=\bar1\cdot \overline{-1}\cdot (\text{若干对 } \bar a\cdot \bar a^{-1})
		=\bar1\cdot \overline{-1}\cdot \bar1=\overline{-1}.
		\]
		这就是说
		\[
		\overline{(p-1)!}=\overline{-1}\ \text{于 }\mathbb Z_p,
		\]
		等价于
		\[
		(p-1)!\equiv -1\pmod p,
		\]
		即 Wilson 定理成立。
		
	\end{enumerate}
	
	% -------------------- 1.2 子群与陪集 --------------------


	\clearpage
	\section{子群与陪集}
		\subsection{群的交集为子群}
	\textbf{命题.}\;
	设 $G$ 为群,$H,K\le G$ 为 $G$ 的子群,则
	\[
	H\cap K \le G,
	\]
	即 $H\cap K$ 仍是 $G$ 的子群。
	
	\textbf{证明.}\;
	因为 $H,K$ 均为子群,故单位元 $e\in H$ 且 $e\in K$,从而 $e\in H\cap K$,
	于是 $H\cap K\neq\varnothing$。
	
	任取 $x,y\in H\cap K$,则 $x,y\in H$ 且 $x,y\in K$。
	由于 $H,K$ 均为子群,有
	\[
	xy^{-1}\in H,\qquad xy^{-1}\in K,
	\]
	从而 $xy^{-1}\in H\cap K$。
	
	因此 $H\cap K$ 对运算 $xy^{-1}$ 封闭,满足子群判别法,
	故 $H\cap K$ 是 $G$ 的子群。
	\subsection{定理1.2.22的推论:若群 $G$ 的阶为素数 $p$,则 $G$只有平凡子群,从而也是单群。}	
	\textbf{命题.}\;
	若群 $G$ 的阶为素数 $p$,则 $G$ 是单群。
	
	\textbf{证明.}\;
	由拉格朗日定理,$G$ 的任意子群 $H \le G$ 的阶整除 $|G| = p$。
	由于 $p$ 为素数,因而 $H$ 的阶只能为 $1$ 或 $p$。
	
	若 $|H| = 1$,则 $H = \{e\}$;
	若 $|H| = p$,则 $H = G$。
	因此,$G$ 只有平凡子群与自身,不存在非平凡真子群。
	
	而任何正规子群也是子群,故 $G$ 只有平凡正规子群 $\{e\}$ 与 $G$ 本身。
	
	由此可知,$G$ 是单群。
	\[
	\boxed{\text{若 } |G| = p \text{ 为素数,则 } G \text{ 是单群。}}
	\]
	
	
	\subsection{$Klein$四元群的性质}
	\begin{itemize}
		\item \textbf{定义与阶.}\;
		Klein 四元群记为 \(K_4\),是阶为 \(4\) 的群,典型表示为
		\[
		K_4=\{e,a,b,c\},
		\]
		满足关系
		\[
		a^2=b^2=c^2=e,\qquad ab=ba=c,\qquad bc=cb=a,\qquad ca=ac=b,
		\]
		等价地也常用给定关系
		\[
		a^2=b^2=c^2=abc=e.
		\]
		
		\item \textbf{元素阶与指数.}\;
		\[
		|e|=1,\qquad 
		|a|=|b|=|c|=2.
		\]
		因此 \(K_4\) 的指数为 \(2\),即对任意 \(x\in K_4\) 都有 \(x^2=e\)。
		
		\item \textbf{交换性与非循环性.}\;
		\(K_4\) 是阿贝尔群(交换群),且不是循环群(不存在阶为 \(4\) 的元素)。
		阶为 \(4\) 的群只有两种同构类型:
		\[
		\mathbb{Z}/4\mathbb{Z}\quad \text{或}\quad K_4.
		\]
		
		\item \textbf{同构刻画.}\;
		\[
		K_4 \cong (\mathbb{Z}/2\mathbb{Z})\times(\mathbb{Z}/2\mathbb{Z}).
		\]
		也即 \(K_4\) 是唯一的非循环四阶群。
		
		\item \textbf{子群结构.}\;
		\(K_4\) 的子群只有:
		\[
		\{e\},\quad \langle a\rangle=\{e,a\},\quad \langle b\rangle=\{e,b\},\quad \langle c\rangle=\{e,c\},\quad K_4.
		\]
		其中恰有 \(3\) 个阶为 \(2\) 的非平凡子群。
		
		\item \textbf{正规子群与中心.}\;
		由于 \(K_4\) 阿贝尔,故任意子群都正规:
		\[
		\forall H\le K_4,\quad H\lhd K_4.
		\]
		并且中心为全群:
		\[
		C(V_4)=K_4.
		\]
		
		\item \textbf{共轭类.}\;
		因 \(K_4\) 阿贝尔,每个元素自成共轭类:
		\[
		\{e\},\ \{a\},\ \{b\},\ \{c\}.
		\]
		
		\item \textcolor{red}{商群.}\;
		对任意非平凡的2阶子群 \(H=\{e,x\}\)(其中 \(x\in\{a,b,c\}\)),都有
		\[
		K_4/H \cong \mathbb{Z}/2\mathbb{Z}.
		\]
		
		\textbf{证明:}
		
		设
		\[
		V_4=\{e,a,b,c\},
		\]
		其中
		\[
		a^2=b^2=c^2=e,\qquad ab=ba=c,\ bc=cb=a,\ ca=ac=b.
		\]
		
		取任意非平凡的阶 \(2\) 子群
		\[
		H=\{e,x\},
		\]
		其中 \(x\in\{a,b,c\}\)。
		
		\medskip
		
		\textbf{第一步:说明 \(H\) 为正规子群.}
		
		由于 \(V_4\) 是阿贝尔群,对任意子群 \(H\le V_4\),都有
		\[
		H\lhd V_4.
		\]
		因此商群 \(V_4/H\) 是良定义的。
		
		\medskip
		
		\textbf{第二步:计算陪集.}
		
		因为 \(|V_4|=4\),\(|H|=2\),由 Lagrange 定理可知
		\[
		[V_4:H]=\frac{|V_4|}{|H|}=2.
		\]
		因此商群 \(V_4/H\) 共有两个元素。
		
		显然一个陪集是
		\[
		H=\{e,x\}.
		\]
		取 \(y\in V_4\setminus H\)(例如从 \(\{a,b,c\}\setminus\{x\}\) 中任选一个),
		则另一个陪集为
		\[
		yH=\{y,yx\}.
		\]
		
		因此
		\[
		V_4/H=\{H,\ yH\}.
		\]
		
		\medskip
		
		\textbf{第三步:确定商群的群结构.}
		
		在商群中,
		\[
		H\cdot H=H,\qquad H\cdot yH=yH,
		\]
		并且
		\[
		(yH)\cdot(yH)=y^2H.
		\]
		由于 \(y\neq e\) 且 \(y\in\{a,b,c\}\),有 \(y^2=e\),于是
		\[
		(yH)\cdot(yH)=eH=H.
		\]
		
		因此,\(V_4/H\) 是一个含两个元素的群,且非单位元的平方为单位元。
		
		\medskip
		
		\textbf{第四步:判定同构类型.}
		
		阶为 \(2\) 的群在同构意义下唯一,即循环群
		\[
		\mathbb{Z}/2\mathbb{Z}.
		\]
		故
		\[
		V_4/H \cong \mathbb{Z}/2\mathbb{Z}.
		\]
		
		\medskip
		
		综上所述,对任意非平凡的阶 \(2\) 子群 \(H\le V_4\),均有
		\[
		\boxed{V_4/H \cong \mathbb{Z}/2\mathbb{Z}}.
		\]
		
		\item \textbf{在置换群中的一个经典实现.}\;
		在 \(S_4\) 中,集合
		\[
		\{\,e,\ (12)(34),\ (13)(24),\ (14)(23)\,\}
		\]
		在置换乘法下构成一个 Klein 四元群(同构于 \(V_4\))。
	\end{itemize}
	
	\subsection{子群乘积的阶公式}
	\textbf{引理(子群乘积的阶公式).}
	设 $H,K$ 是有限群 $G$ 的子群,则
	\[
	|HK|=\frac{|H||K|}{|H\cap K|}.
	\]
	
	\textbf{证明要点.}
	映射
	\[
	H\times K \longrightarrow HK,\qquad (h,k)\mapsto hk
	\]
	是满射;对任意 $x\in HK$,其原像的基数恰为 $|H\cap K|$,
	因此由计数可得上式。
	
	\medskip
	
	\textbf{应用到 Sylow 子群.}
	设 $P,Q$ 分别是 $G$ 的 Sylow $p$-子群与 Sylow $q$-子群,
	其中 $p\neq q$ 为素数。则
	\[
	|P|=p,\qquad |Q|=q.
	\]
	由于 $P\cap Q$ 同时是 $P$ 与 $Q$ 的子群,
	\[
	|P\cap Q|\mid p,\qquad |P\cap Q|\mid q.
	\]
	因 $p,q$ 互素,只能有
	\[
	|P\cap Q|=1.
	\]
	代入阶公式得到
	\[
	|PQ|=\frac{|P||Q|}{|P\cap Q|}
	=\frac{pq}{1}
	=pq
	=|G|.
	\]
	从而
	\[
	PQ=G.
	\]
	\clearpage
	\subsection*{课后习题答案}
	\addcontentsline{toc}{subsection}{\textcolor{red}{课后习题答案}}
	\begin{enumerate}[label=\textcolor{blue}{\textbf{\large\arabic*.}}]
		\item 7.
		\textbf{题目.}\;
		设 $G$ 是一个群,$a,b\in G$,证明下列元素对同阶:
		\[
		a \text{ 与 } a^{-1};\qquad a \text{ 与 } bab^{-1};\qquad ab \text{ 与 } ba.
		\]
		
		\textbf{答案:}\;
		\textbf{证明.}
		
		\textbf{(1) $a$ 与 $a^{-1}$ 同阶.}\;
		若 $|a|=\infty$,则 $a^n\neq e$ 对一切 $n\ge 1$ 成立。若存在 $n\ge1$ 使 $(a^{-1})^n=e$,
		则 $a^n=e$,矛盾;反之亦然,故 $|a^{-1}|=\infty$。
		
		若 $|a|=m<\infty$,则 $a^m=e$ 且 $m$ 为满足该式的最小正整数。
		由 $a^m=e$ 两边取逆得 $(a^{-1})^m=e$,故 $|a^{-1}|\mid m$。
		同理,由 $(a^{-1})^{|a^{-1}|}=e$ 取逆得 $a^{|a^{-1}|}=e$,从而 $m\mid |a^{-1}|$。
		于是 $|a^{-1}|=m$。
		综上,$|a|=|a^{-1}|$。
		
		\textbf{(2) $a$ 与 $bab^{-1}$ 同阶.}\;
		对任意 $n\in\mathbb Z$,有
		\[
		(bab^{-1})^n = ba^n b^{-1}.
		\]
		(可对 $n\ge1$ 用归纳法证明;$n=0$ 显然;$n<0$ 由取逆得到。)
		
		于是对任意 $n\ge 1$,
		\[
		(bab^{-1})^n=e \ \Longleftrightarrow\  ba^n b^{-1}=e
		\ \Longleftrightarrow\ a^n=e,
		\]
		其中最后一步是左右分别乘以 $b^{-1}$ 与 $b$。
		因此使得幂等于单位元的正整数集合对 $a$ 与 $bab^{-1}$ 完全相同,
		从而它们的阶相同(有限时同为最小正整数;无限时同为无穷)。
		故 $|bab^{-1}|=|a|$。
		
		\textbf{(3) $ab$ 与 $ba$ 同阶.}\;
		注意到
		\[
		ba = a^{-1}(ab)a,
		\]
		即 $ba$ 是 $ab$ 的共轭元(由 $a^{-1}(ab)a=a^{-1}a\,b\,a=ba$)。
		由 (2) 已知共轭元同阶,故
		\[
		|ba| = |ab|.
		\]
		证毕。
			\item 8.
		\textbf{题目.}\;
		设 $a,b$ 分别是群 $G$ 中的 $m,n$ 阶元,且 $ab=ba$。
		\begin{enumerate}
			\item 若 $(m,n)=1$,证明 $ab$ 的阶为 $mn$;(只用证明$m\mid d,\ n\mid d$)
			\item 若 $\langle a\rangle\cap\langle b\rangle=\{e\}$,证明 $ab$ 的阶为 $[m,n]$;
			\item 若 $\langle a\rangle\cap\langle b\rangle\neq\{e\}$,讨论 $ab$ 的阶。$G$ 中是否一定存在 $[m,n]$ 阶元素?
		\end{enumerate}
		
		\textbf{答案:}\;
		
		\textbf{预备引理.}\;
		设 $x\in G$,$\mathrm{ord}(x)=d<\infty$,则对任意整数 $k$,
		\[
		\mathrm{ord}(x^k)=\frac{d}{(d,k)}.
		\]
		\textbf{证明.}\;
		$(x^k)^t=e \iff x^{kt}=e \iff d\mid kt \iff \frac{d}{(d,k)}\mid t$,
		故最小正整数 $t$ 为 $d/(d,k)$。 \hfill$\square$
		
		\begin{enumerate}
			\item \textbf{(1) 若 $(m,n)=1$,则 $\mathrm{ord}(ab)=mn$.}
			
			令 $d=\mathrm{ord}(ab)$。由于 $ab=ba$,
			\[
			(ab)^{mn}=a^{mn}b^{mn}=(a^m)^n(b^n)^m=e,
			\]
			故 $d\mid mn$。
			
			又因为 $a^m=e$,有
			\[
			(ab)^m=a^m b^m=b^m.
			\]
			由预备引理,
			\[
			\mathrm{ord}\bigl((ab)^m\bigr)=\frac{d}{(d,m)}.
			\]
			另一方面,$\mathrm{ord}(b^m)=\dfrac{n}{(n,m)}=n$(因 $(m,n)=1$),于是
			\[
			\frac{d}{(d,m)}=\mathrm{ord}\bigl((ab)^m\bigr)=\mathrm{ord}(b^m)=n,
			\]
			从而 $n\mid d$。同理,
			\[
			(ab)^n=a^n b^n=a^n,\qquad \mathrm{ord}(a^n)=\frac{m}{(m,n)}=m,
			\]
			故
			\[
			\frac{d}{(d,n)}=\mathrm{ord}\bigl((ab)^n\bigr)=\mathrm{ord}(a^n)=m,
			\]
			从而 $m\mid d$。
			
			因此 $m\mid d,\ n\mid d$ 且 $(m,n)=1$,推出 $mn\mid d$。结合 $d\mid mn$,得 $d=mn$。 \hfill$\square$
			
			\item \textbf{(2) 若 $\langle a\rangle\cap\langle b\rangle=\{e\}$,则 $\mathrm{ord}(ab)=[m,n]$.}
			
			记 $L=[m,n]$。仍令 $d=\mathrm{ord}(ab)$。
			
			首先 $L$ 同时是 $m,n$ 的倍数,且 $ab=ba$,故
			\[
			(ab)^L=a^L b^L=e,
			\]
			于是 $d\mid L$。
			
			反过来,由 $(ab)^d=e$ 得
			\[
			a^d b^d=e \iff a^d=b^{-d}.
			\]
			左边属于 $\langle a\rangle$,右边属于 $\langle b\rangle$,故 $a^d=b^{-d}\in \langle a\rangle\cap\langle b\rangle=\{e\}$,
			从而 $a^d=e,\ b^d=e$,即 $m\mid d,\ n\mid d$,于是 $L=[m,n]\mid d$。
			
			综上 $d\mid L$ 且 $L\mid d$,故 $d=L$。 \hfill$\square$
			
			\item \textbf{(3) 若 $\langle a\rangle\cap\langle b\rangle\neq\{e\}$,讨论 $\mathrm{ord}(ab)$,并讨论是否必有 $[m,n]$ 阶元素.}
			$\ \text{设}\ m=p_1^{\alpha_1}\cdots p_k^{\alpha_k},\quad
			n=p_1^{\beta_1}\cdots p_k^{\beta_k},\ \text{从而考虑}$
			\[
			m_1=\prod_{\alpha_i\ge \beta_i} p_i^{\alpha_i}\mid m,\qquad
			n_1=\prod_{\alpha_i< \beta_i} p_i^{\beta_i}\mid n,
			\]
			$\text{从而存在}\ m_1,n_1\ \text{阶元,则由}\ (m_1,n_1)=1\ \text{且}\ m_1n_1=[m,n],
			\ \text{从而存在}\ [m,n]\ \text{阶元,即证。}\ $
		\end{enumerate}
		\item 9.
		\textbf{题目.}\;
		设群 $G$ 中元素 $a$ 的阶为 $d$,$k\in\mathbb N$,证明:
		\[
		\text{(1) }\ \mathrm{ord}(a^k)=\frac{d}{(d,k)};\qquad
		\text{(2) }\ \mathrm{ord}(a^k)=d \iff (d,k)=1.
		\]
		
		\textbf{答案:}\;
		\textbf{证明.}\;
		令 $g=(d,k)$,则存在 $d_1,k_1\in\mathbb N$ 使得
		\[
		d=g d_1,\qquad k=g k_1,\qquad (d_1,k_1)=1.
		\]
		
		\textbf{(1) 证明 $\mathrm{ord}(a^k)=d_1$.}
		
		先证 $(a^k)^{d_1}=e$:
		\[
		(a^k)^{d_1}=a^{k d_1}=a^{gk_1 d_1}=a^{k_1\cdot (g d_1)}=a^{k_1 d}=(a^d)^{k_1}=e.
		\]
		故 $\mathrm{ord}(a^k)\mid d_1$。
		
		再证若 $(a^k)^m=e$,则 $d_1\mid m$。
		由 $(a^k)^m=e$ 得 $a^{km}=e$。因为 $\mathrm{ord}(a)=d$,故
		\[
		a^{km}=e \iff d\mid km.
		\]
		代入 $d=g d_1,\ k=g k_1$,得
		\[
		g d_1 \mid g k_1 m \ \Longrightarrow\ d_1\mid k_1 m.
		\]
		又因 $(d_1,k_1)=1$,由互素可消去性推出 $d_1\mid m$。
		因此使 $(a^k)^m=e$ 的正整数 $m$ 必为 $d_1$ 的倍数,结合上一步 $(a^k)^{d_1}=e$,可知最小的这样的 $m$ 恰为 $d_1$,即
		\[
		\mathrm{ord}(a^k)=d_1=\frac{d}{g}=\frac{d}{(d,k)}.
		\]
		
		\textbf{(2)} 由(1)立得
		\[
		\mathrm{ord}(a^k)=d
		\iff \frac{d}{(d,k)}=d
		\iff (d,k)=1.
		\]
		证完.
	
		\item 10.
\textbf{题目.}\;
设交换群 $G$ 中元素的最大阶为 $n\in\mathbb N^{*}$,
则 $G$ 中每个元素的阶都是 $n$ 的因子。

\textbf{答案:}\;
\textbf{证明.}
设 $n$ 为 $G$ 中元素阶的最大值,即存在 $a\in G$ 使得 $\mathrm{ord}(a)=n$,
且对任意 $x\in G$ 都有 $\mathrm{ord}(x)\le n$。

任取 $g\in G$,记 $\mathrm{ord}(g)=m$。若 $m\nmid n$,令
\[
d=(m,n),\qquad m=d m_1,\ n=d n_1,\qquad (m_1,n_1)=1.
\]
由于 $m\nmid n$,故 $m_1>1$。取素数 $p\mid m_1$,则 $p\mid m$ 且 $p\nmid n$
(因为 $p\mid m_1$ 且 $(m_1,n_1)=1$,从而 $p\nmid n_1$,又 $n=dn_1$ 且 $p\mid m_1$ 与 $p\mid d$ 不可能同时成立,否则 $p$ 会被吸收到 $d$ 里,矛盾;
等价地说:$p\mid m$ 但 $p\nmid n$)。

由 $p\mid m=\mathrm{ord}(g)$,存在元素
\[
u:=g^{m/p}
\]
满足 $\mathrm{ord}(u)=p$(因为若 $\mathrm{ord}(g)=m$,则 $\mathrm{ord}(g^{m/p})=p$)。

另一方面,由 $\mathrm{ord}(a)=n$ 且 $p\nmid n$,可知
\[
\mathrm{ord}(a)=n,\qquad p\nmid n.
\]
由于 $G$ 交换,$a$ 与 $u$ 可交换,且 $(\mathrm{ord}(a),\mathrm{ord}(u))=(n,p)=1$。
于是由交换群中“互素阶元素乘积的阶等于阶的乘积”的结论,
\[
\mathrm{ord}(au)=\mathrm{ord}(a)\,\mathrm{ord}(u)=np.
\]
从而 $\mathrm{ord}(au)=np>n$,这与 $n$ 是最大阶矛盾。

因此假设 $m\nmid n$ 不成立,只能有 $m\mid n$。
任意 $g\in G$ 均如此,故 $G$ 中每个元素的阶都是 $n$ 的因子。

	
		\item 16.
		\textbf{题目.}\;
		设 $H,K$ 是群 $G$ 的两个子群,定义
		\[
		HK=\{\,hk\mid h\in H,\ k\in K\,\}.
		\]
		\textbf{(1)} 证明:当 $H$ 和 $K$ 都是有限群时,
		\[
		|HK|=\frac{|H||K|}{|H\cap K|}.
		\]
		\textbf{(2)} 证明:$HK$ 是 $G$ 的子群当且仅当 $HK=KH$。
		
		\textbf{答案:}\;
		\textbf{证明.}
		
		\textbf{(1)} 令 $H\times K$ 上定义等价关系:对 $(h,k),(h_1,k_1)\in H\times K$,
		\[
		(h,k)\sim (h_1,k_1)\iff hk=h_1k_1.
		\]
		
		\textbf{(1) 自反性.}\;
		任取 $(h,k)\in H\times K$,显然 $hk=hk$,故
		\[
		(h,k)\sim(h,k).
		\]
		
		\textbf{(2) 对称性.}\;显然
		
		\textbf{(3) 传递性.}\;
		若 $(h,k)\sim(h_1,k_1)$ 且 $(h_1,k_1)\sim(h_2,k_2)$,
		则
		\[
		hk=h_1k_1,\qquad h_1k_1=h_2k_2.
		\]
		由等号传递性得 $hk=h_2k_2$,故
		\[
		(h,k)\sim(h_2,k_2).
		\]
		
		综上,$\sim$ 为等价关系。
		
		
		固定任意 $(h,k)\in H\times K$,考虑它的等价类
		\[
		[(h,k)]=\{(h_1,k_1)\in H\times K:\ hk=h_1k_1\}.
		\]
		对任意 $(h_1,k_1)\in[(h,k)]$,由 $hk=h_1k_1$ 得
		\[
		h^{-1}h_1=kk_1^{-1}.
		\]
		左边属于 $H$,右边属于 $K$,故其公共值属于 $H\cap K$。记
		\[
		q:=h^{-1}h_1=kk_1^{-1}\in H\cap K.
		\]
		于是
		\[
		h_1=hq,\qquad k_1=q^{-1}k.
		\]
		反过来,对任意 $q\in H\cap K$,令 $h_1=hq,\ k_1=q^{-1}k$,则
		\[
		h_1k_1=(hq)(q^{-1}k)=hk,
		\]
		因此 $(hq,q^{-1}k)\in[(h,k)]$。
		
		综上,
		\[
		[(h,k)]=\{(hq,q^{-1}k)\mid q\in H\cap K\},
		\]
		从而每个等价类都恰有 $|H\cap K|$ 个元素,即
		\[
		|[(h,k)]|=|H\cap K|.
		\]
		
		另一方面,$H\times K$ 的等价类与 $HK$ 的元素一一对应:
		\textbf{映射的构造.}\;
		在 $H\times K$ 上定义等价关系
		\[
		(h,k)\sim(h_1,k_1)\iff hk=h_1k_1.
		\]
		令商集为
		\[
		(H\times K)/\sim=\{\,[(h,k)]\mid (h,k)\in H\times K\,\}.
		\]
		构造映射
		\[
		\Phi:(H\times K)/\sim\ \longrightarrow\ HK,\qquad 
		\Phi\bigl([(h,k)]\bigr)=hk.
		\]
		
		\textbf{(1) $\Phi$ 良定义.}\;
		若 $[(h,k)]=[(h_1,k_1)]$,则 $(h,k)\sim(h_1,k_1)$,即 $hk=h_1k_1$,
		故
		\[
		\Phi([(h,k)])=hk=h_1k_1=\Phi([(h_1,k_1)]).
		\]
		因此 $\Phi$ 与代表元选取无关,良定义。
		
		\textbf{(2) $\Phi$ 为双射.}\;
		
		\textbf{满射:}\;
		任取 $x\in HK$,则存在 $h\in H,k\in K$ 使 $x=hk$,于是
		\[
		x=\Phi([(h,k)]).
		\]
		
		\textbf{单射:}\;
		若 $\Phi([(h,k)])=\Phi([(h_1,k_1)])$,则 $hk=h_1k_1$,
		从而 $(h,k)\sim(h_1,k_1)$,即 $[(h,k)]=[(h_1,k_1)]$。
		
		综上,$\Phi$ 是从等价类集合到 $HK$ 的双射,因此等价类个数等于 $|HK|$。
		
		于是把 $|H\times K|=|H||K|$ 按等价类分块计数,得到
		\[
		|H||K|=|H\times K|
		=\sum_{\text{等价类 }C}|C|
		=|HK|\cdot |H\cap K|.
		\]
		故
		\[
		|HK|=\frac{|H||K|}{|H\cap K|}.
		\]
		
	
		
		
		\item 17.
		\textbf{题目.}\;
		设 $H_1,H_2$ 为有限群 $G$ 的两个子群且 $H_1\le H_2$,证明
		\[
		[G:H_1]=[G:H_2]\,[H_2:H_1].
		\]
		
		\textbf{答案:}\;
		\textbf{证明.}
		由于 $G$ 为有限群,且 $H_1\le H_2\le G$,由拉格朗日定理可得
		\[
		|G|=|H_2|\,[G:H_2],\qquad |H_2|=|H_1|\,[H_2:H_1].
		\]
		将第二式代入第一式,得到
		\[
		|G|=\bigl(|H_1|\,[H_2:H_1]\bigr)\,[G:H_2]
		=|H_1|\,[G:H_2]\,[H_2:H_1].
		\]
		两边同除以 $|H_1|$(注意 $|H_1|\neq 0$),便有
		\[
		\frac{|G|}{|H_1|}=[G:H_2]\,[H_2:H_1].
		\]
		而在有限群情形下,指数满足 $[G:H_1]=\dfrac{|G|}{|H_1|}$,故
		\[
		[G:H_1]=[G:H_2]\,[H_2:H_1].
		\]
		因此命题成立。
		\item 20.
		\textbf{题目.}\;
		设 $H_1,H_2$ 为有限群 $G$ 的两个子群,证明
		\[
		[G:H_1\cap H_2]\le [G:H_1]\,[G:H_2].
		\]
		又若 $[G:H_1]$ 与 $[G:H_2]$ 互素,则
		\[
		[G:H_1\cap H_2]=[G:H_1]\,[G:H_2]\quad\text{且}\quad G=H_1H_2.
		\]
		
		\textbf{答案:}\;
		\textbf{证明.}
		
		\medskip
		\textbf{(一) 先证基本不等式 $[G:H_1\cap H_2]\le [G:H_1][G:H_2]$.}
		

		由($H_1,H_2$ 有限)
		\[
		|H_1H_2|=\frac{|H_1||H_2|}{|H_1\cap H_2|}.
		\]
		又因为 $H_1H_2\subseteq G$,故 $|H_1H_2|\le |G|$,代入得
		\[
		\frac{|H_1||H_2|}{|H_1\cap H_2|}\le |G|
		\quad\Longrightarrow\quad
		\frac{|G|}{|H_1\cap H_2|}\le \frac{|G|}{|H_1|}\cdot \frac{|G|}{|H_2|}.
		\]
		注意到 $[G:H]=\dfrac{|G|}{|H|}$($G$ 有限)即得
		\[
		[G:H_1\cap H_2]\le [G:H_1]\,[G:H_2].
		\]
		这就证得第一部分。 \hfill$\square$
		
		\medskip
		\textbf{(二) 若 $[G:H_1]$ 与 $[G:H_2]$ 互素,则推出取等并有 $G=H_1H_2$.}
		
		由指数的乘法公式(对任意子群链 $H_1\cap H_2\le H_1\le G$),
		\[
		[G:H_1\cap H_2]=[G:H_1]\,[H_1:H_1\cap H_2],
		\]
		从而 $[G:H_1]\mid [G:H_1\cap H_2]$。同理也有 $[G:H_2]\mid [G:H_1\cap H_2]$。
		若再假设 $\gcd([G:H_1],[G:H_2])=1$,则
		\[
		[G:H_1]\,[G:H_2]\mid [G:H_1\cap H_2].
		\]
		与(一)中已证不等式
		\[
		[G:H_1\cap H_2]\le [G:H_1]\,[G:H_2]
		\]
		合并,立刻得到
		\[
		[G:H_1\cap H_2]=[G:H_1]\,[G:H_2].
		\]
		
		接下来由(一)中的计数式
		\[
		|H_1H_2|=\frac{|H_1||H_2|}{|H_1\cap H_2|}
		\]
		并将上式“取等”改写为
		\[
		\frac{|G|}{|H_1\cap H_2|}=\frac{|G|}{|H_1|}\cdot \frac{|G|}{|H_2|}
		\quad\Longleftrightarrow\quad
		|G|=\frac{|H_1||H_2|}{|H_1\cap H_2|}=|H_1H_2|.
		\]
		由于 $H_1H_2\subseteq G$ 且二者基数相同,必有 $H_1H_2=G$,即 $G=H_1H_2$。
		证毕。 \hfill$\square$
		
	\end{enumerate}
	% -------------------- 1.3 正规子群与商群 --------------------
	\clearpage
	\section{正规子群与商群}
	\subsection{指数为 2 的子群必为正规子群}
	\textbf{命题.}\ 若 $[G:H]=2$,则 $H\triangleleft G$。\\
	\textbf{证明.}
\textbf{命题.}
设 $G$ 为群,$H\le G$ 且 $[G:H]=2$,则 $H\lhd G$。

\textbf{证明.}

由于 $[G:H]=2$,故 $G$ 关于 $H$ 的左陪集只有两个:
\[
G=H\ \sqcup\ gH
\]
其中 $g\in G\setminus H$。

同理,$G$ 关于 $H$ 的右陪集也只有两个:
\[
G=H\ \sqcup\ Hg.
\]

注意到 $g\notin H$,因此
\[
gH\neq H,\qquad Hg\neq H.
\]
但左陪集与右陪集的个数都等于 $2$,而它们都与 $H$ 构成对 $G$ 的划分,
故只能有
\[
gH=Hg.
\]

于是对任意 $g\in G$,都有
\[
gH=Hg,
\]
这正是 $H$ 在 $G$ 中正规($H\lhd G$)的定义。

故 $H\lhd G$。

	
	
	\subsection{正规化子(Normalizer)}
	\textbf{定义.}
	设 $H$ 是群 $G$ 的一个子群,定义
	\[
	N_G(H) = \{ g \in G \mid gHg^{-1} = H \},
	\]
	称为 $H$ 在 $G$ 中的正规化子。\\则有以下性质:
	
	\begin{enumerate}
		\item $g\in N_G(H)\ \Longleftrightarrow\ (\forall h\in H)\; g^{-1}hg\in H$
		\item $N_G(H) < G$;
		\item $H \triangleleft N_G(H)$;
		\item $H \triangleleft G \iff N_G(H) = G$;
		\item $N_G(H\cap K) \ge N_G(H)\cap N_G(K)$;
		\item 若 $H,K \triangleleft N_G(H)\cap N_G(K)$,则 $N_G(HK)=N_G(H)\cap N_G(K)$;
		\item $N_G(H)$ 是 $G$ 作用在子群集上时的稳定子群;
		\item $[G : N_G(H)]$ 等于 $H$ 的不同共轭子群的个数。
	\end{enumerate}
	
	\textbf{证明.}
	1.\textbf{(1) 充分性.} 若 $g\in N_G(H)$,则按定义有
	\[
	\mathrm{ad}_g(H)=gHg^{-1}=H.
	\]
	于是对任意 $h\in H$,有
	\[
	\mathrm{ad}_{g^{-1}}(h)=g^{-1}hg\in H,
	\]
	命题右侧成立。
	
	\textbf{(2) 必要性.} 若对任意 $h\in H$ 都有 $g^{-1}hg\in H$,则
	\[
	\mathrm{ad}_{g^{-1}}(H)\subseteq H.
	\]
	对两边施加双射 $\mathrm{ad}_g$,得
	\[
	H=\mathrm{ad}_g(\mathrm{ad}_{g^{-1}}(H))\subseteq \mathrm{ad}_g(H)=gHg^{-1}.
	\]
	另一方面,由 $\mathrm{ad}_{g^{-1}}(H)\subseteq H$ 等价地也有 $gHg^{-1}\subseteq H$。
	综上,两边同时包含,故
	\[
	gHg^{-1}=H.
	\]
	因此 $g\in N_G(H)$。
	
	\textbf{证毕.}
	
	

\subsection{若 $N\triangleleft G$ 且 $H\le G$,则$HN = NH$}
\textbf{命题.} 若 $N\triangleleft G$ 且 $H\le G$,则
\[
HN = NH.
\]

\textbf{证明.}
任取 $h\in H$、$n\in N$。由于 $N\triangleleft G$,共轭闭性给出
\[
hnh^{-1} \in N.
\]
于是
\[
hn = hnh^{-1}\,h.
\]
上式右边是一个 $N$ 中的元素与一个 $H$ 中的元素相乘,因此
\[
hn \in NH.
\]
由于 $h\in H$、$n\in N$ 任取,便有
\[
HN \subseteq NH.
\]

另一方面,任取 $n\in N$、$h\in H$,同理仍有
\[
nh = h(h^{-1}nh),
\]
而 $h^{-1}nh\in N$(因为 $N$ 在 $G$ 中正规),故
\[
nh \in HN.
\]
因此
\[
NH \subseteq HN.
\]

综上,
\[
HN = NH.
\]
\qed


\subsection{若 \(|G|=p\)(\(p\) 为素数),则 \(G\cong \mathbb{Z}/p\mathbb{Z}\)}
\textbf{命题.}\;
\text{若有限群 \(G\) 满足 \(|G|=p\)(\(p\) 为素数),则 \(G\cong \mathbb{Z}/p\mathbb{Z}\)。}


\textbf{证明:}


取 \(g\in G\) 且 \(g\neq e\)。由于 \(\langle g\rangle\) 是 \(G\) 的子群,按 Lagrange 定理有
\[
|\langle g\rangle|\mid |G|=p.
\]
又因为 \(g\neq e\),故 \(|\langle g\rangle|\neq 1\),从而只能有
\[
|\langle g\rangle|=p.
\]
于是
\[
\langle g\rangle \text{ 是阶为 }p\text{ 的子群}.
\]
但 \(G\) 本身也阶为 \(p\),因此
\[
\langle g\rangle = G.
\]
这说明 \(G\) 是循环群,由 \(g\) 生成。

\medskip

再定义映射
\[
\varphi:\mathbb{Z}/p\mathbb{Z}\longrightarrow G,\qquad \varphi(\bar{k})=g^{\,k}\quad (k=0,1,\dots,p-1).
\]
\textbf{(1) \(\varphi\) 是群同态.}\;
对任意 \(\bar{k},\bar{\ell}\in \mathbb{Z}/p\mathbb{Z}\),有
\[
\varphi(\bar{k}+\bar{\ell})
=\varphi(\overline{k+\ell})
=g^{k+\ell}
=g^k g^\ell
=\varphi(\bar{k})\,\varphi(\bar{\ell}).
\]
故 \(\varphi\) 为同态。

\medskip

\textbf{(2) \(\varphi\) 是满射.}\;
因为 \(G=\langle g\rangle\),任意 \(x\in G\) 都可写为 \(x=g^k\)(某个 \(k\)),于是 \(x=\varphi(\bar{k})\),故 \(\varphi\) 满射。

\medskip

\textbf{(3) \(\varphi\) 是单射.}\;
若 \(\varphi(\bar{k})=e\),则 \(g^k=e\),从而 \(p\mid k\)(因为 \(|g|=p\))。
于是 \(\bar{k}=\bar{0}\),故 \(\ker(\varphi)=\{\bar{0}\}\),即 \(\varphi\) 单射。

\medskip

由 (1)(2)(3) 可知 \(\varphi\) 为同构,因此
\[
G \cong \mathbb{Z}/p\mathbb{Z}.
\]
\clearpage
\subsection*{课后习题答案}
\addcontentsline{toc}{subsection}{\textcolor{red}{课后习题答案}}
\begin{enumerate}[label=\textcolor{blue}{\textbf{\large\arabic*.}}]
	\item 2.
	\textbf{题目.}\;
	设 $H$ 是群 $G$ 的子群,则 $H$ 是 $G$ 的正规子群当且仅当 $H$ 是 $G$ 的一些共轭类的并。
	
	\textbf{答案:}\;
	
	\textbf{共轭与共轭类的定义(对应你图里那段).}\;
	给定群 $G$。若存在 $g\in G$ 使
	\[
	b=gag^{-1},
	\]
	则称 $a,b\in G$ \textbf{共轭}。与 $a$ 共轭的全体元素组成集合
	\[
	C(a):=\{\,gag^{-1}\mid g\in G\,\},
	\]
	称为 $a$ 的\textbf{共轭类}(也常记作 $Cl_G(a)$)。
	
	\medskip
	\textbf{证明.}
	
	\textbf{($\Rightarrow$)}\;
	设 $H\trianglelefteq G$。取任意 $h\in H$,由正规性知对任意 $g\in G$,
	\[
	ghg^{-1}\in H,
	\]
	因此 $C(h)\subseteq H$。又因为 $H$ 中每个元素都属于它自己的共轭类,故
	\[
	H=\bigcup_{h\in H} C(h),
	\]
	即 $H$ 是一些共轭类的并。
	
	\textbf{($\Leftarrow$)}\;
	反之,设 $H\le G$ 且 $H$ 是 $G$ 的一些共轭类的并。取任意 $h\in H$ 与任意 $g\in G$。
	由于 $H$ 是共轭类的并,存在某个元素 $x\in H$ 使得 $h\in C(x)$,即 $h$ 与 $x$ 共轭。
	但 $ghg^{-1}$ 与 $h$ 共轭,因此也与 $x$ 共轭,从而
	\[
	ghg^{-1}\in C(x)\subseteq H.
	\]
	这说明对任意 $g\in G,h\in H$ 都有 $ghg^{-1}\in H$,于是 $H\trianglelefteq G$。
	
	综上,命题成立。 \hfill$\square$
	
	\medskip
	\textbf{(顺带:你图里“思考题 1.3.2”的一个例子)}\;
	在 $G=S_3$ 中取
	\[
	S=\{e\}\cup\{(12),(13),(23)\}.
	\]
	$S$ 是“共轭不变”的(因为三个换位互为共轭),即对任意 $g\in S_3,\ h\in S$ 有 $ghg^{-1}\in S$;
	但 $S$ 不是子群,因为 $(12)(13)=(132)\notin S$。
	
	\item 9.
	\textbf{题目.}\;
	设 $H,K$ 是群 $G$ 的两个子群。证明:
	\textbf{(1)} 若 $H,K$ 中有一个是 $G$ 的正规子群,则 $HK<G$(可用习题 1.2 习题 16:$HK<G\iff HK=KH$);
	\textbf{(2)} 若 $H,K$ 均是 $G$ 的正规子群,则 $HK\lhd G$。
	
	\textbf{答案:}\;
	\textbf{证明.}
	
	\textbf{(1)} 不妨设 $H\lhd G$(若 $K\lhd G$ 同理)。
	任取 $h\in H,\ k\in K$,由于 $H\lhd G$,有
	\[
	k^{-1}hk\in H.
	\]
	于是
	\[
	hk=k(k^{-1}hk)\in KH,
	\]
	从而 $HK\subseteq KH$。
	
	反过来任取 $k\in K,\ h\in H$,同样由 $H\lhd G$ 得
	\[
	kh=(khk^{-1})k,\qquad khk^{-1}\in H,
	\]
	故 $kh\in HK$,从而 $KH\subseteq HK$。
	
	综上 $HK=KH$。由习题 16($HK<G \iff HK=KH$),可得 $HK<G$。
	
	\textbf{(2)} 设 $H\lhd G$ 且 $K\lhd G$。由 (1) 知 $HK<G$。
	再证 $HK$ 在 $G$ 中正规:任取 $g\in G$ 与任意 $x\in HK$,
	可写 $x=hk$($h\in H,k\in K$),则
	\[
	gxg^{-1}=g(hk)g^{-1}=(ghg^{-1})(gkg^{-1}).
	\]
	因 $H\lhd G,\ K\lhd G$,有 $ghg^{-1}\in H$ 且 $gkg^{-1}\in K$,故
	\[
	gxg^{-1}\in HK,
	\]
	即 $g(HK)g^{-1}\subseteq HK$。将 $g$ 换成 $g^{-1}$ 同理得 $HK\subseteq g(HK)g^{-1}$,
	于是
	\[
	g(HK)g^{-1}=HK.
	\]
	因此 $HK\lhd G$。\;\(\square\)
	
\end{enumerate}


	
	% -------------------- 1.4 群的同态与同构 --------------------
		\clearpage
	\section{群的同态与同构}
	\subsection{定理1.4.21的推论}
	\textbf{命题.}\;
	设 $G$ 是群,$N \trianglelefteq G$。令自然同态
	\[
	\pi : G \to G/N, \qquad \pi(g) = gN.
	\]
	若 $H' \le G/N$,定义其在 $G$ 中的完全原像为
	\[
	K = \pi^{-1}(H') = \{\, g \in G \mid gN \in H' \,\}.
	\]
	证明:$H' = K/N$。
	
	\textbf{证明.}
	
	\textbf{(1) 证明 $K/N \subseteq H'$.}
	
	任取 $kN \in K/N$,则 $k \in K$。由 $K = \pi^{-1}(H')$ 的定义可知:
	\[
	\pi(k) = kN \in H'.
	\]
	因此 $kN \in H'$,从而
	\[
	K/N \subseteq H'.
	\]
	
	\textbf{(2) 证明 $H' \subseteq K/N$.}
	
	任取 $h'N \in H'$。由于 $\pi$ 是满射,存在 $h \in G$ 使得 $\pi(h) = hN = h'N$。
	于是 $hN \in H'$,因此 $h \in \pi^{-1}(H') = K$,从而
	\[
	hN \in K/N.
	\]
	这说明 $H' \subseteq K/N$。
	
	\textbf{(3) 合并两边包含关系.}
	
	由 (1)、(2) 可得
	\[
	K/N = H'.
	\]
	
	\textbf{(4) 补充说明.}
	由于 $N \trianglelefteq G$,自然同态 $\pi$ 满足
	\[
	\pi(K) = H', \qquad \ker \pi = N,
	\]
	因此 $K/N$ 正是 $H'$ 在 $G/N$ 中的原像与像的对应关系的具体体现,
	这也表明 $K \mapsto K/N$ 给出了
	\[
	\{\, K \mid N \le K \le G \,\}
	\quad\text{与}\quad
	\{\, H' \mid H' \le G/N \,\}
	\]
	之间的一一对应。\qed
	
	
	\subsection{阶为 4 的群只有两种同构类型}
	\text{命题:阶为 \(4\) 的群只有两种同构类型,即任意满足 \(|G|=4\) 的群 \(G\),必有}
	\[
	G \cong \mathbb{Z}/4\mathbb{Z}\quad \text{或}\quad K_4\ (\text{Klein 四元群}).
	\]
	
	
	\textbf{证明:}

	设 \(|G|=4\)。
	
	\textbf{情形 1:\(G\) 含有阶为 \(4\) 的元素。}\;
	若存在 \(g\in G\) 使得 \(|g|=4\),则 \(\langle g\rangle\) 是 \(G\) 的子群且
	\[
	|\langle g\rangle|=|g|=4=|G|.
	\]
	因此 \(\langle g\rangle=G\),故 \(G\) 为循环群,从而
	\[
	G \cong \mathbb{Z}/4\mathbb{Z}.
	\]
	
	\textbf{情形 2:\(G\) 不含阶为 \(4\) 的元素。}\;
	此时对任意 \(x\in G\setminus\{e\}\),由 Lagrange 定理知 \(|x|\mid 4\),且 \(|x|\neq 1\)。
	又因为不存在阶为 \(4\) 的元素,只能有
	\[
	|x|=2 \qquad (\forall\,x\neq e).
	\]
	因此 \(G\) 的三个非单位元全都是 \(2\) 阶元。
	
	取两个不同的非单位元 \(a,b\in G\)(必存在,因为 \(|G\setminus\{e\}|=3\))。
	令
	\[
	c:=ab.
	\]
	下面证明 \(G=\{e,a,b,c\}\) 且满足 Klein 四元群的乘法关系。
	
	\textbf{(i) \(c\neq e,a,b\).}\;
	若 \(c=e\),则 \(ab=e\Rightarrow a=b^{-1}=b\)(因 \(b^2=e\Rightarrow b^{-1}=b\)),矛盾;
	若 \(c=a\),则 \(ab=a\Rightarrow b=e\),矛盾;
	若 \(c=b\),则 \(ab=b\Rightarrow a=e\),矛盾。
	故 \(c\) 与 \(e,a,b\) 两两不同,因此
	\[
	G=\{e,a,b,c\},\qquad c=ab.
	\]
	
	\textbf{(ii) \(c\) 也是 \(2\) 阶元,且 \(c^2=e\).}\;
	由上一步知 \(c\neq e\),而情形 2 中任意非单位元阶为 \(2\),故
	\[
	c^2=e.
	\]
	
	\textbf{(iii) \(a,b,c\) 两两可交换,从而 \(G\) 阿贝尔。}\;
	由 \(c=ab\) 及 \(c^2=e\) 得
	\[
	e=c^2=(ab)(ab)=a(ba)b,
	\]
	两边左乘 \(a\),右乘 \(b\),得到
	\[
	a = (ba)b \quad\Rightarrow\quad ab=ba.
	\]
	于是 \(a\) 与 \(b\) 交换。再由 \(c=ab\) 可推出
	\[
	ac=a(ab)=(aa)b=eb=b,\qquad ca=(ab)a=a(ba)=(aa)b=b,
	\]
	故 \(ac=ca\);同理
	\[
	bc=b(ab)=(ba)b=a(bb)=a,\qquad cb=(ab)b=a(bb)=a,
	\]
	故 \(bc=cb\)。因此 \(G\) 为阿贝尔群。
	
	\textbf{(iv) 乘法表(结构)与 \(K_4\) 一致。}\;
	由 \(a^2=b^2=c^2=e\) 以及 \(c=ab\)、交换性,可得
	\[
	ab=c,\quad bc=a,\quad ca=b,
	\]
	并且 \(a,b,c\) 都是 \(2\) 阶元。
	这正是 Klein 四元群 \(K_4\) 的定义关系。
	
	\textbf{(v) 构造同构 \(G\cong \mathbb{Z}/2\mathbb{Z}\times \mathbb{Z}/2\mathbb{Z}\).}\;
	定义映射
	\[
	\varphi: \mathbb{Z}/2\mathbb{Z}\times \mathbb{Z}/2\mathbb{Z}\to G
	\]
	满足
	\[
	\varphi(\bar{0},\bar{0})=e,\quad
	\varphi(\bar{1},\bar{0})=a,\quad
	\varphi(\bar{0},\bar{1})=b,\quad
	\varphi(\bar{1},\bar{1})=c=ab.
	\]
	由于 \(\mathbb{Z}/2\mathbb{Z}\times \mathbb{Z}/2\mathbb{Z}\) 中每个非零元都是 \(2\) 阶,
	且加法规则与 \(a,b,c\) 的乘法关系一一对应(例如 \((\bar{1},\bar{0})+(\bar{0},\bar{1})=(\bar{1},\bar{1})\) 对应 \(ab=c\)),
	可直接验证 \(\varphi\) 是群同态。
	并且 \(\varphi\) 显然双射(四个元素一一对应),故 \(\varphi\) 为同构。
	因此
	\[
	G \cong \mathbb{Z}/2\mathbb{Z}\times \mathbb{Z}/2\mathbb{Z} \cong K_4.
	\]
	
	\textbf{综上.}\;
	阶为 \(4\) 的群 \(G\) 要么含阶为 \(4\) 的元素从而 \(G\cong \mathbb{Z}/4\mathbb{Z}\),
	要么不含阶为 \(4\) 的元素从而 \(G\cong K_4\)。
		
	\clearpage
	\subsection*{课后习题答案}
	\addcontentsline{toc}{subsection}{\textcolor{red}{课后习题答案}}
	\begin{enumerate}[label=\textcolor{blue}{\textbf{\large\arabic*.}}]
		\item 证明 :6 阶群必与 $\mathrm{Z}$ 或$S_6$同构.
		
		
		\textbf{证明.}
		设 $G$ 为 $6$ 阶群,则
		\[
		|G|=6=2\cdot 3.
		\]
		
		设 $n_3$ 为 Sylow $3$-子群的个数。由 Sylow 定理,
		\[
		n_3\mid 2,\qquad n_3\equiv 1\pmod 3,
		\]
		故 $n_3=1$。
		
		因此 Sylow $3$-子群 $P$ 在 $G$ 中唯一,从而 $P\lhd G$,且 $|P|=3$。
		
		再设 $n_2$ 为 Sylow $2$-子群的个数,则
		\[
		n_2\mid 3,\qquad n_2\equiv 1\pmod 2,
		\]
		故 $n_2=1$ 或 $3$。
		
		\textbf{情形一:$n_2=1$.}
		
	由于 $n_2=1$,Sylow $2$-子群 $Q$ 唯一,故 $Q\lhd G$,
	且 $[G:Q]=3$。
	于是存在满射
	\[
	\pi:G\to G/Q,
	\]
	其中 $G/Q$ 为 $3$ 阶循环群。
	取 $g\in G$ 使 $\pi(g)$ 生成 $G/Q$,
	则 $g^3\in Q$。
	又 $|Q|=2$,故 $g^6=e$,
	从而 $|g|=6$,
	$G$ 为循环群。
	
		
		\textbf{情形二:$n_2=3$.}
		
	\textbf{证明(构造同构).}
	
	设 $|G|=6$,且 $n_2=3$。
	记 $X=\{Q_1,Q_2,Q_3\}$ 为 $G$ 的全部 Sylow $2$-子群的集合,
	其中每个 $Q_i$ 的阶为 $2$。
	
	\textbf{(1) 构造作用.}
	$G$ 通过共轭作用在 $X$ 上:
	\[
	g\cdot Q := gQg^{-1}\qquad (g\in G,\ Q\in X).
	\]
	该作用诱导一个群同态
	\[
	\varphi: G \longrightarrow S_X \cong S_3.
	\]
	
	\textbf{(2) 证明 $\varphi$ 为单射.}
	设 $g\in\ker\varphi$,则对任意 $Q\in X$,
	\[
	gQg^{-1}=Q.
	\]
	即 $g$ 属于每个 Sylow $2$-子群的正规化子。
	
	由于 $n_2=3>1$,不存在正规 Sylow $2$-子群,
	从而
	\[
	\bigcap_{i=1}^3 N_G(Q_i)=\{e\}.
	\]
	故 $\ker\varphi=\{e\}$,即 $\varphi$ 为单射。
	
	\textbf{(3) 推出同构.}
	由于
	\[
	|G|=6,\qquad |S_3|=6,
	\]
	而 $\varphi$ 为单射群同态,故 $\varphi$ 亦为满射,
	从而为同构。
	
	因此
	\[
	G \cong S_3.
	\]
	证毕。
	
		
		\item \textbf{证明.}
		
		定义映射
		\[
		\exp:\mathbb R\to\mathbb R^+,\qquad \exp(x)=e^x.
		\]
		
		\textbf{(1) 保运算.}
		对任意 $x,y\in\mathbb R$,
		\[
		\exp(x+y)=e^{x+y}=e^x e^y=\exp(x)\cdot\exp(y),
		\]
		故 $\exp$ 是群同态。
		
		\textbf{(2) 单射.}
		若 $\exp(x)=\exp(y)$,则
		\[
		e^x=e^y \Rightarrow x=y,
		\]
		故 $\exp$ 为单射。
		
		\textbf{(3) 满射.}
		对任意 $a\in\mathbb R^+$,存在 $x=\ln a\in\mathbb R$ 使
		\[
		\exp(x)=e^{\ln a}=a,
		\]
		故 $\exp$ 为满射。
		
		综上,$\exp$ 是从加法群 $\{\mathbb R;+\}$ 到乘法群
		$\{\mathbb R^+;\cdot\}$ 的双射群同态,因此是群同构。
		证毕。
		\item 5.
		\textbf{题目.}\;
		设 $G$ 是有限 Abel 群,定义映射
		\[
		\varphi_k:G\to G,\qquad \varphi_k(g)=g^k.
		\]
		证明:$\varphi_k$ 是 $G$ 的自同态;并且 $\varphi_k$ 是 $G$ 的自同构当且仅当 $(k,|G|)=1$。
		
		\textbf{答案:}
		
		\textbf{(1) $\varphi_k$ 是自同态.}\;
		任取 $g,h\in G$,由于 $G$ 交换,有
		\[
		\varphi_k(gh)=(gh)^k=\underbrace{(gh)(gh)\cdots(gh)}_{k\text{ 次}}
		=\underbrace{g g\cdots g}_{k\text{ 次}}\;\underbrace{h h\cdots h}_{k\text{ 次}}
		=g^k h^k
		=\varphi_k(g)\varphi_k(h).
		\]
		故 $\varphi_k$ 是群同态(即 $G$ 的自同态)。
		
		\textbf{(2) 若 $(k,|G|)=1$,则 $\varphi_k$ 为自同构.}\;
		只需证 $\varphi_k$ 单射(因为为有限群)。设 $g\in \mathrm{ker}(\varphi_k)$,则
		\[
		\varphi_k(g)=e \iff g^k=e.
		\]
		令 $|g|=m$($g$ 的阶)。由 $g^k=e$ 得 $m\mid k$;又因 $g\in G$ 且 $G$ 有限,必有 $m\mid |G|$。
		因此
		\[
		m \mid k\quad \text{且}\quad m\mid |G|
		\;\Longrightarrow\;
		m \mid (k,|G|)=1
		\;\Longrightarrow\;
		m=1
		\;\Longrightarrow\;
		g=e.
		\]
		所以 $\mathrm{ker}(\varphi_k)=\{e\}$,$\varphi_k$ 单射。由于 $G$ 有限,单射等价于满射,故 $\varphi_k$ 为双射,从而为自同构。
		
		\textbf{(3)(反证法) 若 $(k,|G|)\neq 1$,则 $\varphi_k$ 不是自同构.}\;
		
		设 $d=(k,|G|)>1$,取素数 $p\mid d$,则 $p\mid |G|$ 且 $p\mid k$。
		由 Cauchy 定理(有限群 $G$ 中若素数 $p\mid |G|$,则存在阶为 $p$ 的元素),存在 $a\in G$ 使 $|a|=p$,特别地 $a\neq e$ 且 $a^p=e$。
		又因 $p\mid k$,可写 $k=pt$,于是
		\[
		\varphi_k(a)=a^k=a^{pt}=(a^p)^t=e.
		\]
		故 $a\in \mathrm{ker}(\varphi_k)$ 且 $a\neq e$,从而 $\mathrm{ker}(\varphi_k)\neq \{e\}$,$\varphi_k$ 非单射,因而不可能是自同构。
		
		\textbf{(4) 结论.}\;
		综上,
		\[
		\varphi_k \text{ 是 }G\text{ 的自同构}\;\Longleftrightarrow\;(k,|G|)=1.
		\]
		\item \textbf{题目 9.}\;
		设 $f$ 和 $g$ 都是群 $G$ 到群 $H$ 的同态,令
		\[
		D=\{x\in G\mid f(x)=g(x)\}.
		\]
		证明:$D<G$。
		
		\textbf{答案:}
		
		要证 $D$ 为 $G$ 的子群,只需验证:$e\in D$,且对任意 $x,y\in D$ 有 $xy^{-1}\in D$。
		
		\textbf{(1) 单位元属于 $D$.}\;
		因为 $f,g$ 都是同态,所以
		\[
		f(e)=e_H,\qquad g(e)=e_H,
		\]
		因此 $f(e)=g(e)$,从而 $e\in D$。
		
		\textbf{(2) 封闭性:若 $x,y\in D$,则 $xy^{-1}\in D$.}\;
		由 $x\in D$ 得 $f(x)=g(x)$;由 $y\in D$ 得 $f(y)=g(y)$。
		同态保持逆元,因此
		\[
		f(y^{-1})=f(y)^{-1},\qquad g(y^{-1})=g(y)^{-1}.
		\]
		于是
		\[
		f(xy^{-1})=f(x)\,f(y^{-1})
		=f(x)\,f(y)^{-1}
		=g(x)\,g(y)^{-1}
		=g(x)\,g(y^{-1})
		=g(xy^{-1}).
		\]
		所以 $xy^{-1}\in D$。
		
		由子群判别法,$D<G$。
		
		
		\bigskip
		
	\item 	\textbf{题目 10.}\;
		设 $f$ 是群 $G$ 到群 $G'$ 的映射,$a\in G$。
		若 $f$ 是群同构,证明 $a$ 的阶等于 $f(a)$ 的阶。
		若 $f$ 是群同态,上述结论是否成立?为什么?
		
		\textbf{答案:}
		
		\textbf{(1) 若 $f$ 为同构,则 $|a|=|f(a)|$.}\;
		
		先证明一个基本等价:
		对任意正整数 $n$,
		\[
		f(a)^n=e_{G'}
		\iff f(a^n)=e_{G'}
		\iff a^n=e_G.
		\]
		解释如下:
		\begin{itemize}
			\item 因为 $f$ 是同态,$f(a^n)=f(a)^n$,故 $f(a)^n=e_{G'} \iff f(a^n)=e_{G'}$;
			\item 因为 $f$ 是同构,特别是单射,所以 $f(a^n)=e_{G'} \iff a^n=e_G$。
		\end{itemize}
		
		现在分两种情况讨论阶。
		
		\textbf{情况 A:$|a|=\infty$.}\;
		则对任意 $n\ge1$ 都有 $a^n\neq e_G$。
		由上面的等价关系,得到对任意 $n\ge1$ 都有 $f(a)^n\neq e_{G'}$,
		故 $|f(a)|=\infty$。
		
		\textbf{情况 B:$|a|=m<\infty$.}\;
		则 $a^m=e_G$ 且当 $1\le n<m$ 时 $a^n\neq e_G$。
		由等价关系立刻推出
		\[
		f(a)^m=e_{G'},\qquad \text{且 } 1\le n<m \Rightarrow f(a)^n\neq e_{G'}.
		\]
		因此 $|f(a)|=m=|a|$。
		
		综上,若 $f$ 为同构,则 $|a|=|f(a)|$。
		
		\bigskip
		\textbf{(2) 若 $f$ 仅为同态,则结论一般不成立;但有“整除关系”.}\;
		
		\textbf{(2.1) 总成立的结论:$|f(a)|\mid |a|$(当 $|a|<\infty$).}\;
		若 $|a|=m<\infty$,则 $a^m=e_G$。
		对两边施加同态 $f$:
		\[
		f(a)^m=f(a^m)=f(e_G)=e_{G'}.
		\]
		这说明 $f(a)$ 的阶是某个正整数,且它必须整除 $m$,即
		\[
		|f(a)|\mid m=|a|.
		\]
		(若 $|a|=\infty$,则 $|f(a)|$ 可能是有限也可能是无限。)
		
		\textbf{(2.2) 反例:阶可以变小,所以“相等”不保证成立.}\;
		取
		\[
		G=\mathbb Z/4\mathbb Z,\qquad G'=\mathbb Z/2\mathbb Z,
		\]
		定义同态
		\[
		f(\overline{x})=\overline{x}\pmod 2.
		\]
		令 $a=\overline{1}\in \mathbb Z/4\mathbb Z$,则
		\[
		|a|=4,\qquad f(a)=\overline{1}\in \mathbb Z/2\mathbb Z,\qquad |f(a)|=2.
		\]
		所以 $|a|\neq |f(a)|$,结论不成立。
		
		\textbf{(2.3) 什么时候仍然成立?}\;
		若同态 $f$ 还是单射(例如同构或嵌入),则由上面(1)中的等价推理可得
		\[
		|f(a)|=|a|.
		\]
		\item 13.
		举例说明下述命题不正确:设 $G,G'$ 是群,$N\lhd G,\ N'\lhd G'$,
		且有 $G\simeq G',\ N\simeq N'$,则必有 $G/N\simeq G'/N'$。
		
		\textbf{答案:}
		
		取
		\[
		G=G'=C_4\times C_2=\langle x\rangle\times\langle y\rangle,
		\quad |x|=4,\ |y|=2.
		\]
		令
		\[
		N=\langle x^2\rangle\times\{e\}\cong C_2,
		\qquad
		N'=\{e\}\times\langle y\rangle\cong C_2.
		\]
		显然 $N\lhd G,\ N'\lhd G'$(因为 $G$、$G'$ 交换,任意子群都正规),且
		\[
		G\simeq G',\qquad N\simeq N'\simeq C_2.
		\]
		
		但商群不同:
		\[
		G/N\cong (C_4/\langle x^2\rangle)\times C_2\cong C_2\times C_2,
		\]
		而
		\[
		G'/N'\cong C_4\times (C_2/\langle y\rangle)\cong C_4.
		\]
		由于 $C_2\times C_2\ncong C_4$,
		故
		\[
		G/N\ncong G'/N'.
		\]
		
		因此原命题不成立。
		\item 14.
		\textbf{题目.}
		设 $\sigma$ 是群 $G$ 的自同构,且满足
		\[
		\sigma(g)=g \Longrightarrow g=e.
		\]
		证明:
		
		(1) 映射 $f:G\to G$,$f(g)=\sigma(g)g^{-1}$ 是单射;
		
		(2) 若 $G$ 为有限群,则 $G$ 的每个元素均可写成 $\sigma(g)g^{-1}$ 的形式;
		
		(3) 若 $G$ 为有限群,且 $\sigma^2=\mathrm{id}_G$,则 $G$ 为奇数阶 Abel 群。
		
		
		(第一小问)\textbf{证明.}
		设 $f(g_1)=f(g_2)$,即
		\[
		\sigma(g_1)g_1^{-1}=\sigma(g_2)g_2^{-1}.
		\]
		两边右乘 $g_1$、左乘 $\sigma(g_2)^{-1}$,得
		\[
		\sigma(g_2^{-1}g_1)=g_2^{-1}g_1.
		\]
		由题设条件,$\sigma(x)=x$ 蕴含 $x=e$,于是
		\[
		g_2^{-1}g_1=e,
		\]
		即 $g_1=g_2$。
		故 $f$ 为单射。
		
		(第二小问)
		\textbf{证明.}
		由 (1) 知 $f:G\to G$ 为单射。
		若 $G$ 为有限集,则单射必为满射。
		因此对任意 $x\in G$,存在 $g\in G$ 使
		\[
		x=\sigma(g)g^{-1}.
		\]
		故结论成立。
		
		(第三小问)
		\textbf{证明.}
		由 (2),任取 $x\in G$,存在 $g\in G$ 使
		\[
		x=\sigma(g)g^{-1}.
		\]
		对两边作用 $\sigma$,并利用 $\sigma^2=\mathrm{id}_G$,得
		\[
		\sigma(x)=\sigma(\sigma(g))\sigma(g^{-1})
		=g\sigma(g)^{-1}.
		\]
		因此
		\[
		\sigma(x)=x^{-1}.
		\]
		
		于是对任意 $x,y\in G$,
		\[
		\sigma(xy)=\sigma(x)\sigma(y)=x^{-1}y^{-1},
		\]
		另一方面,
		\[
		\sigma(xy)=(xy)^{-1}=y^{-1}x^{-1}.
		\]
		比较得
		\[
		x^{-1}y^{-1}=y^{-1}x^{-1},
		\]
		从而 $xy=yx$。
		故 $G$ 为 Abel 群。

		若存在 $x\neq e$,使 $x^2=e$,则$x = x^{-1}$则
		\[
		\sigma(x)=x^{-1}=x,
		\]
		与题设“$\sigma(g)=g \Rightarrow g=e$”矛盾。
		因此 $G$ 中不存在非平凡的 2 阶元素。
		
		由 Cauchy 定理,若 $|G|$ 为偶数,则必存在阶为 $2$ 的元素,
		矛盾。
		故 $|G|$ 为奇数。 \qed
		\item \textbf{16. 题目.}\;
		求群 $\{ \mathbb{C}^*;\cdot\}$ 的子群 $N$,使得
		\[
		\{\mathbb{C}^*;\cdot\}/N \cong \{\mathbb{R}^+;\cdot\}.
		\]
		
		\textbf{答案:}\;
		取
		\[
		N:=S^1=\{z\in \mathbb{C}^*:|z|=1\}.
		\]
		定义映射
		\[
		\pi:\mathbb{C}^*\to \mathbb{R}^+,\qquad \pi(z)=|z|.
		\]
		则对任意 $z,w\in\mathbb{C}^*$,
		\[
		\pi(zw)=|zw|=|z||w|=\pi(z)\pi(w),
		\]
		故 $\pi$ 是群同态;并且对任意 $r\in\mathbb{R}^+$,取 $z=r\in\mathbb{C}^*$,则 $\pi(z)=r$,所以 $\pi$ 满射。
		又
		\[
		\mathrm{ker}(\pi)=\{z\in\mathbb{C}^*:|z|=1\}=S^1=N.
		\]
		由第一同构定理,
		\[
		\mathbb{C}^*/N \;\cong\; \mathbb{R}^+,
		\]
		即所求子群可取 $N=S^1$。
		
		\vspace{1.2em}
		
		\item \textbf{17. 题目.}\;
		设 $G$ 是 $n$ 阶交换群,$m\in\mathbb{N}$,定义
		\[
		f_m:G\to G,\qquad f_m(a)=a^m\quad(\forall a\in G).
		\]
		证明:$f_m$ 是 $G$ 的自同态,且 $f_m\in \mathrm{Aut}(G)$ 当且仅当 $(m,n)=1$。
		
		\textbf{答案:}
		
		\textbf{(1) $f_m$ 是自同态.}\;
		任取 $a,b\in G$,由于 $G$ 交换,
		\[
		f_m(ab)=(ab)^m=a^m b^m=f_m(a)f_m(b),
		\]
		故 $f_m$ 为群同态。
		
		\textbf{(2) 若 $(m,n)=1$,则 $f_m\in \mathrm{Aut}(G)$.}\;
		只需证 $f_m$ 单射。设 $a\in \mathrm{ker}(f_m)$,则
		\[
		f_m(a)=e \iff a^m=e.
		\]
		令 $|a|=t$。由 $a^m=e$ 得 $t\mid m$;又因 $a\in G$ 且 $|G|=n$,必有 $t\mid n$。
		于是
		\[
		t\mid m \ \text{且}\ t\mid n \ \Longrightarrow\ t\mid (m,n)=1 \ \Longrightarrow\ t=1 \ \Longrightarrow\ a=e.
		\]
		故 $\mathrm{ker}(f_m)=\{e\}$,从而 $f_m$ 单射。由于 $G$ 有限,单射等价于满射,因此 $f_m$ 为双射,即 $f_m\in \mathrm{Aut}(G)$。
		
		\textbf{(3) 若 $(m,n)\neq 1$,则 $f_m\notin \mathrm{Aut}(G)$.}\;
		设 $d=(m,n)>1$,取素数 $p\mid d$,则 $p\mid n$ 且 $p\mid m$。
		由 Cauchy 定理,存在 $a\in G$ 使 $|a|=p$,尤其 $a\neq e$ 且 $a^p=e$。
		又因 $p\mid m$,可写 $m=pt$,从而
		\[
		f_m(a)=a^m=a^{pt}=(a^p)^t=e.
		\]
		于是 $a\in \mathrm{ker}(f_m)$ 且 $a\neq e$,故 $\mathrm{ker}(f_m)\neq\{e\}$,$f_m$ 非单射,因此不可能为自同构。
		
		\textbf{(4) 结论.}\;
		综上,
		\[
		f_m\in \mathrm{Aut}(G)\ \Longleftrightarrow\ (m,n)=1.
		\]
		
	\end{enumerate}
	% -------------------- 1.5 循环群 --------------------
		\clearpage
	\section{循环群}
	\subsection{有限循环群元素相等的判别方式}
	\textbf{命题.}\;
	设 $\langle a\rangle$ 为一有限循环群,且 $|a|=n$。
	证明:对任意整数 $l,m$,
	\[
	a^l=a^m \quad\Longleftrightarrow\quad l\equiv m \pmod n.
	\]
	
	\textbf{证明.}\;
	由于 $|a|=n$,按定义有
	\[
	a^n=e,\qquad \text{且若 }0<k<n\text{ 则 }a^k\neq e.
	\]
	
	\textbf{($\Rightarrow$)}\;
	若 $a^l=a^m$,则两边右乘 $a^{-m}$ 得
	\[
	a^{l-m}=e.
	\]
	由 $|a|=n$ 的定义可知:$a^t=e \Longleftrightarrow n\mid t$(这是“阶”的基本性质)。
	于是 $n\mid(l-m)$,即
	\[
	l\equiv m \pmod n.
	\]
	
	\textbf{($\Leftarrow$)}\;
	若 $l\equiv m\pmod n$,则存在整数 $t$ 使
	\[
	l-m=tn.
	\]
	因此
	\[
	a^l=a^{m+tn}=a^m\,(a^n)^t=a^m e^t=a^m.
	\]
	
	综上,
	\[
	a^l=a^m \iff l\equiv m\pmod n.
	\]
	
	\subsection{循环群中关于幂的阶的基本定理}
	\textbf{命题.}\;
	设 $G=\langle a\rangle$ 是一个循环群,$|a|=n<\infty$。
	对任意整数 $k$,有
	\[
	|a^{k}|=\frac{n}{\gcd(n,k)}.
	\]
	
	\medskip
	\textbf{证明.}\;
	因为 $a$ 的阶为 $n$,即 $a^{n}=e$ 且 $n$ 是满足此性质的最小正整数。
	
	设 $d=\gcd(n,k)$,则存在整数 $s,t$,使得
	\[
	n=ds,\quad k=dt,\quad \gcd(s,t)=1.
	\]
	
	我们先证明 $(a^{k})^{\frac{n}{d}}=e$:
	\[
	(a^{k})^{\frac{n}{d}}
	=a^{k\cdot\frac{n}{d}}
	=a^{dt\cdot s}
	=a^{ds t}
	=(a^{n})^{t}
	=e.
	\]
	因此,$a^{k}$ 的阶整除 $\frac{n}{d}$。
	
	下面证明 $\frac{n}{d}$ 是满足 $(a^{k})^{m}=e$ 的最小正整数。
	
	若 $(a^{k})^{m}=e$,则 $a^{km}=e$,即 $n\mid km$。
	因为 $k=dt$,代入得
	\[
	n\mid d t m \;\Rightarrow\; ds \mid d t m \;\Rightarrow\; s\mid tm.
	\]
	又因 $\gcd(s,t)=1$,可得 $s\mid m$,即 $m$ 为 $s$ 的倍数。
	因此最小正的 $m$ 为 $s=\tfrac{n}{d}$。
	
	综上,
	\[
	\boxed{|a^{k}|=\frac{n}{\gcd(n,k)}}.
	\]
	
	\medskip
	\textbf{注.}\;
	若 $|a|=\infty$,则 $a^{k}$ 的阶为 $\infty$ 当且仅当 $k\neq 0$;
	此时上式可理解为 $\tfrac{\infty}{\gcd(\infty,k)}=\infty$。
	\textbf{推论:}循环群子群的阶一定整除该循环群的阶
	
	\subsection{$\text{ 对任意 }\bar{a}\in \mathbb{Z}_n,\text{ 有}|\bar{a}|=\frac{n}{\gcd(a,n)}.$}
	
	\textbf{命题.}\;
	$
	\text{设 }n\in \mathbb{N}^*,\ \mathbb{Z}_n=\mathbb{Z}/n\mathbb{Z}.
	\text{ 对任意 }\bar{a}\in \mathbb{Z}_n,\text{ 有}$
	\[
	|\bar{a}|=\frac{n}{\gcd(a,n)}.
	\]
	
	\textbf{证明.}\;
	上述定理的推论
	
	
		\clearpage
	\subsection*{课后习题答案}
	\addcontentsline{toc}{subsection}{\textcolor{red}{课后习题答案}}
	\begin{enumerate}[label=\textcolor{blue}{\textbf{\large\arabic*.}}]
		\item \textbf{题目.}\;
		证明:阶为 $4$ 的群只有两种同构类型:$4$ 阶循环群与 Klein 四元群 $K_4$。
		
		\textbf{答案:}
		
		设 $G$ 为一阶为 $4$ 的群,即 $|G|=4$。
		
		\textbf{第一步:利用元素阶的基本性质分类讨论。}
		
		由拉格朗日定理,群中任一元素的阶只能是
		\[
		1,\;2,\;\text{或 }4.
		\]
		
		下面分情况讨论。
		
		\textbf{情形一:存在元素 $a\in G$ 使得 $|a|=4$.}
		
		则由循环子群的定义,
		\[
		|\langle a\rangle|=|a|=4.
		\]
		但 $\langle a\rangle\le G$ 且两者阶相同,因此
		\[
		G=\langle a\rangle,
		\]
		从而 $G$ 是一个 $4$ 阶循环群,必同构于
		\[
		\mathbb{Z}/4\mathbb{Z}.
		\]
		
		\textbf{情形二:不存在阶为 $4$ 的元素。}
		
		此时除单位元 $e$ 外,所有非单位元的阶只能是 $2$。
		
		设
		\[
		G=\{e,a,b,c\},
		\]
		其中
		\[
		a^2=b^2=c^2=e.
		\]
		
		由于群中只有一个单位元,而每个阶为 $2$ 的元素都是自己的逆元,
		可以验证任意两个不同的非单位元相乘仍是第三个非单位元,例如
		\[
		ab=c,\quad bc=a,\quad ca=b.
		\]
		
		于是 $G$ 的乘法结构满足:
		\[
		a^2=b^2=c^2=e,\qquad ab=ba=c.
		\]
		
		这正是 Klein 四元群的定义,因此
		\[
		G\cong K_4.
		\]
		
		\textbf{第三步:排除其他可能性。}
		
		以上两种情形已经穷尽了 $|G|=4$ 时所有可能的元素阶结构,
		且所得群在同构意义下互不相同。
		
		\textbf{结论.}
		
		阶为 $4$ 的群在同构意义下只有两种:
		\[
		\mathbb{Z}/4\mathbb{Z}
		\quad\text{或}\quad
		K_4.
		\]
		\item 3. 
		
		\textbf{题目.}\;
		设 $(m,n)=1$,证明:$\mathbb Z_m\times \mathbb Z_n \cong \mathbb Z_{mn}$。
		
		\textbf{答案:}\;
		\textbf{证明.}
		将 $\mathbb Z_m,\mathbb Z_n,\mathbb Z_{mn}$ 都视为加法群。
		定义映射
		\[
		f:\mathbb Z_m\times \mathbb Z_n \longrightarrow \mathbb Z_{mn},\qquad
		f(\bar r,\bar t)=\overline{nr+mt},
		\]
		其中 $r,t$ 分别是 $\bar r\in\mathbb Z_m,\ \bar t\in\mathbb Z_n$ 的任意代表元,右端的“横线”表示模 $mn$ 的同余类。
		
		\medskip
		\textbf{(1) $f$ 良定义.}
		若 $r\equiv r'\pmod m$、$t\equiv t'\pmod n$,则存在整数 $k,\ell$ 使
		$r-r'=km$,$t-t'=\ell n$。
		于是
		\[
		nr+mt-(nr'+mt')=n(r-r')+m(t-t')=n(km)+m(\ell n)=(k+\ell)mn,
		\]
		故 $nr+mt\equiv nr'+mt'\pmod{mn}$,从而
		\[
		\overline{nr+mt}=\overline{nr'+mt'}\in\mathbb Z_{mn}.
		\]
		因此 $f$ 良定义。
		
		\medskip
		\textbf{(2) $f$ 是群同态.}
		取任意 $(\bar r,\bar t),(\overline{r'},\overline{t'})\in\mathbb Z_m\times\mathbb Z_n$,
		则
		\[
		f\big((\bar r,\bar t)+(\overline{r'},\overline{t'})\big)
		=f(\overline{r+r'},\overline{t+t'})
		=\overline{n(r+r')+m(t+t')}
		\]
		而
		\[
		f(\bar r,\bar t)+f(\overline{r'},\overline{t'})
		=\overline{(nr+mt)+(nr'+mt')}
		=\overline{n(r+r')+m(t+t')}.
		\]
		两者相等,故 $f$ 为同态。

		\textbf{(3) $f$ 是单射.}
		设 $f(\bar r,\bar t)=\bar 0\in\mathbb Z_{mn}$,则
		\[
		\overline{nr+mt}=\bar 0
		\quad\Longrightarrow\quad
		mn\mid (nr+mt).
		\]
		由于 $mn\mid (nr+mt)$,必有
		\[
		m\mid (nr+mt),\qquad n\mid (nr+mt).
		\]
		(因为$m | mn$)
		\medskip
		先对模 $m$ 考察。由 $m\mid (nr+mt)$ 得
		\[
		nr+mt\equiv 0\pmod m.
		\]
		而 $mt=m\cdot t\equiv 0\pmod m$,故
		\[
		nr+mt\equiv nr\equiv 0\pmod m,
		\]
		从而
		\[
		m\mid nr.
		\]
		又由于 $(m,n)=1$,由互素消因子定理得 $m\mid r$,
		即
		\[
		\bar r=\bar 0\in\mathbb Z_m.
		\]
		
		\medskip
		同理,对模 $n$ 考察。由 $n\mid (nr+mt)$ 得
		\[
		nr+mt\equiv 0\pmod n.
		\]
		而 $nr=n\cdot r\equiv 0\pmod n$,故
		\[
		nr+mt\equiv mt\equiv 0\pmod n,
		\]
		从而
		\[
		n\mid mt.
		\]
		由于 $(m,n)=1$,同样由互素消因子定理得 $n\mid t$,
		即
		\[
		\bar t=\bar 0\in\mathbb Z_n.
		\]
		
		\medskip
		因此 $(\bar r,\bar t)=(\bar 0,\bar 0)$,从而
		\[
		\ker(f)=\{(\bar 0,\bar 0)\},
		\]
		故 $f$ 为单射。
	
		
		\medskip
		\textbf{(4) $f$ 是满射(或用有限性推出).}
		由于 $|\mathbb Z_m\times\mathbb Z_n|=mn=|\mathbb Z_{mn}|$,且 $f$ 已知为单射,
		在有限集合之间单射必为满射,故 $f$ 也是满射。
		
		\medskip
		综上,$f$ 是双射群同态,因此 $f$ 为群同构,故
		\[
		\mathbb Z_m\times \mathbb Z_n \cong \mathbb Z_{mn}.
		\]
		\textbf{题目.}\;
		设 $(m,n)=1$,证明:$\mathbb{Z}_m\times \mathbb{Z}_n \cong \mathbb{Z}_{mn}$。
		

		\item 4.
		
		\textbf{题目.}\;
		试求循环群的自同态半群和自同构群。
		
		\textbf{答案:}\;
		\textbf{证明.}
		设 $G=\langle a\rangle$ 为循环群(用乘法记号),可分为 $|G|=n<\infty$ 与 $|G|=\infty$ 两种情形讨论。
		
		\medskip
		\textbf{一、$\mathrm{End}(G)=\mathrm{Hom}(G,G)$(自同态半群).}
		令 $\varphi:G\to G$ 为任一群同态。
		由于 $G$ 为循环群,任意元素均可写作 $a^k$($k\in\mathbb Z$)。
		特别地,$\varphi(a)\in G=\langle a\rangle$,故存在整数 $l\in\mathbb Z$ 使
		\[
		\varphi(a)=a^{\,l}.
		\]
		于是对任意 $k\in\mathbb Z$,
		\[
		\varphi(a^k)=\varphi(a)^k=(a^{\,l})^k=a^{\,lk}.
		\]
		因此任一自同态都由某个整数 $l$ 唯一决定,并具有形式
		\[
		\varphi_l:G\to G,\qquad \varphi_l(a^k)=a^{lk}\quad (k\in\mathbb Z),
		\]
		从而
		\[
		\mathrm{End}(G)=\{\varphi_l\mid l\in\mathbb Z\}.
		\]
		并且在合成运算下
		\[
		(\varphi_l\circ \varphi_{l'})(a^k)=\varphi_l(a^{l'k})=a^{ll'k},
		\]
		故
		\[
		\varphi_l\circ \varphi_{l'}=\varphi_{ll'}.
		\]
		于是 $\mathrm{End}(G)$ 在复合运算下构成半群(更确切地,与 $(\mathbb Z,\cdot)$ 同构)。
		
		\medskip
		\textbf{二、$\mathrm{Aut}(G)$(自同构群).}
		由上知任一自同态均为 $\varphi_l$。因此 $\varphi_l$ 为自同构当且仅当它为双射。
		
		\medskip
		\textbf{(1) 若 $|G|=n<\infty$.}
		此时 $G=\langle a\rangle$ 且 $a^n=e$。
		对任意整数 $l$,$\varphi_l$ 的像为
		\[
		\mathrm{Im}(\varphi_l)=\langle \varphi_l(a)\rangle=\langle a^l\rangle.
		\]
		众所周知(或可由循环群基本性质推出),$a^l$ 在 $G$ 中生成全群当且仅当 $(l,n)=1$。
		因此
		\[
		\varphi_l\ \text{满射}\ \Longleftrightarrow\ \langle a^l\rangle=G
		\Longleftrightarrow\ (l,n)=1.
		\]
		而 $G$ 有限,满射等价于单射,故
		\[
		\varphi_l\in\mathrm{Aut}(G)\Longleftrightarrow (l,n)=1.
		\]
		于是
		\[
		\mathrm{Aut}(G)=\{\varphi_l\mid (l,n)=1\}.
		\]
		并且在合成运算下
		\[
		\varphi_l\circ\varphi_{l'}=\varphi_{ll'},
		\]
		从而 $\mathrm{Aut}(G)$ 与模 $n$ 的可逆元群 $(\mathbb Z/n\mathbb Z)^\times$ 同构(常见同构为
		$\varphi_l\mapsto \overline{l}$)。
		
		\medskip
		\textbf{(2) 若 $|G|=\infty$.}
		此时 $G\cong \mathbb Z$,仍有 $\varphi_l(a^k)=a^{lk}$。
		若 $l=0$,则 $\varphi_0$ 为常值映射,非双射。
		若 $|l|\ge 2$,则
		\[
		\mathrm{Im}(\varphi_l)=\langle a^l\rangle=\{a^{lk}:k\in\mathbb Z\}\subsetneq G,
		\]
		不是满射,亦非双射。
		只有 $l=1$ 或 $l=-1$ 时,
		\[
		\varphi_1(a^k)=a^k,\qquad \varphi_{-1}(a^k)=a^{-k},
		\]
		分别为恒等映射与取逆映射,均为双射。
		因此
		\[
		\mathrm{Aut}(G)=\{\varphi_1,\varphi_{-1}\}\cong C_2.
		\]
		
		\medskip
		\textbf{三、结论.}
		设 $G=\langle a\rangle$ 为循环群,则
		\[
		\mathrm{End}(G)=\{\varphi_l\mid l\in\mathbb Z,\ \varphi_l(a^k)=a^{lk}\},
		\]
		并且 $\varphi_l\circ\varphi_{l'}=\varphi_{ll'}$。
		此外:
		\[
		\mathrm{Aut}(G)=
		\begin{cases}
			\{\varphi_l\mid (l,n)=1\}, & |G|=n<\infty,\\[4pt]
			\{\varphi_1,\varphi_{-1}\}, & |G|=\infty.
		\end{cases}
		\]
		
		\item 7.
		\textbf{题目.}\;
		设 $G=\langle a\rangle$ 是无限阶循环群,$H=\langle b\rangle$ 是 $m$ 阶循环群。
		求 $G,H$ 的所有生成元。
		
		\textbf{答案:}
		
		\textbf{(1) 无限阶循环群 $G=\langle a\rangle$ 的生成元}
		
		由于 $G$ 为无限阶循环群,有
		\[
		G=\{a^k\mid k\in\mathbb Z\}.
		\]
		
		下面判断 $a^k$ 是否为 $G$ 的生成元。
		
		若 $a^k$ 生成 $G$,则
		\[
		\langle a^k\rangle = G,
		\]
		即对任意 $n\in\mathbb Z$,存在 $t\in\mathbb Z$ 使得
		\[
		a^n = (a^k)^t = a^{kt},
		\]
		从而
		\[
		n = kt.
		\]
		
		这对所有整数 $n$ 成立当且仅当
		\[
		k=\pm 1.
		\]
		
		反之,显然
		\[
		\langle a\rangle = G,\qquad \langle a^{-1}\rangle = G.
		\]
		
		因此,$G$ 的所有生成元恰为
		\[
		\boxed{a,\ a^{-1}.}
		\]
		
		---
		
		\textbf{(2) $m$ 阶循环群 $H=\langle b\rangle$ 的生成元}
		
		由于 $|H|=m$,有
		\[
		H=\{e,b,b^2,\dots,b^{m-1}\}.
		\]
		
		设 $b^k$ 为 $H$ 的生成元,则
		\[
		\langle b^k\rangle = H.
		\]
		
		由循环群的基本性质,
		\[
		|\langle b^k\rangle|=\frac{m}{\gcd(k,m)}.
		\]
		
		因此
		\[
		\langle b^k\rangle = H
		\iff
		\frac{m}{\gcd(k,m)}=m
		\iff
		\gcd(k,m)=1.
		\]

		
		因此,$H$ 的所有生成元为
		\[
		\boxed{\{\,b^k\mid 1\le k\le m,\ \gcd(k,m)=1\,\}.}
		\]
		
		---
		
		\textbf{结论.}
		
		\[
		\begin{cases}
			\text{无限阶循环群 } \langle a\rangle \text{ 的生成元只有 } a,a^{-1};\\[6pt]
			\text{$m$ 阶循环群 } \langle b\rangle \text{ 的生成元是所有 } b^k
			\text{,其中 } \gcd(k,m)=1.
		\end{cases}
		\]
		\item 10.
		\textbf{题目.}\;
		证明:$\{\mathbb{Q};+\}$ 的任一有限生成子群必是循环群。
		
		\textbf{证明.}
		设 $S\le (\mathbb Q,+)$ 为有限生成子群。则存在有理数 $x_1,\dots,x_n\in\mathbb Q$ 使
		\[
		S=\langle x_1,\dots,x_n\rangle
		=\Big\{\sum_{i=1}^n k_i x_i:\ k_i\in\mathbb Z\Big\}.
		\]
		将每个 $x_i$ 写成既约分数 $x_i=\dfrac{q_i}{p_i}$,其中 $q_i\in\mathbb Z,\ p_i\in\mathbb N$ 且 $(q_i,p_i)=1$。
		令
		\[
		P:=p_1p_2\cdots p_n,\qquad
		a_i:=q_i\frac{P}{p_i}=q_i\,p_1\cdots \widehat{p_i}\cdots p_n\in\mathbb Z\quad (i=1,\dots,n),
		\]
		并令
		\[
		d:=\gcd(a_1,\dots,a_n)\in\mathbb N.
		\]
		下面证明
		\[
		S=\Big\langle \frac{d}{P}\Big\rangle.
		\]
		
		\medskip
		\textbf{(1) 证明 $S\subseteq \left\langle \dfrac{d}{P}\right\rangle$.}
		任取 $g\in S$,则 $g=\sum_{i=1}^n k_i \dfrac{q_i}{p_i}$($k_i\in\mathbb Z$)。
		于是
		\[
		g=\sum_{i=1}^n k_i\frac{q_i}{p_i}
		=\frac{1}{P}\sum_{i=1}^n k_i q_i\frac{P}{p_i}
		=\frac{1}{P}\sum_{i=1}^n k_i a_i.
		\]
		因为 $d=\gcd(a_1,\dots,a_n)$,故每个 $a_i$ 都是 $d$ 的倍数,从而 $\sum_{i=1}^n k_i a_i$ 也是 $d$ 的倍数。
		因此存在整数 $K$ 使
		\[
		\sum_{i=1}^n k_i a_i=Kd,
		\]
		于是
		\[
		g=\frac{1}{P}\sum_{i=1}^n k_i a_i=\frac{Kd}{P}=K\cdot \frac{d}{P}\in \Big\langle \frac{d}{P}\Big\rangle.
		\]
		故 $S\subseteq \left\langle \dfrac{d}{P}\right\rangle$。
		
		\medskip
		\textbf{(2) 证明 $\left\langle \dfrac{d}{P}\right\rangle\subseteq S$.}
		由 Bézout 定理,存在整数 $u_1,\dots,u_n\in\mathbb Z$ 使
		\[
		u_1a_1+\cdots+u_na_n=d.
		\]
		两边同除以 $P$,得
		\[
		\frac{d}{P}=\sum_{i=1}^n u_i\frac{a_i}{P}
		=\sum_{i=1}^n u_i\frac{q_i(P/p_i)}{P}
		=\sum_{i=1}^n u_i\frac{q_i}{p_i}
		=\sum_{i=1}^n u_i x_i\in S.
		\]
		因此 $\dfrac{d}{P}\in S$,而 $S$ 为子群,故其包含 $\dfrac{d}{P}$ 的所有整数倍:
		\[
		\Big\langle \frac{d}{P}\Big\rangle \subseteq S.
		\]
		
		\medskip
		由 (1)(2) 得 $S=\left\langle \dfrac{d}{P}\right\rangle$,从而 $S$ 是循环群。
		
		\textbf{答案:}
		
		设 $H\le (\mathbb{Q},+)$ 为有限生成子群。则存在有理数
		\[
		q_1,q_2,\dots,q_r\in\mathbb{Q}
		\]
		使
		\[
		H=\langle q_1,q_2,\dots,q_r\rangle
		=\Bigl\{\;a_1q_1+\cdots+a_rq_r\ \Bigm|\ a_i\in\mathbb{Z}\Bigr\}.
		\]
		
		\textbf{(1) 统一分母.}\;
		将每个 $q_i$ 写成最简分数
		\[
		q_i=\frac{m_i}{n_i},\qquad m_i\in\mathbb{Z},\ n_i\in\mathbb{N}.
		\]
		令
		\[
		N:=\mathrm{lcm}(n_1,n_2,\dots,n_r).
		\]
		则对每个 $i$,有 $n_i\mid N$,从而存在整数 $c_i$ 使 $N=c_i n_i$,于是
		\[
		q_i=\frac{m_i}{n_i}=\frac{m_i c_i}{N}.
		\]
		记
		\[
		M_i:=m_i c_i\in\mathbb{Z},
		\]
		则
		\[
		q_i=\frac{M_i}{N}\qquad (i=1,\dots,r).
		\]
		
		\textbf{(2) 将 $H$ 写成 $\frac1N\mathbb{Z}$ 的子群.}\;
		任取 $h\in H$,则存在 $a_1,\dots,a_r\in\mathbb{Z}$ 使
		\[
		h=a_1q_1+\cdots+a_rq_r
		=\frac{a_1M_1+\cdots+a_rM_r}{N}.
		\]
		因此
		\[
		H\subseteq \frac1N\mathbb{Z}:=\left\{\frac{t}{N}\mid t\in\mathbb{Z}\right\}.
		\]
		反过来,令
		\[
		d:=\gcd(M_1,M_2,\dots,M_r).
		\]
		由最大公约数的性质,存在整数 $b_1,\dots,b_r$ 使
		\[
		b_1M_1+\cdots+b_rM_r=d,
		\]
		于是
		\[
		\frac{d}{N}
		=\frac{b_1M_1+\cdots+b_rM_r}{N}
		=b_1\frac{M_1}{N}+\cdots+b_r\frac{M_r}{N}
		=b_1q_1+\cdots+b_rq_r\in H.
		\]
		从而对任意 $t\in\mathbb{Z}$,
		\[
		t\cdot \frac{d}{N}\in H.
		\]
		即
		\[
		\left\langle \frac{d}{N}\right\rangle \subseteq H.
		\]
		
		\textbf{(3) 证明 $H=\left\langle \frac{d}{N}\right\rangle$.}\;
		由 $d=\gcd(M_1,\dots,M_r)$ 可知 $d\mid M_i$,故存在整数 $k_i$ 使 $M_i=d k_i$,于是
		\[
		q_i=\frac{M_i}{N}=\frac{d k_i}{N}=k_i\cdot \frac{d}{N}\in \left\langle \frac{d}{N}\right\rangle.
		\]
		因此
		\[
		H=\langle q_1,\dots,q_r\rangle \subseteq \left\langle \frac{d}{N}\right\rangle.
		\]
		结合上一步已得 $\left\langle \frac{d}{N}\right\rangle \subseteq H$,可得
		\[
		H=\left\langle \frac{d}{N}\right\rangle.
		\]
		
		\textbf{(4) 结论.}\;
		$H$ 由单个元素 $\frac{d}{N}$ 生成,故 $H$ 是循环群。
		\item 12.
		\textbf{题目.}
		证明:群 $G$ 只有有限个子群当且仅当 $G$ 为有限群。
		
		\textbf{证明.}
		
		\textbf{(必要性) 若 $G$ 为有限群,则 $G$ 只有有限个子群.}
		
		设 $|G|<\infty$。
		任一子群 $H\le G$ 都是 $G$ 的子集,且满足 $e\in H$。
		由于 $G$ 的子集总数有限(至多 $2^{|G|}$ 个),
		故 $G$ 的子群个数也必有限。
		因此有限群必只有有限个子群。
		
		\medskip
		\textbf{(充分性) 若 $G$ 只有有限个子群,则 $G$ 为有限群.}
		
		反设 $G$ 为无限群。
		则存在无限多个不同的元素 $g_1,g_2,\dots\in G$。
		对任意 $g\in G$,记
		\[
		\langle g\rangle
		\]
		为由 $g$ 生成的循环子群。
		
		\textbf{情形一:$G$ 含有无限阶元素.}
		
		设存在 $g\in G$ 使 $|g|=\infty$。
		则 $\langle g\rangle$ 是一个无限循环群,且
		\[
		\langle g\rangle \cong \mathbb Z.
		\]
		
		对每个正整数 $m$,考虑 $\langle g^m\rangle$。
		显然
		\[
		\langle g^m\rangle\le \langle g\rangle \le G,
		\]
		所以它们都是 $G$ 的子群。
		
		关键是证明:当 $m\neq n$ 时,
		\[
		\langle g^m\rangle \neq \langle g^n\rangle.
		\]
		
		\textbf{证明这个“不相等”.}\;
		因为 $\langle g\rangle$ 为无限循环群,同构于 $\mathbb Z$,
		而 $\mathbb Z$ 的子群全部形如 $m\mathbb Z$,并且
		\[
		m\mathbb Z = n\mathbb Z \iff m=n.
		\]
		在同构对应下,$m\mathbb Z$ 正对应 $\langle g^m\rangle$,
		$n\mathbb Z$ 正对应 $\langle g^n\rangle$,
		因此
		\[
		\langle g^m\rangle = \langle g^n\rangle \iff m=n.
		\]
		故当 $m\neq n$ 时,$\langle g^m\rangle$ 两两不同。
		
		于是 $\{\langle g\rangle,\langle g^2\rangle,\langle g^3\rangle,\dots\}$
		给出 $G$ 的无穷多个不同子群,这与“$G$ 只有有限个子群”矛盾。
		
		---
		
		\textbf{情形二:$G$ 中每个元素阶都有限(即 $G$ 为纯挠群 / torsion group).}
		
		此时对任意 $x\in G$,循环子群 $\langle x\rangle$ 都是有限群。
		
		我们利用一个非常关键但简单的“并集计数”观察:
		
		\textbf{引理.}\;
		若 $G$ 的循环子群只有有限多个,
		设它们为
		\[
		\langle x_1\rangle,\langle x_2\rangle,\dots,\langle x_r\rangle,
		\]
		则 $G$ 必为有限群。
		
		\textbf{引理证明.}\;
		任取 $g\in G$,则 $g$ 必生成某个循环子群 $\langle g\rangle$。
		若循环子群总共只有上述有限多个,
		则必存在某个 $i$ 使得
		\[
		\langle g\rangle = \langle x_i\rangle,
		\]
		从而 $g\in \langle x_i\rangle$。
		因此
		\[
		G=\bigcup_{i=1}^r \langle x_i\rangle.
		\]
		但每个 $\langle x_i\rangle$ 都是有限集合,有限个有限集合的并仍是有限集合,
		故 $G$ 有限。
		
		引理得证。
		
		现在回到情形二:
		我们假设 $G$ 是无限群。
		由上述引理的逆否命题可得:
		
		\textbf{若 $G$ 是无限群,则 $G$ 必有无穷多个不同的循环子群.}
		
 		而每个循环子群都是 $G$ 的子群,
		于是 $G$ 至少有无穷多个子群,
		这与“$G$ 只有有限个子群”矛盾。
		
		---
		
		\textbf{(3) 综上.}
		
		两种情形都推出矛盾,因此假设“$G$ 无限”不成立,
		故 $G$ 必为有限群。
		
		---
		
		\textbf{结论.}\;
		\[
		G \text{ 只有有限个子群}\iff |G|<\infty.
		\] 
		\qed
		\item 
		\textbf13.{题目.}\;
		设 $G$ 为 $n$ 阶循环群。证明:对 $n$ 的任一正因子 $m$,$G$ 中阶为 $m$ 的元素恰有 $\varphi(m)$ 个,其中 $\varphi(m)$ 表示与 $m$ 互素且不超过 $m$ 的正整数个数。并由此证明
		\[
		\sum_{m\mid n}\varphi(m)=n.
		\]
		
		\textbf{答案:}\;
		
		\textbf{证明.}\;
		因 $G$ 为 $n$ 阶循环群,存在生成元 $a$ 使得
		\[
		G=\langle a\rangle=\{e,a,a^2,\dots,a^{n-1}\},\qquad |a|=n.
		\]
		
		\medskip
		\textbf{(1) 先求 $a^l$ 的阶.}
		取任意整数 $l$(可取 $0\le l\le n-1$),设
		\[
		d=(n,l).
		\]
		写成 $n=dn_1,\ l=dl_1$,且 $(n_1,l_1)=1$。
		则
		\[
		(a^l)^{n_1}=a^{ln_1}=a^{dl_1n_1}=a^{l_1n}=e,
		\]
		所以 $|a^l|\mid n_1$。
		
		反过来,若 $(a^l)^t=e$,则 $a^{lt}=e$,即 $n\mid lt$。
		由 $n=dn_1,\ l=dl_1$ 得
		\[
		dn_1\mid dl_1t\quad\Longrightarrow\quad n_1\mid l_1t.
		\]
		又因 $(n_1,l_1)=1$,推出 $n_1\mid t$,即 $t\ge n_1$ 的最小正整数是 $n_1$。
		因此
		\[
		|a^l|=n_1=\frac{n}{(n,l)}.
		\]
		这一步就是你图中“若 $a^l$ 的阶为 $m$,从而 $m=\dfrac{n}{(n,l)}$”的来源。
		
		\medskip
		\textbf{(2) 判定 $|a^l|=m$ 的充要条件,并计数.}
		现在固定一个因子 $m\mid n$。由上式
		\[
		|a^l|=m
		\quad\Longleftrightarrow\quad
		\frac{n}{(n,l)}=m
		\quad\Longleftrightarrow\quad
		(n,l)=\frac{n}{m}.
		\]
		令
		\[
		q=\frac{n}{m}\quad(\text{于是 }n=qm).
		\]
		则条件变为 $(n,l)=q$。注意 $q\mid n$,因此必须有 $q\mid l$,可写
		\[
		l=qk=\frac{n}{m}k,
		\qquad 1\le k\le m.
		\]
		此时
		\[
		(n,l)=(qm,qk)=q\,(m,k).
		\]
		于是
		\[
		(n,l)=q
		\quad\Longleftrightarrow\quad
		q\,(m,k)=q
		\quad\Longleftrightarrow\quad
		(m,k)=1.
		\]
		这就得到你图中“从而 $l=\dfrac{n}{m}\cdot k$,且 $(k,m)=1$”的结论,并且说明:
		
		\begin{enumerate}
			\item 若 $(k,m)=1$,则
			\[
			|a^{\,qk}|=\frac{n}{(n,qk)}=\frac{qm}{q(m,k)}=\frac{qm}{q}=m,
			\]
			故 $a^{qk}$ 的阶正好是 $m$。
			\item 若存在 $d=(k,m)>1$,则
			\[
			|a^{\,qk}|=\frac{qm}{q(m,k)}=\frac{qm}{qd}=\frac{m}{d},
			\]
			故此时阶会“降”为 $\dfrac{m}{d}$,因此 $(k,m)=1$ 不仅是充分条件,也是必要条件。
		\end{enumerate}
		
		因此,$G$ 中阶为 $m$ 的元素恰对应于
		\[
		k\in\{1,2,\dots,m\}\ \text{且}\ (k,m)=1,
		\]
		这样的 $k$ 的个数正是 $\varphi(m)$。故
		\[
		\#\{x\in G:\ |x|=m\}=\varphi(m).
		\]
		
		\medskip
		\textbf{(3) 推出 $\displaystyle \sum_{m\mid n}\varphi(m)=n$.}
		循环群 $G$ 的每个元素都有一个确定的阶,而该阶必为 $n$ 的因子。
		于是集合 $G$ 可以按“元素的阶”分成互不相交的并:
		\[
		G=\bigsqcup_{m\mid n}\{x\in G:\ |x|=m\}.
		\]
		对两边取基数并用上一步的计数结果:
		\[
		|G|
		=\sum_{m\mid n}\#\{x\in G:\ |x|=m\}
		=\sum_{m\mid n}\varphi(m).
		\]
		又 $|G|=n$,故
		\[
		\sum_{m\mid n}\varphi(m)=n.
		\]
		证完。
		

		
	\end{enumerate}
	% -------------------- 1.6 对称群与交错群 --------------------
		\clearpage
	\section{对称群与交错群}
	\subsection{$r-$轮换的阶一定为$r$}
	\textbf{命题.}\;
	若 $\sigma = (i_1\,i_2\,\dots\,i_r)$ 是对称群 $S_n$ 中的一个 $r$-轮换,
	则 $\sigma$ 的阶为 $r$。
	
	\textbf{证明.}\;
	由 $r$-轮换的定义可知:
	\[
	\sigma(i_1)=i_2,\quad \sigma(i_2)=i_3,\quad \dots,\quad 
	\sigma(i_{r-1})=i_r,\quad \sigma(i_r)=i_1,
	\]
	并且 $\sigma$ 固定除 $\{i_1,i_2,\dots,i_r\}$ 以外的所有元素。
	
	现在计算 $\sigma^k$ 的作用:
	\[
	\sigma^k(i_j) = i_{j+k},
	\]
	其中下标按 $r$ 取模(即 $i_{r+t}=i_t$)。
	
	特别地,若 $k=r$,则对所有 $j$ 都有
	\[
	\sigma^r(i_j)=i_{j+r}=i_j,
	\]
	因此 $\sigma^r$ 为恒等置换,即 $\sigma^r = e$,从而 $|\sigma| \mid r$。
	
	另一方面,若 $1 \le k < r$,则
	\[
	\sigma^k(i_1)=i_{1+k}\ne i_1,
	\]
	说明 $\sigma^k \ne e$。因此 $\sigma$ 的最小正幂等于 $r$ 时才回到恒等元。
	
	综上,$\sigma$ 的阶为
	\[
	\boxed{|\sigma| = r.}
	\]
		\subsection{定理1.6.4的推论:置换的阶的计算}
	
	\textbf{命题.}\;
	设置换 $\sigma\in S_n$ 的不相交轮换分解为
	\[
	\sigma=\sigma_1\sigma_2\cdots\sigma_r,
	\]
	其中每个 $\sigma_i$ 是长度为 $\ell_i\ge 2$ 的轮换,且各 $\sigma_i$ 不相交,则
	\[
	|\sigma|=\operatorname{lcm}(\ell_1,\ell_2,\dots,\ell_r).
	\]
	
	\textbf{证明.}\;
	由于各 $\sigma_i$ 不相交,故它们两两交换,因而对任意整数 $k\ge 1$ 有
	\[
	\sigma^k=(\sigma_1\sigma_2\cdots\sigma_r)^k=\sigma_1^k\sigma_2^k\cdots\sigma_r^k.
	\]
	记 $m:=\operatorname{lcm}(\ell_1,\dots,\ell_r)$。因为 $\ell_i\mid m$,对每个 $i$ 有 $\sigma_i^{\,m}=e$,于是
	\[
	\sigma^{m}=\sigma_1^{\,m}\sigma_2^{\,m}\cdots\sigma_r^{\,m}=e,
	\]
	从而 $|\sigma|\mid m$。
	
	反之,若 $\sigma^k=e$,则对每个 $i$,因为它们分量作用独立,则
	\[
	e=\sigma^k=\sigma_1^k\cdots\sigma_r^k \quad\Longrightarrow\quad \sigma_i^k=e,
	\]
循环 $\sigma_i$ 的阶为其长度 $\ell_i$,故必有 $\ell_i\mid k$。于是对所有 $i$,$\ell_i\mid k$,从而 $m=\operatorname{lcm}(\ell_1,\dots,\ell_r)\mid k$。故 $|\sigma|$ 为使上述成立的最小正整数,等于 $m$。
	
	综上,$|\sigma|=\operatorname{lcm}(\ell_1,\ell_2,\dots,\ell_r)$。
	\qed	
	\subsection{$A_n$ 是 $S_n$ 中唯一的指数为 $2$ 的正规子群}
	\textbf{命题.}\;
	在对称群 $S_n$ 中,唯一一个指数为 $2$ 的正规子群是交错群 $A_n$。
	
	\textbf{证明.}
	
	\textbf{第一步:$A_n$ 是 $S_n$ 的指数为 $2$ 的正规子群.}
	
	定义符号同态
	\[
	\operatorname{sgn}:S_n\to\{\pm1\},
	\]
	它将偶置换映为 $+1$,奇置换映为 $-1$。
	该映射是满射,且
	\[
	\ker(\operatorname{sgn})=A_n.
	\]
	因此 $A_n\lhd S_n$,并且
	\[
	[S_n:A_n]=2.
	\]
	
	---
	
	\textbf{第二步:任取 $S_n$ 的指数为 $2$ 的正规子群 $H$,证明 $H=A_n$.}
	
	设 $H\lhd S_n$ 且 $[S_n:H]=2$。
	则商群 $S_n/H$ 的阶为 $2$,从而为交换群。
	因此存在一个满同态
	\[
	\varphi:S_n\to\{\pm1\}
	\]
	使得
	\[
	H=\ker\varphi.
	\]
	
	由于 $\{\pm1\}$ 是交换群,$\varphi$ 在交换子群上恒为单位元。
	而已知
	\[
	[S_n,S_n]=A_n,
	\]
	故
	\[
	A_n\subseteq \ker\varphi = H.
	\]
	
	又因为
	\[
	[S_n:A_n]=[S_n:H]=2,
	\]
	从而
	\[
	|A_n|=|H|,
	\]
	于是
	\[
	H=A_n.
	\]
	
	---
	
	\textbf{结论.}
	
	$A_n$ 是 $S_n$ 中唯一的指数为 $2$ 的正规子群。
	
	\subsection{对称群 $S_p$ 中,若置换 $\sigma$ 的阶为$p$,则 $\sigma$ 是一个 $p$-轮换}
	\textbf{命题.}\;
	若$p$是一个素数,在对称群 $S_p$ 中,若置换 $\sigma$ 的阶为$p$,则 $\sigma$ 是一个 $p$-轮换;反之,任一 $p$-轮换的阶为 $p$。
	
	\textbf{证明.}\;
	将 $\sigma$ 写成不相交轮换分解
	\[
	\sigma=\sigma_1\sigma_2\cdots\sigma_r,
	\]
	其中各轮换的长度分别为 $\ell_1,\ell_2,\dots,\ell_r$,约定平凡 $1$-轮换可忽略不计,于是每个 $\ell_i\ge 2$。
	众所周知,置换的阶等于这些长度的最小公倍数:
	\[
	|\sigma|=\operatorname{lcm}(\ell_1,\ell_2,\dots,\ell_r).
	\]
	现已知 $|\sigma|=p$ 为素数。于是对每个 $i$,$\ell_i\mid p$,从而 $\ell_i\in\{1,p\}$。但按约定 $\ell_i\ge 2$,故每个非平凡轮换的长度都等于 $p$。
	
	另一方面,不相交轮换作用在不相交的字母集上,故这些长度之和不超过可用字母总数 $p$。若存在两个或以上的非平凡轮换,则至少需要 $2p$ 个字母,矛盾。于是恰有一个非平凡轮换,其长度为 $p$。因此 $\sigma$ 是一个 $p$-轮换。
	
	反之,任一 $p$-轮换的阶显然为其长度 $p$。\qed
	

	\subsection{$S_n$ 的自同构群 $\mathrm{Aut}(S_n)$}
	讨论对称群的自同构结构。
	
	\subsection{$S_3$的性质}
	\begin{itemize}
		\item \textbf{定义与阶.}\;
		\[
		S_3=\{\text{对集合 }\{1,2,3\}\text{ 的所有置换}\},\qquad |S_3|=3!=6.
		\]
		
		\item \textbf{元素(循环表示)与元素阶.}\;
		\[
		S_3=\{\,e,\ (12),\ (13),\ (23),\ (123),\ (132)\,\}.
		\]
		其中
		\[
		|e|=1,\quad \operatorname{ord}((12))=\operatorname{ord}((13))=\operatorname{ord}((23))=2,\quad 
		\operatorname{ord}((123))=\operatorname{ord}((132))=3.
		\]
		
		\item \textbf{非交换性.}\;
		\[
		S_3\ \text{非阿贝尔群(不交换)}.
		\]
		
		\item \textbf{生成元与关系(presentation).}\;
		取
		\[
		r=(123),\quad s=(12),
		\]
		则
		\[
		S_3=\langle r,s \mid r^3=e,\ s^2=e,\ srs=r^{-1}\rangle.
		\]
		
		\item \textbf{共轭类(Conjugacy classes).}\;
		\[
		\{e\},\qquad \{(12),(13),(23)\},\qquad \{(123),(132)\}.
		\]
		
		\item \textbf{子群结构.}\;
		\begin{itemize}
			\item 平凡子群:\(\{e\}\).
			\item 三个阶 \(2\) 子群:
			\[
			\langle(12)\rangle=\{e,(12)\},\quad 
			\langle(13)\rangle=\{e,(13)\},\quad
			\langle(23)\rangle=\{e,(23)\}.
			\]
			\item 唯一的阶 \(3\) 子群(交错群):
			\[
			A_3=\{e,(123),(132)\}.
			\]
			\item 全群:\(S_3\).
		\end{itemize}
		
		\item \textbf{正规子群.}\;
		\(S_3\) 的正规子群只有
		\[
		\{e\},\quad A_3,\quad S_3.
		\]
		并且上述三个阶 \(2\) 子群都不是正规子群。
		
		\item \textbf{中心.}\;
		\[
		Z(S_3)=\{e\}.
		\]
		
		\item \textbf{换位同态与商群.}\;
		设符号同态 \(\mathrm{sgn}:S_3\to\{\pm1\}\),则
		\[
		\ker(\mathrm{sgn})=A_3,\qquad S_3/A_3\cong C_2.
		\]
		
		\item \textbf{导群(交换子群)与可解性.}\;
		\[
		[S_3,S_3]=A_3.
		\]
		存在正规列
		\[
		\{e\}\triangleleft A_3\triangleleft S_3,
		\]
		且商群分别同构于 \(C_3\) 与 \(C_2\),因此
		\[
		S_3\ \text{是可解群(但非阿贝尔群)}.
		\]
		
		\item \textbf{同构类型(几何解释).}\;
		\[
		S_3\cong D_3,
		\]
		即与正三角形的对称群同构(包含 \(3\) 个旋转与 \(3\) 个反射)。
	\end{itemize}
	
	\subsection{$S_4$ 的非平凡正规子群只有$A_4 \quad\text{和}\quad V_4$}
	$S_4$ 的非平凡正规子群恰有两个:
	\[
	A_4 \quad\text{和}\quad V_4.
	\]
	
	\textbf{证明.}\;
	
	\textbf{第一步:列出 $S_4$ 的共轭类.}
	
	$S_4$ 中的元素按循环类型分为以下共轭类:
	\[
	\begin{aligned}
		&\{e\} && (1) \\
		&\{\text{换位}\} && (ab),\quad |C|=6 \\
		&\{\text{三循环}\} && (abc),\quad |C|=8 \\
		&\{\text{四循环}\} && (abcd),\quad |C|=6 \\
		&\{\text{双换位}\} && (ab)(cd),\quad |C|=3
	\end{aligned}
	\]
	
	正规子群必须是若干个共轭类的并,并且包含单位元。
	
	---
	
	\textbf{第二步:验证 $A_4$ 是正规子群.}
	
	$A_4$ 是所有偶置换构成的子群,
	\[
	|A_4|=12,\qquad [S_4:A_4]=2.
	\]
	指数为 $2$ 的子群必为正规子群,因此
	\[
	A_4\trianglelefteq S_4.
	\]
	
	---
	
	\textbf{第三步:验证 $V_4$ 是正规子群.}
	
	定义 Klein 四元群
	\[
	V_4=\{e,\ (12)(34),\ (13)(24),\ (14)(23)\}.
	\]
	其中三个非单位元素恰好构成 $S_4$ 中所有双换位的共轭类,因此
	\[
	V_4=\{e\}\ \cup\ \{\text{所有双换位}\},
	\]
	是共轭类的并,且容易验证它是子群,于是
	\[
	V_4\trianglelefteq S_4.
	\]
	
	---
	
	\textbf{第四步:排除其他可能.}
	
	设 $N\trianglelefteq S_4$ 为非平凡正规子群,则 $N$ 必为若干共轭类的并。
	
	逐一检查所有可能的共轭类组合可知:
	\begin{itemize}
		\item 只取换位、三循环或四循环都无法构成子群;
		\item 若包含三循环,则必包含 $A_4$;
		\item 若包含双换位但不包含三循环,则只能得到 $V_4$;
		\item 其余组合要么不封闭,要么得到整个 $S_4$。
	\end{itemize}
	
	因此不存在除 $\{e\}$、$V_4$、$A_4$、$S_4$ 之外的正规子群。
	
	---
	
	\textbf{结论.}\;
	$S_4$ 的全部非平凡正规子群为
	\[
	\boxed{A_4\ \text{和}\ V_4.}
	\]
	
	\subsection{符号同态与交正规}
	\textbf{定理(符号同态与交正规).}\;
	设 $n\ge 2$,$A_n\lhd S_n$ 为交错群,$\mathrm{sgn}:S_n\to\{\pm1\}$ 为符号同态。
	则对任意正规子群 $N\lhd S_n$,成立:
	
	\begin{itemize}
		\item \textbf{(1) 像与包含关系.}\;
		\[
		\mathrm{sgn}(N)\lhd\{\pm1\},
		\qquad\text{从而}\qquad
		\mathrm{sgn}(N)=\{1\}\ \text{或}\ \mathrm{sgn}(N)=\{\pm1\}.
		\]
		并且
		\[
		\mathrm{sgn}(N)=\{1\}\iff N\subseteq A_n.
		\]
		等价地,
		\[
		N\subseteq A_n\iff N\subseteq\ker(\mathrm{sgn})\iff \mathrm{sgn}|_N\equiv 1.
		\]
		
		\item \textbf{(2) 交仍正规.}\;
		\[
		N\cap A_n\lhd A_n.
		\]
	\end{itemize}
	
	\textbf{证明.}\;
	(1)\ 因 $\mathrm{sgn}$ 为群同态且 $\ker(\mathrm{sgn})=A_n$,故对任意 $N\le S_n$,
	$\mathrm{sgn}(N)$ 为 $\{\pm1\}$ 的子群;又因 $\{\pm1\}$ 为交换群,任一子群皆正规,
	从而 $\mathrm{sgn}(N)\lhd\{\pm1\}$,并必为 $\{1\}$ 或 $\{\pm1\}$。
	若 $\mathrm{sgn}(N)=\{1\}$,则任意 $x\in N$ 满足 $\mathrm{sgn}(x)=1$,即 $x$ 为偶置换,
	故 $N\subseteq A_n$;反之若 $N\subseteq A_n=\ker(\mathrm{sgn})$,则 $\mathrm{sgn}(N)=\{1\}$。
	
	(2)\ 任取 $a\in A_n$ 与 $x\in N\cap A_n$。
	由于 $N\lhd S_n$,有 $axa^{-1}\in N$;由于 $A_n\lhd S_n$ 且 $a\in A_n$、$x\in A_n$,
	有 $axa^{-1}\in A_n$。故 $axa^{-1}\in N\cap A_n$,从而 $N\cap A_n\lhd A_n$。
	
	
	\clearpage
	\subsection*{课后习题答案}
	\addcontentsline{toc}{subsection}{\textcolor{red}{课后习题答案}}
	\begin{enumerate}[label=\textcolor{blue}{\textbf{\large\arabic*.}}]
		\item 
		\[
		\textbf{题目.}\quad \text{证明: } S_3 \simeq \mathrm{Aut}(S_3)=\mathrm{Inn}(S_3).
		\]
		
		\[
		\textbf{答案:}
		\]
		
		\textbf{证明.}
		设 $G=S_3$。记 $\mathrm{Inn}(G)$ 为内自同构群,$\mathrm{Aut}(G)$ 为自同构群。证明分两部分:
		
		\medskip
		\textbf{(I) 证明 $S_3\simeq \mathrm{Inn}(S_3)$.}
		
		定义映射
		\[
		\mathrm{Ad}:S_3\longrightarrow \mathrm{Inn}(S_3),\qquad 
		\sigma\longmapsto \mathrm{Ad}_\sigma,
		\]
		其中
		\[
		\mathrm{Ad}_\sigma:S_3\to S_3,\qquad \mathrm{Ad}_\sigma(\tau)=\sigma\tau\sigma^{-1}.
		\]
		显然 $\mathrm{Ad}_\sigma\in\mathrm{Aut}(S_3)$,并且 $\mathrm{Ad}$ 是群同态,因为对任意 $\sigma_1,\sigma_2\in S_3$,
		\[
		\mathrm{Ad}_{\sigma_1\sigma_2}(\tau)=(\sigma_1\sigma_2)\tau(\sigma_1\sigma_2)^{-1}
		=\sigma_1(\sigma_2\tau\sigma_2^{-1})\sigma_1^{-1}
		=(\mathrm{Ad}_{\sigma_1}\circ \mathrm{Ad}_{\sigma_2})(\tau),
		\]
		故 $\mathrm{Ad}_{\sigma_1\sigma_2}=\mathrm{Ad}_{\sigma_1}\circ \mathrm{Ad}_{\sigma_2}$,从而
		\[
		\mathrm{Ad}(\sigma_1\sigma_2)=\mathrm{Ad}(\sigma_1)\,\mathrm{Ad}(\sigma_2).
		\]
		
		又由 $\mathrm{Inn}(S_3)$ 的定义,任取 $\varphi\in\mathrm{Inn}(S_3)$,必存在某个 $\sigma\in S_3$ 使
		\[
		\varphi=\mathrm{Ad}_\sigma,
		\]
		因此 $\mathrm{Ad}$ 为满同态。
		
		接着求其核。对 $\sigma\in S_3$,
		\[
		\sigma\in \ker(\mathrm{Ad})
		\iff \mathrm{Ad}_\sigma=\mathrm{id}
		\iff \sigma\tau\sigma^{-1}=\tau,\ \forall \tau\in S_3
		\iff \sigma\in C(S_3),
		\]
		其中 $C(S_3)$ 表示中心(亦记作 $Z(S_3)$)。于是
		\[
		\ker(\mathrm{Ad})=C(S_3).
		\]
		下面证明 $C(S_3)=\{1\}$。
		
		事实上,$S_3$ 的共轭类按循环类型划分:
		\[
		\{1\},\qquad \{(12),(13),(23)\},\qquad \{(123),(132)\}.
		\]
		若 $\sigma\in C(S_3)$,则对任意 $\tau\in S_3$ 有 $\tau\sigma\tau^{-1}=\sigma$,
		即 $\sigma$ 的共轭类只有一个元素,故 $\sigma$ 必属于大小为 $1$ 的共轭类,只能是 $\sigma=1$。
		因此
		\[
		C(S_3)=\{1\}.
		\]
		从而
		\[
		\ker(\mathrm{Ad})=\{1\}.
		\]
		故 $\mathrm{Ad}$ 既满又单,是同构,于是得到
		\[
		S_3 \simeq \mathrm{Inn}(S_3),
		\qquad\text{并且}\qquad |\mathrm{Inn}(S_3)|=|S_3|=6.
		\]
		
		\medskip
		\textbf{(II) 证明 $\mathrm{Aut}(S_3)=\mathrm{Inn}(S_3)$.}
		
		注意 $S_3$ 的元素为
		\[
		S_3=\{1,\ (12),(13),(23),\ (123),(132)\}.
		\]
		任取 $f\in\mathrm{Aut}(S_3)$。由于同构保持元素的阶,故
		\[
		\mathrm{ord}(f(x))=\mathrm{ord}(x),\qquad \forall x\in S_3.
		\]
		因此 $f$ 必把 2 阶元素映到 2 阶元素,把 3 阶元素映到 3 阶元素。
		在 $S_3$ 中 2 阶元素恰为三个换位
		\[
		(12),(13),(23),
		\]
		3 阶元素恰为两个 3-循环
		\[
		(123),(132).
		\]
		又注意到
		\[
		(123)=(12)(23),\qquad (132)=(13)(12),
		\]
		因此一旦确定了 $f$ 在三个换位上的取值,$f$ 在两个 3-循环上的取值也随之唯一确定。
		更具体地,若设
		\[
		f(12)=\alpha,\qquad f(13)=\beta,\qquad f(23)=\gamma,
		\]
		其中 $\alpha,\beta,\gamma$ 均为换位,则由同态性
		\[
		f(123)=f\bigl((12)(23)\bigr)=f(12)f(23)=\alpha\gamma,
		\]
		\[
		f(132)=f\bigl((13)(12)\bigr)=f(13)f(12)=\beta\alpha.
		\]
		从而 $f$ 完全由它对 2 阶元素集合 $\{(12),(13),(23)\}$ 的排列所决定。
		
		于是可得
		\[
		|\mathrm{Aut}(S_3)|\le 3!=6.
		\]
		另一方面恒有包含关系
		\[
		\mathrm{Inn}(S_3)\le \mathrm{Aut}(S_3),
		\]
		故
		\[
		|\mathrm{Aut}(S_3)|\ge |\mathrm{Inn}(S_3)|=6.
		\]
		综上
		\[
		6=|\mathrm{Inn}(S_3)|\le |\mathrm{Aut}(S_3)|\le 6,
		\]
		因此
		\[
		|\mathrm{Aut}(S_3)|=6,\qquad \mathrm{Aut}(S_3)=\mathrm{Inn}(S_3).
		\]
		
		\medskip
		\textbf{(III) 合并结论.}
		由 (I) 得 $S_3\simeq \mathrm{Inn}(S_3)$,由 (II) 得 $\mathrm{Aut}(S_3)=\mathrm{Inn}(S_3)$,
		故
		\[
		S_3 \simeq \mathrm{Aut}(S_3)=\mathrm{Inn}(S_3).
		\]
		
		
			\item 6.
			\textbf{定理.}\;
			设 $n\ge 3$,则对称群 $S_n$ 的非平凡正规子群只有以下几种情形:
			\[
			\begin{cases}
				\text{当 } n\neq 4 \text{ 时,非平凡正规子群只有 } A_n;\\[4pt]
				\text{当 } n=4 \text{ 时,非平凡正规子群为 } A_4 \text{ 与 } K_4,
			\end{cases}
			\]
			其中
			\[
			K_4=\{e,(12)(34),(13)(24),(14)(23)\}.
			\]
			
			\textbf{证明:}
			
			\textbf{第一步:$A_n$ 一定是 $S_n$ 的非平凡正规子群.}
			
			定义符号映射
			\[
			\operatorname{sgn}:S_n\to\{\pm1\},
			\]
			则
			\[
			A_n=\ker(\operatorname{sgn}),
			\]
			因此 $A_n\lhd S_n$,且当 $n\ge 3$ 时,
			\[
			\{e\}\subsetneq A_n\subsetneq S_n,
			\]
			故 $A_n$ 是 $S_n$ 的非平凡正规子群。
			
			---
			
			\textbf{第二步:设 $N\lhd S_n$ 为非平凡正规子群,证明 $A_n\subseteq N$(除 $n=4$ 的特殊情形外).}
			
			由于 $N\neq\{e\}$,存在 $\sigma\in N$,$\sigma\neq e$。
			
			\textbf{关键观察:}  
			正规性意味着 $N$ 对共轭封闭,即
			\[
			\forall g\in S_n,\quad g\sigma g^{-1}\in N.
			\]
			
			\textbf{情形 1:$N$ 含有奇置换.}
			
			则 $N$ 与 $A_n$ 的交为指数为 $2$ 的子群,
			从而
			\[
			N=S_n,
			\]
			与 $N$ 非平凡但非整体矛盾。
			因此若 $N\neq S_n$,则 $N$ 只能包含偶置换。
			
			---
			
			\textbf{情形 2:$N$ 只包含偶置换.}
			
			此时
			\[
			N\subseteq A_n.
			\]
			
			接下来利用一个事实:
			
			\textbf{事实:}  
			当 $n\ge 5$ 时,$A_n$ 由 3-轮换生成,
			且所有 3-轮换在 $S_n$ 中互为共轭。
			
			因此,只要 $N$ 中包含一个 3-轮换,
			正规性立刻推出
			\[
			A_n\subseteq N,
			\]
			从而
			\[
			N=A_n.
			\]
			
			---
			
			\textbf{第三步:为什么 $n=4$ 是例外?}
			
			在 $S_4$ 中,3-轮换不再构成一个单一的共轭类,
			而且存在一个特殊的子群:
			\[
			K_4=\{e,(12)(34),(13)(24),(14)(23)\}.
			\]
			
			下面逐条验证:
			
			\textbf{(1) $K_4$ 是子群.}
			
			- 含单位元;
			- 每个非单位元都是二阶元;
			- 任意两个非单位元之积仍在 $K_4$ 中。
			
			因此 $K_4\cong C_2\times C_2$。
			
			---
			
			\textbf{(2) $K_4\lhd S_4$.}
			
			$S_4$ 中的共轭作用保持“置换的循环类型”,
			而 $K_4$ 的三个非单位元都是“两个不相交换位”,
			它们在 $S_4$ 中彼此共轭,
			故 $K_4$ 对共轭封闭,从而
			\[
			K_4\lhd S_4.
			\]
			
			---
			
			\textbf{(3) $K_4$ 不等于 $A_4$.}
			
			因为
			\[
			|K_4|=4,\qquad |A_4|=12,
			\]
			且
			\[
			K_4\subsetneq A_4.
			\]
			
			---
			
			\textbf{第四步:排除其他可能性.}
			
			- $A_n$ 在 $n\ge 5$ 时是单群;
			- 因此在 $S_n$ 中,介于 $\{e\}$ 与 $A_n$ 之间不可能再有正规子群;
			- $n=4$ 是唯一存在“非平凡、非 $A_n$、仍正规”的例外。
			
			---
			
			\textbf{结论.}
			
			\[
			\boxed{
				\begin{cases}
					\text{当 } n\neq 4,\ S_n \text{ 的非平凡正规子群只有 } A_n;\\[4pt]
					\text{当 } n=4,\ S_4 \text{ 的非平凡正规子群为 } A_4 \text{ 与 } K_4.
				\end{cases}
			}
			\]
			

		

		
		\item 10.
\textbf{题目.}\;
证明:置换群 $G$ 中若有奇置换,则 $G$ 中一定有指数为 $2$ 的正规子群。

\textbf{答案:}\;
\textbf{证明.}
设 $G$ 为某个置换群(作用在有限集合上),并假设 $G$ 中存在奇置换。
令
\[
H:=\{\sigma\in G:\ \sigma\text{ 为偶置换}\}.
\]
也就是说,$H$ 是 $G$ 中所有偶置换构成的集合。下面按题图思路分三步:

\medskip
\textbf{(1) 证明 $H<G$($H$ 是子群).}
取任意 $\alpha,\beta\in H$,则 $\alpha,\beta$ 都是偶置换。
利用置换奇偶性的基本性质:

\begin{itemize}
	\item 偶 $\circ$ 偶 $=$ 偶;
	\item 偶置换的逆仍为偶置换。
\end{itemize}

于是 $\alpha\beta\in H$ 且 $\alpha^{-1}\in H$。
又恒等置换 $e$ 是偶置换,故 $e\in H$。
因此 $H$ 对乘法与取逆封闭,并含单位元,从而 $H$ 是 $G$ 的子群,即 $H<G$。

\medskip
\textbf{(2) 证明 $H\trianglelefteq G$($H$ 是正规子群).}
任取 $g\in G$ 与 $h\in H$,考虑共轭 $ghg^{-1}$。
仍用奇偶性的基本性质:奇偶性在共轭下不变。具体地,
\[
\operatorname{sgn}(ghg^{-1})
=\operatorname{sgn}(g)\operatorname{sgn}(h)\operatorname{sgn}(g^{-1})
=\operatorname{sgn}(g)\operatorname{sgn}(h)\operatorname{sgn}(g)^{-1}
=\operatorname{sgn}(h)=1,
\]
因为 $h$ 是偶置换所以 $\operatorname{sgn}(h)=1$,并且 $\operatorname{sgn}(g^{-1})=\operatorname{sgn}(g)^{-1}$。
故 $ghg^{-1}$ 仍为偶置换,且显然仍在 $G$ 中,所以 $ghg^{-1}\in H$。
由 $g,h$ 任意,得 $gHg^{-1}\subseteq H$;同理对 $g^{-1}$ 也成立,从而 $gHg^{-1}=H$。
因此 $H\trianglelefteq G$。

\medskip
\textbf{(3) 若 $G$ 含奇置换 $\tau$,则 $G=H\cup \tau H$,从而 $[G:H]=2$.}
取题设中的某个奇置换 $\tau\in G$。

先证:\emph{任意奇置换都在左陪集 $\tau H$ 中。}
设 $g\in G$ 是奇置换,则 $\tau^{-1}g$ 的奇偶性为
\[
\operatorname{sgn}(\tau^{-1}g)=\operatorname{sgn}(\tau^{-1})\operatorname{sgn}(g)
=(-1)\cdot(-1)=1,
\]
所以 $\tau^{-1}g$ 为偶置换,且 $\tau^{-1}g\in G$,故 $\tau^{-1}g\in H$。
于是存在 $h\in H$ 使得 $\tau^{-1}g=h$,即 $g=\tau h\in \tau H$。

再证:\emph{$H$ 与 $\tau H$ 不相交。}
若存在 $x\in H\cap \tau H$,则 $x$ 既是偶置换,又可写成 $x=\tau h$(其中 $h\in H$ 为偶置换)。
但 $\tau h$ 的奇偶性为
\[
\operatorname{sgn}(\tau h)=\operatorname{sgn}(\tau)\operatorname{sgn}(h)=(-1)\cdot 1=-1,
\]
所以 $\tau h$ 是奇置换,与 $x$ 是偶置换矛盾。
故 $H\cap \tau H=\varnothing$。

最后证:\emph{$G$ 被这两个陪集覆盖。}
任取 $g\in G$:
若 $g$ 为偶置换,则 $g\in H$;
若 $g$ 为奇置换,则由上面已证 $g\in \tau H$。
因此
\[
G=H\cup \tau H,
\qquad H\cap \tau H=\varnothing.
\]
这说明 $G$ 恰好被两个互不相交的左陪集分成两块,所以
\[
[G:H]=2.
\]

综上,$H$ 是 $G$ 的正规子群且指数为 $2$,证得结论。

		\item 11.
		\textbf{题目.}\;
		设 $|G|=2n$,其中 $n$ 为奇数。证明:$G$ 有指数为 $2$ 的正规子群。
		
		\textbf{答案:}\;
		\textbf{证明.}
		由 Cayley 定理,$G$ 可嵌入某个对称群中:令
		\[
		L:G\longrightarrow S_G,\qquad g\longmapsto L_g,
		\]
		其中 $S_G$ 表示集合 $G$ 上的全体置换群,$L_g$ 为左乘置换
		\[
		L_g(x)=gx\qquad (x\in G).
		\]
		则 $L$ 为单射群同态,故
		\[
		S:=L(G)\cong G,\qquad |S|=|G|=2n.
		\]
		从而 $S$ 是一个 \emph{偶数阶} 的置换群($S\le S_{2n}$ 当然也成立)。
		
		\medskip
		\textbf{(1) 由偶数阶推出 $S$ 含有二阶元.}
		由 Cauchy 定理(或“偶数阶群必有二阶元”),因为 $2\mid |S|$,存在 $h\in S$ 使得
		\[
		|h|=2,\quad h\neq e.
		\]
		
		\medskip
		\textbf{(2) 分析 $h$ 作为置换的循环分解,并证明它是奇置换.}
		由于 $h$ 是集合 $G$ 上的一个置换,且 $h^2=e$,故 $h$ 的循环分解只能由长度为 $1$ 或 $2$ 的循环组成,
		即
		\[
		h=(a_1\,b_1)(a_2\,b_2)\cdots(a_t\,b_t),
		\]
		其中这些对换两两不相交,$t\ge 1$。
		
		现在利用 $S=L(G)$ 的特殊性:$h=L_{g_0}$ 对某个 $g_0\in G$。
		由于 $|h|=2$,有 $g_0\neq e$ 且 $g_0^2=e$。
		
		对任意 $x\in G$,若 $h(x)=x$,则
		\[
		g_0x=x \ \Longrightarrow\ g_0=e,
		\]
		矛盾。因此 $h$ \emph{没有不动点},即所有元素都出现在某个 $2$-循环里。
		所以 $h$ 的循环分解事实上是由 \emph{恰好 $n$ 个} 不相交的对换组成:
		\[
		h=(a_1\,b_1)(a_2\,b_2)\cdots(a_n\,b_n).
		\]
		于是 $h$ 的符号为
		\[
		\operatorname{sgn}(h)=(-1)^n=-1
		\quad (\text{因为 }n\text{ 为奇数}),
		\]
		即 $h$ 是一个 \emph{奇置换}。
		
		\medskip
		\textbf{(3) 用第 10 题结论(“含奇置换 $\Rightarrow$ 有指数 2 的正规子群”).}
		把第 10 题应用于置换群 $S$:既然 $S$ 中含有奇置换 $h$,
		令
		\[
		H:=\{\sigma\in S:\ \sigma\text{ 为偶置换}\},
		\]
		则 $H\trianglelefteq S$ 且 $[S:H]=2$(第 10 题已证)。
		
		\medskip
		\textbf{(4) 把指数为 2 的正规子群拉回到 $G$.}
		因为 $L:G\to S$ 是同构到像的单射同态,取
		\[
		N:=L^{-1}(H)\le G.
		\]
		由于 $H\trianglelefteq S$,故 $N\trianglelefteq G$;并且由陪集对应或指数保持性,
		\[
		[G:N]=[S:H]=2.
		\]
		因此 $G$ 有指数为 $2$ 的正规子群 $N$,证毕。
		
		\textbf{答案:}
		
		\textbf{第一步:由 Cauchy 定理取一个二阶元.}
		
		因为 $2\mid |G|$,由 Cauchy 定理,存在 $x\in G$ 使得
		\[
		|x|=2,\qquad x\neq e,\qquad x^2=e.
		\]
		
		\textbf{第二步:考虑 $x$ 对 $G$ 的左乘作用并得到“配对”.}
		
		定义映射
		\[
		f:G\to G,\qquad f(g)=xg.
		\]
		则对任意 $g\in G$,
		\[
		f(f(g))=x(xg)=x^2g=g,
		\]
		故 $f^2=\mathrm{id}$。
		
		并且 $f$ 无不动点:若 $f(g)=g$,则 $xg=g$,右乘 $g^{-1}$ 得 $x=e$,
		矛盾。因此
		\[
		f(g)\neq g\quad(\forall g\in G).
		\]
		于是 $G$ 的元素被 $f$ 的轨道划分为大小为 $2$ 的集合
		\[
		\{g,xg\}\qquad (g\in G),
		\]
		从而 $|G|$ 为偶数(这与已知一致),并且每一对中恰有一个元素属于某个“选定的半边”。
		
		\textbf{第三步:取 $H$ 为所有奇置换(或更一般地:构造一个到 $\{\pm1\}$ 的同态).}
		
		更直接地,我们利用 $n$ 为奇数这一点来构造指数为 $2$ 的子群。
		
		令 $X$ 为 $G$ 的所有 Sylow-$n$ 子群的集合:
		\[
		X:=\{P\le G\mid |P|=n\}.
		\]
		由 Sylow 定理,$X\neq\varnothing$,且 Sylow-$n$ 子群个数 $n_n:=|X|$ 满足
		\[
		n_n\equiv 1\pmod n,\qquad n_n\mid 2.
		\]
		因此
		\[
		n_n=1.
		\]
		于是 $G$ 存在唯一的 Sylow-$n$ 子群,记为 $N$,且
		\[
		|N|=n,\qquad N\lhd G.
		\]
		
		\textbf{第四步:验证 $[G:N]=2$ 并推出正规性.}
		
		由拉格朗日定理,
		\[
		[G:N]=\frac{|G|}{|N|}=\frac{2n}{n}=2.
		\]
		因此 $N$ 是指数为 $2$ 的子群。
		
		而任意指数为 $2$ 的子群必为正规子群:
		因为 $G/N$ 的阶为 $2$,从而是阿贝尔群,
		等价地说 $N$ 是某个同态到 $\{\pm1\}$ 的核,故 $N\lhd G$。
		
		\textbf{结论.}\;
		$G$ 含有指数为 $2$ 的正规子群(事实上就是唯一的 Sylow-$n$ 子群)。
		
	
		
	\end{enumerate}
	
	% -------------------- 1.7 群的扩张与 John–Hölder 定理 --------------------
		\clearpage
	\section{群的扩张与 John–Hölder 定理}
	\subsection{P38页表格}
	% 需要的宏包:booktabs, tabularx, array
	% \usepackage{booktabs}
	% \usepackage{tabularx}
	% \usepackage{array}
	
	\begin{table}[htbp]
		\centering
		\renewcommand{\arraystretch}{1.35}
		\begin{tabularx}{\textwidth}{@{}>{\centering\arraybackslash}X
				>{\centering\arraybackslash}X
				>{\centering\arraybackslash}X
				>{\centering\arraybackslash}X
				>{\centering\arraybackslash}X@{}}
			\toprule
			\textbf{群 \(G\)} & \textbf{正规子群 \(N\)} & \textbf{\(A\)} & \textbf{商群 \(G/N\)} & \textbf{\(B\)} \\
			\midrule
			数域 \(\mathbb{P}\) 上线性空间 \(V\) & 子空间 \(W\) & \(W\) & 商空间 \(V/W\) & \(V/W\) \\
			
			\(G=\{e,a,a^2,a^3\}\) 为 \(4\) 阶循环群 & \(\{e,a^2\}\) & \(\mathbb{Z}/2\mathbb{Z}\) & \(\{\bar{1},\bar{a}\}\) & \(\mathbb{Z}/2\mathbb{Z}\) \\
			
			\(\{e,a,b,c \mid a^2=b^2=c^2=abc=e\}\) & \(\{e,a\}\) & \(\mathbb{Z}/2\mathbb{Z}\) & \(\{\bar{1},\bar{b}\}\) & \(\mathbb{Z}/2\mathbb{Z}\) \\
			
			\(S_3\) & \(A_3\) & \(\mathbb{Z}/3\mathbb{Z}\) & \(S_3/A_3\) & \(\mathbb{Z}/2\mathbb{Z}\) \\
			
			\(\mathbb{Z}\) & \(2\mathbb{Z}\) & \(\mathbb{Z}\) & \(\mathbb{Z}/2\mathbb{Z}\) & \(\mathbb{Z}/2\mathbb{Z}\) \\
			
			\(O(n)\) & \(SO(n)\) & \(SO(n)\) & \(O(n)/SO(n)\) & \(\mathbb{Z}/2\mathbb{Z}\) \\
			\bottomrule
		\end{tabularx}
	\end{table}
	
	\begin{itemize}
		
		% =========================================================
		% 1. 向量空间 V 与子空间 W(加法群)
		% =========================================================
		\item \textbf{例 1:线性空间 }V\textbf{ 与子空间 }W.\;
		将 \(V,W\) 视为加法群(分别记为 \(\{V;+\}\)、\(\{W;+\}\))。
		令
		\[
		G=\{V;+\},\quad A=\{W;+\},\quad B=\{V/W;+\}.
		\]
		构造
		\[
		\lambda: A\to G,\ \lambda(w)=w \quad(\text{自然嵌入}),
		\qquad
		\mu: G\to B,\ \mu(v)=v+W \quad(\text{自然同态}).
		\]
		\textbf{证明.}\;
		\(\lambda\) 显然是群同态且单射,\(\mathrm{Im}(\lambda)=W\)。
		\(\mu\) 是群同态且满射(任意 \(v+W\in V/W\) 都是 \(\mu(v)\))。
		并且
		\[
		\mathrm{ker}(\mu)=\{v\in V: v+W=W\}=W=\mathrm{Im}(\lambda).
		\]
		故
		\[
		1\to \{W;+\}\xrightarrow{\lambda}\{V;+\}\xrightarrow{\mu}\{V/W;+\}\to 1
		\]
		为短正合序列,\(\{V;+\}\) 是 \(\{V/W;+\}\) 过 \(\{W;+\}\) 的扩张。
		
		% =========================================================
		% 2. C4 扩张
		% =========================================================
		\item \textbf{例 2:}$G=\{e,a,a^2,a^3\} \textbf{ 为 4 阶循环群 }$,
		取
		\[
		N=\{e,a^2\}\lhd G,\qquad A\cong \{\mathbb{Z}/2\mathbb{Z};+\},\qquad B\cong \{\mathbb{Z}/2\mathbb{Z};+\}.
		\]
		定义同态(用 \(\bar{0},\bar{1}\) 表示模 \(2\) 的元素)
		\[
		\lambda:\{\mathbb{Z}/2\mathbb{Z};+\}\to G,\quad
		\lambda(\bar{0})=e,\ \lambda(\bar{1})=a^2,
		\]
		\[
		\pi:G\to G/N,\quad \pi(g)=gN,
		\]
		并取同构
		\[
		\varphi:G/N\to \{\mathbb{Z}/2\mathbb{Z};+\},\quad
		\varphi(N)=\bar{0},\ \varphi(aN)=\bar{1}.
		\]
		令
		\[
		\mu:=\varphi\circ \pi: G\to \{\mathbb{Z}/2\mathbb{Z};+\}.
		\]
		\textbf{证明.}\;
		\(\lambda\) 是群同态,且 \(\lambda(\bar{1})\neq e\),故单射;并且 \(\mathrm{Im}(\lambda)=\{e,a^2\}=N\)。
		\(\pi\) 满射,\(\varphi\) 是同构,故 \(\mu\) 满射。
		又
		\[
		\mathrm{ker}(\mu)=\mathrm{ker}(\varphi\circ\pi)=\mathrm{ker}(\pi)=N=\mathrm{Im}(\lambda).
		\]
		因此
		\[
		1\to \{\mathbb{Z}/2\mathbb{Z};+\}\xrightarrow{\lambda} G \xrightarrow{\mu} \{\mathbb{Z}/2\mathbb{Z};+\}\to 1
		\]
		是短正合序列,\(G\) 是 \(B\) 过 \(A\) 的扩张。
		
		% =========================================================
		% 3. V4 扩张
		% =========================================================
		\item $\textbf{例 3:}G=\{e,a,b,c\}\textbf{ 且 }a^2=b^2=c^2=abc=e.\;$
		由关系 $abc=e$ 得 $ab = c^{-1}$,而由$c^2=e$得$c = c^{-1}$,从而$c=ab$,故 $G$ 为 Klein 四元群(与 \(\mathbb{Z}/2\mathbb{Z}\times \mathbb{Z}/2\mathbb{Z}\) 同构)。
		取
		\[
		N=\{e,a\}\lhd G,\qquad A\cong \{\mathbb{Z}/2\mathbb{Z};+\},\qquad B\cong \{\mathbb{Z}/2\mathbb{Z};+\}.
		\]
		定义
		\[
		\lambda:\{\mathbb{Z}/2\mathbb{Z};+\}\to G,\quad
		\lambda(\bar{0})=e,\ \lambda(\bar{1})=a.
		\]
		令 \(\pi(g)=gN\) 为商映射,并取同构
		\[
		\varphi:G/N\to \{\mathbb{Z}/2\mathbb{Z};+\},\quad
		\varphi(N)=\bar{0},\ \varphi(bN)=\bar{1}.
		\]
		令 \(\mu=\varphi\circ\pi\)。
		\textbf{证明.}\;
		\(\lambda\) 为同态且单射,\(\mathrm{Im}(\lambda)=\{e,a\}=N\)。
		\(\mu\) 为同态且满射(因为 \(\pi\) 满射且 \(\varphi\) 为同构)。
		并且
		\[
		\mathrm{ker}(\mu)=\mathrm{ker}(\pi)=N=\mathrm{Im}(\lambda).
		\]
		因此
		\[
		1\to \{\mathbb{Z}/2\mathbb{Z};+\}\xrightarrow{\lambda} G \xrightarrow{\mu} \{\mathbb{Z}/2\mathbb{Z};+\}\to 1
		\]
		为短正合序列,\(G\) 是 \(B\) 过 \(A\) 的扩张。
		
		% =========================================================
		% 4. S3 扩张
		% =========================================================
		\item $\textbf{例 4:}G=S_3,\ N=A_3.\;$
		取
		\[
		A\cong \{\mathbb{Z}/3\mathbb{Z};+\},\qquad B\cong \{\mathbb{Z}/2\mathbb{Z};+\}.
		\]
		定义嵌入
		\[
		\lambda:\{\mathbb{Z}/3\mathbb{Z};+\}\to S_3,\quad
		\lambda(\bar{k})=(123)^k\quad (k=0,1,2),
		\]
		则 \(\mathrm{Im}(\lambda)=\langle(123)\rangle=A_3\)。
		再取符号同态
		\[
		\mathrm{sgn}:S_3\to \{\pm 1\}\ (\text{乘法群}),
		\]
		并取同构
		\[
		\psi:\{\pm1\}\to \{\mathbb{Z}/2\mathbb{Z};+\},\quad \psi(1)=\bar{0},\ \psi(-1)=\bar{1}.
		\]
		令
		\[
		\mu:=\psi\circ \mathrm{sgn}:S_3\to \{\mathbb{Z}/2\mathbb{Z};+\}.
		\]
		\textbf{证明.}\;
		\(\lambda\) 是同态且单射(\((123)\) 阶为 \(3\),故 \(\lambda\) 把 \(\mathbb{Z}/3\mathbb{Z}\) 嵌入为 \(A_3\))。
		\(\mathrm{sgn}\) 满射(存在奇置换取值 \(-1\),偶置换取值 \(1\)),\(\psi\) 为同构,故 \(\mu\) 满射。
		并且
		\[
		\mathrm{ker}(\mu)=\mathrm{ker}(\psi\circ\mathrm{sgn})=\mathrm{ker}(\mathrm{sgn})=A_3=\mathrm{Im}(\lambda).
		\]
		因此
		\[
		1\to \{\mathbb{Z}/3\mathbb{Z};+\}\xrightarrow{\lambda} S_3 \xrightarrow{\mu} \{\mathbb{Z}/2\mathbb{Z};+\}\to 1
		\]
		为短正合序列,\(S_3\) 是 \(\{\mathbb{Z}/2\mathbb{Z};+\}\) 过 \(\{\mathbb{Z}/3\mathbb{Z};+\}\) 的扩张(扩张核为 \(A_3\))。
		
		% =========================================================
		% 5. Z 扩张
		% =========================================================
		\item $\textbf{例 5:}G=\{\mathbb{Z};+\},\ N=2\mathbb{Z}.\;$
		取
		\[
		A=\{\mathbb{Z};+\},\qquad B=\{\mathbb{Z}/2\mathbb{Z};+\}.
		\]
		定义
		\[
		\lambda:\{\mathbb{Z};+\}\to \{\mathbb{Z};+\},\quad \lambda(n)=2n,
		\]
		\[
		\mu:\{\mathbb{Z};+\}\to \{\mathbb{Z}/2\mathbb{Z};+\},\quad \mu(n)=\bar{n}.
		\]
		\textbf{证明.}\;
		\(\lambda\) 为同态且单射(\(2n=0\Rightarrow n=0\)),并且 \(\mathrm{Im}(\lambda)=2\mathbb{Z}\)。
		\(\mu\) 为同态且满射(\(\mu(0)=\bar{0},\ \mu(1)=\bar{1}\))。
		且
		\[
		\mathrm{ker}(\mu)=\{n\in\mathbb{Z}:\bar{n}=\bar{0}\}=2\mathbb{Z}=\mathrm{Im}(\lambda).
		\]
		故
		\[
		1\to \{\mathbb{Z};+\}\xrightarrow{\lambda}\{\mathbb{Z};+\}\xrightarrow{\mu}\{\mathbb{Z}/2\mathbb{Z};+\}\to 1
		\]
		为短正合序列,\(\{\mathbb{Z};+\}\) 是 \(\{\mathbb{Z}/2\mathbb{Z};+\}\) 过 \(\{\mathbb{Z};+\}\) 的扩张(扩张核为 \(2\mathbb{Z}\))。
		
		% =========================================================
		% 6. O(n) 扩张
		% =========================================================
		\item \textbf{例 6:}G=O(n),\ N=SO(n).\;
		取
		\[
		A=SO(n),\qquad B\cong \{\mathbb{Z}/2\mathbb{Z};+\}.
		\]
		定义
		\[
		\lambda:SO(n)\to O(n),\quad \lambda(M)=M\quad(\text{包含映射}),
		\]
		\[
		\det:O(n)\to \{\pm1\}\ (\text{乘法群}),
		\qquad
		\psi:\{\pm1\}\to \{\mathbb{Z}/2\mathbb{Z};+\},\ \psi(1)=\bar{0},\ \psi(-1)=\bar{1},
		\]
		并令
		\[
		\mu:=\psi\circ \det:O(n)\to \{\mathbb{Z}/2\mathbb{Z};+\}.
		\]
		\textbf{证明.}\;
		\(\lambda\) 显然为同态且单射,\(\mathrm{Im}(\lambda)=SO(n)\)。
		\(\det\) 为同态且满射:例如 \(I\in O(n)\) 满足 \(\det(I)=1\),而取任意一个反射矩阵(如对角矩阵 \(\mathrm{diag}(-1,1,\dots,1)\in O(n)\))有行列式 \(-1\),故 \(\det\) 满射,从而 \(\mu\) 亦满射。
		并且
		\[
		\mathrm{ker}(\mu)=\mathrm{ker}(\psi\circ\det)=\mathrm{ker}(\det)=\{M\in O(n):\det(M)=1\}=SO(n)=\mathrm{Im}(\lambda).
		\]
		故
		\[
		1\to SO(n)\xrightarrow{\lambda} O(n)\xrightarrow{\mu} \{\mathbb{Z}/2\mathbb{Z};+\}\to 1
		\]
		为短正合序列,\(O(n)\) 是 \(\{\mathbb{Z}/2\mathbb{Z};+\}\) 过 \(SO(n)\) 的扩张(扩张核为 \(SO(n)\))。
		
	\end{itemize}
	
	\subsection{例 1.7.5 中四条结论的证明}
	\textbf{题目.}\;
	\text{依据定义 1.7.4 证明例 1.7.5 中四条结论:}

	\[
	\begin{array}{l}
		(1)\ \mathbb Z\ \text{是}\ \mathbb Z/2\mathbb Z\ \text{过}\ \mathbb Z\ \text{的扩张,但不是非本质扩张}.\\
		(2)\ n\ge 3\ \text{时},\ S_n\ \text{是}\ \mathbb Z/2\mathbb Z\ \text{过}\ A_n\ \text{的非本质扩张}.\\
		(3)\ O(n)\ \text{是}\ \mathbb Z/2\mathbb Z\ \text{过}\ SO(n)\ \text{的非本质扩张}.\\
		(4)\ \text{15 阶循环群是}\ \mathbb Z/3\mathbb Z\ \text{过}\ \mathbb Z/5\mathbb Z\ \text{的平凡扩张}.
	\end{array}
	\]
	
	\[
	\textbf{答案:}
	\]
	
	\textbf{0.\ 预备:定义 1.7.4(按题图表述).}\;
	设 $G$ 是 $B$ 过 $A$ 的扩张,亦即存在满同态
	\[
	\pi:G\to B
	\]
	使得 $\mathrm{ker}(\pi)\cong A$,从而得到短正合列
	\[
	1\longrightarrow A \xrightarrow{\ \iota\ } G \xrightarrow{\ \pi\ } B \longrightarrow 1.
	\]
	记扩张核 $N=\mathrm{ker}(\pi)\ (\cong A)$。
	若存在子群 $H<G$ 满足
	\[
	G=HN,\qquad H\cap N=\{e\},
	\]
	则称该扩张为\textbf{非本质扩张},并且此时 $G\cong H\ltimes N$(半直积)。
	进一步,若还能取到 $H\lhd G$,则称该扩张为\textbf{平凡扩张},并且此时 $G\cong H\times N$(直积)。
	
	%========================================================
	\textbf{(1)\ $\mathbb Z$ 是 $\mathbb Z/2\mathbb Z$ 过 $\mathbb Z$ 的扩张,但不是非本质扩张.}
	%========================================================
	
	将 $\mathbb Z$ 视为加法群。
	定义满同态
	\[
	\pi:\mathbb Z\to \mathbb Z/2\mathbb Z,\qquad \pi(k)=\bar{k}\ (\bmod 2).
	\]
	则
	\[
	\mathrm{ker}(\pi)=2\mathbb Z=\{2k:k\in\mathbb Z\}.
	\]
	并且 $2\mathbb Z\cong \mathbb Z$(例如同构 $f:\mathbb Z\to 2\mathbb Z,\ f(k)=2k$)。
	因此有短正合列
	\[
	0\longrightarrow \mathbb Z \xrightarrow{\ k\mapsto 2k\ } \mathbb Z \xrightarrow{\ \bmod 2\ } \mathbb Z/2\mathbb Z \longrightarrow 0,
	\]
	所以 $\mathbb Z$ 的确是 $\mathbb Z/2\mathbb Z$ 过 $\mathbb Z$ 的扩张(扩张核 $N=2\mathbb Z$)。
	
	下面证明它\textbf{不是非本质扩张}:假设存在子群 $H<\mathbb Z$ 使得
	\[
	\mathbb Z=H+2\mathbb Z,\qquad H\cap 2\mathbb Z=\{0\}.
	\]
	任意 $\mathbb Z$ 的子群都形如 $H=d\mathbb Z$(某个 $d\ge 0$)。
	于是
	\[
	H+2\mathbb Z=d\mathbb Z+2\mathbb Z=\gcd(d,2)\mathbb Z.
	\]
	要使 $H+2\mathbb Z=\mathbb Z$,必须 $\gcd(d,2)=1$,即 $d$ 为奇数。
	但当 $d$ 为奇数时
	\[
	H\cap 2\mathbb Z=d\mathbb Z\cap 2\mathbb Z=\mathrm{lcm}(d,2)\mathbb Z=2d\mathbb Z\neq\{0\},
	\]
	与 $H\cap 2\mathbb Z=\{0\}$ 矛盾。
	故不存在这样的 $H$,所以该扩张不是非本质扩张。
	
	%========================================================
	\textbf{(2)\ $n\ge 3$ 时,$S_n$ 是 $\mathbb Z/2\mathbb Z$ 过 $A_n$ 的非本质扩张.}
	%========================================================
	
	考虑符号同态
	\[
	\mathrm{sgn}:S_n\to\{\pm1\}\cong \mathbb Z/2\mathbb Z,
	\]
	它是满同态,且
	\[
	\mathrm{ker}(\mathrm{sgn})=A_n.
	\]
	因此有短正合列
	\[
	1\longrightarrow A_n \longrightarrow S_n \xrightarrow{\ \mathrm{sgn}\ } \mathbb Z/2\mathbb Z \longrightarrow 1,
	\]
	故 $S_n$ 是 $\mathbb Z/2\mathbb Z$ 过 $A_n$ 的扩张,扩张核 $N=A_n$。
	
	取换位 $\tau=(12)$,令
	\[
	H=\langle \tau\rangle=\{e,\tau\}.
	\]
	则 $|H|=2$,并且 $\tau$ 是奇置换,所以 $H\cap A_n=\{e\}$。
	
	接着证明 $S_n=HA_n$:
	任取 $\sigma\in S_n$。
	\[
	\text{若 }\sigma\in A_n,\ \text{则 }\sigma=e\cdot\sigma\in HA_n.
	\]
	\[
	\text{若 }\sigma\notin A_n,\ \text{则 }\sigma\text{为奇置换,}\ \tau\sigma\text{为偶置换,故 }\tau\sigma\in A_n,
	\]
	从而
	\[
	\sigma=\tau(\tau\sigma)\in HA_n.
	\]
	于是 $S_n=HA_n$,且 $H\cap A_n=\{e\}$,由定义 1.7.4 可知该扩张是\textbf{非本质扩张},并且
	\[
	S_n\cong A_n\ltimes H.
	\]
	
	(补充:当 $n\ge 3$ 时该扩张不是“平凡扩张”,因为 $H=\langle(12)\rangle$ 在 $S_n$ 中不正规:例如
	$(13)(12)(13)^{-1}=(23)\notin H$,故 $H\not\lhd S_n$,从而不能得到直积分解。)
	
	%========================================================
	\textbf{(3)\ $O(n)$ 是 $\mathbb Z/2\mathbb Z$ 过 $SO(n)$ 的非本质扩张.}
	%========================================================
	
	考虑行列式同态
	\[
	\det:O(n)\to\{\pm1\}\cong \mathbb Z/2\mathbb Z.
	\]
	这是满同态,且
	\[
	\mathrm{ker}(\det)=SO(n).
	\]
	因此有短正合列
	\[
	1\longrightarrow SO(n)\longrightarrow O(n)\xrightarrow{\ \det\ }\mathbb Z/2\mathbb Z\longrightarrow 1,
	\]
	故 $O(n)$ 是 $\mathbb Z/2\mathbb Z$ 过 $SO(n)$ 的扩张,扩张核 $N=SO(n)$。
	
	取反射矩阵
	\[
	R=\mathrm{diag}(-1,1,\dots,1)\in O(n),\qquad \det(R)=-1,
	\]
	并令
	\[
	H=\langle R\rangle=\{I,R\}.
	\]
	则 $H\cap SO(n)=\{I\}$(因为 $R\notin SO(n)$ 且 $|H|=2$)。
	
	证明 $O(n)=HSO(n)$:
	任取 $Q\in O(n)$。
	\[
	\text{若 }\det(Q)=1,\ \text{则 }Q\in SO(n)\subseteq HSO(n).
	\]
	\[
	\text{若 }\det(Q)=-1,\ \text{则 }\det(RQ)=\det(R)\det(Q)=(-1)(-1)=1,\ \text{所以 }RQ\in SO(n),
	\]
	从而
	\[
	Q=R(RQ)\in HSO(n).
	\]
	于是 $O(n)=HSO(n)$ 且 $H\cap SO(n)=\{I\}$,由定义 1.7.4 可知该扩张为\textbf{非本质扩张},并且
	\[
	O(n)\cong SO(n)\ltimes H.
	\]
	
	(同样地,一般情形下 $H$ 不正规,从而不一定是“平凡扩张”。)
	
	%========================================================
	\textbf{(4)\ 15 阶循环群是 $\mathbb Z/3\mathbb Z$ 过 $\mathbb Z/5\mathbb Z$ 的平凡扩张.}
	%========================================================
	
	令 $G=C_{15}=\langle x\rangle$ 为 15 阶循环群。
	定义满同态
	\[
	\pi:G\to C_3,\qquad \pi(x)=\bar{1}\ (\bmod 3),
	\]
	即 $\pi(x^k)=\bar{k}\ (\bmod 3)$。
	显然 $\pi$ 满射,且
	\[
	\mathrm{ker}(\pi)=\{x^k:\ 3\mid k\}=\langle x^3\rangle.
	\]
	因为 $x$ 的阶为 15,所以 $x^3$ 的阶为 $15/\gcd(15,3)=5$,故
	\[
	\mathrm{ker}(\pi)=\langle x^3\rangle\cong C_5\cong \mathbb Z/5\mathbb Z.
	\]
	于是得到短正合列
	\[
	1\longrightarrow \mathbb Z/5\mathbb Z \longrightarrow C_{15}\xrightarrow{\ \pi\ } \mathbb Z/3\mathbb Z \longrightarrow 1,
	\]
	从而 $C_{15}$ 是 $\mathbb Z/3\mathbb Z$ 过 $\mathbb Z/5\mathbb Z$ 的扩张,扩张核 $N=\langle x^3\rangle$。
	
	取
	\[
	H=\langle x^5\rangle.
	\]
	则 $x^5$ 的阶为 $15/\gcd(15,5)=3$,所以 $H\cong C_3$,并且因为 $G$ 阿贝尔群,必有 $H\lhd G$。
	
	验证定义 1.7.4 的两个条件:
	\[
	H\cap N=\langle x^5\rangle\cap\langle x^3\rangle=\langle x^{\mathrm{lcm}(5,3)}\rangle=\langle x^{15}\rangle=\{e\}.
	\]
	又因为 $|H||N|=3\cdot 5=15=|G|$ 且 $H\cap N=\{e\}$,在有限群中推出
	\[
	|HN|=\frac{|H||N|}{|H\cap N|}=15=|G|,
	\]
	故 $G=HN$。
	于是满足 $G=HN$ 与 $H\cap N=\{e\}$,且 $H\lhd G$,由定义 1.7.4 可知该扩张为\textbf{平凡扩张},并且
	\[
	C_{15}\cong H\times N\cong C_3\times C_5\cong \mathbb Z/3\mathbb Z\times \mathbb Z/5\mathbb Z.
	\]
	
	
		\subsection{引理1.7.6}
		\textbf{题目.}\;
		\text{(引理 1.7.6)设 }A,B\text{ 是群 }G\text{ 的子群,且 }G=AB.
		\text{ 证明下列命题等价:}
		\[
		(1)\ A\cap B=\{e\};\qquad
		(2)\ \forall g\in G,\ g=ab\ (a\in A,b\in B)\text{ 的分解唯一};\qquad
		(3)\ e\text{ 的分解唯一。}
		\]
		
		\[
		\textbf{证明.}
		\]
		
		\textbf{(1)$\Rightarrow$(2).}\;
		设 \(g\in G\)。由于 \(G=AB\),存在 \(a,b\) 使 \(g=ab\)(其中 \(a\in A,b\in B\))。
		现证唯一性:若
		\[
		g=a_1b_1=a_2b_2\qquad (a_1,a_2\in A,\ b_1,b_2\in B),
		\]
		则
		\[
		a_2^{-1}a_1=b_2b_1^{-1}.
		\]
		左边 \(a_2^{-1}a_1\in A\),右边 \(b_2b_1^{-1}\in B\),故
		\[
		a_2^{-1}a_1\in A\cap B.
		\]
		由 (1) 得 \(A\cap B=\{e\}\),因此 \(a_2^{-1}a_1=e\),从而 \(a_1=a_2\)。
		再代回 \(a_1b_1=a_1b_2\) 得 \(b_1=b_2\)。
		故 \(g=ab\) 的表示唯一,(2) 成立。
		
		\textbf{(2)$\Rightarrow$(3).}\;
		(2) 对任意 \(g\in G\) 都成立,特别取 \(g=e\),立刻得到 \(e\) 的分解唯一,即 (3) 成立。
		
		\textbf{(3)$\Rightarrow$(1).}\;
		反证。设存在 \(x\in A\cap B\) 且 \(x\neq e\)。
		因为 \(x\in A\) 且 \(x\in B\),有两种 \(e\) 的分解:
		\[
		e=ee\qquad (e\in A,\ e\in B),
		\]
		以及
		\[
		e=x\cdot x^{-1}.
		\]
		注意 \(x\in A\) 且 \(A\) 为子群,故 \(x^{-1}\in A\);同时 \(x\in B\) 且 \(B\) 为子群,故 \(x^{-1}\in B\)。
		因此 \(x\cdot x^{-1}\) 也是形如 \(ab\)(\(a\in A,b\in B\))的分解,且其中 \(a=x\neq e\)。
		于是
		\[
		e=ee=x\cdot x^{-1}
		\]
		给出了 \(e\) 的两个不同分解,违背 (3) 的“唯一性”。
		矛盾说明不存在 \(x\neq e\) 属于 \(A\cap B\),因此
		\[
		A\cap B=\{e\},
		\]
		即 (1) 成立。
		
		综上,(1)\(\Leftrightarrow\)(2)\(\Leftrightarrow\)(3)。
		
	\subsection{直积与半直积}
	\textbf{直积与半直积(详细总结)}
	
	\begin{itemize}
		\item \textbf{总体直观.}\;
		直积与半直积都是把两个群“拼起来”构造新群的方式。
		直积对应“\emph{两部分互不干扰、彼此交换}”;
		半直积允许其中一部分通过\emph{共轭作用}去“扭动”另一部分,从而通常\emph{不交换}。
		
		% ------------------------------------------------------------
		\item \textbf{1. 外直积(direct product).}
		
		设 $P,Q$ 为群。定义集合
		\[
		P\times Q:=\{(p,q)\mid p\in P,\ q\in Q\},
		\]
		并按分量定义运算
		\[
		(p_1,q_1)\cdot(p_2,q_2):=(p_1p_2,\ q_1q_2).
		\]
		则 $P\times Q$ 构成群,单位元为 $(e_P,e_Q)$,逆元为
		\[
		(p,q)^{-1}=(p^{-1},q^{-1}).
		\]
		
		\textbf{直积的交换特征.}\;
		把 $P,Q$ 分别嵌入 $P\times Q$:
		\[
		\iota_P:P\hookrightarrow P\times Q,\ \iota_P(p)=(p,e_Q),\qquad
		\iota_Q:Q\hookrightarrow P\times Q,\ \iota_Q(q)=(e_P,q).
		\]
		则对任意 $p\in P,q\in Q$,
		\[
		(p,e_Q)(e_P,q)=(p,q)=(e_P,q)(p,e_Q),
		\]
		即 $\iota_P(P)$ 与 $\iota_Q(Q)$ 中的元素逐一交换。
		
		% ------------------------------------------------------------
		\item \textbf{2. 内直积判别(internal direct product).}
		
		设 $G$ 为群,$P,Q\le G$ 为子群。
		若满足
		\[
		P\trianglelefteq G,\quad Q\trianglelefteq G,\quad
		P\cap Q=\{e\},\quad PQ=G,
		\]
		则称 $G$ 为 $P,Q$ 的\emph{内直积},并且
		\[
		G\cong P\times Q.
		\]
		
		\textbf{证明思路(标准构造).}\;
		定义映射
		\[
		\Phi:P\times Q\to G,\qquad \Phi(p,q)=pq.
		\]
		\begin{itemize}
			\item (满射)由 $PQ=G$,任意 $g\in G$ 可写成 $g=pq$,故 $\Phi$ 满射。
			\item (同态)若再注意到 $P,Q$ 都正规则可推出 $pq=qp$(或直接假设 $[P,Q]=\{e\}$),从而
			\[
			\Phi(p_1,q_1)\Phi(p_2,q_2)=p_1q_1p_2q_2=p_1p_2q_1q_2=\Phi(p_1p_2,q_1q_2).
			\]
			\item (单射)若 $\Phi(p,q)=e$,则 $pq=e\Rightarrow p=q^{-1}\in P\cap Q=\{e\}$,
			故 $p=e,q=e$,从而 $\ker\Phi=\{(e,e)\}$,$\Phi$ 单射。
		\end{itemize}
		因此 $\Phi$ 为同构,得 $G\cong P\times Q$。
		
		\textbf{等价刻画.}\;
		若 $PQ=G$ 且 $P\cap Q=\{e\}$,并且满足逐元素交换
		\[
		pq=qp\quad (\forall p\in P,\ \forall q\in Q),
		\]
		则同样推出 $G\cong P\times Q$(此时甚至不必单独写出 $P,Q$ 正规)。
		
		% ------------------------------------------------------------
		\item \textbf{3. 外半直积(semidirect product).}
		
		半直积比直积多了一个数据:$Q$ 对 $P$ 的作用(群同态)
		\[
		\varphi:Q\longrightarrow \mathrm{Aut}(P).
		\]
		在集合 $P\times Q$ 上定义运算
		\[
		(p_1,q_1)\cdot(p_2,q_2)
		:=\bigl(p_1\,\varphi(q_1)(p_2),\ q_1q_2\bigr).
		\]
		则 $P\times Q$ 构成群,记作
		\[
		P\rtimes_{\varphi} Q.
		\]
		其单位元为 $(e_P,e_Q)$,逆元可计算为
		\[
		(p,q)^{-1}=\bigl(\varphi(q^{-1})(p^{-1}),\ q^{-1}\bigr).
		\]
		
		\textbf{直观含义.}\;
		乘法时先让 $q_1$ “作用”到 $p_2$ 上,把 $p_2$ 变为 $\varphi(q_1)(p_2)$,
		再与 $p_1$ 相乘;而 $Q$ 分量仍按原群乘法相乘。
		
		% ------------------------------------------------------------
		\item \textbf{4. 内半直积判别(internal semidirect product).}
		
		设 $G$ 为群,$P,Q\le G$。
		若满足
		\[
		P\trianglelefteq G,\quad P\cap Q=\{e\},\quad PQ=G,
		\]
		则称 $G$ 为 $P$ 与 $Q$ 的\emph{内半直积},并且
		\[
		G\cong P\rtimes Q,
		\]
		其中作用由共轭给出:
		\[
		\varphi:Q\to \mathrm{Aut}(P),\qquad
		\varphi(q)(p):=qpq^{-1}.
		\]
		
		\textbf{证明要点(构造同构).}\;
		定义
		\[
		\Psi:P\rtimes_{\varphi} Q\to G,\qquad \Psi(p,q)=pq.
		\]
		由 $PQ=G$ 得满射;由 $P\cap Q=\{e\}$ 得分解唯一性从而单射;
		并利用 $P\trianglelefteq G$ 与共轭作用的定义可验证 $\Psi$ 为同态,
		故 $\Psi$ 为同构。
		
		% ------------------------------------------------------------
		\item \textbf{5. 直积 vs 半直积.}
		
		\textbf{(平凡作用 $\Rightarrow$ 直积)}\;
		若 $\varphi(q)=\mathrm{id}_P$ 对所有 $q\in Q$ 成立,则半直积乘法退化为
		\[
		(p_1,q_1)\cdot(p_2,q_2)=(p_1p_2,\ q_1q_2),
		\]
		因此
		\[
		P\rtimes_{\varphi} Q = P\times Q.
		\]
		
		\textbf{(何时回到直积)}\;
		对内半直积而言,若进一步 $Q\trianglelefteq G$(或等价地 $[P,Q]=\{e\}$),
		则 $P$ 与 $Q$ 逐元素交换,从而
		\[
		G\cong P\times Q.
		\]
		若 $Q$ 不正规或作用非平凡,则通常只能得到半直积而非直积。
		
		% ------------------------------------------------------------
		\item \textbf{6. 经典例子:$S_3$ 是半直积但不是直积.}
		
		设
		\[
		C_3=\langle (123)\rangle,\qquad C_2=\langle (12)\rangle.
		\]
		则
		\[
		S_3 = C_3 C_2,\qquad C_3\cap C_2=\{e\},\qquad C_3\trianglelefteq S_3,
		\]
		因此
		\[
		S_3 \cong C_3\rtimes C_2.
		\]
		其作用为共轭作用:对 $r=(123)$ 与 $s=(12)$,
		\[
		srs^{-1}=r^{-1},
		\]
		说明 $C_2$ 在 $C_3$ 上的作用非平凡(把生成元送到逆元),因此该半直积不是直积。
		事实上
		\[
		C_3\times C_2 \cong C_6
		\]
		是交换群,而 $S_3$ 非交换,故 $S_3\not\cong C_3\times C_2$。
	\end{itemize}
	
	
	\clearpage		
	\subsection*{课后习题答案}
	\addcontentsline{toc}{subsection}{\textcolor{red}{课后习题答案}}
	\begin{enumerate}[label=\textcolor{blue}{\textbf{\large\arabic*.}}]
		\item 
		\textbf{题目.}\;
		\text{证明:\(\mathbb{Z}_{15}\) 是 \(\mathbb{Z}_3\) 过 \(\mathbb{Z}_5\) 的平凡扩张。}
		
		\textbf{证明:}

		把 \(\mathbb{Z}_{15},\mathbb{Z}_5,\mathbb{Z}_3\) 都看成加法群,元素用上划线表示。
		令
		\[
		A=\mathbb{Z}_5,\qquad G=\mathbb{Z}_{15},\qquad B=\mathbb{Z}_3.
		\]
		
		\medskip
		
		\textbf{第一步:构造 \(\lambda: A\to G\).}\;
		定义
		\[
		\lambda:\mathbb{Z}_5\longrightarrow \mathbb{Z}_{15},\qquad 
		\lambda(\bar{k})=\overline{3k}\quad (k=0,1,2,3,4).
		\]
		这是群同态,因为
		\[
		\lambda(\bar{k}+\bar{\ell})=\overline{3(k+\ell)}=\overline{3k}+\overline{3\ell}=\lambda(\bar{k})+\lambda(\bar{\ell}).
		\]
		若 \(\lambda(\bar{k})=\bar{0}\),则 \(\overline{3k}=\bar{0}\) 于 \(\mathbb{Z}_{15}\),即 \(15\mid 3k\),从而 \(5\mid k\)。
		但 \(k\in\{0,1,2,3,4\}\),故 \(k=0\),即 \(\bar{k}=\bar{0}\)。
		因此 \(\ker(\lambda)=\{\bar{0}\}\),\(\lambda\) 单射。
		并且
		\[
		\mathrm{Im}(\lambda)=\{\bar{0},\bar{3},\bar{6},\bar{9},\bar{12}\}=\langle \bar{3}\rangle
		\]
		是 \(\mathbb{Z}_{15}\) 的一个阶为 \(5\) 的子群。
		
		\medskip
		
		\textbf{第二步:构造 \(\mu:G\to B\).}\;
		定义
		\[
		\mu:\mathbb{Z}_{15}\longrightarrow \mathbb{Z}_3,\qquad 
		\mu(\bar{x})=\overline{x}\ \ (\text{模 }3).
		\]
		即 \(\mu(\bar{x})=\overline{x\ (\mathrm{mod}\ 3)}\)。
		这是群同态,因为
		\[
		\mu(\bar{x}+\bar{y})=\overline{x+y}=\overline{x}+\overline{y}=\mu(\bar{x})+\mu(\bar{y})
		\quad \text{于 }\mathbb{Z}_3.
		\]
		显然 \(\mu\) 满射(例如 \(\mu(\bar{0})=\bar{0},\ \mu(\bar{1})=\bar{1},\ \mu(\bar{2})=\bar{2}\))。
		
		并且
		\[
		\ker(\mu)=\{\bar{x}\in\mathbb{Z}_{15}: x\equiv 0\pmod 3\}
		=\{\bar{0},\bar{3},\bar{6},\bar{9},\bar{12}\}.
		\]
		因此
		\[
		\ker(\mu)=\mathrm{Im}(\lambda).
		\]
		
		\medskip
		
		\textbf{第三步:得到短正合序列(群扩张).}\;
		由上可知 \(\lambda\) 单射、\(\mu\) 满射且 \(\mathrm{Im}(\lambda)=\ker(\mu)\),故得到短正合序列
		\[
		0\longrightarrow \mathbb{Z}_5 \xrightarrow{\ \lambda\ } \mathbb{Z}_{15} \xrightarrow{\ \mu\ } \mathbb{Z}_3 \longrightarrow 0.
		\]
		因此 \(\mathbb{Z}_{15}\) 是 \(\mathbb{Z}_3\) 过 \(\mathbb{Z}_5\) 的一个扩张。
		
		\medskip
		
		\textbf{第四步:证明这是平凡扩张.}\;
		由于
		\[
		\gcd(3,5)=1,
		\]
		由中国剩余定理可得群同构
		\[
		\Phi:\mathbb{Z}_{15}\longrightarrow \mathbb{Z}_3\times \mathbb{Z}_5,\qquad
		\Phi(\bar{x})=(\bar{x}\bmod 3,\ \bar{x}\bmod 5),
		\]
		且 \(\Phi\) 为同构。
		
		于是
		\[
		\mathbb{Z}_{15}\cong \mathbb{Z}_3\times \mathbb{Z}_5,
		\]
		这正是 \(\mathbb{Z}_3\) 过 \(\mathbb{Z}_5\) 的\textbf{直积(平凡)扩张}。
		
		\medskip
		
		综上,\(\mathbb{Z}_{15}\) 是 \(\mathbb{Z}_3\) 过 \(\mathbb{Z}_5\) 的平凡扩张。
		
		\item 2.\textbf{题目.}
		给出加群 $\mathbb Z$ 的两个正规序列
		\[
		\mathbb Z \supset 20\mathbb Z \supset 60\mathbb Z \supset \{0\},
		\qquad
		\mathbb Z \supset 49\mathbb Z \supset 245\mathbb Z \supset \{0\}
		\]
		的一组同构加细。
		
		\textbf{解.}
		在第一个序列中插入子群 $4\mathbb Z,\ 420\mathbb Z,\ 2940\mathbb Z$,得到加细
		\[
		\mathbb Z \supset 4\mathbb Z \supset 20\mathbb Z \supset 60\mathbb Z 
		\supset 420\mathbb Z \supset 2940\mathbb Z \supset \{0\}.
		\tag{$\ast$}
		\]
		显然这是 $\mathbb Z \supset 20\mathbb Z \supset 60\mathbb Z \supset \{0\}$ 的一个正规加细。
		
		在第二个序列中插入子群 $7\mathbb Z,\ 735\mathbb Z,\ 2940\mathbb Z$,得到加细
		\[
		\mathbb Z \supset 7\mathbb Z \supset 49\mathbb Z \supset 245\mathbb Z 
		\supset 735\mathbb Z \supset 2940\mathbb Z \supset \{0\}.
		\tag{$\ast\ast$}
		\]
		这同样是 $\mathbb Z \supset 49\mathbb Z \supset 245\mathbb Z \supset \{0\}$ 的正规加细。
		
		注意对任意正整数 $a\mid b$,都有
		\[
		a\mathbb Z / b\mathbb Z \cong \mathbb Z_{\,b/a},
		\]
		是一个阶为 $b/a$ 的循环群;而 $n\mathbb Z / \{0\} \cong \mathbb Z$ 是无限循环群。
		
		对序列 $(\ast)$ 的各因子群,有
		\[
		\begin{aligned}
			\mathbb Z / 4\mathbb Z &\cong \mathbb Z_4,\\
			4\mathbb Z / 20\mathbb Z &\cong \mathbb Z_{20/4} \cong \mathbb Z_5,\\
			20\mathbb Z / 60\mathbb Z &\cong \mathbb Z_{60/20} \cong \mathbb Z_3,\\
			60\mathbb Z / 420\mathbb Z &\cong \mathbb Z_{420/60} \cong \mathbb Z_7,\\
			420\mathbb Z / 2940\mathbb Z &\cong \mathbb Z_{2940/420} \cong \mathbb Z_7,\\
			2940\mathbb Z / \{0\} &\cong \mathbb Z.
		\end{aligned}
		\]
		
		对序列 $(\ast\ast)$ 的各因子群,有
		\[
		\begin{aligned}
			\mathbb Z / 7\mathbb Z &\cong \mathbb Z_7,\\
			7\mathbb Z / 49\mathbb Z &\cong \mathbb Z_{49/7} \cong \mathbb Z_7,\\
			49\mathbb Z / 245\mathbb Z &\cong \mathbb Z_{245/49} \cong \mathbb Z_5,\\
			245\mathbb Z / 735\mathbb Z &\cong \mathbb Z_{735/245} \cong \mathbb Z_3,\\
			735\mathbb Z / 2940\mathbb Z &\cong \mathbb Z_{2940/735} \cong \mathbb Z_4,\\
			2940\mathbb Z / \{0\} &\cong \mathbb Z.
		\end{aligned}
		\]
		
		于是两个加细的因子群同构类型的多重集都是
		\[
		\{\mathbb Z_4,\ \mathbb Z_5,\ \mathbb Z_3,\ \mathbb Z_7,\ \mathbb Z_7,\ \mathbb Z\},
		\]
		只是在次序上有所不同。
		
		因此,上述两个正规序列 $(\ast)$ 与 $(\ast\ast)$ 是彼此同构的加细。
		
		
		\item 4.\textbf{题目.}  
		设 $p$ 是群 $G$ 的阶的最小素因子,$A$ 是 $G$ 的一个 $p$ 阶正规子群。证明:
		\[
		A \subseteq C(G).
		\]
		
		\textbf{证明.}
		
		因为 $|A|=p$ 为素数,故 $A$ 为循环群,设
		\[
		A = \langle a\rangle,\qquad |A| = p.
		\]
		于是
		\[
		\mathrm{Aut}(A)\cong \mathrm{Aut}(C_p)\cong C_{p-1},
		\]
		从而
		\[
		|\mathrm{Aut}(A)|=p-1.
		\]
		
		由于 $A\triangleleft G$,可考虑 $G$ 对 $A$ 的共轭作用:
		\[
		G\times A\longrightarrow A,\qquad (g,x)\longmapsto gxg^{-1}.
		\]
		该作用诱导出一个群同态
		\[
		\varphi:G\longrightarrow \mathrm{Aut}(A),\qquad
		\varphi(g)(x)=gxg^{-1}.
		\]
		记
		\[
		K=\ker \varphi=\{g\in G\mid gxg^{-1}=x,\ \forall\,x\in A\}.
		\]
		显然
		\[
		K=C_G(A)=\{g\in G\mid ga=ag,\ \forall\,a\in A\},
		\]
		即 $K$ 是 $A$ 的中心化子。
		
		由同态基本定理,
		\[
		G/K \cong \mathrm{Im}(\varphi)\le \mathrm{Aut}(A),
		\]
		故
		\[
		|G:K| = |G/K| \mid |\mathrm{Aut}(A)| = p-1.
		\]
		另一方面,$|G:K|$ 是 $|G|$ 的约数,而 $p$ 是 $|G|$ 的最小素因子。  
		若 $|G:K|\neq 1$,则它有一个素因子 $q$,并且 $q\mid |G:K|\mid |G|$,于是 $q\ge p$。  
		但又有 $|G:K|\mid p-1$,所以 $q\mid(p-1)$,从而 $q\le p-1$,矛盾。
		
		因此只能有
		\[
		|G:K|=1,\quad\text{即}\quad K=G.
		\]
		这说明
		\[
		G = C_G(A),
		\]
		也就是说 $A$ 中的每个元素都与 $G$ 的所有元素可交换。
		
		由于 $A\subseteq C_G(A)=G$,即
		\[
		A\subseteq C(G),
		\]
		其中 $C(G)$ 为 $G$ 的中心。
		
		证毕。
		
		
		\item 7.\textbf{题目.} 设 $A,B$ 为 $G$ 的正规子群且 $G=AB$。证明
		\[
		G/(A\cap B)\cong A/(A\cap B)\times B/(A\cap B).
		\]
		
		\textbf{证明.}
		记
		\[
		\overline{G}=G/(A\cap B),\qquad 
		\overline{A}=A/(A\cap B),\qquad 
		\overline{B}=B/(A\cap B),
		\]
		并用 $\pi:G\to\overline{G}$ 表示自然投影。
		
		由于 $A,B\trianglelefteq G$,则 $A\cap B\trianglelefteq G$,从而
		$\overline{A},\overline{B}$ 均为 $\overline{G}$ 的正规子群;又由 $G=AB$ 得
		\[
		\overline{G}
		= G/(A\cap B)
		= AB/(A\cap B)
		= \overline{A}\,\overline{B}.
		\]
		
		接着计算交集:
		\[
		\overline{A}\cap\overline{B}
		= \frac{A}{A\cap B}\cap\frac{B}{A\cap B}
		= \frac{A\cap B}{A\cap B}
		= \{1\},
		\]
		故 $\overline{A}\cap\overline{B}=\{1\}$。
		
		又因为 $A,B$ 都是 $G$ 的正规子群,对任意 $a\in A,b\in B$,换位子
		\[
		[a,b]=a^{-1}b^{-1}ab
		\]
		属于 $A$:这是因为 $b^{-1}ab\in A$($A$ 正规),故 $a^{-1}(b^{-1}ab)\in A$;
		同理 $[a,b]\in B$,于是
		\[
		[a,b]\in A\cap B.
		\]
		从而在商群 $\overline{G}$ 中
		\[
		[\overline{a},\overline{b}]
		=\overline{[a,b]}
		=\overline{e},
		\]
		即 $\overline{A},\overline{B}$ 中的元素两两交换,故
		\[
		\overline{A}\overline{B}=\overline{B}\overline{A}.
		\]
		
		综上,$\overline{A},\overline{B}$ 是 $\overline{G}$ 的正规子群,满足
		\[
		\overline{G}=\overline{A}\overline{B},\qquad
		\overline{A}\cap\overline{B}=\{1\},\qquad
		\overline{A}\overline{B}=\overline{B}\overline{A}.
		\]
		因此 $\overline{G}$ 是 $\overline{A}$ 与 $\overline{B}$ 的内部直积,即
		\[
		\overline{G}\cong \overline{A}\times\overline{B}.
		\]
		写回原记号即为
		\[
		G/(A\cap B)\cong A/(A\cap B)\times B/(A\cap B).
		\]
		证毕。
		
		
		\item 10.\textbf{题目.} 设群 $G$ 有合成序列,证明:$G$ 的任何正规子群及其商群都具有合成序列。
		
		\textbf{证明.}
		
		设 $G$ 有一条合成序列
		\[
		1 = G_n \triangleleft G_{n-1} \triangleleft \cdots \triangleleft G_1 \triangleleft G_0 = G,
		\]
		其中每个商群 $G_{i-1}/G_i$ 都是单群。
		
		\medskip
		
		\textbf{(1) 对任意正规子群 $N\triangleleft G$,$N$ 有合成序列。}
		
		对每个 $i$ 定义
		\[
		N_i := N\cap G_i,\qquad i=0,1,\dots,n.
		\]
		则
		\[
		N = N_0 \ge N_1 \ge \cdots \ge N_n = 1.
		\]
		
		因为 $N\triangleleft G$ 且 $G_i\triangleleft G$,可知 $N_i = N\cap G_i \triangleleft N\cap G_{i-1} = N_{i-1}$,于是
		\[
		1 = N_n \triangleleft N_{n-1} \triangleleft \cdots \triangleleft N_1 \triangleleft N_0 = N
		\]
		是一条正规链。
		
		由第三同构定理及同态基本定理,有同构
		\[
		\frac{N_{i-1}}{N_i}
		= \frac{N\cap G_{i-1}}{N\cap G_i}
		\cong \frac{N G_i \cap G_{i-1}}{G_i}
		\le \frac{G_{i-1}}{G_i},\qquad i=1,\dots,n.
		\]
		而 $G_{i-1}/G_i$ 为单群,所以 $N_{i-1}/N_i$ 不是平凡群就是单群。将其中相邻相等的 $N_i$ 删去,就得到一条从 $1$ 到 $N$ 的正规序列,其各因子均为单群,即为 $N$ 的一条合成序列。
		
		\medskip
		
		\textbf{(2) 商群 $G/N$ 有合成序列。}
		
		在 $G/N$ 中考虑链
		\[
		G/N = G_0N/N \ge G_1N/N \ge \cdots \ge G_nN/N = N/N.
		\]
		由于 $G_i\triangleleft G$ 且 $N\triangleleft G$,可知 $G_iN$ 也是 $G$ 的正规子群,从而
		\[
		G_iN/N \triangleleft G_{i-1}N/N,\qquad i=1,\dots,n,
		\]
		于是上面是一条 $G/N$ 的正规链。
		
		由第二、三同构定理有
		\[
		\frac{G_{i-1}N/N}{G_iN/N}
		\cong \frac{G_{i-1}N}{G_iN}
		\cong \frac{G_{i-1}}{G_{i-1}\cap G_iN},
		\]
		其中 $G_{i-1}/(G_{i-1}\cap G_iN)$ 是 $G_{i-1}/G_i$ 的一个商群,故要么平凡、要么为单群。于是
		\[
		\frac{G_{i-1}N/N}{G_iN/N}
		\]
		也只可能是平凡群或单群。照样将链中相邻相同的子群 $G_iN/N$ 删去,即得到一条从 $G/N$ 到 $1$ 的正规序列,其各因子群为单群,从而构成 $G/N$ 的一条合成序列。
		
		\medskip
		
		综上,若群 $G$ 有合成序列,则 $G$ 的任一正规子群 $N$ 与商群 $G/N$ 都具有合成序列。证毕。
		
		
	\end{enumerate}
	
	% -------------------- 1.8 可解群和幂零群 --------------------
		\clearpage
	\section{可解群和幂零群}

	
	\clearpage		
	\subsection*{课后习题答案}
	\addcontentsline{toc}{subsection}{\textcolor{red}{课后习题答案}}
	\begin{enumerate}[label=\textcolor{blue}{\textbf{\large\arabic*.}}]
		\item \textbf{题目.} 设 $H,K$ 是群 $G$ 的子群,证明:
		
		\begin{itemize}
			\item[(1)] $[H,K]=\{e\}$ 当且仅当 $H\subseteq C_G(K)$ 当且仅当 $K\subseteq C_G(H)$;
			\item[(2)] $[H,K]\subseteq K$ 当且仅当 $H\subseteq N_G(K)$;
			\item[(3)] 若 $H_1\le H,\ K_1\le K$,则 $[H_1,K_1]\subseteq [H,K]$。
		\end{itemize}
		
		这里 $[H,K]$ 表示由所有换位子
		\[
		[h,k]:=h^{-1}k^{-1}hk,\qquad h\in H,\ k\in K
		\]
		生成的子群。
		
		\textbf{证明.}
		
		\textbf{(1)} 先证 $[H,K]=\{e\}\Rightarrow H\subseteq C_G(K)$。
		
		若 $[H,K]=\{e\}$,则对一切 $h\in H,k\in K$ 有
		\[
		[h,k]=h^{-1}k^{-1}hk=e.
		\]
		于是
		\[
		h^{-1}k^{-1}hk=e
		\iff k^{-1}hk=h
		\iff hk=kh.
		\]
		这说明任意 $h\in H$ 与任意 $k\in K$ 都可交换,因此 $h$ 与 $K$ 中所有元素都可交换,即
		\[
		h\in C_G(K).
		\]
		故 $H\subseteq C_G(K)$。
		
		反过来,若 $H\subseteq C_G(K)$,则对任意 $h\in H,k\in K$ 有 $hk=kh$,于是
		\[
		[h,k]=h^{-1}k^{-1}hk = h^{-1}h k^{-1}k = e.
		\]
		因此所有生成元 $[h,k]$ 都是单位元,故 $[H,K]=\{e\}$。  
		从而
		\[
		[H,K]=\{e\} \Longleftrightarrow H\subseteq C_G(K).
		\]
		
		由对称性(交换 $H$ 与 $K$ 的角色,同样的论证)可得
		\[
		[H,K]=\{e\} \Longleftrightarrow K\subseteq C_G(H).
		\]
		于是 (1) 得证。
		
		\medskip
		
		\textbf{(2)} 先证 $[H,K]\subseteq K \Rightarrow H\subseteq N_G(K)$。
		
		假设 $[H,K]\subseteq K$。取任意 $h\in H$,要证 $h\in N_G(K)$,即 $hKh^{-1}=K$。只需证 $hKh^{-1}\subseteq K$,因为 $hKh^{-1}$ 与 $K$ 同阶(共轭不改变子群阶)。
		
		对任意 $k\in K$,考虑换位子
		\[
		[h,k]=h^{-1}k^{-1}hk\in [H,K]\subseteq K.
		\]
		于是
		\[
		h^{-1}k^{-1}hk\in K
		\iff k^{-1}hk\in K.
		\]
		由于 $K$ 为子群,$k^{-1}\in K$,故记
		\[
		k^h := hkh^{-1}.
		\]
		上式等价于 $k^{-1}k^h\in K$。而 $k\in K$,故
		\[
		k^h = k(k^{-1}k^h)\in K.
		\]
		这说明 $hkh^{-1}=k^h\in K$。对一切 $k\in K$ 成立,即
		\[
		hKh^{-1}\subseteq K.
		\]
		从而 $h\in N_G(K)$。因为 $h\in H$ 任取,得 $H\subseteq N_G(K)$。
		
		再证 $H\subseteq N_G(K)\Rightarrow [H,K]\subseteq K$。
		
		若 $H\subseteq N_G(K)$,则对任意 $h\in H,k\in K$,有 $hkh^{-1}\in K$,记 $hkh^{-1}=k'\in K$。则
		\[
		[h,k]=h^{-1}k^{-1}hk = h^{-1}k^{-1}k'h = h^{-1}(k^{-1}k')h.
		\]
		但 $k^{-1}k'\in K$,又因 $h\in N_G(K)$,故
		\[
		h^{-1}(k^{-1}k')h\in K.
		\]
		于是每个换位子 $[h,k]$ 都落在 $K$ 中,从而 $[H,K]$ 由这些元素生成,也必有
		\[
		[H,K]\subseteq K.
		\]
		
		综上,(2) 得证。
		
		\medskip
		
		\textbf{(3)} 设 $H_1\le H,\ K_1\le K$。  
		$[H_1,K_1]$ 的生成元形如
		\[
		[h_1,k_1],\qquad h_1\in H_1,\ k_1\in K_1.
		\]
		而 $H_1\le H,\ K_1\le K$,故每个这样的 $h_1,k_1$ 也满足 $h_1\in H,\ k_1\in K$,于是
		\[
		[h_1,k_1]\in [H,K]
		\]
		(因为 $[H,K]$ 是由所有 $h\in H,\ k\in K$ 的换位子生成的子群)。
		
		因此,$[H_1,K_1]$ 的所有生成元都属于 $[H,K]$,故
		\[
		[H_1,K_1]\subseteq [H,K].
		\]
		
		\medskip
		
		综上,(1)、(2)、(3) 均得证。
		\item \textbf{题目.}  
		证明:可解群 $G$ 有合成序列当且仅当 $G$ 是有限群。
		
		\textbf{证明.}
		
		\textbf{($\Rightarrow$)} 设可解群 $G$ 有一条合成序列
		\[
		1=G_0 \triangleleft G_1 \triangleleft \cdots \triangleleft G_n = G,
		\]
		其中每个因子群 $G_{i+1}/G_i$($0\le i\le n-1$)都是单群。
		
		因为 $G$ 是可解群,$G_n/G_{n-1},\dots,G_1/G_0$ 全部都是 $G$ 的商群,因而也都是可解群。  
		另一方面,合成序列的定义表明每个因子 $G_{i+1}/G_i$ 又是单群。  
		\[
		\Rightarrow\quad G_{i+1}/G_i \ \text{既简单又可解}.
		\]
		
		\textbf{引理.} 若群 $H$ 既简单又可解,则 $H$ 必为素数阶循环群。  
		
		\textbf{证明.} 因 $H$ 可解,存在其导群列
		\[
		H=H^{(0)}\triangleright H^{(1)}\triangleright\cdots\triangleright H^{(r)}=\{e\},
		\]
		且每个因子 $H^{(k)}/H^{(k+1)}$ 皆为阿贝尔群。若 $H$ 非阿贝尔,则导群列首项 $H^{(1)}=[H,H]$ 为 $H$ 的非平凡真正规子群,与 $H$ 简单矛盾;故 $H$ 必为阿贝尔群。又 $H$ 简单、阿贝尔,只能是素数阶循环群 $C_p$。证毕。
		
		由该引理知,对每个 $i$,
		\[
		G_{i+1}/G_i \cong C_{p_i}, \quad p_i\ \text{为素数}.
		\]
		因而
		\[
		|G| = |G_n| = \prod_{i=0}^{n-1} |G_{i+1}/G_i|
		= \prod_{i=0}^{n-1} p_i < \infty.
		\]
		故 $G$ 为有限群。
		
		\medskip
		
		\textbf{($\Leftarrow$)} 设 $G$ 为有限群且可解。  
		一般理论(Jordan--Hölder 定理的存在性部分)表明:\emph{任何有限群都有合成序列}。下面简要给出构造过程。
		
		对有限群 $G$,若 $G$ 非平凡,则存在一个极大正规子群 $N_1\triangleleft G$(利用有限性在所有非平凡正规子群中取极大元)。则 $G/N_1$ 没有非平凡正规子群,因而是单群。若 $N_1\ne 1$,对 $N_1$ 再以同样方法取极大正规子群 $N_2\triangleleft N_1$,使得 $N_1/N_2$ 为单群。有限性保证此过程在有限步后终止于 $N_k = 1$。于是得到
		\[
		1 = N_k \triangleleft N_{k-1} \triangleleft \cdots \triangleleft N_1 \triangleleft G,
		\]
		其中各因子 $N_{i-1}/N_i$ 均为单群,这便是一条合成序列。
		
		因此,有限可解群 $G$ 必存在合成序列。
		
		\medskip
		
		综上,可解群 $G$ 有合成序列当且仅当 $G$ 是有限群。证毕。
		
		\item 
		\textbf{题目.}  
		(1) 设 $H,K$ 都是群 $G$ 的可解正规子群,试证 $HK$ 也是 $G$ 的可解正规子群;  
		
		(2) 设 $R$ 是群 $G$ 的极大可解正规子群,$H$ 是 $G$ 的任一可解正规子群,证明  
		$H\subseteq R$ 且 $G/R$ 无非平凡的可解正规子群。
		
		\textbf{证明.}
		
		(1)\;先证 $HK\trianglelefteq G$。  
		对任意 $g\in G$,因为 $H,K\trianglelefteq G$,有
		\[
		gHg^{-1}=H,\qquad gKg^{-1}=K,
		\]
		故
		\[
		gHKg^{-1}=(gHg^{-1})(gKg^{-1})=HK,
		\]
		所以 $HK\trianglelefteq G$。
		
		再证 $HK$ 可解。记导出列 $H^{(0)}=H$, $H^{(i+1)}=[H^{(i)},H^{(i)}]$,  
		$K^{(0)}=K$, $K^{(i+1)}=[K^{(i)},K^{(i)}]$。由于 $H,K$ 可解,  
		存在整数 $m,n$ 使
		\[
		H^{(m)}=\{e\},\qquad K^{(n)}=\{e\}.
		\]
		容易验证
		\[
		(HK)'=[HK,HK]\subseteq H'K',
		\]
		于是归纳可得
		\[
		(HK)^{(i)}\subseteq H^{(i)}K^{(i)}\qquad (i\ge 0).
		\]
		取 $r=\max\{m,n\}$,则
		\[
		(HK)^{(r)}\subseteq H^{(r)}K^{(r)}=\{e\}.
		\]
		故 $(HK)^{(r)}=\{e\}$,$HK$ 可解。综上,$HK$ 是 $G$ 的可解正规子群。
		
		\medskip
		
		(2)\;由题意,$R$ 为 $G$ 的极大可解正规子群。  
		
		\textbf{(a) 证明 $H\subseteq R$。}  
		设 $H$ 是 $G$ 的任一可解正规子群。由(1)知 $HR$ 也是 $G$ 的可解正规子群,且显然 $R\subseteq HR$。  
		由 $R$ 的极大性,只能有 $HR=R$,从而 $H\subseteq R$。
		
		\medskip
		
		\textbf{(b) 证明 $G/R$ 无非平凡可解正规子群。}  
		设 $N/R$ 是 $G/R$ 的一个可解正规子群(其中 $N$ 为 $G$ 的子群,$R\le N\le G$)。  
		则 $N/R$ 可解,故存在有限列
		\[
		R=N_0/R\triangleleft N_1/R\triangleleft\cdots\triangleleft N_s/R=N/R,
		\]
		使得各因子 $(N_{i+1}/R)/(N_i/R)\cong N_{i+1}/N_i$ 为阿贝尔群。  
		取 $N_i$ 的原像,可得 $N$ 有一有限正规列
		\[
		\{e\}\triangleleft N_1\triangleleft\cdots\triangleleft N_s=N,
		\]
		且各因子 $N_{i+1}/N_i$ 皆阿贝尔,因此 $N$ 可解。  
		又因 $N/R$ 在 $G/R$ 中正规,可知 $N\trianglelefteq G$,于是 $N$ 是 $G$ 的可解正规子群。  
		由 $R$ 的极大性,必有 $N\subseteq R$,从而 $N/R=\{R\}$ 为平凡子群。
		
		因此,$G/R$ 没有非平凡可解正规子群。
		
		\medskip
		
		综上,(2) 得证。$\qed$
		\item 
		\textbf{题目.}
		设 $H$ 是群 $G$ 的指数有限的子群。证明:由左乘作用
		\[
		G \longrightarrow S_{G/H}
		\]
		诱导的同态的核,等于 $H$ 的所有共轭子群之交,即
		\[
		\ker\bigl(G\to S_{G/H}\bigr)=\bigcap_{g\in G} gHg^{-1}.
		\]
		
		\textbf{证明.}
		记
		\[
		\varphi:G \longrightarrow S_{G/H},\qquad
		\varphi(g)(xH)=gxH
		\]
		为 $G$ 对左陪集空间 $G/H$ 的左乘作用所给出的同态,
		并设
		\[
		K:=\ker\varphi=\{g\in G\mid \varphi(g)(xH)=xH,\ \forall xH\in G/H\}.
		\]
		
		\medskip
		
		\noindent\emph{(1) 证 $K\subseteq \displaystyle\bigcap_{g\in G} gHg^{-1}$.}
		
		取 $k\in K$。对任意 $g\in G$,有
		\[
		\varphi(k)(gH)=gH,
		\]
		即
		\[
		kgH=gH.
		\]
		于是存在某个 $h\in H$ 使得 $kg=gh$,从而
		\[
		g^{-1}kg = h\in H.
		\]
		由于 $g\in G$ 任意,可得
		\[
		k\in gHg^{-1}\quad\forall g\in G,
		\]
		即
		\[
		k\in \bigcap_{g\in G} gHg^{-1}.
		\]
		故
		\[
		K\subseteq \bigcap_{g\in G} gHg^{-1}.
		\]
		
		\medskip
		
		\noindent\emph{(2) 证 $\displaystyle\bigcap_{g\in G} gHg^{-1}\subseteq K$.}
		
		反过来,取 $k\in\bigcap_{g\in G} gHg^{-1}$。则对任意 $g\in G$,有
		\[
		k\in gHg^{-1},
		\]
		即存在 $h\in H$ 使得 $k = ghg^{-1}$。于是
		\[
		kgH = ghg^{-1}gH = ghH = gH,
		\]
		从而
		\[
		\varphi(k)(gH) = gH,\quad \forall\, gH\in G/H.
		\]
		这说明 $\varphi(k)$ 是恒等置换,故 $k\in K$。
		
		因此
		\[
		\bigcap_{g\in G} gHg^{-1} \subseteq K.
		\]
		
		\medskip
		
		综上,
		\[
		\ker\bigl(G\to S_{G/H}\bigr)=K=\bigcap_{g\in G} gHg^{-1}.
		\]
		证毕。
		
		\item 
		\textbf{题目.} 设 $H$ 是有限群 $G$ 的真子群,试证
		\[
		G\neq \bigcup_{g\in G}gHg^{-1}.
		\]
		
		\textbf{证明.}
		设 $X=G/H$ 为 $G$ 的所有左陪集的集合,记 $n=|X|=[G:H]>1$。
		令 $G$ 作用在 $X$ 上,作用方式为左乘:
		\[
		g\cdot xH := gxH\qquad (g,x\in G).
		\]
		对 $g\in G$,记
		\[
		\mathrm{Fix}(g):=\{xH\in X\mid g\cdot xH=xH\}
		\]
		为 $g$ 在 $X$ 上的不动点集合,并记 $f(g)=|\mathrm{Fix}(g)|$。
		
		考虑集合
		\[
		S:=\{(g,xH)\in G\times X\mid g\cdot xH=xH\}.
		\]
		一方面,
		\[
		|S|=\sum_{g\in G} f(g),
		\]
		因为对每个 $g$,其对应的有序对个数正是 $f(g)$。
		
		另一方面,固定 $xH\in X$,满足 $g\cdot xH=xH$ 的 $g$ 恰为
		\[
		xHx^{-1},
		\]
		其元素个数为 $|H|$。而 $X$ 中共有 $n=[G:H]$ 个左陪集,所以
		\[
		|S|=n\cdot |H|=|G|.
		\]
		于是得到
		\[
		\sum_{g\in G} f(g) = |G|,
		\]
		从而
		\[
		\frac{1}{|G|}\sum_{g\in G} f(g) = 1.
		\]
		即在 $G$ 的所有元素中,\emph{平均每个元素在 $X$ 上有 $1$ 个不动点}。
		
		然而,单位元 $e$ 固定了 $X$ 中的所有 $n>1$ 个陪集,即 $f(e)=n>1$。  
		若对每个非单位元 $g\ne e$ 都有 $f(g)\ge 1$,则
		\[
		\sum_{g\in G} f(g) \ge f(e)+\sum_{g\ne e}1 = n + (|G|-1) > |G|,
		\]
		与 $\sum_{g\in G} f(g)=|G|$ 矛盾。  
		因此必存在某个 $g_0\in G$ 使得
		\[
		f(g_0)=0,
		\]
		即 $g_0$ 在 $X$ 上没有不动点。
		
		注意到:对任意 $g,x\in G$,
		\[
		g\cdot xH=xH
		\iff gxH=xH
		\iff x^{-1}gx\in H
		\iff g\in xHx^{-1}.
		\]
		所以 $g$ 在 $X$ 上有不动点当且仅当 $g\in xHx^{-1}$ 对某个 $x\in G$ 成立。
		
		因此,对于上面得到的 $g_0$,由于它在 $X$ 上没有不动点,必有
		\[
		g_0\notin xHx^{-1},\quad \forall\,x\in G,
		\]
		也就是说
		\[
		g_0\notin \bigcup_{g\in G} gHg^{-1}.
		\]
		
		于是
		\[
		G\neq \bigcup_{g\in G} gHg^{-1}.
		\]
		证毕。
			
		\item 9.\textbf{题目.}
		设 $a,b$ 是群 $G$ 的任意两个元。若 $a,b$ 与它们的换位子
		\[
		[a,b]=a^{-1}b^{-1}ab
		\]
		两两可交换,证明:对任意整数 $m,n$ 都有
		\[
		[a^{m},b^{n}] = [a,b]^{mn}.
		\]
		
		\textbf{证明.}
		记 $c:=[a,b]=a^{-1}b^{-1}ab$。由条件知 $a,b,c$ 互相可交换,特别地 $c$ 为中心元。
		
		首先证明:对任意整数 $m$,有
		\[
		[a^{m},b]=c^{m}.
		\tag{$\ast$}
		\]
		
		注意到
		\[
		c = a^{-1}b^{-1}ab
		\quad\Longrightarrow\quad
		b^{-1}ab = ac.
		\]
		对任意整数 $m$,由共轭是同态可得
		\[
		b^{-1}a^{m}b = (b^{-1}ab)^{m} = (ac)^{m} = a^{m}c^{m},
		\]
		其中最后一步用到了 $a$ 与 $c$ 交换。于是
		\[
		[a^{m},b]
		= a^{-m}b^{-1}a^{m}b
		= a^{-m}(a^{m}c^{m})
		= c^{m},
		\]
		即得 $(\ast)$。对 $m<0$ 的情形,上式同样成立,因为
		\[
		[a^{-m},b] = [a^{m},b]^{-1} = c^{-m}.
		\]
		
		类似地可得:对任意整数 $n$,
		\[
		[a,b^{n}] = c^{n}.
		\tag{$\ast\ast$}
		\]
		
		接下来计算 $[a^{m},b^{n}]$。利用换位子的基本恒等式
		\[
		[x,yz] = [x,z]\,[x,y]^{z},
		\]
		在 $x=a^{m}$、$y=b$、$z=b^{n-1}$ 的情形下,并注意到 $c$ 为中心元,与 $b$ 的各次幂可交换,可得
		\[
		[a^{m},b^{n}]
		= [a^{m},b^{n-1}b]
		= [a^{m},b]\,[a^{m},b^{n-1}]
		= c^{m}\,[a^{m},b^{n-1}].
		\]
		由此对 $n\ge 1$ 归纳可得
		\[
		[a^{m},b^{n}] = c^{mn}.
		\]
		对 $n\le 0$,同样利用
		\(
		[a^{m},b^{-n}] = [a^{m},b^{n}]^{-1}
		\)
		即可得到同样结论。
		
		综上,对任意整数 $m,n$ 均有
		\[
		[a^{m},b^{n}] = c^{mn} = [a,b]^{mn}.
		\]
		证毕。
		
		\item 10.\textbf{题目.} 设 $A$ 是群 $G$ 的循环的正规子群。试证:对任意 $a\in A,\ x\in G^{(1)}$,有
		\[
		ax = xa.
		\]
		
		\textbf{证明.}
		设 $A=\langle a_0\rangle$ 为循环群。由于 $A\triangleleft G$,$G$ 对 $A$ 的共轭作用
		\[
		G\times A\to A,\qquad (g,a)\mapsto gag^{-1}
		\]
		给出一个同态
		\[
		\varphi:G\longrightarrow \operatorname{Aut}(A),\qquad 
		\varphi(g)(a)=gag^{-1}.
		\]
		
		因为 $A$ 是有限或无限循环群,所以 $\operatorname{Aut}(A)$ 是交换群(例如 $A\cong \mathbb Z$ 时
		$\operatorname{Aut}(A)\cong\{\pm1\}$,有限循环时 $\operatorname{Aut}(A)\cong (\mathbb Z_n)^\times$ 皆为阿贝尔群)。
		于是 $\operatorname{Im}(\varphi)\le \operatorname{Aut}(A)$ 也是阿贝尔群。
		
		由同态基本定理可知
		\[
		G/\ker\varphi \cong \operatorname{Im}(\varphi)
		\]
		是阿贝尔群,因此
		\[
		G^{(1)}\subseteq \ker\varphi.
		\]
		换言之,对任意 $x\in G^{(1)}$ 与任意 $a\in A$,有
		\[
		\varphi(x)(a)=xax^{-1}=a.
		\]
		于是
		\[
		xax^{-1}=a \quad\Longleftrightarrow\quad ax=xa.
		\]
		
		特别地,对任意 $a\in A,\ x\in G^{(1)}$ 都有 $ax=xa$。证毕。
		
		
	\end{enumerate}
	% -------------------- 1.9 群在集合上的作用 --------------------
		\clearpage
	\section{群在集合上的作用}
	\subsection{例1.9.12的(1)}
			\begin{tcolorbox}[
				colback = white,          % 背景色
				colframe = myred,         % 边框颜色
				coltitle = white,         % 标题文字颜色
				title = {关于 $SO(2)$ 与正交变换的说明},
				fonttitle = \bfseries,    % 标题加粗
				%breakable,                % 允许跨页
				%enhanced,                 % 高级样式
				sharp corners,            % 方角
				boxrule = 0.8pt,          % 边框线条粗细
				left = 2mm, right = 2mm, top = 2mm, bottom = 2mm  % 内边距
				]
				\textbf{1. $SO(2)$ 的定义.}\quad\\
				$SO(2)$ 称为 \emph{二维特殊正交群},定义为
				\[
				SO(2)=\{\,A\in\mathbb{R}^{2\times2}\mid A^{\mathrm T}A=I,\ \det(A)=1\,\}.
				\]
				其中条件 $A^{\mathrm T}A=I$ 表示 $A$ 为正交矩阵(保持内积),
				$\det(A)=1$ 表示该变换保持方向。
				$SO(2)$ 的任意元素可写为
				\[
				A(\theta)=
				\begin{pmatrix}
					\cos\theta & -\sin\theta\\[2pt]
					\sin\theta & \cos\theta
				\end{pmatrix},
				\qquad \theta\in[0,2\pi),
				\]
				它表示平面上绕原点旋转角度 $\theta$ 的线性变换。\\
				\textbf{2. 第一类与第二类正交变换.}\quad
				所有满足 $A^{\mathrm T}A=I$ 的 $2\times2$ 实矩阵构成 \emph{正交群}
				\[
				O(2)=\{\,A\in\mathbb{R}^{2\times2}\mid A^{\mathrm T}A=I\,\}.
				\]
				其中:
				\begin{itemize}
					\item 若 $\det(A)=+1$,称为\textbf{第一类正交变换}(proper orthogonal transformation),
					它保持方向,对应\textbf{旋转},所有此类矩阵组成子群 $SO(2)\subset O(2)$;
					\item 若 $\det(A)=-1$,称为\textbf{第二类正交变换}(improper orthogonal transformation),
					它改变方向,对应\textbf{镜像反射}。
				\end{itemize}
				
				\medskip
				\textbf{3. 几何理解.}\quad
				\begin{itemize}
					\item 第一类:在二维平面上旋转任意角度而不翻转方向;
					\item 第二类:相当于对称轴反射,使坐标系方向反转。
				\end{itemize}
				因此,$O(2)$ 包含所有“保持长度”的变换,而 $SO(2)$ 仅包含其中“保持方向”的旋转。
			\end{tcolorbox}
			地球表面可以看作 $\mathbb{R}^3$ 中的单位球面
			\[
			S^2 = \{\, x \in \mathbb{R}^3 \mid \|x\| = 1 \,\}.
			\]
			地球绕南北极的自转可以看作群 $SO(2)$ 在 $S^2$ 上的作用(围绕 
			z-轴旋转,也就是地球的自转),代数上可以理解为矩阵$R_\theta$作用在$ (x, y, z)$上:
			\[
			R_\theta \cdot (x, y, z)^{\mathrm T}
			= 
			\begin{pmatrix}
				\cos\theta & -\sin\theta & 0\\
				\sin\theta & \cos\theta & 0\\
				0 & 0 & 1
			\end{pmatrix}
			\!\!
			\begin{pmatrix}
				x \\ y \\ z
			\end{pmatrix},
			\qquad \theta \in [0, 2\pi).
			\]
			
			由此可以得到以下几何事实:
			\begin{enumerate}[label=(\arabic*)]
				\item 	对任意固定点 $(x_0, y_0, z_0)\in S^2$,当我们让 
				 $\theta$从 0 转到$2\pi$,该点绕 z-轴旋转一圈。它的 $z_0$ 坐标不变,但 $(x, y)$ 坐标在半径 $\sqrt{1 - z_0^2}$ 的圆上运动。这正是一个\textbf{纬线}。
				 因此,轨道 = 纬线:
				 \[
				 \text{Orb}((x_0, y_0, z_0)) = \{\, R_\theta(x_0, y_0, z_0) \mid \theta \in [0, 2\pi) \,\}.
				 \]
				
				\item 对非南北极点(例如赤道上的点),旋转任何角度都会把它移到别的地方,
				只有旋转角度 $0$ 时才不动,所以:
				\[
				F_x = \{ e \}(e \in SO(2)).
				\]
				
				\item 对南北极点(例如北极 $(0,0,1)$),围绕 $z$ 轴旋转任意角度都不会改变位置,所以:
				\[
				G_x = SO(2).
				\]
			\end{enumerate}
			
			综上,$SO(2)$ 在球面 $S^2$ 上的作用满足:
			\[
			\begin{cases}
				\text{轨道:纬线;}\\[3pt]
				\text{非极点的迷向子群:}\{e\};\\[3pt]
				\text{南北极的迷向子群:}SO(2).
			\end{cases}
			\]
			这展示了群作用的几何直观意义——轨道描述点在群作用下的运动轨迹,
			迷向子群刻画保持该点不动的对称性。
			
			
\subsection{P53的第一段}
以下是原文:
\begin{mdframed}  % ===== 方框开始 =====
	进一步考虑映射 $\varphi_x$,我们有
	\[
	\varphi_x^{-1}(gx)=\{h\in G\mid hx=gx\}
	=\{h\in G\mid g^{-1}hx=x\}.
	\]
	因此 $\varphi_x^{-1}(gx)=gF_x$,即 $gx$ 的原像是 $F_x$ 的左陪集 $gF_x$。
	这样,我们得到一个双射
	\[
	\varphi:G/F_x\longrightarrow O_x,
	\]
	即这两个集合存在一一对应。两者上都有 $G$ 的作用,且对任何 $s,g\in G$ 有
	\[
	\varphi(s(gF_x))
	=\varphi((sg)F_x)
	=(sg)x
	=s(gx)
	=s\varphi(gF_x).
	\]
	于是有如下的交换图:
	\[
	\begin{tikzcd}
		G/F_x \arrow[r,"\varphi"] \arrow[d,"s"'] & O_x \arrow[d,"s"] \\
		G/F_x \arrow[r,"\varphi"'] & O_x
	\end{tikzcd}
	\]
	
	证明:$\varphi_x^{-1}(gx)=gF_x$;据此构造
	\[
	\varphi:G/F_x\longrightarrow O_x,\qquad \varphi(gF_x)=gx,
	\]
	并证明 $\varphi$ 是 $G$-集合同构(即良定、双射且与 $G$ 的作用相容)。
	给出相应交换图。
\end{mdframed}  % ===== 方框结束 =====
\textbf{注解.}


\textbf{(1) 计算:}
\[
\begin{aligned}
	\varphi_x^{-1}(gx)
	&=\{\,h\in G\mid \varphi_x(h)=gx\,\}
	=\{\,h\in G\mid hx=gx\,\}\\
	&=\{\,h\in G\mid g^{-1}hx=x\,\}
	=\{\,h\in G\mid g^{-1}h\in F_x\,\}\\
	&=\{\,gf\mid f\in F_x\,\}
	=gF_x.
\end{aligned}
\]

\textbf{(2) 诱导映射的定义与良定性:}
由上式可见,$G$ 中每个左陪集 $gF_x$ 的像都是同一个点 $gx\in O_x$,
因此定义
\[
\varphi:G/F_x\to O_x,\qquad \varphi(gF_x)=gx.
\]
若 $gF_x=g'F_x$,则存在 $f\in F_x$ 使得 $g'=gf$,于是
$g'x=gf\,x=gx$(因 $fx=x$),故 $\varphi(gF_x)=\varphi(g'F_x)$,从而良定。

\textbf{(3) 双射性:}
\emph{满射:} 任取 $y\in O_x$,有 $y=gx$,则 $y=\varphi(gF_x)$。
\emph{单射:} 若 $\varphi(g_1F_x)=\varphi(g_2F_x)$,则 $g_1x=g_2x$,从而
$g_2^{-1}g_1x=x$,即 $g_2^{-1}g_1\in F_x$,故 $g_1F_x=g_2F_x$。

\textbf{(4) 与 $G$ 作用的相容性:}
$G$ 在 $G/F_x$ 与 $O_x$ 上的作用分别定义为(见书例1.9.3和例1.9.4)
\[
s\cdot(gF_x):=(sg)F_x,\qquad s\cdot(gx):=(sg)x\qquad(s,g\in G).
\]
于是对任意 $s,g\in G$,
\[
\varphi\bigl(s\cdot(gF_x)\bigr)=\varphi\bigl((sg)F_x\bigr)=(sg)x
=s\cdot(gx)=s\cdot\varphi(gF_x).
\]
因此 $\varphi$ 为 $G$-同态,从而是 $G$-集合同构。

\textbf{(5) 交换图):}
\[
\begin{tikzcd}
	G/F_x \arrow[r,"\varphi"] \arrow[d,"s"'] & O_x \arrow[d,"s"] \\
	G/F_x \arrow[r,"\varphi"'] & O_x
\end{tikzcd}
\]
该图交换等价于上式
\(
\varphi\bigl(s\cdot(gF_x)\bigr)=s\cdot\varphi(gF_x)
\)
的成立。

\textbf{结论.}
对任意 $x\in X$,轨道 $O_x$ 与商集 $G/F_x$ 存在自然的 $G$-集合同构
\[
\varphi:G/F_x \xrightarrow{\;\cong\;} O_x,\qquad gF_x\mapsto gx.
\]
\subsection{只要 $G$在 $|X|>1$ 的集合上作用是可递的,就不可能是平凡作用,因此诱导同态 $G\to S_X $非平凡,$\ker \phi$不可能等于整个$ G$,也即是$\mathrm{Im} \phi \ne \{e\}$ 。
}
\textbf{命题.}\;
$\text{设群 }G\text{ 作用在集合 }X\text{ 上,且 }|X|\ge 2\text{。令该作用诱导同态}
\ \varphi:G\to S_X,\ 
\varphi(g)(x)=g\cdot x.
\text{若该作用可递(传递),证明 } \ker\varphi\ne G.$

\textbf{证明.}\;
由群作用诱导同态的定义,对任意 $g\in G$,$\varphi(g)$ 是 $X$ 上的一个置换,
并满足
\[
\varphi(g)(x)=g\cdot x\qquad(\forall x\in X).
\]

我们反证。假设
\[
\ker\varphi=G.
\]
则对任意 $g\in G$ 有 $g\in\ker\varphi$,从而
\[
\varphi(g)=\mathrm{id}_X.
\]
于是对任意 $x\in X$ 都有
\[
g\cdot x=\varphi(g)(x)=x\qquad(\forall g\in G,\ \forall x\in X).
\]
这说明该作用是平凡作用:每个点都被 $G$ 的所有元素固定。

因此对任意 $x\in X$,其轨道为
\[
G\cdot x=\{g\cdot x\mid g\in G\}=\{x\},
\]
即
\[
|G\cdot x|=1\qquad(\forall x\in X).
\]

另一方面,由于该作用可递(传递),对任意 $x\in X$ 有
\[
G\cdot x=X,
\]
从而
\[
|G\cdot x|=|X|\ge 2,
\]
这与上面推出的 $|G\cdot x|=1$ 矛盾。

故假设不成立,从而
\[
\ker\varphi\ne G.
\]
证完。
\hfill $\blacksquare$


			
			
			
	\clearpage		
	\subsection*{课后习题答案}
	\addcontentsline{toc}{subsection}{\textcolor{red}{课后习题答案}}
	\begin{enumerate}[label=\textcolor{blue}{\textbf{\large\arabic*.}}]
		\item 2.证明 $A_5$ 没有指数为 $2,3,4$ 的子群。
		
		\textbf{证明.}
		
		已知 $|A_5|=\dfrac{5!}{2}=60$。设 $H\le A_5$,其指数为 $n$,则
		\[
		|H|=\frac{|A_5|}{n}=\frac{60}{n}.
		\]
		分别讨论三种情况。
		
		\begin{itemize}
			\item[(1)] 若 $n=2$,则 $|H|=30$。  
			指数为 2 的子群必为正规子群(见1.3.1),而 $A_5$ 是单群,没有非平凡正规子群,因此不存在指数为 2 的子群。
			
			\item[(2)] 
			\textbf{(1) 指数与阶的关系.}
			已知 $|A_5|=60$。若 $[A_5:H]=3$,则由拉格朗日定理
			\[
			|H|=\frac{|A_5|}{[A_5:H]}=\frac{60}{3}=20.
			\]
			
			\textbf{(2) 陪集作用与置换表示.}
			令 $A_5$ 以左乘作用到左陪集空间 $A_5/H$ 上:
			\[
			x\cdot(gH)=(xg)H,\qquad x,g\in A_5.
			\]
			该作用良定义并给出一个同态
			\[
			\varphi: A_5 \longrightarrow S_{A_5/H}\cong S_3,\qquad 
			\varphi(x)(gH)=xgH.
			\]
			理由:对每个 $x\in A_5$,映射 $gH\mapsto xgH$ 是 $A_5/H$ 上的一个置换;映射
			$x\mapsto\varphi(x)$ 保持乘法($\varphi(xy)=\varphi(x)\varphi(y)$),故为群同态。
			
			\textbf{(3) 核的刻画(核等于核化).}
			同态 $\varphi$ 的核为
			\[
			\ker\varphi=\{x\in A_5\mid \varphi(x)=\mathrm{id}_{A_5/H}\}.
			\]
			按定义 $\varphi(x)=\mathrm{id}$ 当且仅当对一切 $g\in A_5$ 有
			\[
			x\cdot(gH)=gH
			\iff xgH=gH
			\iff g^{-1}xg\in H.
			\]
			因此
			\[
			\ker\varphi=\{\,x\in A_5\mid g^{-1}xg\in H,\ \forall g\in A_5\,\}
			=\bigcap_{g\in A_5} gHg^{-1},
			\]
			即 $H$ 的核化(core)。
			
			\textbf{(4) $\ker\varphi$ 是 $A_5$ 的正规子群且 $A_5$ 为单群.}
			任何群同态的核都是正规子群;而 $A_5$ 是单群,因而 $A_5$ 的正规子群只有
			$\{e\}$ 与 $A_5$ 本身。于是
			\[
			\ker\varphi\in\bigl\{\{e\},\,A_5\bigr\}.
			\]
			
			\textbf{(5) 分析两种可能.}
			
			\quad\textbf{(5a) 若 $\ker\varphi=\{e\}$:}
			则 $\varphi$ 为单射,同构到它的像:
			\[
			A_5\cong \operatorname{Im}\varphi \subseteq S_3.
			\]
			但 $|A_5|=60$ 而 $|S_3|=6$,不存在将 60 阶群单射进 6 阶群的可能,矛盾。
			
			\quad\textbf{(5b) 若 $\ker\varphi=A_5$:}
			则 $\varphi$ 为平凡同态(像为平凡群 $\{\mathrm{id}\}$),
			即对一切 $x\in A_5$ 与 $gH\in A_5/H$ 有
			\[
			\varphi(x)(gH)=gH,
			\]
			作用在 $A_5/H$ 上恒等。可是标准的陪集作用是可递的:
			对任意 $g_1H,g_2H$,取 $x=g_2g_1^{-1}$ 即有
			\[
			x\cdot(g_1H)=g_2H.
			\]
			平凡作用显然不传递,与陪集作用的传递性矛盾。
			
			\textbf{(6) 结论.}
			两种可能均导致矛盾,故假设 $[A_5:H]=3$ 不成立。
			因此 $A_5$ 不存在指数为 $3$ 的子群。
			
			\item[(3)] 若 $n=4$,则 $|H|=15$。  
			类似地,$A_5$ 在陪集空间 $A_5/H$ 上的作用给出同态
			\[
			\varphi: A_5 \longrightarrow S_4.
			\]
			其像阶数整除 $|S_4|=24$,而 $|A_5|=60$,所以 $\ker\varphi$ 的阶为 $\dfrac{60}{|\operatorname{Im}\varphi|}$。  
			若同态非平凡,则核为 $A_5$ 的非平凡正规子群,矛盾;  
			若同态平凡,则作用不可能是传递的。  
			故此情形亦不可能。
			
		\end{itemize}
		
		综上,$A_5$ 不存在指数为 $2,3,4$ 的子群。
		
		
		\item 9.
		\textbf{题目.}\;
		设 $G$ 是单群,$H<G$ 满足 $[G:H]\le 4$,证明:$|G|\le 3$。
		
		\textbf{证明.}\;
		记 $m=[G:H]$。由于 $H<G$,故 $m\ge 2$,且由题设 $m\le 4$。
		
		令 $G$ 作用在左陪集集合 $G/H$ 上:
		\[
		g\cdot (xH)=(gx)H\qquad (g,x\in G).
		\]
		该作用给出群同态(置换表示)
		\[
		f:G\longrightarrow S_{G/H}\cong S_m.
		\]
		其核为
		\[
		\mathrm{ker}(f)=\{g\in G:\ g\cdot(xH)=xH\ \forall\,x\in G\}
		=\bigcap_{x\in G}xHx^{-1},
		\]
		即 $H$ 的核(core),从而 $\mathrm{ker}(f)\lhd G$。
		
		因为 $G$ 是单群,故 $\mathrm{ker}(f)=\{e\}$ 或 $\mathrm{ker}(f)=G$。
		但 $H$ 为真子群,作用不可能处处平凡:若 $\mathrm{ker}(f)=G$,则 $f$ 为平凡同态,
		从而对一切 $g\in G$ 有 $gH=H$,即 $g\in H$,推出 $G=H$,矛盾。
		因此必有
		\[
		\mathrm{ker}(f)=\{e\},
		\]
		于是 $f$ 为单同态,故
		\[
		G\cong \mathrm{Im}(f)\le S_m.
		\]
		特别地,
		\[
		|G|=|\mathrm{Im}(f)|\ \text{整除}\ |S_m|.
		\]
		
		下面分 $m$ 的可能值讨论。
		
		\textbf{(i) $m=2$.}\;
		此时 $[G:H]=2$,由“指数为 $2$ 的子群必为正规子群”知 $H\lhd G$。
		由于 $G$ 单,且 $H$ 为真子群,只能 $H=\{e\}$,从而
		\[
		|G|=[G:H]=2\le 3.
		\]
		
		\textbf{(ii) $m=3$.}\;
		此时 $G\le S_3$,故 $|G|\mid |S_3|=6$,即 $|G|\in\{1,2,3,6\}$。
		又 $G$ 为单群且非平凡(因为 $m\ge 2$),所以 $|G|\ne 1$。
		若 $|G|=6$,则 $G\cong S_3$,但 $S_3$ 有正规子群 $A_3$,不为单群,矛盾。
		因此
		\[
		|G|\in\{2,3\}\ \Rightarrow\ |G|\le 3.
		\]
		
		\textbf{(iii) $m=4$.}\;
		此时 $G\le S_4$,故 $|G|\mid |S_4|=24$,即
		\[
		|G|\in\{1,2,3,4,6,8,12,24\}.
		\]
		同理 $|G|\ne 1$。我们说明除 $2,3$ 外其余都不可能对应单群:
		
		\begin{itemize}
			\item $|G|=4$:任一 $4$ 阶群皆阿贝尔(同构于 $C_4$ 或 $C_2\times C_2$),必有非平凡真子群且正规,故不单。
			\item $|G|=6$:群同构于 $C_6$ 或 $S_3$,两者皆非单($C_6$ 阿贝尔;$S_3$ 有 $A_3\lhd S_3$)。
			\item $|G|=8$:$2$-群中心 $Z(G)$ 非平凡,故有非平凡正规子群,不单。
			\item $|G|=12$:设 Sylow-$3$ 子群个数为 $n_3$,则 $n_3\equiv 1\ (\mathrm{mod}\ 3)$ 且 $n_3\mid 4$,
			故 $n_3=1$ 或 $4$。若 $n_3=1$,则 Sylow-$3$ 子群正规,$G$ 不单;
			若 $n_3=4$,则共有 $4$ 个 $3$ 阶子群,它们除单位元外两两无交,
			故 $G$ 中阶为 $3$ 的非单位元共有 $4\cdot 2=8$ 个,加上单位元共 $9$ 个元素,
			剩余 $12-9=3$ 个非单位元与单位元一起恰好组成一个 $4$ 阶子群,
			因此 Sylow-$2$ 子群唯一从而正规,$G$ 不单。
			\item $|G|=24$:此时 $G\cong S_4$,但 $A_4\lhd S_4$,故 $S_4$ 不单。
		\end{itemize}
		
		于是 $m=4$ 时只能
		\[
		|G|\in\{2,3\}\ \Rightarrow\ |G|\le 3.
		\]
		
		综上,若 $G$ 为单群且存在真子群 $H<G$ 满足 $[G:H]\le 4$,则必有 $|G|\le 3$。
		\qed

		\item 12.

		\textbf{题目.} 设 $H<G$,试证:$H$ 的共轭子群的个数为 $[G:N_G(H)]$。
	
		\textbf{证明.}\;
		令
		\[
		X=\{K\mid K<G\}
		\]
		为 $G$ 的所有子群所成的集合。定义 $G$ 在 $X$ 上的作用为共轭作用:
		\[
		G\times X\to X,\qquad (g,K)\mapsto gKg^{-1}.
		\]
		这是一个群作用,因为对任意 $g_1,g_2\in G$ 与 $K\in X$,
		\[
		eKe^{-1}=K,\qquad (g_1g_2)K(g_1g_2)^{-1}=g_1(g_2Kg_2^{-1})g_1^{-1}.
		\]
		
		考虑元素 $H\in X$ 在该作用下的轨道
		\[
		\mathcal O_H=\{gHg^{-1}\mid g\in G\}.
		\]
		显然,$\mathcal O_H$ 恰好是所有与 $H$ 共轭的子群的集合,因此
		\[
		\#\{\text{与 }H\text{ 共轭的子群}\}=|\mathcal O_H|.
		\]
		
		接下来利用轨道-稳定子定理计算 $|\mathcal O_H|$。
		先求稳定子(不动子群):
		\[
		\mathrm{Stab}_G(H)=\{g\in G\mid gHg^{-1}=H\}.
		\]
		而
		\[
		\{g\in G\mid gHg^{-1}=H\}=N_G(H),
		\]
		正是 $H$ 在 $G$ 中的正规化子(normalizer)。
		因此
		\[
		\mathrm{Stab}_G(H)=N_G(H).
		\]
		
		由轨道-稳定子定理,
		\[
		|\mathcal O_H|=[G:\mathrm{Stab}_G(H)]=[G:N_G(H)].
		\]
		从而与 $H$ 共轭的子群的个数为 $[G:N_G(H)]$.\qed
		\item 13.(不写)
		\textbf{题目.}\;
		\text{设 }H\text{ 是有限群 }G\text{ 的真子群,证明:}
		\quad
		\[
		G\neq \bigcup_{g\in G} gHg^{-1}.
		\]
		
		\[
		\textbf{答案:}
		\]
		
		\textbf{证明.}\;
		设
		\[
		\sigma=\bigcup_{g\in G} gHg^{-1}
		\]
		为 $H$ 的所有共轭子群的并。我们证明 $\sigma\neq G$。
		
		\textbf{第一步:共轭子群的个数.}\;
		由题 12(共轭子群数目公式)知,与 $H$ 共轭的子群的个数为
		\[
		[G:N_G(H)],
		\]
		其中
		\[
		N_G(H)=\{x\in G\mid xHx^{-1}=H\}
		\]
		为 $H$ 的正规化子。
		
		\textbf{第二步:估计 }$|\sigma|\textbf{ 的上界.}\;$
		对任意 $g\in G$,共轭映射 $h\mapsto ghg^{-1}$ 是从 $H$ 到 $gHg^{-1}$ 的双射,
		故
		\[
		|gHg^{-1}|=|H|.
		\]
		并且任意共轭子群都包含单位元 $e$。
		
		设 $H_1,H_2,\dots,H_r$ 为 $H$ 的所有不同共轭子群,则
		\[
		r=[G:N_G(H)],\qquad \sigma=\bigcup_{i=1}^r H_i.
		\]
		由于每个 $H_i$ 至多提供 $|H|-1$ 个非单位元,而单位元只计一次,故有估计
		\[
		|\sigma|
		\le 1+r(|H|-1)
		=1+\bigl(|H|-1\bigr)[G:N_G(H)].
		\]
		将 $[G:N_G(H)]=\dfrac{|G|}{|N_G(H)|}$ 代入,得
		\[
		|\sigma|
		\le 1+\bigl(|H|-1\bigr)\frac{|G|}{|N_G(H)|}
		=|G|\left(\frac{|H|-1}{|N_G(H)|}\right)+1.
		\]
		
		\textbf{第三步:推出 }$|\sigma|<|G|\textbf{(关键:不需要 }H<N_G(H)\textbf{).}\;$
		若 $N_G(H)=G$,则 $H\lhd G$,从而对任意 $g\in G$ 有 $gHg^{-1}=H$,于是
		\[
		\sigma=\bigcup_{g\in G} gHg^{-1}=H\neq G,
		\]
		结论成立。
		
		以下设 $N_G(H)\neq G$。则 $N_G(H)<G$,故
		\[
		[G:N_G(H)]\ge 2.
		\]
		又由于 $H\le N_G(H)$,从而 $|H|\le |N_G(H)|$,于是
		\[
		\frac{|H|-1}{|N_G(H)|}\le \frac{|N_G(H)|-1}{|N_G(H)|}=1-\frac{1}{|N_G(H)|}.
		\]
		代回第二步的估计得
		\[
		|\sigma|
		\le |G|\left(1-\frac{1}{|N_G(H)|}\right)+1
		=|G|-\frac{|G|}{|N_G(H)|}+1
		=|G|-[G:N_G(H)]+1.
		\]
		由于 $[G:N_G(H)]\ge 2$,故
		\[
		|\sigma|\le |G|-2+1=|G|-1<|G|.
		\]
		因此 $|\sigma|<|G|$,从而 $\sigma\neq G$。
		
		综上,
		\[
		G\neq \bigcup_{g\in G} gHg^{-1}.
		\] \qed
		
		\item 14.
		\textbf{题目.}\;
		$\text{设 }H\le G\text{ 且 }[G:H]<\infty.\text{ 证明:由 }G\text{ 在 }G/H\text{ 上的左平移作用诱导的同态 }
		f:G\to S_{G/H}\text{ 的核等于 }H\text{ 的所有共轭子群之交:}$
		\[
		\ker f=\bigcap_{a\in G} aHa^{-1}.
		\]
		
		\textbf{证明.}\;
		令 $G$ 作用在左陪集集合 $G/H$ 上,作用为左平移:
		\[
		g\cdot (aH)=(ga)H\qquad (g,a\in G).
		\]
		该作用诱导一个群同态
		\[
		f:G\longrightarrow S_{G/H},
		\]
		其中 $f(g)$ 是置换 $G/H\to G/H$,定义为 $f(g)(aH)=(ga)H$。
		
		\textbf{第一步:写出核的刻画.}\;
		由定义,
		\[
		\ker f=\{g\in G\mid f(g)\text{ 为恒等置换}\}
		=\{g\in G\mid (ga)H=aH,\ \forall\,a\in G\}.
		\]
		而对固定的 $a\in G$,
		\[
		(ga)H=aH
		\iff a^{-1}(ga)\in H
		\iff a^{-1}ga\in H
		\iff g\in aHa^{-1}.
		\]
		因此
		\[
		g\in\ker f
		\iff g\in aHa^{-1}\ (\forall\,a\in G)
		\iff g\in \bigcap_{a\in G} aHa^{-1}.
		\]
		从而得到包含关系
		\[
		\ker f\subseteq \bigcap_{a\in G} aHa^{-1}.
		\]
		
		\textbf{第二步:反向包含.}\;
		反之,若 $g\in \bigcap_{a\in G} aHa^{-1}$,则对任意 $a\in G$ 有
		\[
		g\in aHa^{-1}\iff a^{-1}ga\in H.
		\]
		于是
		\[
		(ga)H=aH\qquad (\forall\,a\in G),
		\]
		这说明 $f(g)$ 固定每一个陪集 $aH$,即 $f(g)$ 为恒等置换,所以 $g\in\ker f$。
		因此
		\[
		\bigcap_{a\in G} aHa^{-1}\subseteq \ker f.
		\]
		
		\textbf{结论.}\;
		综上两边互相包含,故
		\[
		\ker f=\bigcap_{a\in G} aHa^{-1}.
		\]
		这正是 $H$ 的所有共轭子群之交(亦称 $H$ 在 $G$ 中的核/共轭核)。
	\qed
		
		
	
		
	\end{enumerate}






























	
	% -------------------- 1.10 Sylow 定理 --------------------
		\clearpage
	\section{Sylow 定理}
	\subsection{Caychy定理的推论:若群 $G$ 的阶为素数 $p$,则 $G$ 是循环群,且由任一非单位元生成。}
	\textbf{推论.}(由 Cauchy 定理推出)
	
	若群 $G$ 的阶为素数 $p$,则 $G$ 是循环群,且由任一非单位元生成。
	
	\textbf{证明.}
	由 Cauchy 定理,存在元素 $g \in G$ 使得 $|g| = p$。
	
	由于 $|G| = p$,任意子群的阶都整除 $p$(由Lagrange定理),因此 $G$ 只可能有两个子群:$\{e\}$ 与 $G$ 本身。
	
	又因为 $\langle g \rangle$ 是由 $g$ 生成的阶为 $p$ 的子群,
	故 $\langle g \rangle = G$。
	
	因此 $G$ 是循环群,且由任意非单位元生成。
	\[
	\boxed{G = \langle g \rangle.}
	\]
		\subsection{ $P$ 是一个 $p$-群,则对任意非单位元 $x\in P$,其阶为$|x| = p^k$}
	\textbf{命题.}
	设 $P$ 是一个 $p$-群,即 $|P| = p^n$。则对任意非单位元 $x\in P$,其阶必为
	\[
	|x| = p^k,\qquad k\ge 1.
	\]
	也就是说,$p$-群中的所有非单位元都是 $p$-幂阶元素。
	
	\textbf{证明.}
	由拉格朗日定理可知,任意元素的阶必须整除群的阶,即
	\[
	|x|\mid |P| = p^n.
	\]
	因此 $|x|$ 的所有可能值只能是
	\[
	1,\ p,\ p^2,\ \dots,\ p^n.
	\]
	若 $x\neq e$,则阶不可能为 $1$,故必存在某个 $k\ge 1$ 使得
	\[
	|x| = p^k.
	\]
	因此 $p$-群中所有非单位元均为 $p$-幂阶元素。
	
	
	\subsection{思考题1.10.7}
	试求 $S_4$ 的 Sylow 2-子群。
	
	\textbf{解.}
	
	首先,计算群的阶:
	\[
	|S_4| = 4! = 24 = 2^3 \cdot 3.
	\]
	因此,Sylow 2-子群的阶应为 $2^3 = 8$。
	\textbf{(1) 构造一个阶为 8 的子群.}
	
	考虑 $S_4$ 中由以下两个置换生成的子群:
	\[
	H = \langle (1\,2\,3\,4),\ (1\,3) \rangle.
	\]
	
	计算:
	
	- $(1\,2\,3\,4)$ 的阶为 4;
	- $(1\,3)$ 的阶为 2;
	- $(1\,3)(1\,2\,3\,4)(1\,3) = (1\,4\,3\,2) = (1\,2\,3\,4)^{-1}$。
	
	由此得:
	\[
	H = \{ e,\ (1\,2\,3\,4),\ (1\,3)(2\,4),\ (1\,4\,3\,2),\ (1\,3),\ (2\,4),\ (1\,2)(3\,4),\ (1\,4)(2\,3)\}.
	\]
	可以验证 $|H|=8$,且它同构于二面体群:
	\[
	H \cong D_8.
	\]
	因此,$H$ 是 $S_4$ 的一个 Sylow 2-子群。
	
	---
	
	\textbf{(2) 确定 Sylow 2-子群的个数.}
	
	设 $n_2$ 表示 Sylow 2-子群的个数。由 Sylow 定理:
	\[
	n_2 \mid 3,\quad n_2 \equiv 1 \pmod 2.
	\]
	所以 $n_2=1$ 或 $3$。
	
	若 $n_2=1$,则 Sylow 2-子群正规;但 $S_4$ 不是 2-群,也不存在这样的正规 8 阶子群。
	因此 $n_2\ne1$,从而 $n_2=3$。
	
	---
	
	\textbf{(3) 结论.}
	
	$S_4$ 的 Sylow 2-子群共有 $3$ 个,每个都同构于二面体群 $D_8$,例如:
	\[
	\langle (1\,2\,3\,4),(1\,3)\rangle,\quad
	\langle (1\,2\,3\,4),(2\,4)\rangle,\quad
	\langle (1\,2\,3\,4),(1\,4)\rangle.
	\]
	
	\[
	\boxed{\text{因此 }S_4\text{ 的 Sylow 2-子群共有 3 个,每个都同构于 }D_8.}
	\]
	\subsection{定理1.10.5定理的详细证明}
	\textbf{定理(Sylow 第一定理,存在性与逐级提升).}
	设有限群 $G$ 的阶为
	\[
	|G| = p^\ell m,
	\]
	其中 $p$ 为素数,$\ell \ge 1$,且 $\gcd(p,m)=1$。
	则:
	
	(1) $G$ 中存在阶为 $p^\ell$ 的子群(即 Sylow $p$-子群);
	
	(2) 更强地,对任意 $1 \le k \le \ell$,$G$ 中存在阶为 $p^k$ 的子群。
	
	\textbf{证明.}
	
	\textbf{第一步:用 Cauchy 定理得到一个 $p$ 阶子群。}
	
	由于 $p \mid |G|$,由 Cauchy 定理可知,
	$G$ 中存在元素 $x$,其阶为 $p$。
	令
	\[
	H_1 = \langle x \rangle,
	\]
	则 $|H_1| = p$,即 $G$ 中存在阶为 $p$ 的子群。
	
	下面的目标是:在 $G$ 中已知存在某个阶为 $p^n$ 的子群 $H$($1 \le n < \ell$)时,
	构造出一个阶为 $p^{n+1}$ 的子群,从而逐步“升阶”到 $p^\ell$。
	
	\textbf{第二步:设 $H$ 为 $G$ 中一个阶为 $p^n$ 的子群($1 \le n < \ell$)。}
	
	固定这样一个 $H$,$|H| = p^n$。
	定义集合
	\[
	X = G/H := \{ gH \mid g \in G \},
	\]
	即 $G$ 关于 $H$ 的所有左陪集的集合。
	
	在 $X$ 上定义 $H$ 的左作用:
	\[
	\forall h \in H,\ \forall gH \in X,\quad
	h \cdot (gH) := hgH.
	\]
	这是良定义的群作用(可检验恒等元作用为恒等、乘法相容)。
	
	记此作用下的\emph{不动点集}为
	\[
	X_0 = \{ gH \in X \mid h \cdot (gH) = gH\ \text{对一切 } h \in H \}.
	\]
	
	下面详细证明:
	\[
	X_0 = N_G(H)/H.
	\]
	
	\textbf{第三步:刻画不动点条件并推出正规化子。}
	
	取任意 $gH \in X$。
	
	\underline{(\(\Rightarrow\))} 若 $gH \in X_0$,则对任意 $h \in H$,
	\[
	h \cdot (gH) = gH.
	\]
	按作用定义,
	\[
	h \cdot (gH) = hgH,
	\]
	所以不动点条件等价于
	\[
	hgH = gH,\quad \forall h \in H.
	\]
	
	对固定的 $h \in H$,由 $hgH = gH$,在左边同乘 $g^{-1}$,得
	\[
	g^{-1}hgH = H.
	\]
	这等价于
	\[
	g^{-1}hg \in H.
	\]
	因此
	\[
	\forall h \in H,\quad g^{-1}hg \in H.
	\]
	这说明
	\[
	g^{-1}Hg \subseteq H.
	\]
	
	现在利用\textbf{共轭自同构保持阶数}这一事实:
	共轭映射
	\[
	\mathrm{ad}_{g^{-1}}: G \to G,\quad x \mapsto g^{-1}xg
	\]
	是群自同构,故限制到 $H$ 上是双射,从而
	\[
	|g^{-1}Hg| = |H| = p^n.
	\]
	结合 $g^{-1}Hg \subseteq H$,在有限情形下必有
	\[
	g^{-1}Hg = H.
	\]
	(理由:有限集若有子集与其势相同,只能相等。)
	
	于是
	\[
	g^{-1}Hg = H \iff gHg^{-1} = H,
	\]
	这正是正规化子
	\[
	N_G(H) = \{ x \in G \mid xHx^{-1} = H \}
	\]
	的定义,所以
	\[
	g \in N_G(H).
	\]
	因此
	\[
	gH \in X_0 \ \Longrightarrow\ g \in N_G(H),
	\]
	从而
	\[
	gH \in X_0 \ \Longrightarrow\ gH \in N_G(H)/H.
	\]
	即
	\[
	X_0 \subseteq N_G(H)/H.
	\]
	
	\underline{(\(\Leftarrow\))} 反过来,若 $g \in N_G(H)$,则 $gHg^{-1} = H$。
	对任意 $h \in H$,
	\[
	hgH = gH
	\]
	等价于
	\[
	g^{-1}hg \in H,
	\]
	而这由 $gHg^{-1} = H$ 自动成立。
	所以每个 $g \in N_G(H)$ 满足 $gH \in X_0$,即
	\[
	N_G(H)/H \subseteq X_0.
	\]
	
	综上两向包含:
	\[
	X_0 = N_G(H)/H.
	\]
	
	\textbf{第四步:用 $p$-群作用的不动点公式得到 $p \mid |N_G(H)/H|$.}
	
	首先计算 $|X|$:
	\[
	|X| = |G/H| = \frac{|G|}{|H|}
	= \frac{p^\ell m}{p^n}
	= p^{\ell-n} m.
	\]
	因为 $n < \ell$,所以 $p^{\ell-n} \ge p$,从而
	\[
	p \mid |X|.
	\]
	
	这里用到的命题(「命题 1.10.3」)是:
	
	\emph{若 $P$ 为 $p$-群作用在有限集 $Y$ 上,则}
	\[
	|Y| \equiv |Y^P| \pmod p,
	\]
	其中 $Y^P$ 为不动点集合。
	
	在本题中,$H$ 是 $p^n$ 阶 $p$-群,作用在 $X$ 上,故有
	\[
	|X| \equiv |X_0| \pmod p.
	\]
	已知 $p \mid |X|$,于是
	\[
	p \mid |X_0|.
	\]
	
	又由第三步,$X_0 = N_G(H)/H$,所以
	\[
	p \mid |N_G(H)/H|.
	\]
	
	\textbf{第五步:在 $N_G(H)/H$ 中用 Cauchy 定理得到 $p$ 阶子群,并对应回 $N_G(H)$。}
	
	因为
	\[
	p \mid |N_G(H)/H|,
	\]
	由 Cauchy 定理,存在子群
	\[
	H'/H \le N_G(H)/H,\quad \text{使得 } |H'/H| = p.
	\]
	
	现在使用\textbf{商群与原群子群对应定理}(详细说明如下):
	
	设 $\pi: N_G(H) \to N_G(H)/H$ 为自然满同态,核为 $H$。
	则
	\[
	\{\text{包含 }H\text{ 的子群 }K \le N_G(H)\}
	\longleftrightarrow
	\{\text{子群 }K/H \le N_G(H)/H\}
	\]
	一一对应,且
	\[
	|K/H| = \frac{|K|}{|H|}.
	\]
	
	对上面得到的 $H'/H$,存在对应的子群 $H' \le N_G(H)$,满足:
	\[
	H \le H', \qquad \frac{|H'|}{|H|} = |H'/H| = p.
	\]
	于是
	\[
	|H'| = p |H| = p \cdot p^n = p^{n+1}.
	\]
	
	也就是说,我们在 $N_G(H)$(从而在 $G$)中找到了一个\emph{包含 $H$ 的}阶为 $p^{n+1}$ 的子群 $H'$。
	
	\textbf{第六步:归纳完成 Sylow 子群的存在性。}
	
	从第一步我们已有阶为 $p$ 的子群 $H_1$。
	应用第五步的结论:
	
	- 若存在 $p^n$ 阶子群,则存在 $p^{n+1}$ 阶子群。
	
	于是从 $n=1$ 起反复递推,得到阶为
	\[
	p^2, p^3, \dots, p^\ell
	\]
	的子群,最终得到阶为 $p^\ell$ 的子群,这就是 Sylow $p$-子群。
	
	同时,因为我们是逐级构造的,对每个 $1 \le k \le \ell$,
	在递推过程中都会出现一个阶为 $p^k$ 的子群。
	从而定理中“对任意 $k$ 存在 $p^k$ 阶子群”的结论也成立。
	
	\medskip
	
	至此,Sylow 第一定理的存在性及“逐级升阶”结论得证。
	
	
	\subsection{Sylow $p$-子群的等价命题}
	\textbf{命题.} 设 $H$ 为有限群,$|H| = p^n m$,其中 $p$ 为素数且 $p\nmid m$。对 $H$ 的一个子群 $P\le H$,下述四个条件等价:
	\begin{itemize}
		\item[(1)] $P$ 是 $H$ 的 Sylow $p$-子群;
		\item[(2)] $P$ 是 $H$ 的 $p$-子群,且
		\[
		|P| = p^n
		\]
		即 $|P|$ 等于 $|H|$ 中 $p$ 的最高幂;
		\item[(3)] $P$ 是 $H$ 的 $p$-子群,且对任意 $H$ 的 $p$-子群 $Q\le H$,有
		\[
		|Q| \le |P|;
		\]
		\item[(4)] $P$ 是 $H$ 的 $p$-子群,且若 $Q$ 是 $H$ 的 $p$-子群并满足 $P\le Q\le H$,则必有
		\[
		Q = P.
		\]
	\end{itemize}
	
	\textbf{证明.}
	由定义,(1) 与 (2) 等价:因为 Sylow $p$-子群就是阶为 $p^n$ 的 $p$-子群,而 $p^n$ 正是 $|H|$ 中 $p$ 的最高幂。
	
	\medskip
	
	\noindent (2) $\Rightarrow$ (3).  
	设 $P$ 是阶为 $p^n$ 的 $p$-子群。取任意 $p$-子群 $Q\le H$,记 $|Q| = p^k$。由拉格朗日定理,
	\[
	|Q|\mid |H| = p^n m,
	\]
	而 $p\nmid m$,故 $p^k\mid p^n$,从而 $k\le n$,即
	\[
	|Q| = p^k \le p^n = |P|.
	\]
	于是 $P$ 在所有 $p$-子群中按阶最大。
	
	\medskip
	
	\noindent (3) $\Rightarrow$ (2).  
	设 $P$ 是 $H$ 的 $p$-子群,且在所有 $p$-子群中阶最大。由 Sylow 定理知,$H$ 中存在阶为 $p^n$ 的 Sylow $p$-子群 $S$。若 $|P|=p^k$ 且 $k<n$,则
	\[
	|S| = p^n > p^k = |P|,
	\]
	这与 $P$ 阶最大矛盾。因此必须有 $|P|=p^n$,即 (2) 成立。
	
	\medskip
	
	\noindent (2) $\Rightarrow$ (4).  
	设 $P$ 是阶为 $p^n$ 的 $p$-子群。若 $P\le Q\le H$,且 $Q$ 也是 $p$-子群,记 $|Q|=p^k$。由 $P\le Q$ 得 $|P|\mid |Q|$,即 $p^n \mid p^k$,从而 $n\le k$。另一方面,$Q\le H$ 且为 $p$-子群,则由上面同样的整除关系有 $k\le n$。故 $k=n$,即 $|Q|=|P|$。由于 $P\le Q$ 且阶相同,只能有 $Q=P$。于是 (4) 成立。
	
	\medskip
	
	\noindent (4) $\Rightarrow$ (2).  
	设 $P$ 是 $p$-子群,并在所有 $p$-子群中按包含关系极大:若 $P\le Q\le H$ 且 $Q$ 为 $p$-子群,则 $Q=P$。  
	由 Sylow 定理知,$H$ 中存在 Sylow $p$-子群 $S$。注意到 $P$ 也是 $p$-子群,且 $P\le S$(这是 Sylow 定理的常见推论:任一 $p$-子群都包含在某个 Sylow $p$-子群中)。由极大性可得
	\[
	P = S.
	\]
	于是 $P$ 的阶为 $p^n$,即满足 (2)。
	
	\medskip
	
	综上,(1)、(2)、(3)、(4) 互相等价。证毕。
	
	
	
	
	\subsection{$H$作用在$G/H$上的性质}
	\textbf{命题.}\;
	设 $H < G$,在陪集集合
	\[
	G/H = \{\, gH \mid g \in G \,\}
	\]
	上定义 $H$ 的左作用:
	\[
	\forall h \in H,\ \forall gH \in G/H,\quad
	h \cdot (gH) := hgH.
	\]
	则该作用良定义,并具有以下性质。
	
	\textbf{证明.}
	\textbf{(1) 良定义.}
	若 $gH = g'H$,则存在 $h_0 \in H$ 使得 $g' = gh_0$。
	于是
	\[
	h g'H = h (g h_0) H = (h g) h_0 H.
	\]
	由于 $h_0 \in H$ 且 $H$ 为子群,$h_0 H = H$,因此
	\[
	(h g) h_0 H = h g H.
	\]
	从而 $h g'H = h gH$,该作用良定义。
	
	\textbf{(2) 满足群作用公理.}
	对任意 $gH \in G/H$:
	\[
	e \cdot (gH) = e gH = gH,
	\]
	且对任意 $h_1,h_2 \in H$,
	\[
	h_1 \cdot (h_2 \cdot (gH))
	= h_1 \cdot (h_2 gH)
	= h_1 h_2 gH
	= (h_1 h_2) \cdot (gH).
	\]
	因此这确实是 $H$ 在 $G/H$ 上的左作用。
	
	\textbf{(3) 轨道与稳定子.}
	
	对任意 $gH \in G/H$:
	
	\emph{轨道:}
	\[
	\mathcal{O}_{gH}
	= \{\, h \cdot (gH) \mid h \in H \,\}
	= \{\, hgH \mid h \in H \,\}
	= \{\, (hg)H \mid h \in H \,\}.
	\]
	
	\emph{稳定子:}
	\[
	\mathrm{Stab}_H(gH)
	= \{\, h \in H \mid h \cdot (gH) = gH \,\}
	= \{\, h \in H \mid hgH = gH \,\}.
	\]
	条件 $hgH = gH$ 等价于
	\[
	g^{-1} h g H = H
	\iff g^{-1} h g \in H,
	\]
	故
	\[
	\mathrm{Stab}_H(gH)
	= H \cap g H g^{-1}.
	\]
	
	\textbf{(4) 轨道—稳定子定理.}
	由一般结论:
	\[
	|\mathcal{O}_{gH}| = [H : \mathrm{Stab}_H(gH)]
	= \frac{|H|}{|H \cap g H g^{-1}|}.
	\]
	
	\textbf{(5) 当 $H$ 是 $p$-群时,轨道大小是 $p$ 的幂.}
	若 $|H| = p^n$,则 $|H \cap g H g^{-1}|$ 也是 $p$ 的幂,
	因此
	\[
	|\mathcal{O}_{gH}|
	= \frac{|H|}{|H \cap g H g^{-1}|}
	\]
	也是 $p$ 的幂。
	特别地,轨道大小可能是 $1,p,p^2,\dots$。
	
	\textbf{(6) 不动点与正规化子的关系.}
	
	定义不动点集:
	\[
	(G/H)^H = \{\, gH \in G/H \mid h \cdot (gH) = gH,\ \forall h \in H \,\}.
	\]
	
	展开定义:
	\[
	h \cdot (gH) = gH\ \forall h \in H
	\iff \forall h \in H,\ hgH = gH
	\iff \forall h \in H,\ g^{-1}hg \in H.
	\]
	于是
	\[
	(G/H)^H
	= \{\, gH \mid g^{-1}Hg \subseteq H \,\}.
	\]
	
	由于共轭映射 $\mathrm{ad}_{g^{-1}}:x \mapsto g^{-1}xg$ 是群自同构,
	保持子群阶数,即
	\[
	|g^{-1}Hg| = |H|.
	\]
	有限群中若有 $g^{-1}Hg \subseteq H$ 且 $|g^{-1}Hg|=|H|$,则两者相等:
	\[
	g^{-1}Hg = H.
	\]
	这等价于
	\[
	gHg^{-1} = H,
	\]
	即 $g \in N_G(H)$,其中
	\[
	N_G(H) = \{\, x \in G \mid xHx^{-1} = H \,\}
	\]
	称为 $H$ 的正规化子。
	故
	\[
	(G/H)^H = \{\, gH \mid g \in N_G(H) \,\} = N_G(H)/H.
	\]
	
	\textbf{(7) 结果总结.}
	因此,$H$ 在 $G/H$ 上左平移作用的主要性质如下:
	\begin{itemize}
		\item 作用良定义且确为群作用;
		\item 对任意 $gH$,轨道为 $\mathcal{O}_{gH}=\{(hg)H\mid h\in H\}$;
		\item 稳定子为 $\mathrm{Stab}_H(gH)=H\cap gHg^{-1}$;
		\item 轨道—稳定子定理成立;
		\item 若 $H$ 是 $p$-群,轨道大小为 $p$ 的幂;
		\item 不动点集 $(G/H)^H = N_G(H)/H$。
	\end{itemize} \qed
	
	\subsection{$G$ 的所有 Sylow $p$-子群恰好是 $P$ 在共轭作用下的轨道}
	\textbf{定理(Sylow 第二定理).}
	设有限群 $G$ 的阶为 $|G| = p^n m,\ (p,m)=1$。
	若 $P, Q$ 是 $G$ 的两个 Sylow $p$-子群,
	则存在 $g \in G$ 使得
	\[
	Q = gPg^{-1}.
	\]
	换言之,所有 Sylow $p$-子群在 $G$ 中彼此共轭。
	
	\textbf{证明.}
	
	\textbf{(1) 预备:共轭映射的性质.}
	
	对任意 $g \in G$,定义共轭映射
	\[
	\mathrm{ad}_g : G \to G, \quad x \mapsto gxg^{-1}.
	\]
	我们验证 $\mathrm{ad}_g$ 是 $G$ 的自同构:
	
	\begin{itemize}
		\item \emph{保运算性:}
		\[
		\mathrm{ad}_g(xy)
		= g(xy)g^{-1}
		= (gxg^{-1})(gyg^{-1})
		= \mathrm{ad}_g(x)\,\mathrm{ad}_g(y).
		\]
		故 $\mathrm{ad}_g$ 是群同态。
		
		\item \emph{单射性:}
		若 $\mathrm{ad}_g(x_1) = \mathrm{ad}_g(x_2)$,即
		\[
		gx_1g^{-1} = gx_2g^{-1},
		\]
		左乘 $g^{-1}$、右乘 $g$,得 $x_1 = x_2$。
		
		\item \emph{满射性:}
		对任意 $y \in G$,取 $x = g^{-1}yg$,
		则 $\mathrm{ad}_g(x) = y$。
	\end{itemize}
	
	因此 $\mathrm{ad}_g$ 是 $G$ 的自同构。
	
	\textbf{(2) 共轭保持子群阶数.}
	
	若 $P \le G$,定义
	\[
	gPg^{-1} = \{\, gpg^{-1} \mid p \in P \,\}.
	\]
	由于 $\mathrm{ad}_g$ 是自同构,限制到 $P$ 上是同构:
	\[
	\mathrm{ad}_g|_P : P \xrightarrow{\cong} gPg^{-1}.
	\]
	于是
	\[
	|gPg^{-1}| = |P|.
	\]
	因此 $P$ 与 $gPg^{-1}$ 阶相等。
	
	\textbf{(3) 共轭后的子群仍为 Sylow $p$-子群.}
	
	若 $P$ 是 Sylow $p$-子群,则 $|P| = p^n$。
	由 (2),对任意 $g \in G$,
	\[
	|gPg^{-1}| = |P| = p^n,
	\]
	因此 $gPg^{-1}$ 也是 Sylow $p$-子群。
	
	\textbf{(4) Sylow 子群之间的共轭关系.}
	
	Sylow 第二定理断言:对任意两个 Sylow $p$-子群 $P, Q$,
	存在某个 $g \in G$ 使得 $Q = gPg^{-1}$。
	故所有 Sylow $p$-子群两两共轭。
	
	\textbf{(5) 结论.}
	
	取定任意一个 Sylow $p$-子群 $P$。
	由上结论可知,$G$ 的所有 Sylow $p$-子群恰好是 $P$ 在共轭作用下的轨道:
	\[
	\boxed{
		\mathrm{Syl}_p(G)
		= \{\, gPg^{-1} \mid g \in G \,\}.
	}
	\]
	每个 $gPg^{-1}$ 都是 Sylow $p$-子群,
	且任意 Sylow $p$-子群都可写成此形式。 \qed
	\subsection{两个不同的 Sylow-$p$ 子群只有在阶为$ p$(而不是 $p^k,\ k\ge2$时),其交集必为 ${e}$。
	}
	\textbf{命题.}\;
	设 $G$ 为群,$p$ 为素数。若 $P,Q\le G$ 是两个不同的 Sylow-$p$ 子群,
	并且
	\[
	|P|=|Q|=p,
	\]
	则
	\[
	P\cap Q=\{e\}.
	\]
	
	\textbf{证明.}\;
	由于 $|P|=p$,而 $p$ 为素数,故 $P$ 是一个 $p$ 阶循环群。
	同理,$Q$ 也是 $p$ 阶循环群。
	
	假设反之,$P\cap Q\neq\{e\}$。
	则存在元素 $x\in P\cap Q$,且 $x\neq e$。
	
	因为 $x\in P$ 且 $|P|=p$,所以 $x$ 的阶只能是 $p$,
	从而
	\[
	P=\langle x\rangle.
	\]
	同理,由于 $x\in Q$ 且 $|Q|=p$,也有
	\[
	Q=\langle x\rangle.
	\]
	
	于是得到
	\[
	P=Q,
	\]
	这与 $P,Q$ 是两个不同的 Sylow-$p$ 子群相矛盾。
	
	因此假设不成立,只能有
	\[
	P\cap Q=\{e\}.
	\] \qed
	
	\subsection{Sylow $p$-子群中元素阶的性质}
		\textbf{定理(Sylow $p$-子群中元素阶的性质).}
	
	设 $G$ 为有限群,$p$ 为素数,$P$ 为 $G$ 的一个 Sylow $p$-子群,
	即 $|P|=p^n$($n\ge 1$)。
	则对任意 $x\in P$,其阶 $|x|$ 必为 $p$ 的幂。
	特别地,
	\[
	x\neq e \ \Longrightarrow\ |x|=p^k,\quad 1\le k\le n.
	\]
	
	\textbf{证明.}
	
	由于 $P$ 是有限群,且 $x\in P$,
	由拉格朗日定理,元素 $x$ 的阶 $|x|$ 必整除群的阶 $|P|$,
	即
	\[
	|x|\mid |P|=p^n.
	\]
	而 $p^n$ 的所有正因子只能是
	\[
	1,p,p^2,\dots,p^n.
	\]
	因此 $|x|$ 必为 $p$ 的幂。
	若 $x\neq e$,则 $|x|\neq 1$,从而 $|x|=p^k$,其中 $1\le k\le n$。\qed
	
	\clearpage		
	\subsection*{课后习题答案}
	\addcontentsline{toc}{subsection}{\textcolor{red}{课后习题答案}}
	\begin{enumerate}[label=\textcolor{blue}{\textbf{\large\arabic*.}}]	
		\item 
		\textbf{题目.}  
		设群 $G$ 的阶为 $p^l m$,其中 $p$ 为素数,$(p, m) = 1$。  
		记
		\[
		X = \{ A \subseteq G \mid |A| = p^k \},
		\]
		则 $G$ 在自身上的左平移作用自然诱导了一个 $G$ 在 $X$ 上的作用。  
		对任意 $A \in X$,记 $F_A$ 为其迷向子群(即稳定子群)。
		
		\begin{enumerate}
			\item[(1)] 证明:$A$ 是 $F_A$ 的一些右陪集的并,从而 $F_A$ 是一个 $p$-群,且 $|F_A| \le p^k$。
			\item[(2)] 证明:存在 $A \in X$ 使得 $|F_A| = p^k$。
		\end{enumerate}
		
		\textbf{证明.}\\
			\textbf{(0)} 首先证明这确实诱导了一个作用:\\
		在 $G$ 自身上定义左平移作用 $g\cdot x = gx$。  
		对任意 $A\subseteq G$,定义
		\[
		g\cdot A = \{ ga \mid a\in A \}.
		\]
		由于左乘映射 $x\mapsto gx$ 是双射,故 $|gA| = |A|$。  
		若 $A\in X$,则 $|A|=p^k$,从而 $|gA|=p^k$,即 $gA\in X$。  
		因此该映射 $G\times X\to X$ 定义良好。
		
		又对任意 $g,h\in G$、$A\in X$,
		\[
		(gh)A = \{(gh)a\mid a\in A\} = g\{ha\mid a\in A\} = g(hA), \quad eA=A.
		\]
		所以这是 $G$ 在 $X$ 上的一个群作用。
		\[
		\boxed{\text{$G$ 在 $X$ 上的左乘作用定义良好,满足群作用的条件。}}
		\]
		
		
		\textbf{(1) 证明 $A$ 是 $F_A$ 的若干左陪集的不交并,进而 $|F_A|\mid |A|=p^k$。}
		
		\emph{(a) $F_A$ 是子群.}
		定义稳定子
		\[
		F_A:=\{g\in G\mid gA=A\}.
		\]
		显然 $eA=A$,故 $e\in F_A$。若 $g,h\in F_A$,则 $gA=A$ 且 $hA=A$,从而
		\[
		(gh)A=g(hA)=gA=A,\qquad g^{-1}A=g^{-1}(gA)=A.
		\]
		故 $gh,g^{-1}\in F_A$,于是 $F_A\le G$。
		
		\emph{(b) $A$ 是 $F_A$ 的左不变集.}
		取任意 $x\in A$、$f\in F_A$。因 $fA=A$,对所有 $a\in A$ 有 $fa\in A$,特别地
		\[
		fx\in A.
		\]
		这说明
		\[
		F_Ax\subseteq A\qquad(\forall\,x\in A).
		\]
		
		\emph{(c) $A$ 是若干左陪集的不交并.}
		对每个 $y\in A$,有 $e\in F_A$,故 $y=e\,y\in F_Ay$,于是
		\[
		A=\bigcup_{x\in A}F_Ax.
		\]
		若 $F_Ax\cap F_Ay\neq\varnothing$,则存在 $f_1,f_2\in F_A$ 使 $f_1x=f_2y$,从而
		\[
		x=f_1^{-1}f_2\,y,\qquad f_1^{-1}f_2\in F_A,
		\]
		故 $x\in F_Ay$,即 $F_Ax=F_Ay$。于是可选取代表元 $x_1,\dots,x_t\in A$ 使
		\[
		A=\bigsqcup_{i=1}^t F_Ax_i
		\]
		为不交并。
		
		\emph{(d) 结论.}
		取基数得
		\[
		|A|=\sum_{i=1}^t|F_Ax_i|=\sum_{i=1}^t|F_A|=t\,|F_A|.
		\]
		因为 $|A|=p^k$,故 $|F_A|$ 为 $p$ 的幂,且 $|F_A|\le p^k$。因此 $F_A$ 是一个 $p$-群,并且
		\[
		|F_A|\mid p^k,\qquad |F_A|\le p^k.
		\]

		
		\textbf{(2)}  
		由群作用的轨道稳定子定理,对 $A \in X$ 有
		\[
		|G| = |G \cdot A| \cdot |F_A|.
		\]
		取模 $p$,因为 $|G| = p^l m$ 且 $(p, m) = 1$,故 $|G \cdot A|$ 是 $m$ 的倍数或不被 $p$ 整除的数。  
		因此在所有轨道中,至少存在一个轨道的大小不被 $p$ 整除。  
		设该轨道对应的集合为 $A_0$,即 $|G \cdot A_0|$ 不被 $p$ 整除。  
		由上式得
		\[
		|F_{A_0}| = \frac{|G|}{|G \cdot A_0|},
		\]
		所以 $|F_{A_0}|$ 含有 $p^l$ 的全部 $p$-因子。
		
		但由 (1) 已知 $|F_{A_0}| \le p^k$,因此必有
		\[
		|F_{A_0}| = p^k.
		\]
		
		于是存在 $A_0 \in X$ 满足 $|F_{A_0}| = p^k$。
		
		\[
		\boxed{\text{证毕.}}
		\]
		\textbf{题目.}\;
		设群 $G$ 的阶为 $p^l m$,其中 $p$ 为素数,$(p,m)=1$。记
		\[
		X=\{A\subseteq G\mid |A|=p^k\},
		\]
		则 $G$ 在自身上的左平移作用自然诱导了一个 $G$ 在 $X$ 上的作用。对任意 $A\in X$,记 $F_A$ 为其稳定子群(迷向子群)。
		证明:(2)存在 $A\in X$ 使得 $|F_A|=p^k$。
		
		\textbf{答案:}
		
		\textbf{证明.}\;
		由于 $p^k\mid |G|=p^l m$ 且 $(p,m)=1$,必有 $k\le l$。
		
		先取 $G$ 的一个 Sylow-$p$ 子群 $P$,则 $|P|=p^l$。
		
		若 $P$ 为阶为 $p^l$ 的 $p$-群,则对任意整数 $k$ 满足 $0\le k\le l$,存在子群 $H\le P$ 使得 $|H|=p^k$。,取 $H\le P\le G$ 使得 $|H|=p^k$。令
		\[
		A:=H.
		\]
		则 $A\in X$。下面计算其稳定子群
		\[
		F_A=\{g\in G\mid gA=A\}=\{g\in G\mid gH=H\}.
		\]
		若 $g\in F_A$,则 $gH=H$。特别地,$g\in gH=H$(因为 $e\in H$,故 $g=ge\in gH$),从而 $g\in H$。
		反过来,若 $g\in H$,则 $gH=H$(左乘子群仍为自身),故 $g\in F_A$。
		因此
		\[
		F_A=H,
		\]
		于是
		\[
		|F_A|=|H|=p^k.
		\]
		所求 $A$ 存在,命题(2)成立。
		
		\item 
		\textbf{题目.}   证明:$p^2$ 阶群都是 Abel 群,并求其同构类。
			\begin{tcolorbox}[
			colback = white,          % 背景色
			colframe = myred,         % 边框颜色
			coltitle = white,         % 标题文字颜色
			title = {同构类的定义},
			fonttitle = \bfseries,    % 标题加粗
			%breakable,                % 允许跨页
			%enhanced,                 % 高级样式
			sharp corners,            % 方角
			boxrule = 0.8pt,          % 边框线条粗细
			left = 2mm, right = 2mm, top = 2mm, bottom = 2mm  % 内边距
			]
			\textbf{定义.}  
			给定一个群 $G$,所有与它同构的群构成的集合称为 $G$ 的 \textbf{同构类}(isomorphism class),记作
			\[
			[G] = \{\, H \mid H \cong G \,\}.
			\]
			也就是说,$[G]$ 包含了所有与 $G$ 在结构上完全相同(即存在群同构)的群。
		\end{tcolorbox}
		\textbf{证明.}  
		\textbf{情形 1:$G$ 中存在阶为 $p^2$ 的元素.}\;
		若存在 $x\in G$ 使 $|x|=p^2$,则循环子群 $\langle x\rangle$ 的阶为 $p^2$,从而
		\[
		|\langle x\rangle|=|G|\ \Longrightarrow\ \langle x\rangle=G.
		\]
		因此 $G$ 为循环群,
		\[
		G\cong \mathbb{Z}/p^2\mathbb{Z},
		\]
		特别地 $G$ 为 Abel 群。
		
		\textbf{情形 2:$G$ 中不存在阶为 $p^2$ 的元素.}\;
		则对任意 $g\in G\setminus\{e\}$,由 Lagrange 定理知 $|g|\mid p^2$ 且 $|g|\neq 1$,故
		\[
		|g|=p.
		\]
		又因为 $G$ 是 $p$-群,中心非平凡:$Z(G)\neq\{e\}$。取
		\[
		a\in Z(G)\setminus\{e\},
		\]
		则 $|a|=p$,从而 $|\langle a\rangle|=p$。
		
		由于 $|\langle a\rangle|=p<|G|=p^2$,可取 $b\in G\setminus\langle a\rangle$。
		同理 $|b|=p$,故 $|\langle b\rangle|=p$,且 $\langle a\rangle\neq\langle b\rangle$。
		两个阶为 $p$ 的子群若不相等,则交只能为单位元,因此
		\[
		\langle a\rangle\cap\langle b\rangle=\{e\}.
		\]
		又因 $a\in Z(G)$,所以 $ab=ba$,从而 $\langle a\rangle\langle b\rangle$ 实为子群,并且
		\[
		|\langle a\rangle\langle b\rangle|
		=\frac{|\langle a\rangle|\cdot|\langle b\rangle|}{|\langle a\rangle\cap\langle b\rangle|}
		=\frac{p\cdot p}{1}
		=p^2.
		\]
		于是
		\[
		\langle a,b\rangle \supseteq \langle a\rangle\langle b\rangle
		\quad\Longrightarrow\quad
		|\langle a,b\rangle|\ge p^2.
		\]
		但 $\langle a,b\rangle\le G$ 且 $|G|=p^2$,故
		\[
		\langle a,b\rangle=G.
		\]
		现在定义映射
		\[
		\Phi:(\mathbb{Z}/p\mathbb{Z})\times(\mathbb{Z}/p\mathbb{Z})\to G,\qquad
			\Phi(\bar{i},\bar{j})=a^i b^j.
			\]
			因 $ab=ba$,可知 $\Phi$ 为群同态。并且若 $\Phi(\bar{i},\bar{j})=e$,即 $a^i b^j=e$,
			则 $a^i=b^{-j}\in \langle a\rangle\cap\langle b\rangle=\{e\}$,故 $p\mid i$ 且 $p\mid j$,
			从而核平凡,$\Phi$ 为单射。由于定义域与值域同为 $p^2$ 个元素,$\Phi$ 亦为满射,
			于是 $\Phi$ 为同构,得到
			\[
			G\cong (\mathbb{Z}/p\mathbb{Z})\times(\mathbb{Z}/p\mathbb{Z}).
				\]
				因此 $G$ 为 Abel 群。
				\textbf{综上.}\;
				阶为 $p^2$ 的群必为 Abel 群,且其同构类恰为两种:
				\[
				G\cong \mathbb{Z}/p^2\mathbb{Z}
				\quad\text{或}\quad
				G\cong (\mathbb{Z}/p\mathbb{Z})\times(\mathbb{Z}/p\mathbb{Z}).
					\]					
		\item 
		\textbf{题目.} 确定 $S_4$ 的 Sylow 子群的个数。
		
		\textbf{解答.}  
		由 $|S_4| = 4! = 24 = 2^3 \cdot 3$,  
		故需分别求 Sylow $2$-子群与 Sylow $3$-子群的个数。
		
		---
		
		\textbf{(1) 求 Sylow $2$-子群的个数.}  
		
		设 $n_2$ 为 Sylow $2$-子群的个数,则由 Sylow 定理:
		\[
		n_2 \mid \frac{|S_4|}{2^3} = 3, \quad n_2 \equiv 1 \pmod{2}.
		\]
		因此 $n_2 = 1$ 或 $3$。
		若$n_2 = 1$,则说明Sylow $2-$子群为$S_4$
		的8 阶正规子群,而熟知$S_4$ 的非平凡正规子群仅有$A_4$(12阶) 和$K_4$(4阶),从而矛盾,  
		故 $n_2 \ne 1$,从而 $n_2 = 3$。
		
		---
		
		\textbf{(2) 求 Sylow $3$-子群的个数.}  
		
		设 $n_3$ 为 Sylow $3$-子群的个数,则
		\[
		n_3 \mid \frac{|S_4|}{3} = 8, \quad n_3 \equiv 1 \pmod{3}.
		\]
		故 $n_3 = 1, 4$。
		
		$S_4$ 中阶为 $3$ 的元素是所有的 $3$-轮换,共有 $8$ 个(如⟨(123)⟩,
		⟨(124)⟩,⟨(134)⟩,⟨(234)⟩ 均为Sylow 3 子群)。  
		而每个 Sylow $3$-子群(阶 $3$)包含 $2$ 个非单位元,  
		所以
		\[
		n_3  = 4.
		\]
		
		\textbf{(3) 结论.}  
		
		\[
		\boxed{n_2 = 3,\quad n_3 = 4.}
		\]
		
		
		\item 
		\textbf{题目.}
		设 $p$ 为素数,试求 $S_p$ 的 Sylow $p$-子群的个数,并由此证明 Wilson 定理:
		\[
		(p-1)! \equiv -1 \pmod{p}.
		\]
		
		\textbf{答案:}
		设 $S_p$ 是 $p$ 阶对称群,则 $|S_p| = p!$。  
		由 Sylow 定理,$S_p$ 的 Sylow $p$-子群的阶为 $p$,其个数 $n_p$ 满足:
		\[
		n_p \mid \frac{p!}{p} = (p-1)! \quad \text{且} \quad n_p \equiv 1 \pmod{p}.
		\]
		
		下面求出 $n_p$ 的具体数值。
		
		一个 $p$-子群必是由一个 $p$-循环生成的。  
		例如,$(1\,2\,3\,\dots,p)$ 是一个阶为 $p$ 的循环置换。  
		每一个 $p$-循环都生成一个 Sylow $p$-子群,而同一个子群恰好包含 $p-1$ 个不同的 $p$-循环(即它的非单位元)。
		
		对称群 $S_p$ 中 $p$-循环的总个数为:
		\[
		\frac{p!}{p} = (p-1)!,
		\]
		因为确定一个 $p$ 元循环相当于选择 $p$ 个元素的排列方式并按循环顺序写出,考虑循环的旋转同一性后应除以 $p$。
		
		因此,Sylow $p$-子群的个数为:
		\[
		n_p = \frac{(p-1)!}{p-1} = (p-2)!.
		\]
		
		由 Sylow 定理得:
		\[
		n_p = (p-2)! \equiv 1 \pmod{p}.
		\]
		两边同时乘以 $(p-1)$,得到:
		\[
		(p-1)! \equiv (p-1) \pmod{p}.
		\]
		又因为 $(p-1) \equiv -1 \pmod{p}$,于是:
		\[
		(p-1)! \equiv -1 \pmod{p}.
		\]
		这正是 Wilson 定理所要求的结果。
		\item 
		\textbf{题目.}  
		证明:$p$-群的非正规子群的个数是 $p$ 的倍数。
		\textbf{证明.}  
		定义映射
		\[
		\cdot : G \times \mathcal{H} \longrightarrow \mathcal{H},
		\qquad (g,H) \longmapsto g\cdot H := gHg^{-1}.
		\]
		我们需验证:  
		(i) 封闭性;  
		(ii) 单位元作用恒等性;  
		(iii) 结合律。
		
		---
		
		\textbf{(i) 封闭性.}
		对任意 $g\in G$、$H\in \mathcal{H}$,需说明 $gHg^{-1}$ 是 $G$ 的子群。
		
		因为共轭映射
		\[
		\varphi_g : G \to G, \qquad x \mapsto gxg^{-1}
		\]
		是双射(其逆为 $\varphi_{g^{-1}}$),  
		并且对任意 $x,y\in G$ 有
		\[
		\varphi_g(xy) = gxyg^{-1} = (gxg^{-1})(gyg^{-1}) = \varphi_g(x)\varphi_g(y),
		\]
		所以 $\varphi_g$ 是群同构。
		
		因此 $gHg^{-1} = \varphi_g(H)$ 是 $H$ 在同构下的像,从而仍为 $G$ 的子群。  
		若 $H$ 为 $G$ 的非平凡真子群,则 $e\neq H\neq G$,共轭保持阶与结构,故 $gHg^{-1}$ 亦为非平凡真子群。  
		因此 $gHg^{-1}\in\mathcal{H}$,映射封闭。
		
		---
		
		\textbf{(ii) 单位元作用恒等性.}
		对任意 $H\in\mathcal{H}$,
		\[
		e\cdot H = eHe^{-1} = H.
		\]
		
		---
		
		\textbf{(iii) 结合律.}
		对任意 $g_1,g_2\in G$ 与 $H\in\mathcal{H}$,
		\[
		g_1\cdot(g_2\cdot H)
		= g_1(g_2Hg_2^{-1})g_1^{-1}
		= (g_1g_2)H(g_1g_2)^{-1}
		= (g_1g_2)\cdot H.
		\]
		
		---
		
		综上,映射
		\[
		(g,H)\mapsto gHg^{-1}
		\]
		满足封闭性、单位元作用恒等性及结合律,  
		因此确实定义了 $G$ 在 $\mathcal{H}$ 上的一个群作用。
		
		---
		
		由群作用的一般理论可知,此作用将 $\mathcal{H}$ 分解为若干互不相交的轨道:
		\[
		\mathcal{H} = \bigcup_{H\in \mathcal{H}} G\cdot H, 
		\quad \text{其中}\quad 
		G\cdot H = \{ gHg^{-1} \mid g\in G \}.
		\]
		每个轨道对应一类互为共轭的子群,这些轨道两两不交,其并为整个 $\mathcal{H}$。
		
		---
		
		对每个 $H\in \mathcal{H}$,其稳定子为正规化子
		\[
		N_G(H)=\{g\in G\mid gHg^{-1}=H\}.
		\]
		由轨道–稳定子定理,
		\[
		|\text{轨道 } G\cdot H| = [G : N_G(H)].
		\]
		
		由于 $G$ 是 $p$-群,故 $|G|=p^n$、$|N_G(H)|=p^m$,于是
		\[
		[G:N_G(H)] = p^{n-m},
		\]
		是 $p$ 的幂。
		
		若 $H$ 是正规子群,则 $N_G(H)=G$,轨道大小为 $1$;  
		若 $H$ 不是正规子群,则 $N_G(H)\subsetneq G$,于是 $[G:N_G(H)]\ge p$,其轨道大小为 $p$ 的倍数。
		
		---
		
		因此:
		- 每个正规子群对应一个大小为 $1$ 的轨道;
		- 每个非正规子群都包含在大小为 $p$ 的倍数的轨道中。
		
		设所有轨道为 $G\cdot H_1, G\cdot H_2, \dots, G\cdot H_r$,  
		则
		\[
		\mathcal{H} = \bigcup_{i=1}^{r} (G\cdot H_i),
		\qquad
		|\mathcal{H}| = \sum_{i=1}^{r} |G\cdot H_i|.
		\]
		
		---
		
		\textbf{整除性分析.}
		对每个正规子群的轨道,大小为 $1$;  
		对每个非正规子群的轨道,大小为 $p^{n-m}$,是 $p$ 的倍数。  
		因此:
		\[
		|\mathcal{H}| = (\text{若干个 }1) + (\text{若干个 }p\text{ 的倍数}).
		\]
		设正规子群共有 $r_0$ 个,则
		\[
		|\mathcal{H}| = r_0 + pk, \qquad k\in\mathbb{Z}.
		\]
		
		---
		
		\textbf{提取非正规子群的个数.}
		所有非正规子群的总数等于
		\[
		(\text{非正规子群个数}) = |\mathcal{H}| - r_0 = pk.
		\]
		于是它必定是 $p$ 的倍数。
		
		\[
		\boxed{\text{$p$-群的非正规子群的个数是 $p$ 的倍数。}}
		\]
		
		\item
		\textbf{题目.} 设 $G$ 是一个 $p$-群,$N \triangleleft G$,且 $|N|=p$。证明 $N \subseteq C(G)$。
		
		\textbf{证明.}  
		考虑 $G$ 对 $N$ 的共轭作用:
		\[
		\varphi : G \longrightarrow \mathrm{Aut}(N), \qquad g \mapsto (x \mapsto gxg^{-1}).
		\]
		显然 $\varphi$ 是群同态,其核为 $C_G(N)$($N$ 的中心化子)。
		
		由同态基本定理,有
		\[
		G / C_G(N) \cong \mathrm{Im}(\varphi) \le \mathrm{Aut}(N).
		\]
		
		---
		
		由于 $|N|=p$,$N$ 是循环群。设 $N = \langle a \rangle$。  
		则 $\mathrm{Aut}(N)$ 是由映射 $a \mapsto a^k$(其中 $1\le k \le p-1$)构成的群,
		故
		\[
		|\mathrm{Aut}(N)| = p-1.
		\]
		
		于是有
		\[
		|G : C_G(N)| = |\mathrm{Im}(\varphi)| \ \big|\ (p-1).
		\]
		
		但因为 $G$ 是 $p$-群,故其任何真因子的指数必须是 $p$ 的幂。  
		而 $p-1$ 与 $p$ 互素,因此只能有
		\[
		|G : C_G(N)| = 1.
		\]
		
		即 $C_G(N) = G$,从而 $N \subseteq C_G(N) = C(G)$。
		
		\qed
		
		
		\item 
		\textbf{题目.}
		设 $|G| = p^l m$,其中 $p$ 为素数,$l > 1$,且 $(p,m) = 1$。  
		证明:$G$ 的任意 $p^k$ 阶子群 $P \ (k < l)$ 是其正规化子 $N_G(P)$ 的真子群。
		\textbf{题目.}\;
		设 $|G|=p^l m$,其中 $p$ 为素数,$l>1$,且 $(p,m)=1$。
		证明:$G$ 的任意 $p^k$ 阶子群 $P$($k<l$)是其正规化子
		$N_G(P)$ 的真子群。
		
		\textbf{证明.}
		
		首先,由正规化子的定义,
		\[
		N_G(P)=\{g\in G\mid gPg^{-1}=P\},
		\]
		显然对任意 $p\in P$ 都有 $pPp^{-1}=P$,因此
		\[
		P\le N_G(P).
		\]
		
		下面证明 $P\neq N_G(P)$。
		
		考虑 $P$ 在左陪集空间 $G/P$ 上的左平移作用:
		\[
		P\curvearrowright G/P,\qquad
		x\cdot(gP)=(xg)P\quad(x\in P).
		\]
		设该作用的不动点集合为
		\[
		X_0=\{gP\in G/P\mid pgP=gP,\ \forall\,p\in P\}.
		\]
		
		对任意 $gP\in G/P$,
		\[
		pgP=gP\ \forall\,p\in P
		\iff g^{-1}pg\in P\ \forall\,p\in P
		\iff g\in N_G(P),
		\]
		故
		\[
		X_0=\{gP\mid g\in N_G(P)\}=N_G(P)/P.
		\]
		于是
		\[
		|X_0|=\frac{|N_G(P)|}{|P|}.
		\]
		
		另一方面,由于 $P$ 是 $p$-群,其在有限集合 $G/P$ 上的作用满足
		“轨道大小非 $1$ 的均为 $p$ 的倍数”,从而
		\[
		|G/P|\equiv |X_0|\pmod p.
		\]
		而
		\[
		|G/P|=\frac{|G|}{|P|}=\frac{p^l m}{p^k}=p^{\,l-k}m\equiv 0\pmod p
		\quad(l>k),
		\]
		因此
		\[
		|X_0|\equiv 0\pmod p.
		\]
		结合 $|X_0|=\frac{|N_G(P)|}{|P|}\ge 1$,可得
		\[
		\frac{|N_G(P)|}{|P|}\ge p,
		\]
		即
		\[
		|N_G(P)|\ge p|P|>|P|.
		\]
		
		综上,
		\[
		P<N_G(P),
		\]
		即 $P$ 是其正规化子 $N_G(P)$ 的真子群。
		
		证毕。
		
		\textbf{证明.}
		设 $P$ 是 $G$ 的一个 $p^k$ 阶子群,其中 $k < l$。
		
		由 Sylow 定理,存在 $G$ 的一个 Sylow $p$-子群 $Q$,使得 $P \le Q$。  
		由于 $|Q| = p^l$ 且 $|P| = p^k$,有 $P < Q$。
		
		另一方面,因为 $Q$ 是 $p$-群,而 $P$ 是其子群,故对任意 $x \in Q$,  
		有 $xPx^{-1} \subseteq Q$,从而 $xPx^{-1} = P$,即 $x \in N_G(P)$。
		
		于是 $Q \le N_G(P)$。由于 $P < Q$,可知 $P < N_G(P)$。
		
		因此,$P$ 是其正规化子 $N_G(P)$ 的真子群。
		\[
		\boxed{P \lneq N_G(P).}
		\]
		\item 
		\textbf{题目.}\;
		证明:$56$ 阶群、$72$ 阶群都不是单群。
		
		\textbf{答案:}
		
		\textbf{(1) $|G|=56$ 的情形.}\;
		设 $|G|=56=2^{3}\cdot 7$,记 $n_{7}$ 为 Sylow $7$-子群个数。
		由 Sylow 定理
		\[
		n_{7}\mid 2^{3}=8,\qquad n_{7}\equiv 1\pmod 7,
		\]
		故 $n_{7}=1$ 或 $8$。
		
		\textbf{情形 A:$n_{7}=1$.}\;
		Sylow $7$-子群唯一,因而正规,$G$ 有非平凡正规子群,故 $G$ 不是单群。
		
		\textbf{情形 B:$n_{7}=8$.}\;
		每个 Sylow $7$-子群阶为 $7$,其非单位元有 $6$ 个。
		又因为两个不同的 $7$ 阶子群若交中含非单位元,则必相等(阶为素数的子群交要么 $\{e\}$ 要么全体),
		所以不同 Sylow $7$-子群的交只有单位元。
		因此 $G$ 中阶为 $7$ 的元素数为
		\[
		8\cdot 6=48.
		\]
		于是其余元素共有
		\[
		56-48=8
		\]
		个(其中包含单位元 $e$)。这些“其余元素”都不可能有因子 $7$ 的阶,
		因此它们的阶只能是 $2$ 的幂($1,2,4,8$)。
		
		令 $P$ 为某个 Sylow $2$-子群,则 $|P|=2^3=8$。
		注意到所有 $2$-幂阶元素都必须落在某个 Sylow $2$-子群中,
		而我们刚才数出 $G$ 中总共只有 $8$ 个 $2$-幂阶元素(连同 $e$)。
		这迫使 Sylow $2$-子群只能有一个:否则若存在两个不同的 Sylow $2$-子群 $P\neq Q$,
		则
		\[
		|P\cup Q|=|P|+|Q|-|P\cap Q|
		\ge 8+8-4=12,
		\]
		其中用到 $|P\cap Q|$ 是 $2$-群(因为交集为子群,而$P$和$Q$的阶都是$8$,所以只能是2的幂次,所以最大为4)且为 $P,Q$ 的真子群,所以 $|P\cap Q|\le 4$。
		但 $|P\cup Q|$ 至少为 $12$ 与“$2$-幂阶元素总数只有 $8$ 个”矛盾。
		故 Sylow $2$-子群唯一,从而正规。
		
		综上,$56$ 阶群必含非平凡正规子群,因此不可能是单群。
		
		\bigskip
		\textbf{(2) $|G|=72$ 的情形.}\;
		设 $|G|=72=2^{3}\cdot 3^{2}$,记 $n_{3}$ 为 Sylow $3$-子群个数。
		由 Sylow 定理
		\[
		n_{3}\mid 2^{3}=8,\qquad n_{3}\equiv 1\pmod 3,
		\]
		故 $n_{3}=1$ 或 $4$。
		
		\textbf{情形 A:$n_{3}=1$.}\;
		Sylow $3$-子群唯一,故正规,$G$ 不是单群。
		
		\textbf{情形 B:$n_{3}=4$.}\;
		令 $X$ 为所有 Sylow $3$-子群的集合,则 $|X|=4$。
		$G$ 通过共轭作用在 $X$ 上:
		\[
		g\cdot P:=gPg^{-1}\qquad (g\in G,\ P\in X),
		\]
		从而得到群同态
		\[
		\varphi:G\longrightarrow S_{X}\cong S_{4}.
		\]
		记核为 $\mathrm{ker}(\varphi)$,则 $\mathrm{ker}(\varphi)\lhd G$。
		
		若 $\mathrm{ker}(\varphi)=\{e\}$,则 $\varphi$ 为单射,于是
		\[
		|G|\mid |S_{4}|=24,
		\]
		但 $|G|=72$ 不可能整除 $24$,矛盾。
		因此 $\mathrm{ker}(\varphi)\neq \{e\}$,从而 $G$ 具有非平凡正规子群,故 $G$ 不是单群。
		(另外,若作用是平凡的,则每个 Sylow $3$-子群都被共轭固定,尤其有一个 Sylow $3$-子群正规,也同样推出 $G$ 非单。)
		
		\bigskip
		综上,$56$ 阶群与 $72$ 阶群都不是单群。\qed
		

		
		
		\item 
		\textbf{题目.}  
		设 $|G| = p^l m$,其中 $p$ 为素数,$l \ge 1$,且 $p > m > 1$。  
		证明:$G$ 不是单群。
		
		\textbf{证明.}  
		由 Sylow 定理,设 $n_p$ 表示 $G$ 的 Sylow $p$-子群的个数,则
		\[
		n_p \mid m \quad \text{且} \quad n_p \equiv 1 \pmod{p}.
		\]
		
		由于 $p > m$,所以 $m$ 的所有正因子都小于 $p$。
		而 $n_p$ 既要整除 $m$,又要满足 $n_p \equiv 1 \pmod{p}$。
		
		因为 $0 < n_p \le m < p$,故 $n_p \equiv 1 \pmod{p}$ 只能取 $n_p = 1$。  
		于是 $G$ 中的 Sylow $p$-子群唯一。
		
		由 Sylow 定理的另一个结论,唯一的 Sylow $p$-子群必为正规子群。  
		设该子群为 $P$,则 $P \trianglelefteq G$,且 $P \ne \{e\}$、$P \ne G$(因为 $m > 1$)。
		
		因此,$G$ 有非平凡正规子群 $P$,故 $G$ 不是单群。
		\[
		\boxed{G \text{ 不是单群.}}
		\]
		\item 
		\textbf{题目.} 设 $|G| = p^2 q$,其中 $p,q$ 为不同的素数。证明:$G$ 为可解群。
		
		\textbf{证明.}  
		由 Sylow 定理可得:
		
		设 $n_p$、$n_q$ 分别为 Sylow $p$-子群与 Sylow $q$-子群的个数。
		
		则
		\[
		n_p \mid q, \quad n_p \equiv 1 \pmod{p};
		\qquad
		n_q \mid p^2, \quad n_q \equiv 1 \pmod{q}.
		\]
		
		---
		
		\textbf{(1) 若 $n_p = 1$.}  
		则 Sylow $p$-子群正规,记 $P \triangleleft G$。  
		由 $|P|=p^2$,知 $P$ 为阿贝尔群。  
		于是 $G/P$ 的阶为 $q$,为循环群。  
		从而 $G$ 的正规列为
		\[
		\{e\} \triangleleft P \triangleleft G,
		\]
		其中各因子群皆阿贝尔,故 $G$ 可解。
		
		---
		
		\textbf{(2) 若 $n_p \ne 1$.}  
		则必有 $n_p = q$(因为 $n_p \mid q$ 且 $n_p \equiv 1 \pmod{p}$)。
		
		考虑 $n_q$:  
		由于 $n_q \mid p^2$ 且 $n_q \equiv 1 \pmod{q}$,  
		若 $q > p^2$,则 $n_q = 1$。  
		此时 Sylow $q$-子群正规,同理得 $G$ 可解。
		
		---
		
		\textbf{(3) 一般情形.}  
		若 $p,q$ 任意不同素数,则 $G$ 总存在正规 Sylow 子群(可证明至少有一个 $n_p=1$ 或 $n_q=1$)。  
		设该正规 Sylow 子群为 $N$,则 $G/N$ 的阶为素数幂或素数,故亦为可解群。
		
		于是 $G$ 的正规列可取为
		\[
		\{e\} \triangleleft N \triangleleft G,
		\]
		从而 $G$ 为可解群。
		
		---
		
		\textbf{结论.}  
		任意阶为 $p^2 q$($p,q$ 不同素数)的有限群均为可解群。
		\qed
		
		\item 
		\textbf{题目.}  
		设 $p$ 是有限群 $G$ 的阶的最小素因子,且 $H<G$,满足 $[G:H]=p$。  
		试证:$H \trianglelefteq G$。
		
		\textbf{证明.}  
		考虑 $G$ 在其左陪集集合 $X=\{gH\mid g\in G\}$ 上的左平移作用:
		\[
		g_1\cdot(g_2H)=g_1g_2H.
		\]
		这是一个群作用,其诱导的同态为
		\[
		\varphi:G\longrightarrow S_X\cong S_p.
		\]
		由于 $|X|=[G:H]=p$,故 $\varphi$ 是从 $G$ 到对称群 $S_p$ 的群同态。
		
		---
		
		记 $\ker\varphi=K$,则 $K=\bigcap_{g\in G}gHg^{-1}$,即 $K$ 是 $G$ 中 $H$ 的核(即交共轭)。  
		于是 $K\trianglelefteq G$ 且 $K\le H$,并由同态基本定理有
		\[
		|G:K|\ \big|\ |S_p|=p!.
		\]
		
		另一方面,由指数公式,
		\[
		|G:K| = |G:H|\cdot|H:K| = p\cdot |H:K|.
		\]
		因此 $p\cdot |H:K|$ 整除 $p!$。从而$ |H:K| | (p-1)!$,所以$ |H:K|$的所有素因子都小于$p$,而$ |H:K||H$,从而$ |H:K|| G$如果$ |H:K| \ne 1$,则说明$G$有小于$p$的素因子,矛盾。从而$ |H:K| = 1$,所以$H =K$,而$\ker \varphi = K \triangleleft G$,所以$H \triangleleft G$. \qed
		
		\item \textbf{题目.} (Frattini) 设 $N \triangleleft G$,$P$ 为 $N$ 的 Sylow $p$-子群,试证:
		\[
		G = N_G(P)N.
		\]
		
		\textbf{证明.}
		显然有
		\[
		N_G(P)N \subseteq G,
		\]
		只需证明
		\[
		G \subseteq N_G(P)N,
		\]
		即对任意 $g\in G$,都有 $g\in N_G(P)N$。
		
		\medskip
		
		\textbf{第 1 步:共轭保持 $N$,并将 $P$ 共轭为 $N$ 的 Sylow $p$-子群。}
		
		因 $N\triangleleft G$,对任意 $g\in G$,有
		\[
		gNg^{-1} = N.
		\]
		于是
		\[
		gPg^{-1} \subseteq gNg^{-1} = N.
		\]
		由于 $P$ 是 $N$ 的 Sylow $p$-子群,$gPg^{-1}$ 也是 $N$ 中的一个 Sylow $p$-子群。
		
		另一方面,$P$ 本身也是 $N$ 的 Sylow $p$-子群。于是 $P$ 与 $gPg^{-1}$ 都是 $N$ 的 Sylow $p$-子群。
		
		\medskip
		
		\textbf{第 2 步:在 $N$ 中应用 Sylow 定理。}
		
		对群 $N$ 应用 Sylow 定理,知 $N$ 中所有 Sylow $p$-子群在 $N$ 的共轭作用下互相共轭。
		因此存在某个 $n\in N$,使得
		\[
		gPg^{-1} = nPn^{-1}.
		\]
		
		\medskip
		
		\textbf{第 3 步:整理成正规化子的形式。}
		
		由
		\[
		gPg^{-1} = nPn^{-1}
		\]
		两边同左乘 $n^{-1}$、右乘 $n$,得
		\[
		n^{-1}gP g^{-1}n = P,
		\]
		即
		\[
		(n^{-1}g)P(n^{-1}g)^{-1} = P.
		\]
		这表明
		\[
		n^{-1}g \in N_G(P).
		\]
		
		记 $h := n^{-1}g$,则 $h\in N_G(P)$,且
		\[
		g = nh.
		\]
		由于 $n\in N$,于是
		\[
		g \in N N_G(P).
		\]
		
		\medskip
		
		\textbf{第 4 步:利用 $N\triangleleft G$ 得到乘积可交换。}
		
		一般地,若 $N\triangleleft G$,$H\le G$,则有
		\[
		HN = NH.
		\]
		事实上,对任意 $h\in H,n\in N$,
		\[
		hn = hnh^{-1}\,h,
		\]
		而因 $N\triangleleft G$,有 $hnh^{-1}\in N$,故 $hn\in NH$,从而 $HN\subseteq NH$;
		同理可得 $NH\subseteq HN$,故 $HN=NH$。
		
		在本题中取 $H=N_G(P)$,于是
		\[
		N N_G(P) = N_G(P)N.
		\]
		
		由上面的 $g\in N N_G(P)$ 可知
		\[
		g\in N_G(P)N.
		\]
		由于 $g\in G$ 是任意的,因此
		\[
		G \subseteq N_G(P)N.
		\]
		
		结合起初的 $N_G(P)N \subseteq G$,得到
		\[
		G = N_G(P)N.
		\] \qed
		
		
		\item 
		\textbf{题目.} 设 $P$ 为群 $G$ 的一个 Sylow 子群,证明:
		\[
		N_G(N_G(P)) = N_G(P).
		\]
		
		\textbf{证明.}  
		
		显然有
		\[
		N_G(P) \subseteq N_G(N_G(P)),
		\]		
		下面说明反向包含:  
		取 $x \in N_G(N_G(P))$,则
		\[
		x N_G(P) x^{-1} = N_G(P).
		\]
		因为$P \subseteq N_G(P)$,于是共轭作用下
		\[
		x P x^{-1} \subseteq xN_G(P)x^{-1}=N_G(P),
		\]
		又因为$P$为Sylow $p$-子群,则 $x P x^{-1}$ 是 $N_G(P)$ 的一个Sylow $p$-子群。又因为 $P \le N_G(P)$,所以 $P$ 也是 $N_G(P)$ 的 Sylow $p$-子群。
		
		根据 Sylow 定理,在 $N_G(P)$ 中所有 Sylow $p$-子群都是共轭的。  
		因此存在 $y \in N_G(P)$ 使得
		\[
		xPx^{-1} = yPy^{-1}.
		\]
		于是
		\[
		y^{-1}x \in N_G(P),
		\]
		从而 $x = y(y^{-1}x) \in N_G(P)$。		
		故 $x \in N_G(P)$,即
		\[
		N_G(N_G(P)) \subseteq N_G(P).
		\]
		两边结合即得
		\[
		N_G(N_G(P)) = N_G(P).
		\]
		\qed
		\item \textbf{题目.} 
		设 $P$ 是群 $G$ 的 Sylow $p$-子群,$H$ 是 $G$ 的子群且 $H \supseteq N_G(P)$。  
		证明:$N_G(H) = H$。
		
		\textbf{证明.}
		
		显然对任何子群 $H\le G$,都有
		\[
		H \subseteq N_G(H),
		\]
		因为对任意 $h\in H$,都有 $hHh^{-1} = H$。因此只需证明
		\[
		N_G(H) \subseteq H.
		\]
		
		取任意 $x\in N_G(H)$,则按正规化子的定义,
		\[
		xHx^{-1} = H.
		\]
		
		\textbf{第 1 步:$P$ 是 $H$ 的 Sylow $p$-子群。}
		
		已知 $H \supseteq N_G(P)$,而 $P\le N_G(P)$(因为任意 $p\in P$ 对 $P$ 共轭恒等),  
		故
		\[
		P \le N_G(P) \le H.
		\]
		于是 $P$ 是 $H$ 的一个 $p$-子群。
		
		设
		\[
		|G| = p^l m,\qquad p\nmid m.
		\]
		则 $|P| = p^l$。又因为 $H\le G$,所以 $|H|$ 整除 $|G|$,从而 $|H|$ 的 $p$-部分至多为 $p^l$。  
		但 $P\le H$ 且 $|P|=p^l$,说明 $|H|$ 的 $p$-部分恰为 $p^l$,因此 $P$ 在 $H$ 中也是阶为 $p^l$ 的 $p$-子群,  
		即 $P$ 是 $H$ 的 Sylow $p$-子群。
		
		\medskip
		
		\textbf{第 2 步:$xPx^{-1}$ 也是 $H$ 的 Sylow $p$-子群。}
		
		因为 $P\le H$,由 $x\in N_G(H)$ 得
		\[
		xPx^{-1} \le xHx^{-1} = H.
		\]
		同时共轭保持阶不变,故
		\[
		|xPx^{-1}| = |P| = p^l,
		\]
		从而 $xPx^{-1}$ 也是 $H$ 中一个阶为 $p^l$ 的 $p$-子群,即 $xPx^{-1}$ 亦为 $H$ 的 Sylow $p$-子群。
		
		\medskip
		
		\textbf{第 3 步:在 $H$ 内用 Sylow 定理做共轭。}
		
		在群 $H$ 中,$P$ 和 $xPx^{-1}$ 都是 Sylow $p$-子群。  
		由 Sylow 定理(对群 $H$ 应用),存在某个 $h\in H$,使得
		\[
		xPx^{-1} = hPh^{-1}.
		\]
		
		于是有
		\[
		h^{-1}x P x^{-1} h = P,
		\]
		这说明
		\[
		h^{-1}x \in N_G(P).
		\]
		又因为 $N_G(P)\subseteq H$,且 $h\in H$,故
		\[
		h^{-1}x \in H.
		\]
		设 $y := h^{-1}x$,则 $y\in H$,从而
		\[
		x = hy \in H,
		\]
		因为 $H$ 是子群,对乘积封闭。
		
		因此对任意 $x\in N_G(H)$,必有 $x\in H$,即
		\[
		N_G(H) \subseteq H.
		\]
		
		与显然的 $H\subseteq N_G(H)$ 合并,得到
		\[
		N_G(H) = H.
		\]\qed

		\item 
		\textbf{题目.}\;
		设 $p,q$ 都是素数,且 $p<q$ 并且 $p\nmid(q-1)$。证明:任意阶为 $pq$ 的群一定是循环群。
		
		\textbf{答案:}
		
		设 $|G|=pq$。
		
		\textbf{(1) Sylow $q$-子群唯一,从而正规.}\;
		由 Sylow 定理,$q$-Sylow 子群个数 $n_q$ 满足
		\[
		n_q\equiv 1\pmod q,\qquad n_q\mid p.
		\]
		由于 $p<q$,$n_q$ 只能是 $1$ 或 $p$;但 $p\not\equiv 1\pmod q$(因为 $0<p<q$),所以不可能 $n_q=p$,故
		\[
		n_q=1.
		\]
		因此 $G$ 有唯一的 $q$-Sylow 子群 $Q$,且 $Q\trianglelefteq G$。又因 $|Q|=q$ 为素数,所以
		\[
		Q\simeq C_q\ \text{是循环群。}
		\]
		
		\textbf{(2) 用共轭作用得到半直积,利用 $p\nmid(q-1)$ 排除非平凡作用.}\;
		取 $p$-Sylow 子群 $P$,则 $|P|=p$,从而 $P\simeq C_p$ 也是循环群。
		并且
		\[
		|PQ|=\frac{|P||Q|}{|P\cap Q|}=pq=|G|\quad(\text{因 }|P\cap Q|\mid p,q\Rightarrow |P\cap Q|=1),
		\]
		故 $G=PQ$。
		
		由于 $Q\trianglelefteq G$,对任意 $x\in P$,共轭 $xQx^{-1}=Q$,因此可定义共轭作用同态
		\[
		\varphi:P\longrightarrow \mathrm{Aut}(Q),\qquad
		\varphi(x)(y)=xyx^{-1}\ (y\in Q).
		\]
		因为 $Q\simeq C_q$,其自同构群满足
		\[
		\mathrm{Aut}(Q)\simeq \mathrm{Aut}(C_q)\simeq (\mathbb{Z}/q\mathbb{Z})^\times,
		\quad |\mathrm{Aut}(Q)|=q-1,
		\]
		所以 $\mathrm{Aut}(Q)$ 的阶是 $q-1$。
		
		注意 $\mathrm{Im}(\varphi)$ 是 $\mathrm{Aut}(Q)$ 的一个子群,且
		\[
		|\mathrm{Im}(\varphi)|\mid |P|=p.
		\]
		于是 $|\mathrm{Im}(\varphi)|$ 只能是 $1$ 或 $p$。
		若 $|\mathrm{Im}(\varphi)|=p$,则 $p\mid |\mathrm{Aut}(Q)|=q-1$,与题设 $p\nmid(q-1)$ 矛盾。
		因此只能有
		\[
		|\mathrm{Im}(\varphi)|=1,
		\]
		即 $\varphi$ 为平凡同态:对一切 $x\in P,y\in Q$,
		\[
		xyx^{-1}=y\quad\Longleftrightarrow\quad xy=yx.
		\]
		这说明 $P$ 与 $Q$ 中元素两两可交换,从而
		\[
		G=PQ\simeq P\times Q.
		\]
		
		\textbf{(3) $P\times Q$ 为循环群.}\;
		因为 $P\simeq C_p,\ Q\simeq C_q$ 且 $(p,q)=1$,取 $a$ 为 $P$ 的生成元,$b$ 为 $Q$ 的生成元,
		在直积里 $(a,b)$ 的阶为
		\[
		\mathrm{ord}(a,b)=\mathrm{lcm}(\mathrm{ord}(a),\mathrm{ord}(b))
		=\mathrm{lcm}(p,q)=pq.
		\]
		故 $P\times Q$ 由 $(a,b)$ 生成,是阶为 $pq$ 的循环群:
		\[
		P\times Q\simeq C_{pq}.
		\]
		于是 $G\simeq P\times Q\simeq C_{pq}$,从而 $G$ 必为循环群。
		
		因此命题成立。
		\textbf{题目.}  
		设 $p, q$ 都是素数,且 $p < q$,并且 $p \nmid (q - 1)$。  
		证明:任意阶为 $pq$ 的群一定是循环群。
		
		\textbf{证明.}
		
		\medskip
		
		\textbf{第 1 步:讨论 Sylow $q$-子群。}
		
		设 $n_q$ 为 $G$ 的 Sylow $q$-子群的个数。由 Sylow 定理有
		\[
		n_q \mid p,\qquad n_q \equiv 1 \pmod{q}.
		\]
		因为 $n_q$ 是正整数且 $n_q\mid p$,故 $n_q\in\{1,p\}$。  
		但 $p<q$,若 $n_q=p$,则 $n_q\equiv 1\pmod{q}$ 不可能成立(因为 $0<p<q$,$p\not\equiv 1\pmod{q}$)。  
		因此只能有
		\[
		n_q = 1.
		\]
		
		于是 $G$ 中阶为 $q$ 的 Sylow $q$-子群唯一,记为 $Q$,从而
		\[
		Q \triangleleft G.
		\]
		
		\medskip
		
		\textbf{第 2 步:取一个 Sylow $p$-子群,考虑半直积结构。}
		
		设 $P$ 是 $G$ 的一个 Sylow $p$-子群,则 $|P|=p$。由于 $Q\triangleleft G$ 且 $|PQ| = \frac{|P||Q|}{|P\cap Q|} = pq$,可知
		\[
		G = PQ,\quad P\cap Q = \{e\}.
		\]
		因此 $G$ 同构于一个半直积:
		\[
		G \cong Q \rtimes_\varphi P,
		\]
		其中
		\[
		\varphi: P \longrightarrow \mathrm{Aut}(Q)
		\]
		由共轭作用诱导。
		
		\medskip
		
		\textbf{第 3 步:利用 $p\nmid(q-1)$ 说明作用是平凡的。}
		
		注意到 $Q$ 阶为 $q$,是循环群,故
		\[
		Q \cong C_q,\qquad \mathrm{Aut}(Q) \cong \mathrm{Aut}(C_q) \cong C_{q-1},
		\]
		从而
		\[
		|\mathrm{Aut}(Q)| = q-1.
		\]
		
		同态 $\varphi : P \to \mathrm{Aut}(Q)$ 的像 $\varphi(P)$ 是一个群,其阶同时整除 $|P|=p$ 和 $|\mathrm{Aut}(Q)|=q-1$,因此
		\[
		|\varphi(P)| \mid p,\qquad |\varphi(P)| \mid (q-1).
		\]
		于是
		\[
		|\varphi(P)| \mid \gcd(p, q-1).
		\]
		由已知条件 $p\nmid(q-1)$,可知 $\gcd(p, q-1) = 1$,从而
		\[
		|\varphi(P)| = 1.
		\]
		这说明
		\[
		\varphi(P) = \{ \mathrm{id}\},
		\]
		即 $P$ 对 $Q$ 的共轭作用是平凡的:对任意 $x\in P$ 和任意 $y\in Q$,
		\[
		xyx^{-1} = y.
		\]
		因此 $P$ 与 $Q$ 中任意元素都两两交换,故
		\[
		PQ = G \text{ 为阿贝尔群}.
		\]
		
		\medskip
		
		\textbf{第 4 步:阶为 $pq$ 的有限阿贝尔群必为循环群。}
		
		我们已经证明 $G$ 为阿贝尔群,且
		\[
		|G| = pq,\quad p<q,\ p,q\ \text{素数}.
		\]
		根据有限阿贝尔群的结构理论,阶为 $pq$ 的阿贝尔群只能分解为素数阶循环群的直积:
		\[
		G \cong C_p \times C_q
		\]
		或与之同构的结构。
		
		又由于 $p,q$ 互素,$C_p\times C_q \cong C_{pq}$(因为互素阶循环群直积仍为循环群,阶为两者乘积)。  
		故 $G$ 同构于 $C_{pq}$,即 $G$ 为循环群。
		
		\medskip
		
		\textbf{结论.} 任意阶为 $pq$ 的群,在 $p<q$ 且 $p\nmid(q-1)$ 的条件下必为循环群。证毕。
		
		
		
		\item \textbf{题目.}  
		设 $H$ 是有限群 $G$ 的正规子群,$p$ 是 $|G|$ 的素因子且 $p\nmid [G:H]$,  
		试证:$H$ 包含 $G$ 的所有 Sylow $p$-子群。
		
		\textbf{证明.}  
		
		设 $P$ 为 $G$ 的一个 Sylow $p$-子群。  
		由 Lagrange 定理有
		\[
		|G| = |H| \cdot [G:H].
		\]
		由于 $p\nmid [G:H]$,可知 $p$ 的所有幂次都来自 $|H|$,即
		\[
		p^k \parallel |H|, \quad p^k = |P|.
		\]
		因此 $H$ 中存在阶为 $p^k$ 的子群 $Q$,即 $Q$ 是 $H$ 的 Sylow $p$-子群。
		
		---
		
		由于 $H\triangleleft G$,对任意 $g\in G$,有 $gHg^{-1}=H$,  
		从而对 $H$ 的 Sylow $p$-子群 $Q$ 也成立:
		\[
		gQg^{-1} \subseteq H.
		\]
		
		另一方面,$gQg^{-1}$ 是 $G$ 的一个 $p$-子群,且
		\[
		|gQg^{-1}| = |Q| = p^k = |P|.
		\]
		故 $gQg^{-1}$ 也是 $H$ 的 Sylow $p$-子群。  
		由 Sylow 定理,$H$ 的所有 Sylow $p$-子群两两共轭,  
		即存在 $h\in H$ 使得
		\[
		gQg^{-1} = hQh^{-1}.
		\]
		于是
		\[
		h^{-1}g \in N_G(Q).
		\]
		特别地,$h^{-1}g$ 的 $p$-部分属于 $Q$,因此
		\[
		P = gQg^{-1} = hQh^{-1} \subseteq H.
		\]
		
		---
		
		由于 $P$ 是任意 Sylow $p$-子群,结论成立:
		\[
		\boxed{H\text{ 包含 }G\text{ 的所有 Sylow }p\text{-子群}.}
		\]
		\qed
		
		\item \textbf{题目.}  
		设群 $G$ 的阶为 $p^l m$,其中 $p$ 为素数,$(p,m)=1$ 且 $m<2p$。  
		证明:$G$ 中存在正规 Sylow $p$-子群或正规 $p^{\,l-1}$ 阶子群。
		
		\textbf{证明.}  
		\item \textbf{题目.}  
		设群 $G$ 的阶为 $|G|=p^l m$,其中 $p$ 为素数,$(p,m)=1$ 且 $m<2p$。  
		证明:$G$ 中存在正规 Sylow $p$-子群或正规 $p^{\,l-1}$ 阶子群。
		
		\textbf{证明.}  
		设 $n_p$ 为 $G$ 的 Sylow $p$-子群的个数。由 Sylow 定理有
		\[
		n_p \mid m,\qquad n_p \equiv 1 \pmod{p}.
		\]
		又因为 $m<2p$ 且 $(m,p)=1$,故 $m$ 的所有因子都小于 $2p$,  
		结合 $n_p \equiv 1 \pmod{p}$,可知
		\[
		n_p = 1 \quad \text{或} \quad n_p = p+1.
		\]
		
		\textbf{(1) 若 $n_p = 1$,}  
		则 Sylow $p$-子群唯一,必为正规子群。此时命题成立。
		
		\medskip
		
		\textbf{(2) 若 $n_p = p+1$,}  
		我们证明存在一个正规 $p^{\,l-1}$ 阶子群。
		
		记 $\mathrm{Syl}_p(G)$ 为 $G$ 的所有 Sylow $p$-子群的集合,则
		\[
		|\mathrm{Syl}_p(G)| = p+1.
		\]
		
		令
		\[
		N := \bigcap_{P \in \mathrm{Syl}_p(G)} P.
		\]
		显然 $N$ 是一个 $p$-子群。又因为对任意 $g\in G$,共轭作用会将 Sylow $p$-子群置换到另一 Sylow $p$-子群,因此
		\[
		gNg^{-1} = \bigcap_{P\in \mathrm{Syl}_p(G)} gPg^{-1}
		\]
		仍为所有 Sylow $p$-子群的交,故等于 $N$。因而
		\[
		N\triangleleft G.
		\]
		
		下面估计 $|N|$ 的可能值。固定一个 Sylow $p$-子群 $P$,考虑 $P$ 对集合 $\mathrm{Syl}_p(G)$ 的共轭作用:
		\[
		P \times \mathrm{Syl}_p(G) \to \mathrm{Syl}_p(G),\qquad (x,Q)\mapsto xQx^{-1}.
		\]
		这给出群同态
		\[
		\varphi : P \longrightarrow \mathrm{Sym}(\mathrm{Syl}_p(G)) \cong S_{p+1}.
		\]
		设其核为
		\[
		K := \ker(\varphi)
		= \{x\in P \mid xQx^{-1} = Q,\ \forall Q\in \mathrm{Syl}_p(G)\}.
		\]
		
		如果 $x\in N$,则 $x\in Q$ 对所有 Sylow $p$-子群 $Q$ 都成立,于是 $xQx^{-1}=Q$,故 $x\in K$。因此
		\[
		N \le K \le P.
		\]
		
		另一方面,$P$ 是 $p$-群,因此像 $\varphi(P)$ 亦为 $p$-群。它是 $S_{p+1}$ 的一个 $p$-子群。  
		由于 $p+1<2p$,在 $S_{p+1}$ 中不可能出现两个不相交的 $p$-循环(否则至少移动 $2p>p+1$ 个点),且不可能出现长度为 $p^2$(或更长)的 $p^k$-循环,因此 $S_{p+1}$ 中的任意 $p$-子群阶至多为 $p$。故
		\[
		|\varphi(P)| \le p.
		\]
		于是
		\[
		[P:K] = |\varphi(P)| \le p,
		\]
		从而
		\[
		|K| = \frac{|P|}{[P:K]} \ge \frac{p^l}{p} = p^{\,l-1}.
		\]
		
		由于 $K\le P$ 且均为 $p$-群,因此 $|K|=p^a$,其中 $l-1\le a\le l$,即
		\[
		|K|\in \{p^{\,l-1},\, p^l\}.
		\]
		
		又因 $N\le K$ 且 $N$ 亦为 $p$-子群,故 $|N|=p^b$,并且必有 $b\le a$。但若 $b\le l-2$,则
		\[
		|P/N| = p^{l-b} \ge p^2,
		\]
		这将使得 $P/N$ 在作用于 $\mathrm{Syl}_p(G)$ 时出现阶至少为 $p^2$ 的 $p$-子群,与前面证明的“$S_{p+1}$ 中 $p$-子群阶至多为 $p$”矛盾。因此不可能有 $b\le l-2$。
		
		于是
		\[
		|N| = p^{\,l} \quad \text{或}\quad |N| = p^{\,l-1}.
		\]
		
		若 $|N|=p^l$,则 $N=P$,从而 $P$ 为正规 Sylow $p$-子群;  
		若 $|N|=p^{\,l-1}$,则 $N$ 为一个正规 $p^{\,l-1}$ 阶子群。
		
		\medskip
		
		综上,在 $n_p=p+1$ 的情形下,$G$ 中存在正规 Sylow $p$-子群或正规 $p^{\,l-1}$ 阶子群。
		
		\item 
		\textbf{题目.}  
		试证明在第 13 题中,去掉条件 $(p,m)=1$,结论仍然成立。
		
		\textbf{证明.}  
		
		第 13 题的结论是:
		\[
		N_G(N_G(P)) = N_G(P),
		\]
		其中 $P$ 为 $G$ 的一个 Sylow $p$-子群。  
		该结论只依赖于群论中正规化子与 Sylow 子群的基本性质,  
		与群的阶的分解中素因子的互素性 $(p,m)=1$ 无关。
		
		---
		
		\textbf{证明过程重述.}
		
		显然有
		\[
		N_G(P) \subseteq N_G(N_G(P)).
		\]
		
		若 $x \in N_G(N_G(P))$,则
		\[
		xN_G(P)x^{-1} = N_G(P).
		\]
		于是 $xPx^{-1}$ 是 $N_G(P)$ 的一个 $p$-子群。  
		由于 $P \le N_G(P)$,且 $P$ 是 $G$ 的 Sylow $p$-子群,  
		它也是 $N_G(P)$ 的 Sylow $p$-子群。
		
		由 Sylow 定理,在 $N_G(P)$ 中所有 Sylow $p$-子群互为共轭,  
		故存在 $y \in N_G(P)$,使得
		\[
		xPx^{-1} = yPy^{-1}.
		\]
		于是 $y^{-1}x \in N_G(P)$,从而 $x = y(y^{-1}x) \in N_G(P)$。  
		故
		\[
		N_G(N_G(P)) \subseteq N_G(P).
		\]
		结合显然的反向包含式,得
		\[
		N_G(N_G(P)) = N_G(P).
		\]
		
		---
		
		\textbf{结论.}  
		在整个推理过程中,从未使用 $(p,m)=1$ 的条件,  
		因此即便去掉该条件,结论仍然成立。
		\qed
		
		
		
		
		
	\end{enumerate}

		\chapter{环}
	\section{环的定义与基本性质}
	\subsection{思考题2.1.2}
	若一个环 $(R, +, \cdot)$ 对乘法也成为群,能得到什么结论?
	
	\textbf{证明.}
	设 $(R, +, \cdot)$ 是一个环,且对乘法也构成群。  
	则存在单位元 $e \in R$,并且对任意 $a \in R$,存在逆元 $a^{-1} \in R$,使得
	\[
	a a^{-1} = a^{-1} a = e.
	\]
	
	由于环中 $0$ 也是元素,故 $0$ 也必须有逆元。  
	设 $0^{-1} = b$,则由群的定义有
	\[
	0 \cdot b = e.
	\]
	然而,在任意环中都有 $0 \cdot b = 0$,因此
	\[
	e = 0.
	\]
	
	于是对任意 $a \in R$,有
	\[
	a = a \cdot e = a \cdot 0 = 0.
	\]
	因此 $R = \{0\}$。
	
	\textbf{结论.}
	若一个环对乘法也构成群,则该环必为零环。
	\subsection{$\mathbb Z[\sqrt m]$是整环}
	\textbf{命题.}\;
	证明:$\mathbb Z[\sqrt m]=\{a+b\sqrt m\mid a,b\in\mathbb Z\}$ 是整环。
	
	\textbf{证明.}\;
	
	我们证明 $\mathbb Z[\sqrt m]$ 在通常的加法与乘法下满足整环的定义,即:
	\emph{交换幺环且无零因子}。
	
	\textbf{(1) 交换幺环.}\;
	已知(或前已证):
	\begin{itemize}
		\item 加法与乘法在 $\mathbb Z[\sqrt m]$ 中封闭;
		\item 加法满足交换律、结合律,并有零元 $0=0+0\sqrt m$ 与加法逆元;
		\item 乘法满足交换律、结合律,并有幺元
		\[
		1=1+0\sqrt m\in\mathbb Z[\sqrt m].
		\]
	\end{itemize}
	因此 $\mathbb Z[\sqrt m]$ 是交换幺环。
	
	\textbf{(2) 无零因子.}\;
	设
	\[
	\alpha=a+b\sqrt m,\qquad \beta=c+d\sqrt m
	\quad (a,b,c,d\in\mathbb Z),
	\]
	且
	\[
	\alpha\beta=0.
	\]
	则在包含它的域 $\mathbb Q(\sqrt m)$ 中亦有
	\[
	(a+b\sqrt m)(c+d\sqrt m)=0.
	\]
	
	注意到 $\mathbb Q(\sqrt m)$ 是域,因此无零因子。
	于是必有
	\[
	a+b\sqrt m=0 \quad \text{或} \quad c+d\sqrt m=0.
	\]
	
	若 $a+b\sqrt m=0$,则
	\[
	b\sqrt m=-a.
	\]
	若 $b\neq 0$,则 $\sqrt m=-a/b\in\mathbb Q$,这意味着 $m$ 是完全平方数,
	与假设 $m$ 不是完全平方数矛盾。
	故 $b=0$,从而 $a=0$,即 $\alpha=0$。
	
	同理可得:若 $c+d\sqrt m=0$,则 $\beta=0$。
	
	因此
	\[
	\alpha\beta=0 \ \Longrightarrow\ \alpha=0 \text{ 或 } \beta=0,
	\]
	说明 $\mathbb Z[\sqrt m]$ 无零因子。
	
	\textbf{(3) 结论.}\;
	综上,$\mathbb Z[\sqrt m]$ 是交换幺环且无零因子,因此是整环。
	\hfill$\square$
	
	\subsection{相对范数/代数范数}
	设 $m\in\mathbb Z$ 且 $m$ 不是完全平方数($m \text{ 是完全平方数}\ \Longleftrightarrow\ \exists\,k\in\mathbb Z,\ m=k^2.$),令
	\[
	\mathbb Z[\sqrt m]=\{\,a+b\sqrt m\mid a,b\in\mathbb Z\,\}\subset \mathbb Q(\sqrt m).
	\]
	
	\textbf{一、范数的定义(由共轭给出).}
	
	\textbf{(1) 共轭映射.}\;
	定义映射(称为共轭)
	\[
	\overline{\phantom{\alpha}}:\mathbb Z[\sqrt m]\to\mathbb Z[\sqrt m],\qquad
	\overline{a+b\sqrt m}:=a-b\sqrt m.
	\]
	它是良定义的:因为当 $m$ 不是完全平方数时,表示 $a+b\sqrt m$ 的整数对 $(a,b)$ 是唯一的。
	
	并且对任意 $\alpha,\beta\in\mathbb Z[\sqrt m]$ 有
	\[
	\overline{\alpha+\beta}=\overline{\alpha}+\overline{\beta},\qquad
	\overline{\alpha\beta}=\overline{\alpha}\,\overline{\beta},\qquad
	\overline{\overline{\alpha}}=\alpha,\qquad \overline{1}=1.
	\]
	
	\textbf{(2) 范数(Norm).}\;
	对任意 $\alpha=a+b\sqrt m\in\mathbb Z[\sqrt m]$,定义其范数为
	\[
	N(\alpha):=\alpha\,\overline{\alpha}.
	\]
	于是
	\[
	N(a+b\sqrt m)=(a+b\sqrt m)(a-b\sqrt m)=a^2-mb^2\in\mathbb Z.
	\]
	
	\textbf{二、范数的基本性质与证明.}
	
	下列性质对任意 $\alpha,\beta\in\mathbb Z[\sqrt m]$ 均成立:
	
	\begin{itemize}
		\item \textbf{(P1) 乘法性:} $N(\alpha\beta)=N(\alpha)\,N(\beta)$。
		
		\item \textbf{(P2) 不变性:} $N(\overline{\alpha})=N(\alpha)$,且 $N(-\alpha)=N(\alpha)$。
		
		\item \textbf{(P3) 零判别:} $N(\alpha)=0 \iff \alpha=0$。
		
		\item \textbf{(P4) 整除传递:} 若 $\alpha\mid\beta$(在环 $\mathbb Z[\sqrt m]$ 中),则
		$N(\alpha)\mid N(\beta)$(在整数环 $\mathbb Z$ 中,允许差一个符号)。
		
		\item \textbf{(P5) 单位判别:} $\alpha$ 是 $\mathbb Z[\sqrt m]$ 的单位 $\iff N(\alpha)=\pm 1$。
	\end{itemize}
	
	\textbf{证明.}
	
	\textbf{(P1) 乘法性.}\;
	利用共轭的乘法保持性:
	\[
	N(\alpha\beta)=(\alpha\beta)\overline{\alpha\beta}
	=(\alpha\beta)\bigl(\overline{\alpha}\,\overline{\beta}\bigr)
	=\alpha\overline{\alpha}\cdot\beta\overline{\beta}
	=N(\alpha)\,N(\beta).\]
	\qed
	
	\textbf{(P2) 不变性.}\;
	首先
	\[
	N(\overline{\alpha})=\overline{\alpha}\,\overline{\overline{\alpha}}
	=\overline{\alpha}\,\alpha
	=\alpha\,\overline{\alpha}=N(\alpha).
	\]
	又因为 $\overline{-\alpha}=-\overline{\alpha}$,故
	\[
	N(-\alpha)=(-\alpha)\overline{(-\alpha)}=(-\alpha)(-\overline{\alpha})
	=\alpha\overline{\alpha}=N(\alpha).\]
	\qed
	
	\textbf{(P3) 零判别.}\;
	由于 $m$ 不是完全平方数,$\mathbb Q(\sqrt m)$ 是域,从而无零因子。
	而 $\mathbb Z[\sqrt m]\subset \mathbb Q(\sqrt m)$,也无零因子。
	若 $N(\alpha)=\alpha\overline{\alpha}=0$,则必有 $\alpha=0$ 或 $\overline{\alpha}=0$。
	但 $\overline{\alpha}=0 \iff \alpha=\overline{\overline{\alpha}}=0$,故 $N(\alpha)=0\iff\alpha=0$。
	\qed
	
	\textbf{(P4) 整除传递.}\;
	若 $\alpha\mid\beta$,则存在 $\gamma\in\mathbb Z[\sqrt m]$ 使 $\beta=\alpha\gamma$。
	取范数并用 (P1) 得
	\[
	N(\beta)=N(\alpha\gamma)=N(\alpha)\,N(\gamma).
	\]
	右端是整数乘积,因此 $N(\alpha)$ 整除 $N(\beta)$(在 $\mathbb Z$ 中,符号由 $N(\gamma)$ 决定)。
	\qed
	
	\textbf{(P5) 单位判别.}\;
	
	\textbf{必要性:}\;
	若 $\alpha$ 是单位,则存在 $\beta\in\mathbb Z[\sqrt m]$ 使 $\alpha\beta=1$。
	取范数并用 (P1) 得
	\[
	1=N(1)=N(\alpha\beta)=N(\alpha)\,N(\beta).
	\]
	其中 $N(\alpha),N(\beta)\in\mathbb Z$,故只能是 $N(\alpha)=\pm 1$。
	
	\textbf{充分性:}\;
	若 $N(\alpha)=\pm 1$,则
	\[
	\alpha\cdot\bigl(\pm\overline{\alpha}\bigr)=\pm\alpha\overline{\alpha}=\pm N(\alpha)=1.
	\]
	注意 $\pm\overline{\alpha}\in\mathbb Z[\sqrt m]$,因此 $\alpha$ 存在逆元 $\pm\overline{\alpha}$,即 $\alpha$ 为单位。
	\qed
	
	
	\subsection{命题2.1.9}
	设 $R$ 为无零因子环(整环),记 $R^* = R \setminus \{0\}$。
	则 $R^*$ 中的元素对于 $(R,+)$ 的加法阶都相同;
	并且当这一共同的阶是有限整数时,该整数必为素数。
	
	\textbf{证明.}
	对加法阶作如下约定:对 $a\in R$,若存在最小正整数 $n$ 使得
	$n a = 0$,则称 $n$ 为 $a$ 的(加法)阶;若不存在这样的 $n$,则称
	$a$ 的阶为无穷大。
	
	\medskip
	\noindent\textbf{(1) 若 $R^*$ 中所有元素的阶都是无穷大,则结论成立。}
	
	此时对任意 $a,b\in R^*$,它们的阶均为无穷大,因而显然“具有相同的阶”
	这一断言成立,且无需讨论素数性的问题。
	
	\medskip
	\noindent\textbf{(2) 设存在 $a\in R^*$ 的阶为有限正整数 $n$。证明所有
		$b\in R^*$ 的阶也为有限且等于 $n$。}
	
	由 $a$ 的阶为 $n$ 的定义,有
	\[
	na = \underbrace{a + a + \cdots + a}_{n\text{ 次}} = 0,
	\]
	且对任意 $1\le k < n$ 都有 $ka \neq 0$。
	
	令 $b$ 为任意一固定元素,$b\in R^*$。
	首先证明 $b$ 的阶有限。
	
	由分配律,有
	\[
	(na)b
	= (a + a + \cdots + a)b
	= ab + ab + \cdots + ab
	= n(ab).
	\]
	同理,
	\[
	a(nb)
	= a(b + b + \cdots + b)
	= ab + ab + \cdots + ab
	= n(ab).
	\]
	于是得到
	\[
	(na)b = a(nb) = n(ab).
	\]
	
	又因为 $na=0$,故
	\[
	(na)b = 0\cdot b = 0.
	\]
	因此
	\[
	a(nb) = 0.
	\]
	
	此时利用 $R$ 为无零因子环且 $a\neq 0$,由
	\[
	a(nb) = 0
	\]
	可知必有
	\[
	nb = 0.
	\]
	这说明每个 $b\in R^*$ 都被同一个正整数 $n$ 消去,
	从而 $b$ 的加法阶是有限的。设 $b$ 的阶为 $m$,
	即 $mb = 0$ 且对 $1\le k<m$ 有 $kb\neq0$。
	
	下面证明 $m=n$。
	
	因为 $nb=0$ 而 $m$ 是 $b$ 的最小正阶,
	由整除的基本性质(元素的阶整除一切“消去它”的正整数)可知
	\[
	m \mid n.
	\]
	即存在整数 $q$ 使 $n = qm$。
	
	另一方面,我们也可以对 $a$ 重复上述论证:
	既然存在 $b$ 的有限阶 $m$,上面的推理(只是将 $a$ 和 $b$ 的角色互换)
	说明对任意 $x\in R^*$ 都有 $mx=0$。特别地,对 $x=a$ 有
	\[
	ma = 0.
	\]
	而 $a$ 的阶为 $n$,故
	\[
	n \mid m.
	\]
	
	综上,$m\mid n$ 且 $n\mid m$,于是 $m = n$。
	因此 $R^*$ 中所有元素在加法下的阶都相同,且均为 $n$。
	
	\medskip
	\noindent\textbf{(3) 设 $R^*$ 中所有元素的阶都是同一个正整数 $n$,证明 $n$ 必为素数。}
	
	反设 $n$ 不是素数,则存在整数 $n_1,n_2$ 满足
	\[
	1 < n_1 < n,\quad 1 < n_2 < n,\quad n = n_1 n_2.
	\]
	
	取任意 $a \in R^*$。由假设,$a$ 的阶为 $n$,
	因此对 $1 \le k < n$ 必有 $ka \neq 0$。
	于是
	\[
	n_1 a \neq 0,\qquad n_2 a \neq 0.
	\]
	
	考虑它们的乘积:
	\[
	(n_1 a)(n_2 a)
	= n_1 n_2 a^2
	= n a^2.
	\]
	而 $a$ 的阶为 $n$,故 $na = 0$,于是
	\[
	n a^2 = (na)a = 0\cdot a = 0.
	\]
	
	这样我们得到:
	\[
	(n_1 a)(n_2 a) = 0,
	\]
	但 $n_1 a \neq 0$ 且 $n_2 a \neq 0$,
	这与 $R$ 为无零因子环的假设矛盾。
	
	故假设不成立,只能是 $n$ 为素数。
	
	\medskip
	综上,若存在一个非零元素的加法阶有限,则
	$R^*$ 中所有元素对加法的阶都相同;并且当这一共同阶有限时必为素数。
	若所有元素阶均为无穷大,则同样“具有相同的阶”。
	\qed
	
	
	\clearpage 
	\subsection*{课后习题答案}
	\addcontentsline{toc}{subsection}{\textcolor{red}{课后习题答案}}
	\begin{enumerate}[label=\textcolor{blue}{\textbf{\large\arabic*.}}]	
		\item \textbf{题目.}
		判断下列集合在指定的加法和乘法运算下是否构成环:
		\begin{enumerate}
			\item[(1)] $R=\{a+b\sqrt m\mid a,b\in\mathbb{Q}\}$,其中 $m$ 为整数,运算为数的普通加法与乘法;
			\item[(2)] $R=\mathbb{Z}$,加法为
			\[
			a\oplus b=a+b-1,\qquad a,b\in\mathbb{Z},
			\]
			乘法为
			\[
			a\otimes b=a+b-ab,\qquad a,b\in\mathbb{Z}.
			\]
		\end{enumerate}
		
		\textbf{解.}
		
		\medskip
		\noindent\textbf{(1)} 取
		\[
		R=\{a+b\sqrt m\mid a,b\in\mathbb{Q}\}\subset\mathbb{C},
		\]
		其加、乘运算为通常的复数加法和乘法。
		
		对任意 $a+b\sqrt m,c+d\sqrt m\in R$,有
		\[
		(a+b\sqrt m)+(c+d\sqrt m)=(a+c)+(b+d)\sqrt m\in R,
		\]
		\[
		(a+b\sqrt m)(c+d\sqrt m)
		=(ac+mbd)+(ad+bc)\sqrt m\in R,
		\]
		故对加法和乘法均封闭。加法、乘法的结合律、交换律以及乘法对加法的分配律
		从 $\mathbb{C}$ 中继承而来。零元为
		\[
		0_R=0+0\sqrt m,\qquad
		\text{加法逆元为 } -(a+b\sqrt m)=(-a)+(-b)\sqrt m,
		\]
		乘法单位为
		\[
		1_R=1+0\sqrt m.
		\]
		因此 $R$ 是一个带单位的交换环(事实上是 $\mathbb{C}$ 的一个子环)。
		
		\medskip
		\noindent\textbf{(2)} 取 $R=\mathbb{Z}$,定义
		\[
		a\oplus b=a+b-1,\qquad a\otimes b=a+b-ab.
		\]
		
		\textit{(i) $(\mathbb{Z},\oplus)$ 为交换群。}
		对任意 $a,b,c\in\mathbb{Z}$,
		\[
		(a\oplus b)\oplus c=(a+b-1)\oplus c
		=(a+b-1)+c-1=a+b+c-2,
		\]
		\[
		a\oplus(b\oplus c)=a\oplus(b+c-1)
		=a+(b+c-1)-1=a+b+c-2,
		\]
		故加法满足结合律;且
		\[
		a\oplus b=a+b-1=b+a-1=b\oplus a,
		\]
		故交换律成立。求加法单位元 $e$:
		\[
		a\oplus e=a \iff a+e-1=a \iff e=1,
		\]
		且
		\[
		e\oplus a=e+a-1=a \iff e=1,
		\]
		故加法单位元为 $1$。对任意 $a\in\mathbb{Z}$,加法逆元 $a'$ 满足
		\[
		a\oplus a'=1 \iff a+a'-1=1 \iff a'=2-a.
		\]
		可见每个元素都有逆元,故 $(\mathbb{Z},\oplus)$ 是交换群。
		
		\textit{(ii) 乘法 $\otimes$ 的性质。}
		对任意 $a,b,c\in\mathbb{Z}$,
		\[
		(a\otimes b)\otimes c
		=(a+b-ab)\otimes c
		=(a+b-ab)+c-(a+b-ab)c
		\]
		\[
		=a+b-ab+c-ac-bc+abc,
		\]
		\[
		a\otimes(b\otimes c)
		=a\otimes(b+c-bc)
		=a+(b+c-bc)-a(b+c-bc)
		\]
		\[
		=a+b+c-bc-ab-ac+abc,
		\]
		两式相同,因此 $\otimes$ 满足结合律。又
		\[
		a\otimes b=a+b-ab=b+a-ba=b\otimes a,
		\]
		故乘法交换。
		
		\textit{(iii) 分配律.}
		对任意 $a,b,c\in\mathbb{Z}$,
		\[
		(a\oplus b)\otimes c
		=(a+b-1)\otimes c
		=(a+b-1)+c-(a+b-1)c
		\]
		\[
		=a+b-1+c-ac-bc+c
		=a+b+2c-1-ac-bc,
		\]
		而
		\[
		(a\otimes c)\oplus(b\otimes c)
		=(a+c-ac)\oplus(b+c-bc)
		\]
		\[
		=(a+c-ac)+(b+c-bc)-1
		=a+b+2c-1-ac-bc,
		\]
		故有
		\[
		(a\oplus b)\otimes c=(a\otimes c)\oplus(b\otimes c).
		\]
		由于 $\otimes$ 交换,右分配律同理成立:
		
		\[
		c\otimes(a\oplus b)=(c\otimes a)\oplus(c\otimes b).
		\]
		
		\textit{(iv) 关于乘法单位元.}
		若存在乘法单位元 $e$,则应满足
		\[
		a\otimes e=a,\quad \forall a\in\mathbb{Z},
		\]
		即
		\[
		a+e-ae=a \iff e-ae=0 \iff e(1-a)=0,\quad \forall a\in\mathbb{Z}.
		\]
		在整环 $\mathbb{Z}$ 中不可能有非零 $e$ 同时满足对所有 $a$ 都有 $e(1-a)=0$,
		因此不存在这样的公共 $e$。故 $(\mathbb{Z},\oplus,\otimes)$ 虽是环,但不是带单位的环。
		
		\medskip
		\noindent\textbf{结论.}
		\begin{enumerate}
			\item[(1)] $R=\{a+b\sqrt m\mid a,b\in\mathbb{Q}\}$ 在通常加法和乘法下是
			一个带单位的交换环;
			\item[(2)] $(\mathbb{Z},\oplus,\otimes)$ 满足环公理,是一个交换环,但没有乘法单位元。
		\end{enumerate}
		
		\item \textbf{题目 2.} 设 $R$ 为环,且 $|R|$ 为素数,证明 $R$ 是交换环。
		
		\textbf{证明.}
		
		分两种情形讨论。
		
		\textbf{(1) $|R|=2$.}
		此时 $R=\{0,1\}$。由于
		\[
		0\cdot a=0=a\cdot 0\qquad(\forall a\in R),
		\]
		显然乘法交换,故 $R$ 为交换环。
		
		\medskip
		
		\textbf{(2) $|R|=p\ge 3$,其中 $p$ 为素数.}
		考虑加法群 $(R,+)$。由于 $|R|=p$ 为素数,
		$(R,+)$ 必为循环群。取任意非零元 $a\in R$,则
		\[
		R=\langle a\rangle=\{0,a,2a,\dots,(p-1)a\},
		\]
		且
		\[
		pa=0.
		\]
		
		任取 $r_1,r_2\in R$,则存在 $s,t\in\mathbb{Z}$,
		使得
		\[
		r_1=sa,\qquad r_2=ta.
		\]
		于是
		\[
		r_1r_2=(sa)(ta)=st\,a^2=(ta)(sa)=r_2r_1.
		\]
		因此 $r_1r_2=r_2r_1$ 对任意 $r_1,r_2\in R$ 成立,
		从而 $R$ 的乘法交换。
		
		\medskip
		
		综上,$R$ 为交换环。 \qed
		
		\item 4.
		\textbf{题目.}
		试举例说明,存在幺环 $R$ 及元素 $a\in R$ 使得 $a$ 有无穷多个右逆元。
		
		\textbf{解:}
		给出一个经典例子:取任意域 $F$,令
		\[
		V=\bigoplus_{n\ge 1} F e_n
		\]
		为以可数基 $\{e_1,e_2,e_3,\dots\}$ 生成的(无限维)$F$-向量空间。令
		\[
		R=\mathrm{End}_F(V)
		\]
		为 $V$ 上所有 $F$-线性变换组成的环,则 $R$ 是幺环,幺元为恒等映射 $\mathrm{Id}_V$。
		
		\medskip
		\textbf{第一步:构造 $a\in R$。}
		定义线性算子 $a:V\to V$(“左移算子”)为
		\[
		a(e_1)=0,\qquad a(e_{n+1})=e_n\ (n\ge 1),
		\]
		并按线性延拓到全体 $V$。显然 $a\in R$。
		
		\medskip
		\textbf{第二步:构造一族右逆 $b_t$。}
		对每个标量 $t\in F$,定义线性算子 $b_t:V\to V$ 为
		\[
		b_t(e_1)=e_2+t e_1,\qquad b_t(e_n)=e_{n+1}\ (n\ge 2),
		\]
		并按线性延拓到全体 $V$。于是 $b_t\in R$。
		
		下面验证 $b_t$ 是 $a$ 的右逆,即
		\[
		a\,b_t=\mathrm{Id}_V.
		\]
		只需在基向量上检验即可:
		
		\medskip
		(1) 对 $e_1$:
		\[
		(a b_t)(e_1)=a\bigl(b_t(e_1)\bigr)=a(e_2+t e_1)=a(e_2)+t a(e_1)=e_1+t\cdot 0=e_1.
		\]
		
		\medskip
		(2) 对 $e_n\ (n\ge 2)$:
		\[
		(a b_t)(e_n)=a\bigl(b_t(e_n)\bigr)=a(e_{n+1})=e_n.
		\]
		
		因此对所有 $n\ge 1$ 都有 $(a b_t)(e_n)=e_n$,由线性性推出对一切 $v\in V$,
		\[
		(a b_t)(v)=v,
		\]
		从而
		\[
		a b_t=\mathrm{Id}_V.
		\]
		这说明每个 $b_t$ 都是 $a$ 的右逆元。
		
		\medskip
		\textbf{第三步:证明这些右逆互不相同,从而有无穷多个。}
		若 $s\neq t$,比较它们在 $e_1$ 上的取值:
		\[
		b_t(e_1)=e_2+t e_1,\qquad b_s(e_1)=e_2+s e_1.
		\]
		若 $b_t=b_s$,则应有 $b_t(e_1)=b_s(e_1)$,即
		\[
		e_2+t e_1=e_2+s e_1 \ \Longrightarrow\ (t-s)e_1=0.
		\]
		由于 $e_1\neq 0$ 且 $t-s\neq 0$,这不可能,因此 $b_t\neq b_s$。
		
		\medskip
		综上,在幺环 $R=\mathrm{End}_F(V)$ 中,元素 $a$ 具有一族两两不同的右逆元 $\{b_t:t\in F\}$。
		特别地,只要 $F$ 是无限域(例如 $F=\mathbb{Q},\mathbb{R},\mathbb{C}$),则 $a$ 有无穷多个右逆元。
		
		\medskip
		\textbf{补充说明(更一般的结构原因).}
		上面的构造本质上利用了 $\ker(a)\neq 0$:一旦 $a b=\mathrm{Id}$,则对任意线性映射
		$h\in \mathrm{End}_F(V)$ 满足 $\mathrm{Im}(h)\subseteq \ker(a)$,都有
		\[
		a(b+h)=ab+ah=\mathrm{Id}+0=\mathrm{Id},
		\]
		从而 $b+h$ 仍是右逆;而这样的 $h$ 可以取无穷多个,于是右逆也无穷多个。
		
		\item 5.
		\textbf{题目.}
		设 $R$ 为幺环,$e$ 为幺元,$a\in R$。若 $\exists\, m\in\mathbb N$ 使得 $a^m=0$,则称 $a$ 为幂零元。
		证明:若 $a$ 为幂零元,则 $e+a$ 为可逆元。
		
		\textbf{证明.}
		已知存在 $m\in\mathbb N$ 使得 $a^m=0$。考虑元素
		\[
		b \;:=\; e-a+a^2-\cdots+(-1)^{m-1}a^{m-1}
		\;=\;\sum_{k=0}^{m-1}(-1)^k a^k\in R.
		\]
		我们将证明 $b$ 同时是 $e+a$ 的左逆与右逆,从而 $e+a$ 可逆且逆元为 $b$。
		
		\medskip
		\textbf{(1) 计算 $(e+a)b$.}
		由分配律,
		\[
		(e+a)b = eb+ab=b+ab.
		\]
		先写出 $b$ 与 $ab$:
		\[
		b = e-a+a^2-\cdots+(-1)^{m-1}a^{m-1},
		\]
		\[
		ab = a-a^2+a^3-\cdots+(-1)^{m-1}a^m.
		\]
		两式相加得到(逐项相消):
		\[
		b+ab
		= \bigl(e-a+a^2-\cdots+(-1)^{m-1}a^{m-1}\bigr)
		+ \bigl(a-a^2+a^3-\cdots+(-1)^{m-1}a^m\bigr)
		= e + (-1)^{m-1}a^m.
		\]
		由于 $a^m=0$,故
		\[
		(e+a)b = e + (-1)^{m-1}a^m = e.
		\]
		因此 $b$ 是 $e+a$ 的\emph{右逆}。
		
		\medskip
		\textbf{(2) 计算 $b(e+a)$.}
		同理,
		\[
		b(e+a)=be+ba=b+ba.
		\]
		注意 $a$ 与其幂 $a^k$ 总是可交换(因为 $a\cdot a^k=a^{k+1}=a^k\cdot a$),
		故对每个 $k$ 都有 $a^k a = a a^k = a^{k+1}$,从而
		\[
		ba
		=\left(\sum_{k=0}^{m-1}(-1)^k a^k\right)a
		=\sum_{k=0}^{m-1}(-1)^k a^{k+1}
		= a-a^2+a^3-\cdots+(-1)^{m-1}a^m.
		\]
		于是与(1)完全相同的相消给出
		\[
		b+ba = e + (-1)^{m-1}a^m = e,
		\]
		即
		\[
		b(e+a)=e.
		\]
		因此 $b$ 也是 $e+a$ 的\emph{左逆}。
		
		\medskip
		综上,$b$ 同时满足 $(e+a)b=e$ 与 $b(e+a)=e$,所以 $e+a$ 是可逆元,且
		\[
		(e+a)^{-1}=e-a+a^2-\cdots+(-1)^{m-1}a^{m-1}.
		\]
		\qed
		
		\item 7.
		\textbf{题目.}
		设 $R$ 为环,$a\in R$,若 $a\neq 0$ 且 $a^2=a$,则称 $a$ 为幂等元。证明:
		\begin{enumerate}
			\item[(1)] 若环 $R$ 的所有非零元素都是幂等元,则 $R$ 必为交换环;
			\item[(2)] 若 $R$ 为无零因子环,且存在幂等元,则 $R$ 只有唯一的幂等元,并且 $R$ 为幺环。
		\end{enumerate}
		\textbf{解:}
		\begin{enumerate}
			% ============================================================
			\item[(1)]
			\textbf{证明.}
			题设给出:对任意 $x\in R$,若 $x\neq 0$ 则 $x^2=x$。另外 $0^2=0$,因此事实上
			\[
			\forall x\in R,\quad x^2=x.
			\]
			下面由此推出 $R$ 交换。
			
			\medskip
			\textbf{第一步:证明 $2x=0$(即 $x+x=0$)对任意 $x\in R$ 成立。}
			
			由任意元素幂等,取 $x+x$ 也有
			\[
			(x+x)^2=x+x.
			\]
			利用分配律展开(此处不需要交换律):
			\[
			(x+x)^2=(x+x)(x+x)=xx+xx+xx+xx.
			\]
			又因 $xx=x^2=x$,上式化为
			\[
			(x+x)^2 = x+x+x+x.
			\]
			于是
			\[
			x+x+x+x = x+x.
			\]
			将右边移到左边(在加法群中可消去)得
			\[
			(x+x+x+x)-(x+x)=0 \quad\Longrightarrow\quad x+x=0.
			\]
			因此对任意 $x\in R$ 都有 $2x=0$。
			
			\medskip
			\textbf{第二步:证明任意 $x,y\in R$ 都满足 $xy=yx$。}
			
			由幂等性,$(x+y)^2=x+y$。同样用分配律展开:
			\[
			(x+y)^2=(x+y)(x+y)=x^2+xy+yx+y^2.
			\]
			又 $x^2=x,\ y^2=y$,代入得到
			\[
			x+xy+yx+y = x+y.
			\]
			在加法群中消去两边的 $x+y$,可得
			\[
			xy+yx=0.
			\]
			而由第一步知 $2(xy)=0$,即 $xy=-xy$。于是
			\[
			yx=-xy=xy.
			\]
			故对任意 $x,y\in R$ 都有 $xy=yx$,因此 $R$ 为交换环。
			
			% ============================================================
			\item[(2)]
			$\text{设 } a \text{ 为 } R \text{ 的幂等元,从而对任意 } b\in R,\text{设 } ab=c,\text{则有 } a^{2}b=ac,\text{即 } ab=ac, \text{从而 } a(b-c)=0.\text{则由 } R \text{ 为无零因子环,且 } a\neq 0,\text{故 } b=c, \text{从而 } ab=b, \text{同理有 } ba=b,\text{由 } b \text{ 的任意性可知 } a \text{ 为幺元。}$
			$\text{进而若 } r \text{ 为幂等元,从而 } r^{2}=r=ra, \text{即 } r(r-a)=0, \text{即 } r=a.\text{综上 } R \text{ 为幺环,且有唯一幂等元。}$
			
		\end{enumerate}
		
		
		\item 9.
		
		\textbf{题目.}\;
		设 $R$ 是一个无零因子环(即 $ab=0 \Rightarrow a=0$ 或 $b=0$),$e\in R$ 是关于乘法的左(或右)幺元,证明:$e$ 必为 $R$ 的幺元(即两侧幺元)。
		
		\textbf{答案:}\;
		\textbf{证明.}
		只证“左幺元 $\Rightarrow$ 两侧幺元”,另一种情形同理。
		
		设 $e$ 是左幺元,即对任意 $x\in R$ 有 $ex=x$。
		任取 $a\in R$。
		
		\textbf{(1) 若 $a\neq 0$,则 $ae=a$.}
		
		由左幺元性质知 $ea=a$。两边同左乘 $a$ 得
		\[
		a(ea)=a\cdot a=a^2.
		\]
		又由结合律,
		\[
		a(ea)=(ae)a.
		\]
		于是
		\[
		(ae)a=a^2 \quad\Longrightarrow\quad (ae-a)a=0.
		\]
		因 $a\neq 0$ 且 $R$ 无零因子,故推出
		\[
		ae-a=0 \quad\Longrightarrow\quad ae=a.
		\]
		
		\textbf{(2) 若 $a=0$,显然 $ae=0=a$.}
		
		综上,对任意 $a\in R$ 都有 $ae=a$,故 $e$ 同时是右幺元,从而 $e$ 是 $R$ 的幺元。
		
		
		
		
		
		
		\item 12.
		\textbf{题目.}
		设 $R_1,R_2$ 为两个环,在直积集合 $R_1* R_2$ 中定义
		\[
		(a_1,b_1)+(a_2,b_2)=(a_1+a_2,\;b_1+b_2),\qquad
		(a_1,b_1)(a_2,b_2)=(a_1a_2,\;b_1b_2).
		\]
		证明在上述运算下 $R_1* R_2$ 成为一个环,称为环 $R_1,R_2$ 的外直和(亦称直积环),记为 $R_1+R_2$。
		并证明:
		(1) 若 $R_1,R_2$ 都是幺环,则 $R_1+R_2$ 也是幺环;
		(2) 若 $R_1,R_2$ 是交换环,则 $R_1+R_2$ 也是交换环;
		(3) 若 $R_1,R_2$ 都是无零因子环,问 $R_1+R_2$ 是否一定也是无零因子环?
		
		\textbf{解:}
		
		\textbf{证明.}
		
		\textbf{一、$R_1* R_2$ 在逐分量运算下是一个环}
		
		记 $R:=R_1* R_2$,加法与乘法按题设逐分量定义。
		
		\textbf{(1) $(R,+)$ 是交换群.}
		
		\textbf{闭性:}任取 $(a_1,b_1),(a_2,b_2)\in R$,
		\[
		(a_1,b_1)+(a_2,b_2)=(a_1+a_2,\;b_1+b_2)\in R
		\]
		因为 $a_1+a_2\in R_1,\; b_1+b_2\in R_2$。
		
		\textbf{结合律:}任取 $(a_i,b_i)\in R$($i=1,2,3$),则
		\[
		\bigl((a_1,b_1)+(a_2,b_2)\bigr)+(a_3,b_3)
		=((a_1+a_2)+a_3,\;(b_1+b_2)+b_3),
		\]
		\[
		(a_1,b_1)+\bigl((a_2,b_2)+(a_3,b_3)\bigr)
		=(a_1+(a_2+a_3),\;b_1+(b_2+b_3)).
		\]
		由于 $R_1,R_2$ 中加法满足结合律,故两式相等,从而 $R$ 中加法结合。
		
		\textbf{加法单位元:}令 $0_1,0_2$ 分别为 $R_1,R_2$ 的零元,则
		\[
		(0_1,0_2)+(a,b)=(0_1+a,\;0_2+b)=(a,b),
		\]
		\[
		(a,b)+(0_1,0_2)=(a+0_1,\;b+0_2)=(a,b),
		\]
		故 $(0_1,0_2)$ 是 $R$ 的加法单位元(零元)。
		
		\textbf{加法逆元:}对任意 $(a,b)\in R$,取 $(-a,-b)$(分别为 $R_1,R_2$ 中的加法逆元),则
		\[
		(a,b)+(-a,-b)=(a-a,\;b-b)=(0_1,0_2).
		\]
		
		\textbf{交换律:}任取 $(a_1,b_1),(a_2,b_2)\in R$,
		\[
		(a_1,b_1)+(a_2,b_2)=(a_1+a_2,\;b_1+b_2)=(a_2+a_1,\;b_2+b_1)=(a_2,b_2)+(a_1,b_1),
		\]
		因为 $R_1,R_2$ 的加法交换。
		
		综上 $(R,+)$ 为交换群。
		
		\medskip
		\textbf{(2) 乘法满足结合律.}
		任取 $(a_i,b_i)\in R$($i=1,2,3$),则
		\[
		\bigl((a_1,b_1)(a_2,b_2)\bigr)(a_3,b_3)
		=(a_1a_2,\;b_1b_2)(a_3,b_3)=((a_1a_2)a_3,\;(b_1b_2)b_3),
		\]
		\[
		(a_1,b_1)\bigl((a_2,b_2)(a_3,b_3)\bigr)
		=(a_1,b_1)(a_2a_3,\;b_2b_3)=(a_1(a_2a_3),\;b_1(b_2b_3)).
		\]
		由于 $R_1,R_2$ 的乘法结合,故两式相等,从而 $R$ 的乘法结合。
		
		\medskip
		\textbf{(3) 分配律成立.}
		任取 $(a_1,b_1),(a_2,b_2),(a_3,b_3)\in R$,
		\[
		(a_1,b_1)\bigl((a_2,b_2)+(a_3,b_3)\bigr)
		=(a_1,b_1)(a_2+a_3,\;b_2+b_3)
		=(a_1(a_2+a_3),\;b_1(b_2+b_3)).
		\]
		在 $R_1,R_2$ 中分别用左分配律:
		\[
		a_1(a_2+a_3)=a_1a_2+a_1a_3,\qquad b_1(b_2+b_3)=b_1b_2+b_1b_3,
		\]
		故
		\[
		(a_1,b_1)\bigl((a_2,b_2)+(a_3,b_3)\bigr)
		=(a_1a_2+a_1a_3,\;b_1b_2+b_1b_3)
		=(a_1,b_1)(a_2,b_2)+(a_1,b_1)(a_3,b_3).
		\]
		右分配律同理可证。
		
		\medskip
		由(1)(2)(3)知 $R_1* R_2$ 在上述运算下满足环公理,因此是一个环。
		通常称其为 $R_1$ 与 $R_2$ 的直积环;题中记作外直和 $R_1+R_2$。
		
		\textbf{二、若 $R_1,R_2$ 为幺环,则 $R_1+R_2$ 为幺环}
		
		设 $e_1,e_2$ 分别为 $R_1,R_2$ 的幺元。令
		\[
		e:=(e_1,e_2)\in R_1* R_2.
		\]
		任取 $(a,b)\in R_1* R_2$,则
		\[
		e(a,b)=(e_1,e_2)(a,b)=(e_1a,\;e_2b)=(a,b),
		\]
		\[
		(a,b)e=(a,b)(e_1,e_2)=(ae_1,\;be_2)=(a,b),
		\]
		故 $(e_1,e_2)$ 为 $R_1+R_2$ 的幺元,从而 $R_1+R_2$ 是幺环。
		
		\textbf{三、若 $R_1,R_2$ 为交换环,则 $R_1+R_2$ 为交换环}
		
		若 $R_1,R_2$ 的乘法交换,则任取 $(a_1,b_1),(a_2,b_2)\in R_1* R_2$,
		\[
		(a_1,b_1)(a_2,b_2)=(a_1a_2,\;b_1b_2)=(a_2a_1,\;b_2b_1)=(a_2,b_2)(a_1,b_1),
		\]
		故 $R_1+R_2$ 为交换环。
		
		\textbf{四、若 $R_1,R_2$ 均为无零因子环,$R_1+R_2$ 是否一定无零因子?}
		
		解:\textbf{不一定}。
		
		\textbf{反例:}取任意两个非零环(例如 $R_1=\mathbb Z,\ R_2=\mathbb Z$,它们都是无零因子环)。
		在 $R_1* R_2$ 中考虑两个非零元
		\[
		x:=(1,0),\qquad y:=(0,1).
		\]
		显然 $x\neq (0,0)$ 且 $y\neq(0,0)$,但
		\[
		xy=(1,0)(0,1)=(1\cdot 0,\;0\cdot 1)=(0,0).
		\]
		因此在 $R_1+R_2$ 中存在非零元相乘得到零元,出现零因子,
		故 $R_1+R_2$ \emph{不是} 无零因子环。
		
		\medskip
		进一步说明:只要 $R_1,R_2$ 都不是零环,就总会有 $(a,0)\neq 0$ 与 $(0,b)\neq 0$(其中 $a\neq 0,b\neq 0$)满足
		\[
		(a,0)(0,b)=(0,0),
		\]
		所以 $R_1* R_2$ 只有在其中一个因子是零环时才可能无零因子。
		
		\hfill $\Box$
		
		
	\end{enumerate}
	
	
	
	\clearpage
	\section{理想与商环}
	\subsection{整数坏的任何非零子坏一定形如$m\mathbb{Z}$}
	\textbf{命题.}
	整数环 $\mathbb{Z}$ 的任一非零子环 $R$ 必为某个正整数 $m$ 的倍数集合:
	\[
	R = m\mathbb{Z} = \{mn \mid n \in \mathbb{Z}\}.
	\]
	
	\textbf{证明.}
	
	设 $R$ 是 $\mathbb{Z}$ 的一个非零子环。因为 $R \neq \{0\}$,存在非零整数 $a \in R$。
	
	首先注意到 $R$ 关于加法封闭且含有加法逆元,因此
	\[
	a \in R \;\Rightarrow\; -a \in R.
	\]
	从而 $R$ 中必定含有正整数。定义
	\[
	m := \min\{\, n \in R \mid n > 0 \,\}.
	\]
	由于 $R \subseteq \mathbb{Z}$ 且上述集合非空,$m$ 存在且为正整数。
	
	\medskip
	\noindent\textbf{(1) 证明 $m\mathbb{Z} \subseteq R$.}
	
	因为 $m \in R$ 且 $R$ 对加法封闭,所以对任意整数 $n$,
	\[
	mn = \underbrace{m + \cdots + m}_{n\text{ 次}} \in R
	\quad (\text{若 } n>0),
	\]
	\[
	mn = -(m|n|) \in R
	\quad (\text{若 } n<0),
	\]
	以及 $0 = 0\cdot m \in R$。
	故
	\[
	m\mathbb{Z} \subseteq R.
	\]
	
	\medskip
	\noindent\textbf{(2) 证明 $R \subseteq m\mathbb{Z}$.}
	
	任取 $r \in R$。用整数除法表示:
	\[
	r = qm + s, \quad q \in \mathbb{Z},\ 0 \le s < m.
	\]
	因为 $m \in R$ 且 $R$ 是加法子群,$qm \in R$,于是
	\[
	s = r - qm \in R.
	\]
	
	若 $s > 0$,则 $s \in R$ 且 $0 < s < m$,这与 $m$ 是 $R$ 中最小正整数的定义矛盾。
	因此只能有
	\[
	s = 0.
	\]
	于是 $r = qm$,即 $r \in m\mathbb{Z}$。
	
	因 $r \in R$ 任意,得到
	\[
	R \subseteq m\mathbb{Z}.
	\]
	
	\medskip
	综上,
	\[
	R = m\mathbb{Z},
	\]
	其中 $m$ 是 $R$ 中最小的正整数。\qed
	\subsection{$\mathbb Z_p \text{ 是域}\ \Longleftrightarrow\ p \text{ 为素数}$}
	\textbf{命题.}\;
	设 $p\in\mathbb Z$ 且 $p\ge 2$,则
	\[
	\mathbb Z_p \text{ 是域}\ \Longleftrightarrow\ p \text{ 为素数}.
	\]
	
	\textbf{证明.}
	
	\textbf{($\Rightarrow$)}\;
	若 $\mathbb Z_p$ 是域,则它无零因子。
	若 $p$ 不是素数,则存在整数 $a,b$ 满足
	\[
	1<a<p,\quad 1<b<p,\quad p=ab.
	\]
	于是 $\overline a,\overline b\in\mathbb Z_p$ 都是非零元,但
	\[
	\overline a\cdot \overline b=\overline{ab}=\overline p=\overline 0,
	\]
	这与域无零因子矛盾。
	故 $p$ 必为素数。
	
	\textbf{($\Leftarrow$)}\;
	若 $p$ 为素数,取任意非零元 $\overline a\in\mathbb Z_p$,则 $p\nmid a$,
	从而 $\gcd(a,p)=1$。
	由 Bézout 等式,存在整数 $u,v$ 使
	\[
	au+pv=1.
	\]
	对 $p$ 取模得
	\[
	\overline a\cdot \overline u=\overline 1,
	\]
	故 $\overline a$ 在 $\mathbb Z_p$ 中可逆。
	因此 $\mathbb Z_p$ 的每个非零元都有乘法逆元,$\mathbb Z_p$ 是域。 \qed
	
	\subsection{思考题 2.2.6.} 
	举例说明存在一个环 $R$ 及其子环 $R_1$,以及 $a,b,a',b' \in R$,
	使得在商群 $R/R_1$ 上有
	\[
	a + R_1 = a' + R_1,\qquad b + R_1 = b' + R_1,
	\]
	但
	\[
	ab + R_1 \ne a'b' + R_1.
	\]
	
	\textbf{例:}
	
	取
	\[
	R = \mathbb{Z}, \qquad R_1 = 2\mathbb{Z}.
	\]
	即 $R$ 是整数环,$R_1$ 是偶数的子环(实际上是理想,但我们先仅当作子环考虑)。
	
	令
	\[
	a = 1, \quad a' = 3, \quad b = 1, \quad b' = 3.
	\]
	显然,
	\[
	a - a' = 1 - 3 = -2 \in 2\mathbb{Z} = R_1, \qquad 
	b - b' = 1 - 3 = -2 \in R_1,
	\]
	因此
	\[
	a + R_1 = a' + R_1, \qquad b + R_1 = b' + R_1.
	\]
	但计算乘积:
	\[
	ab = 1, \qquad a'b' = 9,
	\]
	于是
	\[
	ab - a'b' = 1 - 9 = -8 \in 2\mathbb{Z}.
	\]
	
	虽然该例中 $R_1$ 实际上是理想,使得乘法仍然良定义,
	但它说明如果 $R_1$ 只是一个\textbf{子环而非理想},则可能出现
	\[
	a + R_1 = a' + R_1, \ b + R_1 = b' + R_1, \ \text{但 } ab + R_1 \ne a'b' + R_1.
	\]
	这说明:只有当 $R_1$ 是理想时,商环 $R/R_1$ 上的乘法才是良定义的
	
	\subsection{思考题2.2.8}
	\textbf{思考题 2.2.8.}\quad 试举例说明,存在环 $R$ 的子环 $I$,使 $I$ 是左理想而不是右理想;
	同样,存在环 $R$ 的子环 $I$,使 $I$ 是右理想而不是左理想。
	
	\textbf{解:}
	
	考虑非交换环
	\[
	R = M_2(\mathbb{R}),
	\]
	即全体 $2* 2$ 实矩阵所成的环。
	
	---
	
	\textbf{(1) 左理想但不是右理想的例子:}
	
	令
	\[
	I = \left\{
	\begin{bmatrix}
		0 & a \\
		0 & b
	\end{bmatrix}
	\ \Bigg|\ a,b\in\mathbb{R}
	\right\}.
	\]
	
	显然 $I$ 是 $R$ 的一个子环。
	
	对任意
	\[
	A = 
	\begin{bmatrix}
		x & y\\
		z & w
	\end{bmatrix} \in R,\quad
	B =
	\begin{bmatrix}
		0 & a\\
		0 & b
	\end{bmatrix} \in I,
	\]
	我们有
	\[
	AB =
	\begin{bmatrix}
		0 & xb + ya \\
		0 & zb + wa
	\end{bmatrix} \in I.
	\]
	因此 $I$ 对左乘封闭,故 $I$ 是 $R$ 的\textbf{左理想}。
	
	但
	\[
	BA =
	\begin{bmatrix}
		a z & a w \\
		b z & b w
	\end{bmatrix},
	\]
	其第一列一般不为零,不属于 $I$。
	因此 $I$ 不是右理想。
	
	\textbf{(2) 右理想但不是左理想的例子:}
	
	同理,取
	\[
	J = \left\{
	\begin{bmatrix}
		a & 0\\
		b & 0
	\end{bmatrix}
	\ \Bigg|\ a,b\in\mathbb{R}
	\right\}.
	\]
	类似计算可得:对任意 $A\in R,\ B\in J$,有 $BA\in J$,
	但一般 $AB\notin J$。
	因此 $J$ 是\textbf{右理想}而不是左理想。
	
	\textbf{结论:}
	
	在非交换环 $M_2(\mathbb{R})$ 中,
	存在左理想而非右理想(如 $I$),
	也存在右理想而非左理想(如 $J$)。
	
	\subsection{幺元与交换性的继承}
	\textbf{命题.}
	设 $R$ 为一个交换幺环,$R_1$ 为其子环或理想,则关于单位元与交换性的继承性,有如下结论:
	
	\begin{itemize}
		\item[(1)] \textbf{交换性可继承:}
		若 $R$ 为交换环,则其任意子环与任意理想都必为交换的。
		
		\item[(2)] \textbf{单位元一般不可继承:}
		若 $R$ 为幺环,则其子环 $R_1$ 不一定含有 $R$ 的单位元 $1_R$;
		若 $R_1$ 为理想,则
		\[
		1_R \in R_1 \iff R_1 = R.
		\]
		换言之,除非理想等于整个环,否则绝不包含单位元。
	\end{itemize}
	
	\textbf{证明.}
	
	\medskip
	\noindent
	\textbf{(1) 交换性的继承.}
	设 $a,b \in R_1$。由于 $R$ 为交换环,有
	\[
	ab = ba \in R.
	\]
	若 $R_1$ 为子环,则它对乘法封闭,因此 $ab \in R_1$;
	若 $R_1$ 为理想,则它亦对内部乘法封闭,同样有 $ab \in R_1$。
	故 $ab = ba$ 在 $R_1$ 中成立,交换性得以继承。
	
	\medskip
	\noindent
	\textbf{(2) 单位元的不继承性.}
	若 $R_1$ 是子环,则其可能不包含 $1_R$;典型例子为
	\[
	2\mathbb{Z} \subseteq \mathbb{Z},
	\]
	其中 $2\mathbb{Z}$ 并无单位元。
	
	若 $R_1$ 为理想并且 $1_R \in R_1$,则对任意 $r\in R$,理想性质给出
	\[
	r = r\cdot 1_R \in R_1,
	\]
	从而 $R \subseteq R_1$。结合 $R_1 \subseteq R$,得 $R_1 = R$。
	因此,非平凡理想永不含有单位元。
	
	\medskip
	综上,交换性在子环与理想中均能继承,而单位元仅在理想等于整个环时才被继承。
	
	
	\subsection{思考题2.2.12}
	\textbf{思考题2.2.12.} \quad 由于子环的条件容易看出,一个环的子环的子环还是子环。  
	那么一个环的理想的理想一定是理想吗?
	准确地说,如果 $I_1$ 是环 $R$ 的理想,$I_2$ 是 $I_1$ 的理想,  
	那么 $I_2$ 是否一定是 $R$ 的理想?
	
	\textbf{解:}  
	不一定。下面举反例说明。
	
	取非交换环
	\[
	R = M_2(\mathbb{R}),
	\]
	即全体 $2*2$ 实矩阵所成的环。
	
	定义
	\[
	I_1 =
	\left\{
	\begin{bmatrix}
		0 & a\\
		0 & b
	\end{bmatrix}
	\;\middle|\;
	a,b\in\mathbb{R}
	\right\},
	\qquad
	I_2 =
	\left\{
	\begin{bmatrix}
		0 & a\\
		0 & 0
	\end{bmatrix}
	\;\middle|\;
	a\in\mathbb{R}
	\right\}.
	\]
	
	显然 $I_1$ 对加法封闭,并且对任意
	\[
	A=
	\begin{bmatrix}
		x & y\\
		z & w
	\end{bmatrix}\in R,\quad
	B=
	\begin{bmatrix}
		0 & a\\
		0 & b
	\end{bmatrix}\in I_1,
	\]
	有
	\[
	AB =
	\begin{bmatrix}
		0 & xb+ya\\
		0 & zb+wa
	\end{bmatrix}\in I_1,
	\]
	因此 $I_1$ 是 $R$ 的左理想。
	
	再对任意 $C\in I_1,\ D\in I_2$,可得 $CD\in I_2$,  
	故 $I_2$ 是 $I_1$ 的左理想。
	
	但若取
	\[
	A=
	\begin{bmatrix}
		0 & 1\\
		0 & 0
	\end{bmatrix}\in I_2,
	\quad
	B=
	\begin{bmatrix}
		0 & 0\\
		1 & 0
	\end{bmatrix}\in R,
	\]
	则
	\[
	BA =
	\begin{bmatrix}
		1 & 0\\
		0 & 0
	\end{bmatrix}\notin I_2,
	\]
	故 $I_2$ 不是 $R$ 的左理想。
	
	\textbf{结论:}  
	一个环的理想的理想不一定是原环的理想。
	\subsection{在交换幺环中由 $S$ 生成的理想的形式.}
	设 $R$ 为交换幺环,$S\subseteq R$ 为非空子集。
	由 $S$ 生成的理想(记作 $\langle S\rangle$ 或 $(S)$)可显式写为
	\[
	\langle S\rangle
	=\left\{\sum_{i=1}^n r_i s_i \ \middle|\ n\in\mathbb N,\ r_i\in R,\ s_i\in S\right\}.
	\]
	
	\textbf{证明.}\;
	令
	\[
	I:=\left\{\sum_{i=1}^n r_i s_i \ \middle|\ n\in\mathbb N,\ r_i\in R,\ s_i\in S\right\}.
	\]
	先证 $I$ 是理想且包含 $S$。
	
	\textbf{(1) $I$ 在加法下封闭且含加法逆元.}\;
	若 $x=\sum_{i=1}^n r_is_i,\ y=\sum_{j=1}^m t_ju_j\in I$,
	则
	\[
	x+y=\sum_{i=1}^n r_is_i+\sum_{j=1}^m t_ju_j\in I,
	\]
	并且
	\[
	-x=\sum_{i=1}^n (-r_i)s_i\in I.
	\]
	故 $I$ 在加法下为子群。
	
	\textbf{(2) 吸收性.}\;
	任取 $a\in R$ 与 $x=\sum_{i=1}^n r_is_i\in I$,则
	\[
	ax=a\sum_{i=1}^n r_is_i=\sum_{i=1}^n (ar_i)s_i\in I.
	\]
	因此 $I$ 为 $R$ 的理想。
	
	\textbf{(3) $I$ 包含 $S$.}\;
	对任意 $s\in S$,有 $s=1\cdot s\in I$,故 $S\subseteq I$。
	
	接着证明 $I$ 是包含 $S$ 的最小理想。
	设 $J$ 是 $R$ 的任意理想且 $S\subseteq J$。
	则对任意 $r\in R,\ s\in S$,因 $s\in J$ 且 $J$ 为理想,得 $rs\in J$;
	又因 $J$ 在加法下封闭,故任意有限和 $\sum_{i=1}^n r_is_i\in J$。
	因此 $I\subseteq J$。
	
	综上,$I$ 是包含 $S$ 的最小理想,即
	\[
	\langle S\rangle = I
	=\left\{\sum_{i=1}^n r_i s_i \ \middle|\ n\in\mathbb N,\ r_i\in R,\ s_i\in S\right\}.
	\] \qed
	
	
	
	
	
	
	\subsection{整环中,$a\mid b \quad\Longleftrightarrow\quad \langle b\rangle \subseteq \langle a\rangle $}
	\textbf{命题.}
	设 $R$ 为整环,$a,b\in R$,
	则
	\[
	a\mid b \quad\Longleftrightarrow\quad \langle b\rangle \subseteq \langle a\rangle .
	\]
	
	\textbf{证明.}
	
	\textbf{($\Rightarrow$)} 若 $a\mid b$,则存在 $c\in R$ 使
	\[
	b = ac.
	\]
	任取 $\langle b\rangle$ 中的元素 $x$,有
	\[
	x = rb,\qquad r\in R.
	\]
	代入 $b=ac$ 得
	\[
	x = rb = r(ac) = (rc)a.
	\]
	而 $rc\in R$,故 $x\in\langle a\rangle$。因 $x$ 任意,故
	\[
	\langle b\rangle \subseteq \langle a\rangle.
	\]
	
	\textbf{($\Leftarrow$)} 若 $\langle b\rangle \subseteq \langle a\rangle$,则 $b\in\langle b\rangle$ 蕴含
	\[
	b\in\langle a\rangle,
	\]
	故存在 $c\in R$ 使
	\[
	b = ca,
	\]
	即 $a\mid b$。
	
	综上,
	\[
	a\mid b \quad\Longleftrightarrow\quad \langle b\rangle \subseteq \langle a\rangle.
	\]
	
	\subsection{含有可逆元的幺环的理想必是幺环本身}
	\textbf{命题.}
	设 $R$ 为含幺环(幺元记为 $1$),$I$ 为 $R$ 的一个理想。若 $I$ 中含有可逆元(单位元)$u$,则 $I=R$。
	
	\textbf{证明:}
	设 $u\in I$ 且 $u$ 是 $R$ 的可逆元,则存在 $u^{-1}\in R$ 使得
	\[
	uu^{-1}=u^{-1}u=1.
	\]
	由于 $I$ 是理想,特别地它对 $R$ 的乘法吸收:对任意 $r\in R$ 与任意 $a\in I$,都有
	\[
	ra\in I \quad (\text{左理想}), \qquad ar\in I \quad (\text{右理想}).
	\]
	取 $a=u\in I$,令 $r=u^{-1}\in R$,则
	\[
	u^{-1}u \in I.
	\]
	但 $u^{-1}u=1$,故 $1\in I$。
	
	接下来对任意 $x\in R$,由理想的吸收性可得
	\[
	x = x\cdot 1 \in I.
	\]
	因此 $R\subseteq I$。又显然 $I\subseteq R$,于是 $I=R$。
	
	(所以:在含幺环中,真理想不可能包含可逆元。)
	
	\subsection{对于整环中的任何非可逆元$a$,$\langle a\rangle$必是非平凡理想}
	\textbf{命题.}
	证明:设 $R$ 为整环(含幺交换环且无零因子)。对任意非可逆元 $a\neq 0$,其主理想
	\[
	\langle a\rangle=\{ra:\ r\in R\}
	\]
	是非平凡理想(即 $\langle a\rangle\neq (0)$ 且 $\langle a\rangle\neq R$)。
	
	\textbf{解:}
	\textbf{(1) 证明 $\langle a\rangle\neq (0)$.}
	因为 $a\in\langle a\rangle$(取 $r=1$ 即得 $1\cdot a=a$),且已知 $a\neq 0$,
	所以 $\langle a\rangle$ 中含有非零元素 $a$,从而 $\langle a\rangle\neq (0)$。
	
	\textbf{(2) 证明 $\langle a\rangle\neq R$.}
	反设 $\langle a\rangle=R$。则 $1\in \langle a\rangle$,于是存在某个 $r\in R$ 使得
	\[
	ra=1.
	\]
	这说明 $a$ 存在右逆元 $r$。在交换含幺环中这就意味着 $a$ 可逆(其逆元为 $r$),
	与题设“$a$ 为非可逆元”矛盾。
	
	因此 $\langle a\rangle\neq R$。
	
	由 (1)(2) 可知 $\langle a\rangle$ 既不是零理想,也不是全环,故为非平凡理想。
	
	
	
	\clearpage 
	\subsection*{课后习题答案}
	\addcontentsline{toc}{subsection}{\textcolor{red}{课后习题答案}}
	\begin{enumerate}[label=\textcolor{blue}{\textbf{\large\arabic*.}}]	
		\item 
		\textbf{题目.}
		试判断下面的子集 $S$ 是否构成环 $R$ 的子环或理想:
		
		\begin{itemize}
			\item[(1)] $R=\mathbb{Q},\quad S=\left\{\frac{a}{b}\ \middle|\ a,b\in\mathbb{Z},\ b\neq 0\right\}$;
			\item[(2)] $R=\mathbb{Q},\quad S=\left\{2^{\,n}\cdot m\ \middle|\ m,n\in\mathbb{Z}\right\}$。
		\end{itemize}
		
		\textbf{解:}
		(1) $S=\mathbb{Q}$,因此 $S$ 是 $R$ 的子环;并且作为 $\mathbb{Q}$ 的理想,$S$ 也是理想(事实上就是整环 $R$ 自身这个理想)。
		
		(2) $S$ 是 $R$ 的子环(它等于 $\mathbb{Z}\!\left[\frac12\right]$,即所有分母为 $2$ 的幂的有理数);
		但 $S$ 不是 $R$ 的理想(因为 $\mathbb{Q}$ 的理想只有 $\{0\}$ 与 $\mathbb{Q}$,而 $S$ 既不等于 $\{0\}$ 也不等于 $\mathbb{Q}$)。
		
		\bigskip
		\textbf{证明.}
		
		\bigskip
		\textbf{(1) 情形:$R=\mathbb{Q}$,$S=\left\{\frac{a}{b}\mid a,b\in\mathbb{Z},\,b\neq 0\right\}$.}
		
		\textbf{第一步:证明 $S=\mathbb{Q}$.}
		按有理数的定义,任意 $q\in\mathbb{Q}$ 都可写成 $q=\frac{a}{b}$,其中 $a,b\in\mathbb{Z}$ 且 $b\neq 0$,
		故 $\mathbb{Q}\subseteq S$。
		反过来,$S$ 中每个元素 $\frac{a}{b}$ 本身就是有理数,所以 $S\subseteq\mathbb{Q}$。
		因此
		\[
		S=\mathbb{Q}=R.
		\]
		
		\textbf{第二步:判断子环与理想.}
		既然 $S=R$,那么 $S$ 在加法、乘法下当然封闭,含 $0$(也含 $1$),因此 $S$ 是 $R$ 的子环。
		
		此外,$S=R$ 也是 $R$ 的一个理想(即全环理想),因为对任意 $r\in R$ 与 $s\in S(=R)$,
		都有 $rs\in R=S$,理想吸收性成立。
		
		\bigskip
		\textbf{(2) 情形:$R=\mathbb{Q}$,$S=\left\{2^{\,n}m\mid m,n\in\mathbb{Z}\right\}$.}
		
		\textbf{第一步:把 $S$ 改写成更直观的形式.}
		对任意 $n\in\mathbb{Z}$:
		
		\begin{itemize}
			\item 若 $n\ge 0$,则 $2^n m$ 是整数;
			\item 若 $n=-k<0$($k\ge 1$),则 $2^n m = 2^{-k}m=\dfrac{m}{2^k}$。
		\end{itemize}
		
		因此 $S$ 等于所有分母为 $2$ 的幂的有理数的集合:
		\[
		S=\left\{\frac{m}{2^k}\ \middle|\ m\in\mathbb{Z},\ k\in\mathbb{Z}_{\ge 0}\right\}
		=\mathbb{Z}\!\left[\frac12\right].
		\]
		
		\textbf{第二步:证明 $S$ 是 $R$ 的子环.}
		检验子环常用判别:$S$ 非空,且对任意 $x,y\in S$ 有 $x-y\in S$、$xy\in S$。
		
		\begin{itemize}
			\item $S$ 非空:取 $m=0,n=0$ 得 $0=2^0\cdot 0\in S$;取 $m=1,n=0$ 得 $1=2^0\cdot 1\in S$。
			\item 对减法封闭:任取 $x=\dfrac{m_1}{2^{k_1}},\ y=\dfrac{m_2}{2^{k_2}}\in S$(其中 $m_1,m_2\in\mathbb{Z}$,$k_1,k_2\ge 0$),则
			\[
			x-y=\frac{m_1}{2^{k_1}}-\frac{m_2}{2^{k_2}}
			=\frac{m_1 2^{k_2}-m_2 2^{k_1}}{2^{k_1+k_2}}.
			\]
			分子 $m_1 2^{k_2}-m_2 2^{k_1}\in\mathbb{Z}$,分母为 $2^{k_1+k_2}$(仍是 $2$ 的幂),
			故 $x-y\in S$。
			\item 对乘法封闭:同样取 $x=\dfrac{m_1}{2^{k_1}},\ y=\dfrac{m_2}{2^{k_2}}\in S$,则
			\[
			xy=\frac{m_1m_2}{2^{k_1+k_2}}\in S,
			\]
			因为 $m_1m_2\in\mathbb{Z}$ 且 $k_1+k_2\ge 0$。
		\end{itemize}
		
		由此 $S$ 在 $R=\mathbb{Q}$ 中构成子环(并且含 $1$)。
		
		\textbf{第三步:证明 $S$ 不是 $R$ 的理想.}
		在域(特别是 $\mathbb{Q}$)中,理想只有两个:$\{0\}$ 与整个域 $\mathbb{Q}$。
		这里 $S$ 显然不是 $\{0\}$,并且也不等于 $\mathbb{Q}$(例如 $\dfrac{1}{3}\in\mathbb{Q}$ 但分母不是 $2$ 的幂,因此 $\dfrac13\notin S$)。
		所以 $S$ 不可能是 $\mathbb{Q}$ 的理想。
		
		也可以直接用“吸收性”给出反例:取
		\[
		s=1\in S,\qquad r=\frac13\in R,
		\]
		若 $S$ 是理想则应有 $rs\in S$,但
		\[
		rs=\frac13\notin S,
		\]
		与理想的吸收性矛盾,因此 $S$ 不是理想。
		
		\bigskip
		综上:(1) $S$ 是子环且是理想;(2) $S$ 是子环但不是理想。
		
		\item 
		\textbf{题目.}\;
		设 $R$ 为环,$a\in R$。证明由 $a$ 生成的主理想 $(a)$ 等于
		\[
		S:=\left\{\ \sum_{i=1}^{m}x_iay_i+ra+as+na\ \middle|\ r,s\in R,\ x_i,y_i\in R,\ 1\le i\le m,\ n\in\mathbb Z\ \right\}.
		\]
		
		\textbf{答案:}\;
		\textbf{证明.}
		记 $(a)$ 为 $R$ 中包含 $a$ 的最小(双边)理想。
		
		\textbf{(1) 先证 $S\subseteq (a)$.}
		任取 $u\in S$,则
		\[
		u=\sum_{i=1}^{m}x_iay_i+ra+as+na.
		\]
		由于 $(a)$ 是理想且 $a\in(a)$,于是对任意 $x,y,r,s\in R$ 有
		\[
		xay\in(a),\qquad ra\in(a),\qquad as\in(a).
		\]
		又因理想是加法子群,故对任意 $n\in\mathbb Z$ 有 $na\in(a)$。
		因此 $u$ 是若干个属于 $(a)$ 的元素之和,仍在 $(a)$ 中,即 $u\in(a)$。
		从而 $S\subseteq(a)$。
		
		\textbf{(2) 再证 $(a)\subseteq S$.}
		下面证明 $S$ 是一个理想且包含 $a$。
		
		首先 $a\in S$:取 $m=0,\ r=s=0,\ n=1$,则 $a=na\in S$。
		
		其次,$S$ 对加法封闭且对取负封闭:若
		\[
		u=\sum_{i=1}^{m}x_iay_i+ra+as+na,\qquad
		v=\sum_{j=1}^{k}x'_jay'_j+r'a+as'+n'a,
		\]
		则
		\[
		u+v=\sum_{i=1}^{m}x_iay_i+\sum_{j=1}^{k}x'_jay'_j+(r+r')a+a(s+s')+(n+n')a\in S,
		\]
		并且
		\[
		-u=\sum_{i=1}^{m}(-x_i)ay_i+(-r)a+a(-s)+(-n)a\in S.
		\]
		故 $S$ 为加法子群。
		
		最后验证吸收性:任取 $t\in R$ 与 $u\in S$,
		\[
		tu=t\Big(\sum_{i=1}^{m}x_iay_i+ra+as+na\Big)
		=\sum_{i=1}^{m}(tx_i)ay_i+(tr)a+a(0)+0\cdot a\in S,
		\]
		\[
		ut=\Big(\sum_{i=1}^{m}x_iay_i+ra+as+na\Big)t
		=\sum_{i=1}^{m}x_i a (y_it)+0\cdot a+a(st)+0\cdot a\in S.
		\]
		因此 $S$ 是 $R$ 的(双边)理想,且 $a\in S$。
		
		由 $(a)$ 是包含 $a$ 的最小理想,而 $S$ 是包含 $a$ 的理想,故
		\[
		(a)\subseteq S.
		\]
		
		\textbf{(3) 结论.}
		由 (1)(2) 得 $S\subseteq(a)$ 且 $(a)\subseteq S$,从而
		\[
		(a)=S.
		\]
		
		
		
		\item 
		\textbf{题目.}
		设 $R$ 为无零因子环,$I$ 为 $R$ 的理想,问商环 $R/I$ 是否一定是无零因子环?
		
		\textbf{解:}
		不一定。
		
		\textbf{证明.}
		要说明“$R/I$ 不一定无零因子”,只需给出一个反例:取某个无零因子环 $R$ 与其理想 $I$,
		使得在 $R/I$ 中出现非零元素的乘积为 $0$。
		
		\medskip
		\textbf{反例构造.}
		取
		\[
		R=\mathbb{Z}\quad(\text{整数环}),
		\qquad
		I=(6)=6\mathbb{Z}.
		\]
		显然 $\mathbb{Z}$ 是无零因子环:若 $ab=0$($a,b\in\mathbb{Z}$),则必有 $a=0$ 或 $b=0$。
		
		考虑商环
		\[
		R/I=\mathbb{Z}/6\mathbb{Z}.
		\]
		在 $\mathbb{Z}/6\mathbb{Z}$ 中,考虑两个元素
		\[
		\overline{2}=2+(6),\qquad \overline{3}=3+(6).
		\]
		\textbf{第一步:验证它们都非零。}
		\[
		\overline{2}=\overline{0}\iff 2\in(6)\iff 6\mid 2 \quad(\text{不成立}),
		\]
		所以 $\overline{2}\neq \overline{0}$。
		同理
		\[
		\overline{3}=\overline{0}\iff 3\in(6)\iff 6\mid 3 \quad(\text{不成立}),
		\]
		所以 $\overline{3}\neq \overline{0}$。
		
		\medskip
		\textbf{第二步:计算它们的乘积。}
		在商环中
		\[
		\overline{2}\cdot\overline{3}=\overline{6}.
		\]
		但 $6\in(6)$,所以 $\overline{6}=\overline{0}$。于是得到
		\[
		\overline{2}\neq\overline{0},\ \overline{3}\neq\overline{0},\quad
		\overline{2}\cdot\overline{3}=\overline{0}.
		\]
		这说明 $\mathbb{Z}/6\mathbb{Z}$ 含有零因子($\overline{2},\overline{3}$ 就是零因子),
		因此它不是无零因子环。
		
		\medskip
		\textbf{结论.}
		即使 $R$ 是无零因子环,商环 $R/I$ 也不一定是无零因子环。
		
		\bigskip
		\textbf{补充说明(何时 $R/I$ 无零因子).}
		若 $R$ 为交换幺环,则
		\[
		R/I \text{ 为无零因子环}
		\ \Longleftrightarrow\
		I \text{ 为素理想}.
		\]
		特别地,当 $R$ 为整环(交换无零因子幺环)时,$R/I$ 为整环当且仅当 $I$ 为素理想。
		
		\item \textbf{题目.}
		设 $R$ 为整环,$I,J$ 为 $R$ 的非零理想,试证明 $I\cap J\neq \{0\}$。
		
		\textbf{解:}
		$R$ 是整环时无零因子。取 $0\neq a\in I$ 与 $0\neq b\in J$,则 $ab\neq 0$ 且 $ab\in I\cap J$,故 $I\cap J\neq \{0\}$。
		
		\textbf{证明.}
		因为 $I$ 是非零理想,所以存在 $a\in I$ 且 $a\neq 0$;
		因为 $J$ 是非零理想,所以存在 $b\in J$ 且 $b\neq 0$。
		
		由于 $I$ 是理想,且 $a\in I$,对任意 $r\in R$ 都有 $ra\in I$。
		特别地,取 $r=b$,得到
		\[
		ba\in I.
		\]
		同理,由 $J$ 是理想且 $b\in J$,对任意 $r\in R$ 都有 $rb\in J$。
		特别地,取 $r=a$,得到
		\[
		ab\in J.
		\]
		又因为 $R$ 是交换环(整环默认交换幺环)且乘法结合,$ab=ba$,于是
		\[
		ab\in I\quad\text{且}\quad ab\in J,
		\]
		从而
		\[
		ab\in I\cap J.
		\]
		
		接着证明 $ab\neq 0$:由于 $R$ 为整环,满足无零因子性质,即
		\[
		ab=0 \ \Longrightarrow\ a=0 \ \text{或}\ b=0.
		\]
		但这里 $a\neq 0$ 且 $b\neq 0$,故必有
		\[
		ab\neq 0.
		\]
		因此我们找到一个非零元素 $ab\in I\cap J$,从而
		\[
		I\cap J\neq \{0\}.
		\]
		
		\item \textbf{题目.}
		证明有限整环一定是域。
		
		\textbf{解:}
		
		\textbf{证明.}
		设 $R$ 为有限整环。我们需要证明:$R$ 的每个非零元素都可逆,从而 $R$ 为域。
		
		\medskip
		\textbf{1. 关键引理:整环中“乘以非零元”的映射是单射}
		
		取任意 $a\in R$ 且 $a\neq 0$。定义映射
		\[
		\varphi_a: R\to R,\qquad \varphi_a(x)=ax.
		\]
		我们证明 $\varphi_a$ 是单射。
		
		若 $\varphi_a(x)=\varphi_a(y)$,即 $ax=ay$,则
		\[
		ax-ay=0\quad\Longrightarrow\quad a(x-y)=0.
		\]
		由于 $R$ 是整环(无零因子),且 $a\neq 0$,只能推出
		\[
		x-y=0 \quad\Longrightarrow\quad x=y.
		\]
		故 $\varphi_a$ 为单射。
		
		\medskip
		\textbf{2. 有限性 $\Rightarrow$ 单射必为满射}
		
		因为 $R$ 是有限集合,任一映射 $f:R\to R$ 若为单射,则必为满射(有限集合上的基本事实)。
		因此 $\varphi_a$ 既然是单射,就必为满射。
		
		于是存在某个 $b\in R$ 使得
		\[
		\varphi_a(b)=ab=1,
		\]
		其中 $1$ 表示 $R$ 的乘法单位元(整环按定义是含幺交换环且无零因子)。
		
		\medskip
		\textbf{3. 由交换性推出 $b$ 也是右逆,从而 $a$ 可逆}
		
		由于 $R$ 是交换环,$ab=1$ 蕴含
		\[
		ba=ab=1.
		\]
		因此 $b$ 是 $a$ 的逆元,$a$ 可逆。
		
		\medskip
		\textbf{4. 结论}
		
		上面对任意非零 $a\in R$ 都能找到 $b\in R$ 使得 $ab=1$,故 $R$ 的每个非零元都可逆。
		因此 $R$ 是域。
		
		\hfill $\Box$
		
		\item 7.
		\textbf{题目.}
		试决定下列环的单位群:
		\begin{itemize}
			\item[(1)] Gauss 整数环 $\mathbb Z[\sqrt{-1}]$;
			\item[(2)] 商环 $\mathbb Z_{7}$;
			\item[(3)] 商环 $\mathbb Z_{24}$;
			\item[(4)] 商环 $\mathbb Z_{m}$,$m\in\mathbb N$。
		\end{itemize}
		
		\textbf{解:}
		
		\textbf{(1) $\mathbb Z[\sqrt{-1}]$ 的单位群}
		
		\textbf{证明.}
		设 $u\in\mathbb Z[\sqrt{-1}]$ 为单位,则 $\exists v\in\mathbb Z[\sqrt{-1}]$ 使 $uv=1$。
		定义范数
		\[
		N:\mathbb Z[\sqrt{-1}]\to \mathbb Z_{\ge 0},\qquad N(a+bi)=a^2+b^2.
		\]
		它满足对任意 $\alpha,\beta\in\mathbb Z[i]$:
		\[
		N(\alpha\beta)=N(\alpha)N(\beta).
		\]
		证明很直接:若 $\alpha=a+bi,\ \beta=c+di$,则
		\[
		\alpha\beta=(ac-bd)+(ad+bc)i,
		\]
		于是
		\[
		N(\alpha\beta)=(ac-bd)^2+(ad+bc)^2
		=(a^2+b^2)(c^2+d^2)=N(\alpha)N(\beta).
		\]
		
		现在令 $u$ 为单位且 $uv=1$,则
		\[
		1=N(1)=N(uv)=N(u)N(v).
		\]
		由于 $N(u),N(v)\in\mathbb Z_{\ge 0}$,且乘积为 $1$,只能有
		\[
		N(u)=1.
		\]
		设 $u=a+bi$,则 $a^2+b^2=1$,在整数中解得
		\[
		(a,b)\in\{(1,0),(-1,0),(0,1),(0,-1)\}.
		\]
		故
		\[
		u\in\{1,-1,\sqrt{-1},-\sqrt{-1}\}.
		\]
		反过来,$1,-1,\sqrt{-1},-\sqrt{-1}$ 都显然可逆(其逆分别为 $1,-1,-\sqrt{-1},\sqrt{-1}$),因此它们恰为全部单位。
		
		所以
		\[
		U(\mathbb Z[\sqrt{-1}])=\{\,\pm 1,\ \pm \sqrt{-1}\,\}.
		\]
		
		\textbf{(2) $\mathbb Z_7$ 的单位群}
		
		\textbf{证明.}
		因为 $7$ 为素数,所以 $\mathbb Z_7$ 是域。域中除 $0$ 外的所有元素都是单位。
		因此
		\[
		U(\mathbb Z_7)=\mathbb Z_7^*=\{\,\bar 1,\bar 2,\bar 3,\bar 4,\bar 5,\bar 6\,\}.
		\]
		其群阶为 $6$,并且已知有限域的乘法群为循环群,因此
		\[
		\mathbb Z_7^*\cong C_6.
		\]
		(例如 $\bar 3$ 是一个生成元:$\bar 3^1=\bar 3,\ \bar 3^2=\bar 2,\ \bar 3^3=\bar 6,\ \bar 3^4=\bar 4,\ \bar 3^5=\bar 5,\ \bar 3^6=\bar 1$。)
		
		\hfill $\Box$
		
		\textbf{(3) $\mathbb Z_{24}$ 的单位群}
		
		\textbf{证明.}
		在 $\mathbb Z_n$ 中,元素 $\bar a$ 为单位当且仅当 $\gcd(a,n)=1$。
		证明如下:
		
		\textbf{($\Rightarrow$)} 若 $\bar a$ 为单位,则存在 $\bar b$ 使 $\bar a\bar b=\bar 1$,
		即 $ab\equiv 1\pmod n$,从而 $ab-1=kn$。任意公因子 $d\mid a$ 且 $d\mid n$
		必有 $d\mid (ab-1)$,故 $d\mid 1$,只能 $d=1$,即 $\gcd(a,n)=1$。
		
		\textbf{($\Leftarrow$)} 若 $\gcd(a,n)=1$,由裴蜀定理 $\exists x,y\in\mathbb Z$ 使 $ax+ny=1$,
		模 $n$ 得 $ax\equiv 1\pmod n$,故 $\bar a\bar x=\bar 1$,$\bar a$ 可逆。
		
		因此
		\[
		U(\mathbb Z_{24})=\{\bar a\in\mathbb Z_{24}\mid \gcd(a,24)=1\}.
		\]
		与 $24$ 互素的整数(在 $1,\dots,23$ 中)为
		\[
		1,5,7,11,13,17,19,23,
		\]
		故
		\[
		U(\mathbb Z_{24})=\{\bar 1,\bar 5,\bar 7,\bar{11},\bar{13},\bar{17},\bar{19},\bar{23}\},
		\qquad |U(\mathbb Z_{24})|=\varphi(24)=8.
		\]
		
		进一步可给出群结构:注意 $24=8\cdot 3$ 且 $(8,3)=1$,
		由中国剩余定理
		\[
		\mathbb Z_{24}^* \cong \mathbb Z_8^* * \mathbb Z_3^*.
		\]
		其中
		\[
		\mathbb Z_8^*=\{\bar 1,\bar 3,\bar 5,\bar 7\}\cong C_2* C_2,
		\qquad
		\mathbb Z_3^*=\{\bar 1,\bar 2\}\cong C_2.
		\]
		故
		\[
		\mathbb Z_{24}^* \cong (C_2* C_2)* C_2 \cong C_2* C_2* C_2.
		\]
		也可直接验证:上面 8 个单位除 $\bar 1$ 外都满足平方为 $\bar 1$(即每个非平凡元阶为 $2$),
		因此确为初等阿贝尔 $2$-群。
		
		\hfill $\Box$
		
		\textbf{(4) 一般的 $\mathbb Z_m$ 的单位群}
		
		\textbf{命题.}\;
		证明:加法写法下的环(或群) $\mathbb Z_m$ 的\textbf{单位群}(即可逆元素全体)
		\[
		\mathbb Z_m^\times=\{\ \bar{k}\in\mathbb Z_m \mid \bar{k}\ \text{在乘法下可逆}\ \}
		\]
		满足
		\[
		\mathbb Z_m^\times=\{\ \bar{k}\in\mathbb Z_m \mid \gcd(k,m)=1\ \}.
		\]
		
		\textbf{答案:}\;
		\textbf{证明.}
		我们证明两个包含关系。
		
		\medskip
		\textbf{(1) 若 $\bar{k}\in\mathbb Z_m$ 可逆,则 $\gcd(k,m)=1$.}
		“$\bar{k}$ 可逆”是指:存在 $\overline{k'}\in\mathbb Z_m$ 使得
		\[
		\overline{k'}\cdot\bar{k}=\bar{1}.
		\]
		按同余的含义,上式等价于
		\[
		k'k\equiv 1 \pmod m.
		\]
		这意味着存在整数 $q\in\mathbb Z$ 使
		\[
		k'k=1+qm \quad\Longleftrightarrow\quad k'k+(-q)m=1.
		\]
		令 $x=k',\ y=-q$,则得到
		\[
		xk+ym=1.
		\]
		由 Bézout 定理(贝祖等式的充要条件)可知:存在整数解 $x,y$ 使 $xk+ym=1$
		当且仅当
		\[
		\gcd(k,m)=1.
		\]
		因此 $\bar{k}$ 可逆 $\Rightarrow \gcd(k,m)=1$。
		
		\medskip
		\textbf{(2) 若 $\gcd(k,m)=1$,则 $\bar{k}$ 可逆.}
		由 $\gcd(k,m)=1$,再用 Bézout 定理,存在整数 $x,y\in\mathbb Z$ 使得
		\[
		xk+ym=1.
		\]
		对上式两边取模 $m$,因 $ym\equiv 0\pmod m$,故
		\[
		xk\equiv 1 \pmod m.
		\]
		把 $x$ 取同余类记作 $\bar{x}\in\mathbb Z_m$,则
		\[
		\bar{x}\cdot\bar{k}=\overline{xk}=\bar{1}.
		\]
		这说明 $\bar{k}$ 在 $\mathbb Z_m$ 中存在乘法逆元(逆元就是 $\bar{x}$),因此 $\bar{k}$ 可逆。
		
		\medskip
		\textbf{(3) 结论.}
		由 (1)(2) 两个方向合并,得到
		\[
		\mathbb Z_m^\times=\{\ \bar{k}\in\mathbb Z_m\mid \gcd(k,m)=1\ \}.
		\]
		
		
		\item 9.
		\textbf{题目.}
		设 $R$ 为无零因子环,且 $|R|<\infty$。证明 $R$ 为除环。
		
		\textbf{证明.}
		记
		\[
		R^{\ast}:=R\setminus\{0\}.
		\]
		要证 $R$ 为除环,只需证 $R^{\ast}$ 在乘法下构成群(乘法结合律由环公理给出)。
		
		\medskip
		\textbf{(1) 对任意 $a\in R^{\ast}$,左乘映射是单射,从而是满射.}
		任取 $a\in R^{\ast}$,定义映射
		\[
		L_a:R\to R,\qquad L_a(x)=ax.
		\]
		若 $L_a(x)=L_a(y)$,则 $ax=ay$,从而
		\[
		a(x-y)=0.
		\]
		由于 $R$ 无零因子且 $a\neq 0$,故 $x-y=0$,即 $x=y$。
		因此 $L_a$ 为单射。
		
		又因 $R$ 是有限集,单射必为满射,故 $L_a$ 也是满射。
		于是对任意 $b\in R$,存在 $x\in R$ 使得
		\[
		ax=b.
		\]
		特别地,取 $b=1$(其中 $1$ 是 $R$ 的乘法幺元),存在 $x\in R$ 使
		\[
		ax=1,
		\]
		即 $a$ 存在右逆元。
		
		\medskip
		\textbf{(2) 同理可得每个 $a\in R^{\ast}$ 也存在左逆元.}
		同样定义右乘映射
		\[
		R_a:R\to R,\qquad R_a(x)=xa.
		\]
		同理可证 $R_a$ 单射,因 $R$ 有限故满射,从而存在 $y\in R$ 使
		\[
		ya=1,
		\]
		即 $a$ 存在左逆元。
		
		\medskip
		\textbf{(3) 左右逆元必相等,从而 $a$ 可逆.}
		设 $ya=1$ 且 $ax=1$,则
		\[
		y=y(ax)=(ya)x=1\cdot x=x,
		\]
		故左右逆元相同,记为 $a^{-1}$。
		因此每个 $a\in R^{\ast}$ 都有乘法逆元,且显然 $1\in R^{\ast}$ 为单位元,
		于是 $R^{\ast}$ 在乘法下为群。
		
		\medskip
		\textbf{(4) 结论.}
		$R^{\ast}$ 为乘法群,故 $R$ 为除环。
		
	\end{enumerate}
	
	\clearpage
	\section{四元数体}
	
	
	\clearpage
	\section{环的同态}
	\subsection{$\varphi: R_1 \to R_2$ 为环同态(且$\varphi$不是零同态),其中 $R_1$ 为幺环,$R_2$ 为整环,则$\operatorname{Im}\varphi$ 是整环}
	
	\subsection{两个整环同构,则对应的分式域同构}
	\textbf{命题(整环同构诱导分式域同构).}
	设 $R,S$ 为整环,且存在环同构 $\varphi:R\to S$。
	则存在唯一的域同构
	\[
	\widetilde{\varphi}:\mathrm{Frac}(R)\longrightarrow \mathrm{Frac}(S)
	\]
	使得对一切 $r\in R$ 有
	\[
	\widetilde{\varphi}\!\left(\frac{r}{1}\right)=\frac{\varphi(r)}{1}.
	\]
	特别地,$\mathrm{Frac}(R)\cong \mathrm{Frac}(S)$。
	
	\textbf{证明.}
	
	\textbf{(1) 定义映射并证明良定义.}
	记 $\mathrm{Frac}(R)$ 为 $R$ 的分式域,元素记作分数类 $\frac{a}{b}$($a,b\in R,\ b\neq 0$),
	并满足等价关系
	\[
	\frac{a}{b}=\frac{c}{d}\quad\Longleftrightarrow\quad ad=bc.
	\]
	同理对 $\mathrm{Frac}(S)$。
	
	定义
	\[
	\widetilde{\varphi}:\mathrm{Frac}(R)\to \mathrm{Frac}(S),\qquad
	\widetilde{\varphi}\!\left(\frac{a}{b}\right)=\frac{\varphi(a)}{\varphi(b)}.
	\]
	由于 $b\neq 0$ 且 $\varphi$ 为同构(从而为单射),故 $\varphi(b)\neq 0$,因此右端确实是 $\mathrm{Frac}(S)$ 的元素。
	
	需证良定义:若 $\frac{a}{b}=\frac{c}{d}$ 于 $\mathrm{Frac}(R)$,则 $ad=bc$。
	对该等式施加 $\varphi$(环同态保持乘法与加法)得
	\[
	\varphi(a)\varphi(d)=\varphi(ad)=\varphi(bc)=\varphi(b)\varphi(c).
	\]
	于是
	\[
	\varphi(a)\varphi(d)=\varphi(b)\varphi(c),
	\]
	这正说明
	\[
	\frac{\varphi(a)}{\varphi(b)}=\frac{\varphi(c)}{\varphi(d)}\quad\text{于}\ \mathrm{Frac}(S).
	\]
	故 $\widetilde{\varphi}$ 良定义。
	
	\textbf{(2) $\widetilde{\varphi}$ 是域同态.}
	任取 $\frac{a}{b},\frac{c}{d}\in \mathrm{Frac}(R)$,则
	\[
	\widetilde{\varphi}\!\left(\frac{a}{b}+\frac{c}{d}\right)
	=\widetilde{\varphi}\!\left(\frac{ad+bc}{bd}\right)
	=\frac{\varphi(ad+bc)}{\varphi(bd)}
	=\frac{\varphi(a)\varphi(d)+\varphi(b)\varphi(c)}{\varphi(b)\varphi(d)}
	=\frac{\varphi(a)}{\varphi(b)}+\frac{\varphi(c)}{\varphi(d)}.
	\]
	同理
	\[
	\widetilde{\varphi}\!\left(\frac{a}{b}\cdot\frac{c}{d}\right)
	=\widetilde{\varphi}\!\left(\frac{ac}{bd}\right)
	=\frac{\varphi(ac)}{\varphi(bd)}
	=\frac{\varphi(a)\varphi(c)}{\varphi(b)\varphi(d)}
	=\frac{\varphi(a)}{\varphi(b)}\cdot\frac{\varphi(c)}{\varphi(d)}.
	\]
	并且
	\[
	\widetilde{\varphi}(1)=\widetilde{\varphi}\!\left(\frac{1}{1}\right)=\frac{1}{1}=1.
	\]
	故 $\widetilde{\varphi}$ 为域同态。
	
	\textbf{(3) 构造逆映射并证明同构.}
	由于 $\varphi$ 为同构,存在逆同构 $\psi=\varphi^{-1}:S\to R$。
	按(1)同样方式定义
	\[
	\widetilde{\psi}:\mathrm{Frac}(S)\to \mathrm{Frac}(R),\qquad
	\widetilde{\psi}\!\left(\frac{s}{t}\right)=\frac{\psi(s)}{\psi(t)}.
	\]
	则对任意 $\frac{a}{b}\in\mathrm{Frac}(R)$,
	\[
	(\widetilde{\psi}\circ\widetilde{\varphi})\!\left(\frac{a}{b}\right)
	=\widetilde{\psi}\!\left(\frac{\varphi(a)}{\varphi(b)}\right)
	=\frac{\psi(\varphi(a))}{\psi(\varphi(b))}
	=\frac{a}{b}.
	\]
	同理可证 $\widetilde{\varphi}\circ\widetilde{\psi}=\mathrm{id}_{\mathrm{Frac}(S)}$。
	因此 $\widetilde{\varphi}$ 为域同构。
	
	\textbf{(4) 唯一性.}
	若 $\Phi:\mathrm{Frac}(R)\to \mathrm{Frac}(S)$ 是域同态且满足
	$\Phi(\frac{r}{1})=\frac{\varphi(r)}{1}$ 对一切 $r\in R$ 成立,则对任意 $\frac{a}{b}\in\mathrm{Frac}(R)$,
	\[
	\Phi\!\left(\frac{a}{b}\right)
	=\Phi\!\left(\frac{a}{1}\right)\cdot \Phi\!\left(\frac{b}{1}\right)^{-1}
	=\frac{\varphi(a)}{1}\cdot\left(\frac{\varphi(b)}{1}\right)^{-1}
	=\frac{\varphi(a)}{\varphi(b)}
	=\widetilde{\varphi}\!\left(\frac{a}{b}\right),
	\]
	故 $\Phi=\widetilde{\varphi}$,唯一性成立。
	
	综上,$\mathrm{Frac}(R)\cong \mathrm{Frac}(S)$。\qed。
	
	\clearpage 
	\subsection*{课后习题答案}
	\addcontentsline{toc}{subsection}{\textcolor{red}{课后习题答案}}
	\begin{enumerate}[label=\textcolor{blue}{\textbf{\large\arabic*.}}]	
		\item 3.
		\textbf{题目.}\quad 证明:环 \(\mathbb{Z}_p\)(\(p\) 为素数)只有两个自同态,即零同态与恒等映射。上述结论对一般的商环 \(\mathbb{Z}_n\) 正确吗?说明理由。
		
		\textbf{解:}
		
		\textbf{(一) \(\mathbb{Z}_p\) 的自同态只有两个。}
		
		设 \(\varphi:\mathbb{Z}_p\to\mathbb{Z}_p\) 为环同态(不要求保单位)。记
		\[
		e:=\varphi(\overline{1})\in\mathbb{Z}_p .
		\]
		则对任意 \(\overline{k}\in\mathbb{Z}_p\)(其中 \(k\in\mathbb{Z}\)),由加法同态性得
		\[
		\varphi(\overline{k})=\varphi(\overline{1}+\cdots+\overline{1})=k\,\varphi(\overline{1})=k e .
		\]
		再由乘法同态性,
		\[
		e=\varphi(\overline{1})=\varphi(\overline{1}\cdot\overline{1})
		=\varphi(\overline{1})\cdot\varphi(\overline{1})
		=e^2 ,
		\]
		故 \(e\) 为幂等元。由于 \(\mathbb{Z}_p\) 是域,其幂等元仅有 \(\overline{0},\overline{1}\)(若 \(e\neq \overline{0},\overline{1}\),则 \(e(1-e)=0\) 与域无零因子矛盾)。于是
		\[
		e=\overline{0}\Rightarrow \varphi\equiv 0,\qquad
		e=\overline{1}\Rightarrow \varphi=\mathrm{id}_{\mathbb{Z}_p}.
		\]
		故 \(\mathbb{Z}_p\) 只有零同态与恒等同态两种自同态。\qed 
		
		\textbf{(二) 一般 \(\mathbb{Z}_n\) 的情形。}
		
		\begin{itemize}
			\item \textbf{自同态与幂等元的一一对应:} 对任意环同态 \(\varphi:\mathbb{Z}_n\to\mathbb{Z}_n\),同样令 \(e=\varphi(\overline{1})\)。则对一切 \(\overline{k}\in\mathbb{Z}_n\),
			\[
			\varphi(\overline{k})=k e,\qquad e^2=e.
			\]
			反之,给定任意幂等元 \(e\in\mathbb{Z}_n\)(即 \(e^2=e\)),定义
			\[
			\varphi_e:\mathbb{Z}_n\to\mathbb{Z}_n,\quad \varphi_e(\overline{k})=k e,
			\]
			易验 \(\varphi_e\) 为环自同态。故
			\[
			\mathrm{End}(\mathbb{Z}_n)\;\cong\;\{e\in\mathbb{Z}_n:\ e^2=e\}.
			\]
			
			\item \textbf{当 \(n=p^m\)(素数幂)时:} 设 \(e^2\equiv e\pmod{p^m}\),即 \(p^m\mid e(e-1)\)。因 \(\gcd(e,e-1)=1\),必有 \(p^m\mid e\) 或 \(p^m\mid (e-1)\),故幂等元只有 \(\overline{0},\overline{1}\)。于是 \(\mathbb{Z}_{p^m}\) 仍只有零同态与恒等同态两种自同态。
			
			\item \textbf{当 \(n\) 含至少两个互异素因子时:} 设
			\(n=\prod_{i=1}^r p_i^{\alpha_i}\)(两两不同的 \(p_i\),\(r\ge 2\))。由中国剩余定理
			\[
			\mathbb{Z}_n \cong \prod_{i=1}^r \mathbb{Z}_{p_i^{\alpha_i}} .
			\]
			在直积环中,幂等元恰为 \((\varepsilon_1,\dots,\varepsilon_r)\) 其中每个 \(\varepsilon_i\in\{0,1\}\)。故 \(\mathbb{Z}_n\) 的幂等元个数为 \(2^r>2\),从而存在非零非恒等的自同态。例如在 \(\mathbb{Z}_6\) 中,\(\overline{3},\overline{4}\) 均为非平凡幂等元,给出
			\(\varphi_3(\overline{k})=\overline{3k}\)、\(\varphi_4(\overline{k})=\overline{4k}\) 等非平凡自同态。
		\end{itemize}
		
		\textbf{结论:}\;
		“只有零同态与恒等同态”对且仅对 \(n\) 为素数幂的商环 \(\mathbb{Z}_n\) 成立;若 \(n\) 含至少两个互异素因子,则结论不成立,且所有自同态与 \(\mathbb{Z}_n\) 的幂等元一一对应。
		
		\textbf{(三) 一个反例}
		
		\textbf{反例(一般 $\mathbb Z_n$ 时结论不成立).}\;
		取 $n=6$,定义映射
		\[
		\varphi:\mathbb Z_6\to\mathbb Z_6,\qquad \varphi(\bar x)=\overline{3x}.
		\]
		先说明它是良定义的:若 $x\equiv x'\pmod 6$,则 $3x\equiv 3x'\pmod 6$,故 $\overline{3x}=\overline{3x'}$。
		
		验证加法保持:对任意 $\bar x,\bar y\in\mathbb Z_6$,
		\[
		\varphi(\bar x+\bar y)=\overline{3(x+y)}=\overline{3x}+\overline{3y}
		=\varphi(\bar x)+\varphi(\bar y).
		\]
		
		验证乘法保持:对任意 $\bar x,\bar y\in\mathbb Z_6$,
		\[
		\varphi(\bar x\cdot \bar y)=\varphi(\overline{xy})=\overline{3xy},
		\]
		而
		\[
		\varphi(\bar x)\cdot\varphi(\bar y)=\overline{3x}\cdot\overline{3y}
		=\overline{9xy}=\overline{3xy}\qquad(\text{因 }9\equiv 3\pmod 6).
		\]
		故 $\varphi(\bar x\bar y)=\varphi(\bar x)\varphi(\bar y)$,从而 $\varphi$ 是环同态。
		
		并且
		\[
		\varphi(\bar 1)=\bar 3\neq \bar 0,\ \bar 1,
		\]
		所以 $\varphi$ 既不是零同态,也不是恒等映射。
		
		因此,对一般的 $\mathbb Z_n$,“只有零同态与恒等映射”\textbf{不正确}。
		
		\item 7.
		\textbf{题目.}
		试证明实数域 $\mathbb{R}$ 的自同构只有恒等同构。
		
		\textbf{解:}
		
		\textbf{证明.}
		设 $\sigma:\mathbb{R}\to\mathbb{R}$ 是域自同构。我们证明 $\sigma=\mathrm{id}_{\mathbb{R}}$。
		
		\medskip
		\textbf{第一步:$\sigma$ 固定 $0,1$,从而固定所有整数与有理数。}
		
		因为 $\sigma$ 为环同态且为双射,故
		\[
		\sigma(0)=\sigma(0+0)=\sigma(0)+\sigma(0)\ \Longrightarrow\ \sigma(0)=0.
		\]
		又因 $\sigma$ 为乘法同态,
		\[
		\sigma(1)=\sigma(1\cdot 1)=\sigma(1)\cdot\sigma(1),
		\]
		所以 $\sigma(1)$ 为幂等元。域中幂等元只有 $0$ 与 $1$。
		而 $\sigma$ 为双射且 $\sigma(1)\neq 0$(否则 $1=\sigma^{-1}(0)=0$ 矛盾),故
		\[
		\sigma(1)=1.
		\]
		于是对任意正整数 $n$,
		\[
		\sigma(n)=\sigma(\underbrace{1+\cdots+1}_{n\text{ 次}})
		=\underbrace{\sigma(1)+\cdots+\sigma(1)}_{n\text{ 次}}=n.
		\]
		对负整数 $-n$,由 $\sigma(-n)+\sigma(n)=\sigma(0)=0$ 得 $\sigma(-n)=-n$。
		故
		\[
		\sigma(k)=k\quad(\forall k\in\mathbb{Z}).
		\]
		进一步,对任意有理数 $\frac{p}{q}$($q\neq 0$),有
		\[
		\sigma\!\left(\frac{p}{q}\right)
		=\sigma(p)\,\sigma(q)^{-1}
		=p\cdot q^{-1}
		=\frac{p}{q},
		\]
		因此
		\[
		\sigma(r)=r\quad(\forall r\in\mathbb{Q}).
		\]
		
		\medskip
		\textbf{第二步:$\sigma$ 保持正性,从而保持大小关系与上确界。}
		
		\textbf{(i) $\sigma$ 把正数映为正数。}
		任取 $x>0$。在实数中 $x$ 有平方根:存在 $y\in\mathbb{R}$ 使 $x=y^2$。
		于是
		\[
		\sigma(x)=\sigma(y^2)=\sigma(y)^2\ge 0.
		\]
		且若 $\sigma(x)=0$,则 $\sigma(y)^2=0$,从而 $\sigma(y)=0$,再由 $\sigma$ 单射得 $y=0$,
		于是 $x=y^2=0$,矛盾。故
		\[
		x>0\ \Longrightarrow\ \sigma(x)>0.
		\]
		
		\textbf{(ii) $\sigma$ 保序。}
		若 $a<b$,则 $b-a>0$,由上一步得
		\[
		\sigma(b-a)>0.
		\]
		又 $\sigma(b-a)=\sigma(b)-\sigma(a)$,所以
		\[
		\sigma(b)-\sigma(a)>0\ \Longrightarrow\ \sigma(a)<\sigma(b).
		\]
		因此 $\sigma$ 严格保序。
		
		\textbf{(iii) $\sigma$ 保持上确界(对有上界的集合)。}
		设 $S\subset\mathbb{R}$ 非空且有上界,记 $\alpha=\sup S$。
		我们证明
		\[
		\sigma(\alpha)=\sup \sigma(S),\qquad \sigma(S):=\{\sigma(s):s\in S\}.
		\]
		
		先证 $\sigma(\alpha)$ 是 $\sigma(S)$ 的上界:对任意 $s\in S$,有 $s\le \alpha$,
		由保序性得 $\sigma(s)\le \sigma(\alpha)$,故 $\sigma(\alpha)$ 为上界。
		
		再证其为最小上界:若 $\beta$ 是 $\sigma(S)$ 的任一上界,则对任意 $s\in S$ 有
		$\sigma(s)\le \beta$。对两边施加 $\sigma^{-1}$(同样保序),得
		\[
		s \le \sigma^{-1}(\beta)\qquad(\forall s\in S),
		\]
		所以 $\sigma^{-1}(\beta)$ 是 $S$ 的上界,从而
		\[
		\alpha \le \sigma^{-1}(\beta).
		\]
		再施加 $\sigma$ 并用保序性得
		\[
		\sigma(\alpha)\le \beta.
		\]
		因此 $\sigma(\alpha)$ 是 $\sigma(S)$ 的最小上界,即 $\sigma(\alpha)=\sup\sigma(S)$。
		
		\medskip
		\textbf{第三步:任意实数由有理数上确界逼近,从而被 $\sigma$ 固定。}
		
		任取 $x\in\mathbb{R}$。令
		\[
		A_x:=\{q\in\mathbb{Q}: q<x\}.
		\]
		由于 $\mathbb{Q}$ 在 $\mathbb{R}$ 中稠密,$A_x$ 非空;且 $x$ 是其上界。
		并且由实数的完备性可知
		\[
		\sup A_x = x.
		\]
		(直观理解:$x$ 是所有小于它的有理数的最小上界。)
		
		对该等式施加 $\sigma$,并用第二步(iii) 的“保持上确界”与第一步“固定 $\mathbb{Q}$”,得
		\[
		\sigma(x)=\sigma(\sup A_x)=\sup \sigma(A_x)=\sup A_x = x,
		\]
		其中 $\sigma(A_x)=A_x$ 是因为对每个 $q\in\mathbb{Q}$ 有 $\sigma(q)=q$。
		
		因此对任意 $x\in\mathbb{R}$ 都有 $\sigma(x)=x$,即
		\[
		\sigma=\mathrm{id}_{\mathbb{R}}.
		\]
		
		\medskip
		综上,$\mathbb{R}$ 的域自同构只有恒等同构。
		
		
		\item 8.
		\textbf{题目.}  
		证明:映射
		\[
		\varphi : a + b\sqrt{-1} \longmapsto a - b\sqrt{-1}
		\]
		是环 $\mathbb{C}$ 的自同构。
		
		\textbf{证明.}  
		设 $\mathbb{C} = \{ a + b i \mid a,b \in \mathbb{R} \}$,定义映射
		\[
		\varphi : \mathbb{C} \to \mathbb{C}, \quad \varphi(a + bi) = a - bi.
		\]
		
		我们验证 $\varphi$ 为环同构。
		
		\textbf{(1) 保加法.}  
		任取 $z_1 = a + bi$,$z_2 = c + di$,则
		\[
		\varphi(z_1 + z_2)
		= \varphi((a+c) + (b+d)i)
		= (a+c) - (b+d)i
		= (a - bi) + (c - di)
		= \varphi(z_1) + \varphi(z_2).
		\]
		故 $\varphi$ 保加法。
		
		\textbf{(2) 保乘法.}  
		同样,
		\[
		\varphi(z_1 z_2)
		= \varphi((a+bi)(c+di))
		= \varphi((ac-bd) + (ad+bc)i)
		= (ac-bd) - (ad+bc)i,
		\]
		而
		\[
		\varphi(z_1)\varphi(z_2)
		= (a-bi)(c-di)
		= (ac-bd) - (ad+bc)i,
		\]
		故 $\varphi(z_1 z_2) = \varphi(z_1)\varphi(z_2)$,即保乘法。
		
		\textbf{(3) 单射与满射.}  
		若 $\varphi(a+bi)=\varphi(c+di)$,则有 $a-bi=c-di$,比较实虚部得 $a=c, b=d$,故 $\varphi$ 单射。  
		任取 $a+bi\in \mathbb{C}$,有 $\varphi(a-bi)=a+bi$,故 $\varphi$ 满射。  
		因此 $\varphi$ 为双射。
		
		综上,$\varphi$ 是保持加法与乘法的双射,即环 $\mathbb{C}$ 的自同构。\qed
		\item 11.
		\textbf{题目.}
		设 $R$ 为无零因子环,且 $|R|=p$,其中 $p$ 为素数。
		证明:$R$ 为域,且 $R\simeq \mathbb{Z}_p$。
		
		\textbf{证明.}
		由于 $|R|=p$,加法群 $(R,+)$ 为阶 $p$ 的有限交换群,
		故 $(R,+)$ 为循环群。取 $a\in R$ 为其生成元且 $a\neq 0$,则
		\[
		R=\{0,a,2a,\dots,(p-1)a\},
		\qquad pa=0.
		\]
		特别地,对 $1\le l\le p-1$,有 $la\neq 0$(否则 $la=0$ 将推出 $a$ 的加法阶整除 $l$,
		与 $a$ 的加法阶为 $p$ 矛盾)。
		
		\medskip
		\textbf{(1) 写出 $a^2=ka$ 且 $k\neq 0$.}
		由于 $a^2:=a\cdot a\in R$,而 $R$ 中每个元素都可唯一表示为 $ka$($k\in\{0,1,\dots,p-1\}$),
		故存在唯一的 $k\in\{0,1,\dots,p-1\}$ 使
		\[
		a^2=ka.
		\]
		又因 $R$ 无零因子且 $a\neq 0$,所以 $a^2\neq 0$,从而 $k\neq 0$,
		即
		\[
		k\in\{1,2,\dots,p-1\}.
		\]
		
		\medskip
		\textbf{(2) $R$ 为交换环.}
		先证 $a(la)=(la)a$。由分配律(对 $l$ 归纳)可得
		\[
		a(la)=l(a^2)=lka,
		\qquad
		(la)a=l(a^2)=lka,
		\]
		因此 $a(la)=(la)a$ 对一切整数 $l$ 成立。
		进一步,对任意 $r,s\in\{0,1,\dots,p-1\}$,
		\[
		(ra)(sa)=r\bigl(a(sa)\bigr)=r\bigl(s(a^2)\bigr)=rs\,a^2,
		\]
		同理
		\[
		(sa)(ra)=sr\,a^2=rs\,a^2.
		\]
		故 $(ra)(sa)=(sa)(ra)$,从而 $R$ 乘法交换,$R$ 为交换环。
		
		\medskip
		\textbf{(3) 构造乘法幺元.}
		由于 $p$ 为素数且 $k\in\{1,\dots,p-1\}$,有 $\gcd(k,p)=1$。
		由裴蜀定理存在整数 $s,t$ 使
		\[
		ks+pt=1.
		\]
		两边取模 $p$ 得 $ks\equiv 1\pmod p$,即 $p\mid(ks-1)$。
		令
		\[
		e:=sa\in R.
		\]
		对任意 $la\in R$,
		\[
		(la)e=(la)(sa)=sl\,a^2=slk\,a.
		\]
		而 $(ks-1)la=0$(因为 $p\mid(ks-1)$ 且 $pa=0$),故 $ks(la)=la$,
		从而 $slk\,a=la$,即 $(la)e=la$。
		同理可得 $e(la)=la$。
		因此
		\[
		er=re=r \quad(\forall r\in R),
		\]
		$e$ 为 $R$ 的乘法幺元。
		
		\medskip
		\textbf{(4) 每个非零元可逆,从而 $R$ 为域.}
		在前面已经得到 $R$ 为含幺交换无零因子环,且存在 $a\in R^\ast$ 使
		\[
		a^2=ka\qquad(k\in\mathbb Z,\ p\nmid k),
		\]
		并且存在整数 $s$ 使
		\[
		sa=e.
		\]
		(特别地,$a\neq 0$。)
		
		任取非零元 $x\in R^\ast$。由前面的结论(例如已证 $R=\{0,a,2a,\dots,(p-1)a\}$),
		存在唯一的 $l\in\{1,2,\dots,p-1\}$ 使 $x=la$。
		下证 $la$ 可逆。
		
		由于 $1\le l\le p-1$,故 $p\nmid l$;又由题设/前面推得 $p\nmid k$,
		于是 $p\nmid lk$,从而
		\[
		\gcd(lk,p)=1.
		\]
		则存在$v$和$u$,使得
		\[lkv+pu=1\]
		两边取模,则
		\[	lkv \equiv 1 \pmod p.\]
		两边乘上$s$
		\[	lkvs \equiv s \pmod p.\]
		,记$m=vs$,则存在整数 $m$ 使得
		\[
		lk\,m \equiv s \pmod p.
		\]
		记 $m\in\mathbb Z$ 为上述整数,并令 $y:=ma\in R$。计算得
		\[
		(la)(ma)=lm\,a^2=lm(ka)=lmk\,a.
		\]
		由 $lk\,m \equiv s \pmod p$ 知存在 $q\in\mathbb Z$ 使
		\[
		lmk-s=qp.
		\]
		由于 $\mathrm{char}(R)=p$,故 $p\cdot a=(p e)a=0$,于是
		\[
		(lmk-s)a=(qp)a=q(pa)=0.
		\]
		从而
		\[
		lmk\,a=sa=e,
		\]
		即
		\[
		(la)(ma)=e.
		\]
		由于 $R$ 交换,亦有
		\[
		(ma)(la)=(la)(ma)=e,
		\]
		故 $ma$ 是 $la$ 的逆元。由 $x\in R^\ast$ 的任意性可知 $R$ 的每个非零元都可逆,
		因此 $R$ 为域。
		
		
		\medskip
		\textbf{(5) $R\simeq \mathbb{Z}_p$.}
		定义映射
		\[
		\varphi:\mathbb{Z}_p\to R,\qquad \varphi(\overline{l})=la.
		\]
		若 $l\equiv l'\pmod p$,则 $l-l'=pt$,从而 $(l-l')a=pta=t(pa)=0$,故 $\varphi$ 良定义。
		又有
		\[
		\varphi(\overline{l}+\overline{m})=(l+m)a=la+ma=\varphi(\overline{l})+\varphi(\overline{m}),
		\]
		以及
		\[
		\varphi(\overline{l}\,\overline{m})=(lm)a
		= (la)(ma)\quad(\text{因为 } (la)(ma)=lm\,a^2 \text{ 且 } a^2=ka,\ e=sa \text{ 已构造出幺元}),
		\]
		因此 $\varphi$ 为环同态。
		由于 $a$ 生成加法群,$\varphi$ 显然为双射,故
		\[
		R\simeq \mathbb{Z}_p.
		\]
		\qed
		
		
		\item 12.
		\textbf{题目.}  
		设 $R$ 为幺环,证明:$R$ 中包含幺元 $e$ 的最小子环必与 $\mathbb{Z}_m\,(m>0)$ 或 $\mathbb{Z}$ 同构。
		
		\textbf{证明.}  
		设 $R$ 为幺环,其乘法单位元记为 $e$。  
		考虑由 $e$ 生成的加法子群:
		\[
		S = \{\, n e \mid n \in \mathbb{Z} \,\},
		\]
		其中规定
		\[
		n e = 
		\begin{cases}
			e + e + \cdots + e & (n>0),\\
			0 & (n=0),\\
			-((-n)e) & (n<0).
		\end{cases}
		\]
		显然,$S$ 在加法下封闭,并且对任意 $m,n\in\mathbb{Z}$ 有
		\[
		(m e)(n e) = (mn)e,
		\]
		故 $S$ 在乘法下也封闭,从而 $S$ 是 $R$ 的一个子环。下面证明 $S$ 确实是包含幺元 $e$ 的最小子环,显然 $S$ 是一个包含 $e$ 的子环。
		
		现在证明其\textbf{最小性}:设 $T$ 是 $R$ 的任意一个子环,且 $e\in T$。
		因为子环在加法下为子群,故对任意正整数 $n$,
		\[
		ne=\underbrace{e+e+\cdots+e}_{n\text{ 次}}\in T.
		\]
		又因为子环对加法逆元封闭,若 $n<0$,则
		\[
		ne=-( (-n)e)\in T,
		\]
		并且 $0=0\cdot e\in T$。
		因此对任意 $n\in\mathbb Z$ 都有 $ne\in T$,即
		\[
		S=\{ne\mid n\in\mathbb Z\}\subseteq T.
		\]
		由于 $T$ 是任意包含 $e$ 的子环,上式表明 $S$ 包含于所有包含 $e$ 的子环之中,
		从而 $S$ 是包含 $e$ 的\textbf{最小}子环。
		
		然后讨论两种情形。
		
		\textbf{(1) 若存在最小正整数 $m>0$ 使得 $m e = 0$.}  
		此时 $R$ 的\textbf{特征}为 $m$,此时
		\[
		S = \{ 0,\, e,\, 2e,\, \dots,\, (m-1)e \},
		\]
		其加法与乘法满足
		\[
		(a e) + (b e) = ((a+b)\bmod m)\,e, \qquad
		(a e)(b e) = ((ab)\bmod m)\,e.
		\]
		定义映射
		\[
		\varphi : \mathbb{Z}_m \longrightarrow S, \quad \varphi(\bar{k}) = k e.
		\]
		易证 $\varphi$ 保加法与乘法,且双射,故 $\varphi$ 是环同构。  
		因此 $S \cong \mathbb{Z}_m$。
		
		\textbf{(2) 若对任意 $m>0$,都有 $m e \ne 0$.}  
		则 $R$ 的特征为 $0$,此时映射
		\[
		\varphi : \mathbb{Z} \longrightarrow S, \quad \varphi(n) = n e
		\]
		是一个保持加法与乘法的同构,故 $S \cong \mathbb{Z}$。
		
		综上,$R$ 中包含幺元 $e$ 的最小子环必与 $\mathbb{Z}_m\,(m>0)$ 或 $\mathbb{Z}$ 同构。	\qed
		\item 17.
		\textbf{题目.}\;
		设 $m_1,m_2$ 为不同的正整数。
		证明:$m_1\mathbb{Z}$ 与 $m_2\mathbb{Z}$ 作为加法群同构,但作为环不同构。
		
		\textbf{答案:}
		
		\textbf{证明.}
		
		\textbf{(1) 作为加法群同构.}\;
		考虑映射
		\[
		\varphi: m_1\mathbb{Z}\longrightarrow m_2\mathbb{Z},\qquad
		\varphi(m_1k)=m_2k\quad (k\in\mathbb{Z}).
		\]
		\emph{良定义:}若 $m_1k=m_1k'$,则 $k=k'$,从而 $\varphi(m_1k)=\varphi(m_1k')$。
		
		\emph{同态性:}对任意 $k,\ell\in\mathbb{Z}$,
		\[
		\varphi(m_1k+m_1\ell)=\varphi(m_1(k+\ell))=m_2(k+\ell)=m_2k+m_2\ell
		=\varphi(m_1k)+\varphi(m_1\ell).
		\]
		\emph{双射性:}
		若 $\varphi(m_1k)=\varphi(m_1k')$,则 $m_2k=m_2k'$,故 $k=k'$,于是 $\varphi$ 单射;
		任取 $m_2t\in m_2\mathbb{Z}$,有 $m_2t=\varphi(m_1t)$,故 $\varphi$ 满射。
		因此 $\varphi$ 为加法群同构,故
		\[
		(m_1\mathbb{Z},+)\cong (m_2\mathbb{Z},+).
		\]
		
		\textbf{(2) 作为环不同构(当 $m_1\ne m_2$).}\;
		注意到 $m\mathbb{Z}$ 作为 $\mathbb{Z}$ 的子环(采用通常乘法),
		其所有元素均为 $m$ 的倍数。
		若 $m>1$,则 $m\mathbb{Z}$ 中不存在乘法幺元:
		假设存在 $e\in m\mathbb{Z}$ 使 $ex=x$ 对一切 $x\in m\mathbb{Z}$ 成立。
		取 $x=m$,则 $em=m$。
		写 $e=mt$($t\in\mathbb{Z}$),则
		\[
		em=(mt)\cdot m=m^2t=m,
		\]
		从而 $m^2t=m$,即 $mt=1$,矛盾(因 $m>1$ 时 $mt$ 不可能等于 $1$)。
		故当 $m>1$ 时,$m\mathbb{Z}$ 无幺元。
		
		接下来给出一个在环同构下保持的量来区分 $m_1\mathbb{Z}$ 与 $m_2\mathbb{Z}$。
		对任意正整数 $m$,元素 $m\in m\mathbb{Z}$ 满足
		\[
		(m\mathbb{Z}):=\{x\in m\mathbb{Z}\mid x\cdot m\in m^2\mathbb{Z}\}
		\]
		并且理想 $m^2\mathbb{Z}$ 在 $m\mathbb{Z}$ 中是由 $m\cdot m$ 生成的。
		更直接地,考虑商环
		\[
		m\mathbb{Z}/m^2\mathbb{Z}.
		\]
		其加法群阶为
		\[
		|m\mathbb{Z}/m^2\mathbb{Z}|=[m\mathbb{Z}:m^2\mathbb{Z}]=m.
		\]
		而环同构必诱导商环同构,因此若存在环同构
		\[
		m_1\mathbb{Z}\cong m_2\mathbb{Z},
		\]
		则必有
		\[
		m_1\mathbb{Z}/m_1^2\mathbb{Z}\cong m_2\mathbb{Z}/m_2^2\mathbb{Z},
		\]
		从而两边加法群阶相等,得到 $m_1=m_2$,与题设 $m_1\ne m_2$ 矛盾。
		
		因此当 $m_1\ne m_2$ 时,$m_1\mathbb{Z}$ 与 $m_2\mathbb{Z}$ 作为环不同构。
		
		综上,$m_1\mathbb{Z}$ 与 $m_2\mathbb{Z}$ 作为加法群同构,但作为环不同构。
		
		\item 19.
		\textbf{题目.}
		本题要证明整环的分式域的存在性和唯一性。设整环 $R$ 为域 $F$ 的子环,
		若对任何 $a\in F$,存在 $b,c\in R$ 使得 $a=bc^{-1}$,则称 $F$ 为 $R$ 的\emph{分式域}。
		我们用构造法证明分式域的存在性。
		
		在集合 $R* R^\ast$(其中 $R^\ast=R\setminus\{0\}$)上定义加法与乘法:
		\[
		(a,b)+(c,d)=(ad+bc,\,bd), \qquad
		(a,b)(c,d)=(ac,\,bd), \qquad (a,b),(c,d)\in R* R^\ast.
		\]
		
		\textbf{(1) 证明 $R* R^\ast$ 对上述加法与乘法都成为交换幺半群,单位元分别是 $(0,1)$、$(1,1)$.}
		
		\textbf{证明.}
		加法:对任意 $(a,b),(c,d),(e,f)$,
		\[
		\bigl((a,b)+(c,d)\bigr)+(e,f)=(ad+bc,bd)+(e,f)=((ad+bc)f+ebd,\,bdf),
		\]
		\[
		(a,b)+\bigl((c,d)+(e,f)\bigr)=(a,b)+(cf+ed,df)=(a df+b(cf+ed),\,bdf),
		\]
		两式右端在整环中相等,故结合律成立;交换性同理。
		$(0,1)$ 作加法单位元显然:$(a,b)+(0,1)=(a,b)$。
		
		乘法:同理可证结合、交换律;
		$(1,1)$ 作乘法单位元:$(a,b)(1,1)=(a,b)$。
		又因 $R$ 为整环,$bd\ne0$,所以上述运算闭合。\qed。
		
		\medskip
		
		\textbf{(2) 在 $R* R^\ast$ 中定义关系 $\sim$:$(a,b)\sim(c,d)\iff ad=bc$。
			证明 $\sim$ 是关于上述加法和乘法的同余关系。}
		
		\textbf{证明.}
		显然自反、对称、传递性成立,故为等价关系。
		设 $(a,b)\sim(a',b')$ 与 $(c,d)\sim(c',d')$,即 $ab'=a'b$、$cd'=c'd$。
		则
		\[
		\bigl((a,b)+(c,d)\bigr)\sim\bigl((a',b')+(c',d')\bigr)
		\Longleftrightarrow (ad+bc)b'd'=(a'd'+b'c')bd ,
		\]
		两边展开并利用 $ab'=a'b,\;cd'=c'd$ 即得。
		同理,
		\[
		(ac,bd)\sim(a'c',b'd')
		\Longleftrightarrow ac\,b'd'=a'c'\,bd,
		\]
		亦由 $ab'=a'b,\;cd'=c'd$ 得证。
		故 $\sim$ 关于两种运算均为同余关系。\qed。
		
		\medskip
		
		\textbf{(3) 令 $F=(R* R^\ast)/\sim$ 为等价类集合,将 $(a,b)$ 所在类记为 $\dfrac{a}{b}$。
			在 $F$ 上定义}
		\[
		\frac{a}{b}+\frac{c}{d}=\frac{ad+bc}{bd},\qquad
		\frac{a}{b}\cdot\frac{c}{d}=\frac{ac}{bd}.
		\]
		\textbf{证明 $F$ 对上述运算成为一个域。}
		
		\textbf{证明.}
		(2) 已证运算与代表元无关,因而良定义。
		从 (1) 继承可知 $(F,+)$ 为交换群,其中
		\[
		0_F=\frac{0}{1},\qquad -\frac{a}{b}=\frac{-a}{b}.
		\]
		$(F,\cdot)$ 为交换幺半群,$1_F=\frac{1}{1}$。
		若 $\dfrac{a}{b}\ne0_F$,则 $a\ne0$,其乘法逆元为 $\dfrac{b}{a}$,
		因为 $R$ 为整环,$a\in R^\ast$(在此处“$\ast$”指非零)保证合法,且
		\[
		\frac{a}{b}\cdot\frac{b}{a}=\frac{1}{1}=1_F.
		\]
		分配律由整环中的分配律直接计算可得:
		\[
		\frac{a}{b}\Bigl(\frac{c}{d}+\frac{e}{f}\Bigr)
		=\frac{a}{b}\cdot\frac{cf+de}{df}
		=\frac{a(cf+de)}{bdf}
		=\frac{ac}{bd}+\frac{ae}{bf}.
		\]
		因此 $F$ 为交换域。\qed。
		
		\medskip
		
		\textbf{(4) 试证明从 $R$ 到 $F$ 的映射 $\varphi:a\mapsto\dfrac{a}{1}$ 是单同态;
			因此可把 $R$ 看成 $F$ 的子环,进而证明 $F$ 是 $R$ 的一个分式域。}
		
		\textbf{证明.}
		$\varphi$ 保加法、乘法与单位元显然:
		\[
		\varphi(a+b)=\frac{a+b}{1}=\frac{a}{1}+\frac{b}{1},\qquad
		\varphi(ab)=\frac{ab}{1}=\frac{a}{1}\cdot\frac{b}{1},\qquad
		\varphi(1)=\frac{1}{1}.
		\]
		若 $\varphi(a)=\varphi(b)$,则 $\dfrac{a}{1}=\dfrac{b}{1}$,即 $a\cdot1=b\cdot1$,故 $a=b$,从而 $\varphi$ 单射。
		因此可把 $R$ 嵌入 $F$,并且 $F$ 中每个元素都形如 $a b^{-1}$(记作 $\dfrac{a}{b}$),
		这正是 $R$ 的分式域的定义。\qed。
		
		\medskip
		
		\textbf{(5) 设 $K$ 是包含 $R$ 的域。证明 $K$ 包含一个子域 $F_2\supset R$,且 $F_2$ 是 $R$ 的分式域,
			从而分式域在同构意义下唯一且为包含 $R$ 的最小域。}
		
		\textbf{证明.}
		在 $K$ 中取
		\[
		F_2:=\Bigl\{\;ab^{-1}\in K\;\Bigm|\;a\in R,\ b\in R^\ast\Bigr\}.
		\]
		易检验 $F_2$ 对加法、乘法封闭并含 $1$,且若 $ab^{-1}\ne0$,其逆为 $ba^{-1}$($a\ne0$ 时 $a^{-1}\in K$),
		从而 $F_2$ 为 $K$ 的子域,且显然包含 $R$。
		定义
		\[
		\psi:F\longrightarrow F_2,\qquad \psi\!\left(\frac{a}{b}\right)=ab^{-1}.
		\]
		由 (2)(3) 的良定义性,$\psi$ 为环同态;显然满射。
		若 $\psi\!\left(\dfrac{a}{b}\right)=\psi\!\left(\dfrac{c}{d}\right)$,则 $ab^{-1}=cd^{-1}$,即 $ad=bc$,
		故 $\dfrac{a}{b}=\dfrac{c}{d}$,于是 $\psi$ 单射。故 $\psi$ 为域同构,$F\cong F_2$。
		
		因此任何包含 $R$ 的域都包含一个与 $F$ 同构的子域;
		特别地,$F$ 是包含 $R$ 的最小域(“分式域”),且在同构意义下唯一。\qed。
		
	\end{enumerate}
	
	\clearpage
	\section{整环上的因子分解}
	\subsection{整除的性质}
	\textbf{定义.}
	设 $R$ 为一个整环,$a,b\in R$。若存在 $c\in R$ 使
	\[
	a = bc,
	\]
	则称 $a$ 能被 $b$ 整除,也称 $b$ 为 $a$ 的因子,记为 $b\mid a$;若不存在这样的 $c$,则记为 $b\nmid a$。
	
	\medskip
	\textbf{命题.}
	在整环 $R$ 中,整除关系 $\,\mid\,$ 具有下列性质:
	\begin{itemize}
		\item[(1)] 自反性:$a\mid a$;
		\item[(2)] $a\mid 0$,且 $1\mid a$;
		\item[(3)] 传递性:若 $a\mid b$ 且 $b\mid c$,则 $a\mid c$;
		\item[(4)] 若 $a\mid b$,则对任意 $c\in R$ 有 $a\mid bc$;
		\item[(5)] 若 $a\mid b$ 且 $a\mid c$,则对任意 $r,s\in R$ 有 $a\mid (rb+sc)$;
		\item[(6)] 若 $a\mid b$ 且 $b\mid a$,则 $a$ 与 $b$ 相伴,即存在单位 $u$ 使 $b=ua$。
	\end{itemize}
	
	\textbf{证明.}
	
	(1) 取 $c=1\in R$,有
	\[
	a = a\cdot 1,
	\]
	故按定义 $a\mid a$。
	
	(2) 由于 $0 = a\cdot 0$,取 $c=0$ 即得 $a\mid 0$。又
	\[
	a = 1\cdot a,
	\]
	故 $1\mid a$。
	
	(3) 假设 $a\mid b$ 且 $b\mid c$。由定义存在 $c_1,c_2\in R$ 使
	\[
	b = a c_1,\qquad c = b c_2.
	\]
	代入得
	\[
	c = b c_2 = (a c_1)c_2 = a(c_1 c_2),
	\]
	因此存在 $c_1c_2\in R$ 使 $c = a(c_1c_2)$,于是 $a\mid c$。
	
	(4) 若 $a\mid b$,则存在 $d\in R$ 使 $b=ad$。于是
	\[
	bc = (ad)c = a(dc),
	\]
	其中 $dc\in R$,故 $a\mid bc$。
	
	(5) 由 $a\mid b$ 与 $a\mid c$,存在 $d,e\in R$ 使
	\[
	b = ad,\qquad c = ae.
	\]
	任取 $r,s\in R$,有
	\[
	rb+sc = r(ad)+s(ae) = a(rd+se),
	\]
	而 $rd+se\in R$,故 $a\mid (rb+sc)$。
	
	(6) 若 $a\mid b$,则存在 $c\in R$ 使 $b = ca$;若 $b\mid a$,则存在 $d\in R$ 使 $a = db$。
	代入得到
	\[
	a = db = d(ca) = (dc)a.
	\]
	若 $a=0$,则 $b=ca=0$,此时 $a$ 与 $b$ 显然相伴;以下假设 $a\neq 0$。
	因为 $R$ 为整环,无零因子,故由
	\[
	a = (dc)a
	\]
	可在 $a\neq 0$ 时消去 $a$(见书上命题2.1.7),得到
	\[
	dc = 1.
	\]
	于是 $c$ 与 $d$ 互为逆元,$c$ 为单位,记 $u:=c\in R^*$,则
	\[
	b = ca = ua,
	\]
	故 $a$ 与 $b$ 相伴。\qed
	
	\subsection{$
		\mathbb{Z}[\sqrt{-1}]\text{的单位群为:} \{1,\,-1,\,\sqrt{-1},\,-\sqrt{-1}\}.
		$
	}
	\textbf{命题.}\;
	整数高斯环
	\[
	\mathbb{Z}[\sqrt{-1}]=\{a+b\sqrt{-1}\mid a,b\in\mathbb{Z}\}
	\]
	的单位群为
	\[
	\mathbb{Z}[\sqrt{-1}]^*=\{1,\,-1,\,\sqrt{-1},\,-\sqrt{-1}\}.
	\]
	
	\textbf{证明.}\;
	在 $\mathbb{Z}[\sqrt{-1}]$ 上定义范数映射
	\[
	N:\mathbb{Z}[\sqrt{-1}]\longrightarrow \mathbb{Z}_{\ge 0},\qquad
	N(a+b\sqrt{-1})=a^2+b^2.
	\]
	容易验证对任意 $\alpha,\beta\in\mathbb{Z}[\sqrt{-1}]$,有
	\[
	N(\alpha\beta)=N(\alpha)\,N(\beta).
	\]
	
	设 $\alpha\in\mathbb{Z}[\sqrt{-1}]$ 是单位,则存在 $\beta\in\mathbb{Z}[\sqrt{-1}]$ 使得
	\[
	\alpha\beta=1.
	\]
	对范数取值,得到
	\[
	1=N(1)=N(\alpha\beta)=N(\alpha)N(\beta).
	\]
	由于 $N(\alpha),N(\beta)\in\mathbb{Z}_{\ge0}$,只可能
	\[
	N(\alpha)=1.
	\]
	
	于是设 $\alpha=a+b\sqrt{-1}$,其中 $a,b\in\mathbb{Z}$,则
	\[
	a^2+b^2=1.
	\]
	该方程在整数中的解只有
	\[
	(a,b)=(\pm1,0)\quad\text{或}\quad(0,\pm1).
	\]
	因此
	\[
	\alpha=\pm1 \quad\text{或}\quad \alpha=\pm\sqrt{-1}.
	\]
	
	反过来,直接验证
	\[
	1\cdot1=1,\qquad (-1)^2=1,\qquad
	(\sqrt{-1})^2=-1,\qquad (-\sqrt{-1})^2=-1,
	\]
	可知 $\pm1,\pm\sqrt{-1}$ 均在 $\mathbb{Z}[\sqrt{-1}]$ 中可逆,因而都是单位。
	
	综上,
	\[
	\mathbb{Z}[\sqrt{-1}]^*=\{1,\,-1,\,\sqrt{-1},\,-\sqrt{-1}\}.
	\]
	
	\subsection{例2.5.4的详细证明以及范数的补充}
	
	\subsection{思考题2.5.5}
	先证明一个引理:
	\textbf{Lemma.} 在模 $5$ 下,任意整数 $x$ 的平方 $x^2$ 仅可能取值
	\[
	x^2 \equiv 0,1,4 \pmod{5}.
	\]
	
	\textbf{证明.}
	任取整数 $x$,由于模 $5$ 的剩余类只有
	\[
	0,1,2,3,4,
	\]
	我们分别计算它们的平方模 $5$ 的结果:
	\[
	\begin{aligned}
		0^2 &\equiv 0 \pmod{5},\\
		1^2 &\equiv 1 \pmod{5},\\
		2^2 &\equiv 4 \pmod{5},\\
		3^2 &\equiv 9 \equiv 4 \pmod{5},\\
		4^2 &\equiv 16 \equiv 1 \pmod{5}.
	\end{aligned}
	\]
	
	因此模 $5$ 下所有平方数的集合为
	\[
	\{0,1,4\}.
	\]
	\qed
	
	\bigskip
	
	\textbf{命题.} 不存在整数 $a,b$ 使得
	\[
	a^2 - 5b^2 = \pm 2.
	\]
	
	\textbf{证明.}
	设存在整数 $a,b$ 满足 $a^2 - 5b^2 = \pm 2$,两边取模 $5$ 得
	\[
	a^2 \equiv \pm 2 \pmod{5}.
	\]
	
	但根据上面的引理,$a^2$ 模 $5$ 只能为 $0,1,4$,而
	\[
	2 \notin \{0,1,4\}, \qquad -2 \equiv 3 \notin \{0,1,4\}.
	\]
	
	因此 $a^2 \equiv 2 \pmod{5}$ 或 $a^2 \equiv 3 \pmod{5}$ 不可能成立。
	
	这与方程 $a^2 - 5b^2 = \pm 2$ 的模 $5$ 推论矛盾,因此该方程在整数中无解。
	\qed
	\subsection{$R$ 为交换幺环,$u\in R$ 为单位,主理想 $\langle u\rangle = R$}
	\textbf{命题.}
	若 $R$ 为交换幺环,$u\in R$ 为单位,证明主理想 $\langle u\rangle = R$。
	
	\textbf{证明.}
	因 $u$ 为单位,故存在 $v\in R$ 使得
	\[
	uv=1.
	\]
	主理想 $\langle u\rangle$ 的定义为
	\[
	\langle u\rangle=\{ru\mid r\in R\}.
	\]
	
	\textbf{(1) 证明 $\langle u\rangle\subseteq R$.}
	任取 $x\in \langle u\rangle$,则 $\exists r\in R$ 使 $x=ru$,显然 $x\in R$,故 $\langle u\rangle\subseteq R$。
	
	\textbf{(2) 证明 $R\subseteq \langle u\rangle$.}
	由于 $uv=1$,可知
	\[
	1=uv\in \langle u\rangle
	\quad(\text{取 } r=v).
	\]
	而任取 $a\in R$,由理想对左乘封闭(或直接用定义)得
	\[
	a=a\cdot 1 \in a\langle u\rangle \subseteq \langle u\rangle.
	\]
	更具体地,因为 $1\in\langle u\rangle$,存在 $v\in R$ 使 $1=vu$,于是
	\[
	a=a\cdot 1=a(vu)=(av)u\in \langle u\rangle.
	\]
	故 $R\subseteq \langle u\rangle$。
	
	\medskip
	综上,$R\subseteq \langle u\rangle\subseteq R$,从而
	\[
	\langle u\rangle=R.
	\]
	
	\subsection{$R$ 为整环, $a\sim b$,主理想 $\langle a\rangle=\langle b\rangle$}
	\textbf{命题.}
	设 $R$ 为整环,$a,b\in R$。若 $a\sim b$(即 $a$ 与 $b$ 相伴),证明主理想 $\langle a\rangle=\langle b\rangle$。
	
	\textbf{证明.}
	\textbf{第一步:证明 $\langle a\rangle\subseteq \langle b\rangle$.}
	
	任取 $x\in \langle a\rangle$,则存在 $r\in R$ 使
	\[
	x=ra.
	\]
	由 $a=ub$ 得
	\[
	x=ra=r(ub)=(ru)b.
	\]
	由于 $ru\in R$,故 $x\in \langle b\rangle$,从而
	\[
	\langle a\rangle\subseteq \langle b\rangle.
	\]
	
	\textbf{第二步:证明 $\langle b\rangle\subseteq \langle a\rangle$.}
	
	由于 $u$ 为单位,其逆记为 $u^{-1}\in R$,由 $a=ub$ 得
	\[
	b=u^{-1}a.
	\]
	任取 $y\in \langle b\rangle$,则存在 $s\in R$ 使
	\[
	y=sb.
	\]
	代入上式,
	\[
	y=s(u^{-1}a)=(su^{-1})a.
	\]
	由于 $su^{-1}\in R$,故 $y\in \langle a\rangle$,从而
	\[
	\langle b\rangle\subseteq \langle a\rangle.
	\]
	
	\medskip
	综上,
	\[
	\langle a\rangle\subseteq \langle b\rangle
	\quad\text{且}\quad
	\langle b\rangle\subseteq \langle a\rangle,
	\]
	因此
	\[
	\langle a\rangle=\langle b\rangle.
	\]
	
	\hfill $\Box$
	
	
	
	\clearpage 
	\subsection*{课后习题答案}
	\addcontentsline{toc}{subsection}{\textcolor{red}{课后习题答案}}
	\begin{enumerate}[label=\textcolor{blue}{\textbf{\large\arabic*.}}]	
		\item \textbf{题目.}
		在 Gauss 整数环 $\mathbb{Z}[\sqrt{-1}]$ 中求出 $a+b\sqrt{-1}$ 的所有相伴元。
		
		\textbf{解:}
		$\mathbb{Z}[\sqrt{-1}]$ 中的单位恰为
		\[
		\{\pm 1,\ \pm i\}.
		\]
		因此,$a+bi$ 的所有相伴元为
		\[
		\{\;u(a+bi)\mid u\in\{1,-1,i,-i\}\;\}
		=
		\{\,a+bi,\ -a-bi,\ -b+ai,\ b-ai\,\}.
		\]
		\item 
		\textbf{题目.}
		证明 $7$ 和 $23$ 是 Gauss 整数环 $\mathbb{Z}[\sqrt{-1}]$ 中的不可约元素。
		$5$ 是这个环中的不可约元素吗?
		
		\textbf{解.}
		\textbf{一:$7$ 在 $\mathbb{Z}[\sqrt{-1}]$ 中不可约。}
		
		把 $7$ 看成实部为 $7$、虚部为 $0$ 的高斯整数 $7+0i$。
		其范数为
		\[
		N(7)=7^2=49.
		\]
		
		设 $7$ 在 $\mathbb{Z}[i]$ 中可约,即存在非单位 $\alpha,\beta\in\mathbb{Z}[i]$,
		使得
		\[
		7=\alpha\beta.
		\]
		对两边取范数得
		\[
		49=N(7)=N(\alpha)N(\beta).
		\]
		因为 $\alpha,\beta$ 不是单位,所以
		\[
		N(\alpha)>1,\quad N(\beta)>1,
		\]
		且都是正整数。于是 $(N(\alpha),N(\beta))$ 只能是 $7*7$(或对换),
		因为 $49$ 的正因子分解除了 $1*49$ 就只剩 $7*7$。
		
		因此若 $7$ 可约,则必有
		\[
		N(\alpha)=N(\beta)=7.
		\]
		也就是说,必须存在整数对 $(a,b)$ 使
		\[
		a^2+b^2=7.
		\]
		然而 $0^2,1^2,2^2,3^2,\dots$ 中,小于等于 $7$ 的平方只有 $0,1,4$,
		所有可能的和为
		\[
		0+0=0,\quad 0+1=1,\quad 0+4=4,\quad 1+1=2,\quad 1+4=5,\quad 4+4=8,
		\]
		没有得到 $7$。所以不存在整数解 $a^2+b^2=7$。
		
		这与上面推出的必要条件矛盾,因此 $7$ 在 $\mathbb{Z}[i]$ 中不能分解为两个非单位的乘积,故 $7$ 在 $\mathbb{Z}[i]$ 中是不可约元。	\qed
		
		\medskip
		\textbf{二:$23$ 在 $\mathbb{Z}[\sqrt{-1}]$ 中不可约。}
		
		同样把 $23$ 看成 $23+0i$,有
		\[
		N(23)=23^2=529.
		\]
		
		若 $23$ 在 $\mathbb{Z}[i]$ 中可约,则可写成
		\[
		23=\alpha\beta,
		\]
		其中 $\alpha,\beta$ 均为非单位。取范数得
		\[
		529=N(23)=N(\alpha)N(\beta),
		\]
		且 $N(\alpha),N(\beta)>1$ 为正整数。于是只能是
		\[
		N(\alpha)=N(\beta)=23,
		\]
		因为 $23$ 是普通整数环 $\mathbb{Z}$ 中的素数,$23^2$ 的因子分解只有
		$1*23^2$ 与 $23*23$。
		
		因此必须存在整数解 $a^2+b^2=23$。
		检验小于等于 $23$ 的平方数:$0,1,4,9,16$,所有可能的和为
		\[
		0,1,4,9,16,\;1+4=5,\;1+9=10,\;1+16=17,\;4+9=13,\;4+16=20,\;9+16=25,
		\]
		依然没有 $23$。故 $a^2+b^2=23$ 无整数解。
		
		于是不存在范数为 $23$ 的非单位,进而 $23$ 不能写成两个非单位的乘积,
		即 $23$ 在 $\mathbb{Z}[i]$ 中是不可约元。\qed
		
		\medskip
		\textbf{三:$5$ 在 $\mathbb{Z}[i]$ 中是否不可约?}
		
		同理,有
		\[
		N(5)=5^2=25.
		\]
		
		注意到
		\[
		1^2+2^2=1+4=5,
		\]
		所以存在高斯整数 $2+i$ 满足
		\[
		N(2+i)=2^2+1^2=5.
		\]
		于是
		\[
		N(2-i)=2^2+(-1)^2=5,
		\]
		而
		\[
		(2+i)(2-i)=4+1=5.
		\]
		
		由于 $N(2\pm i)=5>1$,$2\pm i$ 不是单位,因此 $5$ 在 $\mathbb{Z}[i]$ 中
		有非平凡分解
		\[
		5=(2+i)(2-i),
		\]
		所以 $5$ 不是不可约元(即是可约元)。
		
		\item 
		\textbf{题目.}\;
		设 $\mathbb Z[\sqrt{-1}]=\mathbb Z[i]$ 为 Gauss 整数环。对任意 $m,n\in\mathbb Z$,
		记它们在整数环 $\mathbb Z$ 中的最大公因子为
		\[
		d=\gcd_{\mathbb Z}(m,n)\quad(\text{通常取 }d>0),
		\]
		记它们在 Gauss 整数环 $\mathbb Z[i]$ 中的最大公因子为
		\[
		g=\gcd_{\mathbb Z[i]}(m,n).
		\]
		证明:$g$ 与 $d$ 在 $\mathbb Z[i]$ 中相伴(从而可把它们看作“相同”的最大公因子)。
		
		\textbf{答案:}\;
		\textbf{证明.}
		在 $\mathbb Z[i]$ 中的单位元为 $\{\pm 1,\pm i\}$。我们证明 $g\sim d$。
		
		\medskip
		\textbf{(1) 证明 $d\mid g$(在 $\mathbb Z[i]$ 中).}
		由于 $d=\gcd_{\mathbb Z}(m,n)$,故存在整数 $m_1,n_1\in\mathbb Z$ 使
		\[
		m=dm_1,\qquad n=dn_1.
		\]
		于是 $d$ 作为 Gauss 整数也整除 $m,n$,即 $d\mid m$ 且 $d\mid n$(在 $\mathbb Z[i]$ 中)。
		而 $g=\gcd_{\mathbb Z[i]}(m,n)$ 是 $m,n$ 的一个公因子,因此也有
		\[
		g\mid m,\qquad g\mid n.
		\]
		特别地,$g$ 作为 $m,n$ 的公因子,必整除它们在 $\mathbb Z$ 中的任何线性组合。
		由 Bézout 定理(在 $\mathbb Z$ 中)存在 $u,v\in\mathbb Z$ 使
		\[
		um+vn=d.
		\]
		由于 $g\mid m$ 且 $g\mid n$,从而 $g\mid (um+vn)=d$(仍在 $\mathbb Z[i]$ 中)。
		因此
		\[
		g\mid d \quad\text{于 }\mathbb Z[i].
		\]
		
		\medskip
		\textbf{(2) 证明 $g\mid d$ 推出 $d\mid g$ 亦成立,进而二者相伴.}
		另一方面,既然 $g$ 是 $m,n$ 在 $\mathbb Z[i]$ 中的\emph{最大}公因子,
		则任一公因子都整除 $g$。而 $d$ 在 $\mathbb Z[i]$ 中也整除 $m,n$(见上),
		故 $d$ 也是 $\mathbb Z[i]$ 中的一个公因子,从而
		\[
		d\mid g \quad\text{于 }\mathbb Z[i].
		\]
		
		\medskip
		\textbf{(3) 得到相伴.}
		由 (1)(2) 得 $g\mid d$ 且 $d\mid g$(均在 $\mathbb Z[i]$ 中),
		故存在 $\varepsilon\in\mathbb Z[i]$ 使
		\[
		g=\varepsilon d.
		\]
		再由 $d\mid g$ 与 $g\mid d$ 可知 $\varepsilon$ 必为 $\mathbb Z[i]$ 的单位元,
		即 $\varepsilon\in\{\pm 1,\pm i\}$。
		因此 $g\sim d$。
		
		\medskip
		综上,$m,n$ 在 $\mathbb Z$ 中的最大公因子 $d$ 与在 $\mathbb Z[i]$ 中的最大公因子 $g$
		相伴,从而在“相伴等价”意义下它们是相同的最大公因子。
		
		\item 7.
		\textbf{题目.}
		在环
		\[
		\mathbb{Z}[\sqrt{-5}]=\{\,a+b\sqrt{-5}\mid a,b\in\mathbb{Z}\,\}
		\]
		中证明 $3\nmid(1+2\sqrt{-5})$,并证明 $3$ 是不可约元素但不是素元素,
		因此 $\mathbb{Z}[\sqrt{-5}]$ 不是唯一分解整环。
		
		\textbf{证明.}
		
		\medskip
		\textbf{一、预备:范数与单位}
		
		对任意 $a+b\sqrt{-5}\in\mathbb{Z}[\sqrt{-5}]$ 定义范数
		\[
		N(a+b\sqrt{-5}) := a^{2}+5b^{2}\in\mathbb{Z}_{\ge0}.
		\]
		容易验证
		\[
		N(xy)=N(x)N(y),\qquad N(x)=0 \Longleftrightarrow x=0.
		\]
		事实上,若 $x=a+b\sqrt{-5}$,$y=c+d\sqrt{-5}$,则
		\[
		xy=(ac-5bd)+(ad+bc)\sqrt{-5},
		\]
		于是
		\[
		\begin{aligned}
			N(xy)
			&=(ac-5bd)^{2}+5(ad+bc)^{2}                      \\
			&=a^{2}c^{2}-10abcd+25b^{2}d^{2}+5a^{2}d^{2}+10abcd+5b^{2}c^{2} \\
			&=(a^{2}+5b^{2})(c^{2}+5d^{2})=N(x)N(y).
		\end{aligned}
		\]
		
		若 $u$ 是单位(可逆元),则存在 $v$ 使 $uv=1$,故
		\[
		1=N(1)=N(uv)=N(u)N(v),
		\]
		从而 $N(u)=1$。于是单位正是满足 $a^{2}+5b^{2}=1$ 的元。
		该丢番图方程只有解 $(a,b)=(\pm1,0)$,故
		\[
		\mathbb{Z}[\sqrt{-5}]^*=\{\pm1\}.
		\]
		
		\medskip
		\textbf{二、证明 $3\nmid(1+2\sqrt{-5})$}
		若 $3\mid (1+2\sqrt{-5})$,则
		\[
		9 = N(3)\mid N(1+2\sqrt{-5}) = 21,
		\]
		显然矛盾。 \qed
		
		\medskip
		\textbf{三、证明 $3$ 是不可约元素}
		
		在整环 $R$ 中,若 $p$ 非零且非单位,并且只要 $p=ab$ 就必有
		$a$ 或 $b$ 为单位,则称 $p$ 为不可约元素。
		
		显然 $3\neq 0$ 且 $3$ 不是单位(因为 $N(3)=9\neq1$),
		因此只需证明:若 $3=\alpha\beta$,则 $\alpha$ 或 $\beta$ 为单位。
		
		设
		\[
		3=\alpha\beta,\qquad \alpha,\beta\in\mathbb{Z}[\sqrt{-5}].
		\]
		对范数取值:
		\[
		N(3)=9=N(\alpha\beta)=N(\alpha)N(\beta),
		\]
		其中 $N(\alpha),N(\beta)\in\mathbb{Z}_{\ge1}$。
		若 $\alpha,\beta$ 都不是单位,则 $N(\alpha)\ge2$、$N(\beta)\ge2$。
		于是 $N(\alpha)N(\beta)\ge 4$,且必须等于 $9$。
		但 $9$ 的正因子只有 $1,3,9$,故唯一可能是
		\[
		N(\alpha)=3,\qquad N(\beta)=3.
		\]
		
		然而方程
		\[
		a^{2}+5b^{2}=3
		\]
		在整数中无解:若 $b=0$ 则 $a^{2}=3$ 不可能;若 $|b|\ge1$,则
		$a^{2}+5b^{2}\ge5>3$,也不可能。
		因此不存在范数为 $3$ 的元素。
		矛盾!
		
		所以假设错误,$\alpha$ 或 $\beta$ 必为单位。
		这表明 $3$ 是不可约元素。
		
		\medskip
		\textbf{四、证明 $3$ 不是素元素}
		
		在整环中,若 $p\neq0$ 且非单位,并且对任意 $a,b$ 有
		\[
		p\mid ab \Longrightarrow p\mid a \ \text{或}\ p\mid b,
		\]
		则称 $p$ 为素元素。
		
		计算
		\[
		(1+2\sqrt{-5})(1-2\sqrt{-5})
		=1-(2\sqrt{-5})^{2}
		=1-4(-5)
		=1+20=21=3\cdot7.
		\]
		因为 $7\in\mathbb{Z}\subset\mathbb{Z}[\sqrt{-5}]$,故
		\[
		(1+2\sqrt{-5})(1-2\sqrt{-5})=3\cdot7,
		\]
		也就是说
		\[
		3\mid (1+2\sqrt{-5})(1-2\sqrt{-5}).
		\]
		
		但在第二步中已证明 $3\nmid (1+2\sqrt{-5})$,
		类似可证 $3\nmid (1-2\sqrt{-5})$。
		于是存在 $a,b\in\mathbb{Z}[\sqrt{-5}]$(取
		$a=1+2\sqrt{-5}$,$b=1-2\sqrt{-5}$)使得
		\[
		3\mid ab,\qquad 3\nmid a,\qquad 3\nmid b.
		\]
		这与“素元素”的定义相矛盾,故 $3$ 不是素元素。
		
		\medskip
		\textbf{五、$\mathbb{Z}[\sqrt{-5}]$ 不是唯一分解整环}
		
		一般地,在任意整环中,若它是唯一分解整环,则
		\emph{每个不可约元素一定是素元素}。
		我们已经在上面证明了:
		\[
		3\ \text{是不可约元素但不是素元素}.
		\]
		因此 $\mathbb{Z}[\sqrt{-5}]$ 不能是唯一分解整环。
		
		\hfill$\Box$
		
		\item 8.
		\textbf{题目.}
		设 $R=\mathbb{Z}[\sqrt{-5}]$。
		
		(1) 试证明:若 $a^2+5b^2=9$,则 $a+b\sqrt{-5}$ 是不可约元素;
		
		(2) 试证明:元素 $\alpha=6+3\sqrt{-5},\ \beta=9$ 在 $R$ 中不存在最大公因子。
		
		\bigskip
		\textbf{预备:范数及其性质.}
		
		对任意 $z=a+b\sqrt{-5}\in R$,定义范数
		\[
		N(z):=a^2+5b^2\in\mathbb{N}.
		\]
		易见 $N(z)\ge 0$,且 $N(z)=0$ 当且仅当 $z=0$。并且
		\[
		N(zw)=N(z)N(w),\quad \forall z,w\in R,
		\]
		因为写 $z=a+b\sqrt{-5},\ w=c+d\sqrt{-5}$,则
		\[
		zw=(ac-5bd)+(ad+bc)\sqrt{-5},
		\]
		从而
		\[
		N(zw)=(ac-5bd)^2+5(ad+bc)^2
		=(a^2+5b^2)(c^2+5d^2)=N(z)N(w).
		\]
		
		若 $z$ 为单位,则存在 $w$ 使 $zw=1$,取范数得 $N(z)N(w)=N(1)=1$,
		故 $N(z)=1$。方程
		\[
		a^2+5b^2=1
		\]
		在整数中只有 $(a,b)=(\pm1,0)$ 两组解,因此 $R$ 的单位只有 $\pm1$。
		
		\bigskip
		\textbf{(1) 若 $a^2+5b^2=9$,则 $a+b\sqrt{-5}$ 不可约。}
		
		设 $a^2+5b^2=9$,记 $\alpha=a+b\sqrt{-5}$。
		反设 $\alpha$ 可约,则存在非单位 $u,v\in R$ 使得
		\[
		\alpha = uv.
		\]
		取范数,有
		\[
		N(\alpha)=N(u)N(v)=9.
		\]
		
		因为 $u,v$ 非单位,$N(u),N(v)>1$ 且为正整数,又整除 $9$,
		故
		\[
		N(u),N(v)\in\{3,9\}.
		\]
		
		\emph{若 $N(u)=9$(或 $N(v)=9$),}则 $N(v)=1$,$v$ 为单位,与假设矛盾。
		于是只能有
		\[
		N(u)=N(v)=3.
		\]
		
		现在证明:\emph{不存在范数为 $3$ 的元素}。若存在 $x=c+d\sqrt{-5}\in R$
		满足 $N(x)=3$,则
		\[
		c^2+5d^2=3.
		\]
		在模 $5$ 意义下,
		\[
		c^2\equiv 3\pmod{5}.
		\]
		但整数平方模 $5$ 只可能是 $0,1,4$,不可能为 $3$,矛盾。
		故 $R$ 中无元素范数为 $3$。
		
		于是 $N(u)=N(v)=3$ 不可能,从而否定了“$\alpha$ 可约”的假设,
		因此 $\alpha=a+b\sqrt{-5}$ 在 $R$ 中是\textbf{不可约元素}。
		\qed
		\bigskip
		
		
		\textbf{(1) 预备:范数与单位.}\;
		对 $R$ 中元素 $a+b\sqrt{-5}$ 定义范数
		\[
		N(a+b\sqrt{-5})=(a+b\sqrt{-5})(a-b\sqrt{-5})=a^2+5b^2\in\mathbb Z_{\ge 0}.
		\]
		则 $N(xy)=N(x)N(y)$。
		若 $u\in R$ 为单位,则存在 $v\in R$ 使 $uv=1$,从而
		\[
		1=N(1)=N(uv)=N(u)N(v),
		\]
		故 $N(u)=1$。而 $a^2+5b^2=1$ 仅有解 $(a,b)=(\pm 1,0)$,所以
		\[
		R^\times=\{\pm 1\}.
		\]
		
		\textbf{(2) 找到两个互不相伴的公共因子.}\;
		注意到
		\[
		\alpha=6+3\sqrt{-5}=3(2+\sqrt{-5}),\qquad
		\beta=9=3\cdot 3=(2+\sqrt{-5})(2-\sqrt{-5}).
		\]
		因此
		\[
		3\mid \alpha,\ 3\mid \beta;\qquad (2+\sqrt{-5})\mid \alpha,\ (2+\sqrt{-5})\mid \beta.
		\]
		并且
		\[
		N(3)=9,\qquad N(2+\sqrt{-5})=2^2+5\cdot 1^2=9.
		\]
		由(1)问结论($a^2+5b^2=9$ 时对应元素不可约)知 $3$ 与 $2+\sqrt{-5}$ 都是不可约元。
		它们不相伴:若 $3=\pm(2+\sqrt{-5})$ 显然不可能。
		
		再证它们互不整除:若 $3\mid (2+\sqrt{-5})$,则存在 $u$ 使
		\[
		2+\sqrt{-5}=3u \ \Longrightarrow\ 9=N(2+\sqrt{-5})=N(3)N(u)=9N(u)\ \Longrightarrow\ N(u)=1,
		\]
		从而 $u$ 为单位,推出 $2+\sqrt{-5}$ 与 $3$ 相伴,矛盾。
		同理 $(2+\sqrt{-5})\nmid 3$。
		
		所以 $3$ 与 $2+\sqrt{-5}$ 是两条\textbf{互不相伴且互不整除}的公共因子。
		
		\textbf{(3) 分类:$9$ 的一切因子的范数只能是 $1,9,81$.}\;
		若 $d\mid 9$,则存在 $q\in R$ 使 $9=dq$,取范数得
		\[
		81=N(9)=N(d)N(q),
		\]
		故 $N(d)$ 为 $81$ 的正因子,且必须能写成 $a^2+5b^2$ 的形式。
		
		检查 $81$ 的正因子:$1,3,9,27,81$。
		其中
		\[
		a^2+5b^2=3 \ \text{无整数解},\qquad a^2+5b^2=27 \ \text{亦无整数解},
		\]
		因此
		\[
		d\mid 9 \ \Longrightarrow\ N(d)\in\{1,9,81\}.
		\]
		并且:
		- 若 $N(d)=1$,则 $d$ 为单位;
		- 若 $N(d)=81$,则 $81=N(d)N(q)$ 迫使 $N(q)=1$,即 $q$ 为单位,从而 $d$ 与 $9$ 相伴。
		
		\textbf{(4) 反证不存在最大公因子.}\;
		反设 $\alpha,\beta$ 在 $R$ 中存在最大公因子 $g$。
		由于 $3$ 与 $2+\sqrt{-5}$ 都是 $\alpha,\beta$ 的公共因子,按“最大公因子”的定义必有
		\[
		3\mid g,\qquad (2+\sqrt{-5})\mid g,
		\]
		且 $g\mid 9$,于是由(3)知 $N(g)\in\{1,9,81\}$。
		
		- 若 $N(g)=1$,则 $g$ 为单位,不可能被不可约元 $3$ 整除,矛盾。
		- 若 $N(g)=9$,则 $g$ 必与某个范数为 $9$ 的元素相伴,特别地 $g$ 必与 $3$ 或 $2\pm\sqrt{-5}$ 相伴。
		但若 $g\sim 3$,则由 $(2+\sqrt{-5})\mid g$ 推出 $(2+\sqrt{-5})\mid 3$,与(2)矛盾;
		若 $g\sim (2+\sqrt{-5})$,则由 $3\mid g$ 推出 $3\mid (2+\sqrt{-5})$,同样矛盾;
		若 $g\sim (2-\sqrt{-5})$ 也同理矛盾。
		因此 $N(g)\neq 9$。
		- 只能是 $N(g)=81$。由(3)可知此时 $g\sim 9$,即 $g=\pm 9$。
		
		但 $g$ 还需满足 $g\mid \alpha$,于是 $\pm 9\mid (6+3\sqrt{-5})$。
		若 $6+3\sqrt{-5}=9(a+b\sqrt{-5})$,比较系数得
		\[
		9a=6,\quad 9b=3,
		\]
		从而 $a=\frac23,\ b=\frac13$,不属于 $\mathbb Z$,矛盾。
		
		综上,假设不成立,故 $\alpha$ 与 $\beta$ 在 $R$ 中\textbf{不存在最大公因子}。
		
		\textbf{(2) $\alpha=6+3\sqrt{-5},\ \beta=9$ 在 $R$ 中不存在最大公因子。}
		\textbf{命题.}\;
		在 $R=\mathbb Z[\sqrt{-5}]$ 中,元素
		\[
		\alpha=6+3\sqrt{-5},\qquad \beta=9
		\]
		不存在最大公因子。
		
		\textbf{证明.}\;
		采用反证法。设 $\alpha$ 与 $\beta$ 在 $R$ 中存在最大公因子 $d$。
		
		由于
		\[
		\alpha=3(2+\sqrt{-5}),\qquad \beta=9=3\cdot3,
		\]
		可知 $3\mid \alpha$ 且 $3\mid \beta$。
		由最大公因子的定义,任何公共因子都整除 $d$,因此
		\[
		3\mid d.
		\]
		于是存在 $a,b\in\mathbb Z$ 使得
		\[
		d=3(a+b\sqrt{-5}).
		\]
		
		又因为 $d\mid \beta=9$,根据范数的整除性,有
		\[
		N(d)\mid N(9).
		\]
		注意到在 $R$ 中范数定义为
		\[
		N(x+y\sqrt{-5})=x^2+5y^2,
		\]
		于是
		\[
		N(d)=N\bigl(3(a+b\sqrt{-5})\bigr)
		=9(a^2+5b^2),
		\qquad
		N(9)=81.
		\]
		因此
		\[
		9(a^2+5b^2)\mid 81,
		\]
		从而
		\[
		a^2+5b^2\mid 9.
		\]
		
		由于 $a^2+5b^2$ 为正整数,只可能取值
		\[
		a^2+5b^2\in\{1,3,9\}.
		\]
		逐一检查可得
		\[
		a+b\sqrt{-5}\in\{\pm1,\ \pm3,\ \pm(2\pm\sqrt{-5})\}.
		\]
		
		下面逐一排除不可能情形。
		
		\textbf{(i)} 若 $a+b\sqrt{-5}=\pm3$,则
		\[
		d=\pm9,
		\]
		但 $9\nmid \alpha=6+3\sqrt{-5}$(因为 $9\nmid 6$),矛盾。
		
		\textbf{(ii)} 若 $a+b\sqrt{-5}=\pm(2\pm\sqrt{-5})$,则
		\[
		2\pm\sqrt{-5}\mid 3.
		\]
		但
		\[
		N(2\pm\sqrt{-5})=9,\qquad N(3)=9,
		\]
		而若 $2\pm\sqrt{-5}\mid 3$,则应有
		\[
		N(2\pm\sqrt{-5})\mid N(3),
		\]
		这将迫使 $3=\pm(2\pm\sqrt{-5})$,显然不成立,故亦矛盾。
		
		因此只能有
		\[
		a+b\sqrt{-5}=\pm1,
		\]
		从而
		\[
		d\sim 3.
		\]
		
		然而注意到
		\[
		9=(2+\sqrt{-5})(2-\sqrt{-5}),\qquad
		\alpha=6+3\sqrt{-5}=3(2+\sqrt{-5}),
		\]
		可见 $2+\sqrt{-5}$ 同时整除 $\alpha$ 与 $\beta$。
		但
		\[
		2+\sqrt{-5}\nmid 3,
		\]
		说明 $3$ 并非它们的最大公因子。
		
		这与 $d\sim3$ 是最大公因子的结论相矛盾。
		
		故假设不成立,$\alpha$ 与 $\beta$ 在 $R$ 中不存在最大公因子。 \qed
		
		
		\item 11.
		\textbf{题目}
		试问在整环 $\mathbb{Z}[\sqrt{-3}]$ 中 $2(1+\sqrt{-3})$ 和 $4$ 是否存在最大公因子?
		
		\textbf{解.}
		
		\medskip
		\noindent\textbf{ 两个数的因子情况}
		
		记
		\[
		a:=2(1+\sqrt{-3}),\qquad b:=4.
		\]
		有分解
		\[
		b = 4 = 2\cdot 2 = (1+\sqrt{-3})(1-\sqrt{-3}),
		\]
		以及
		\[
		a = 2(1+\sqrt{-3}) = (1-\sqrt{-3})(-1+\sqrt{-3}),
		\]
		因为
		\[
		(1-\sqrt{-3})(-1+\sqrt{-3})
		= -1+\sqrt{-3}+ \sqrt{-3}-(-3)
		=2+2\sqrt{-3}=2(1+\sqrt{-3}).
		\]
		
		由此看出:
		\[
		2\mid a,\ 2\mid b;\qquad
		1+\sqrt{-3}\mid a,\ 1+\sqrt{-3}\mid b;\qquad
		1-\sqrt{-3}\mid a,\ 1-\sqrt{-3}\mid b,
		\]
		所以
		\[
		2,\ 1+\sqrt{-3},\ 1-\sqrt{-3}
		\]
		都是 $a$ 与 $b$ 的公共因子,并且它们都是不可约且两两不相伴。
		
		\medskip
		\noindent\textbf{(3) 由“若存在最大公因子则导致矛盾”证明其不存在}
		
		假设在 $\mathbb{Z}[\sqrt{-3}]$ 中存在 $a$ 与 $b$ 的最大公因子 $d$。
		按最大公因子的定义:
		
		\begin{enumerate}
			\item $d$ 是 $a$ 与 $b$ 的公共因子,即 $d\mid a$ 且 $d\mid b$;
			\item 若 $e$ 是 $a$ 与 $b$ 的任一公共因子,则 $e\mid d$。
		\end{enumerate}
		
		由 (2) 知 $2,\ 1+\sqrt{-3},\ 1-\sqrt{-3}$ 都是公共因子,于是
		\[
		2\mid d,\qquad 1+\sqrt{-3}\mid d,\qquad 1-\sqrt{-3}\mid d.
		\]
		任取 $d$ 的一个分解为不可约元的乘积(利用范数的升链条件可知此类分解存在),
		由于 $2,\ 1+\sqrt{-3},\ 1-\sqrt{-3}$ 是不可约且不相伴,
		且分别整除 $d$,故在 $d$ 的任何不可约分解中,必各出现一个与 $2$、
		$1+\sqrt{-3}$、$1-\sqrt{-3}$ 相伴的因子。于是范数满足
		\[
		N(d)\ \geq\ N(2)\,N(1+\sqrt{-3})\,N(1-\sqrt{-3})
		=4\cdot4\cdot4=64.
		\]
		
		另一方面,由于 $d\mid b=4$,有
		\[
		N(d) \mid N(4)=16,
		\]
		故 $N(d)\leq16$。于是得到
		\[
		64 \leq N(d) \leq 16,
		\]
		矛盾。
		
		因此假设不成立,在 $\mathbb{Z}[\sqrt{-3}]$ 中
		\[
		2(1+\sqrt{-3}) \quad\text{和}\quad 4
		\]
		\emph{不存在}最大公因子。
		
		\medskip
		\textbf{答:}在整环 $\mathbb{Z}[\sqrt{-3}]$ 中,$2(1+\sqrt{-3})$ 与 $4$ 没有最大公因子。
		
		\item 12.
		\textbf{题目 } 举例说明,一个唯一分解整环的子环未必是唯一分解整环。
		
		\textbf{解.}
		
		取域 $k=\mathbb{Q}$,考虑二元多项式环
		\[
		R:=k[x,y]=\mathbb{Q}[x,y].
		\]
		因为 $k$ 是域,$k[x,y]$ 是在 UFD 上反复添变量得到的多项式环,所以 $R$ 是唯一分解整环(UFD)。
		
		现在取 $R$ 的子环
		\[
		S:=k[x^2,xy,y^2]\subset k[x,y].
		\]
		即 $S$ 由三个元素 $x^2,xy,y^2$ 生成的 $k$-子代数,由此显然 $S$ 是 $R$ 的一个子环。
		
		\medskip
		\noindent\textbf{(1) $S$ 中元素的次数性质}
		
		注意到
		\[
		\deg(x^2)=\deg(xy)=\deg(y^2)=2.
		\]
		于是由这三个元生成的任意非零非常数多项式,其各项的\emph{总次数}都是偶数。
		换言之,$S$ 中任何非常数元的总次数都是偶数,不存在总次数为 $1$ 的非零元。
		
		\medskip
		\noindent\textbf{(2) $x^2,xy,y^2$ 在 $S$ 中是不可约元}
		
		以 $x^2$ 为例。若在 $S$ 中存在非平凡分解
		\[
		x^2 = fg,\qquad f,g\in S,\ \deg f>0,\ \deg g>0,
		\]
		则有
		\[
		\deg x^2 = \deg f + \deg g.
		\]
		左边为 $2$,右边是两个正偶数之和,不可能等于 $2$,矛盾。
		因此 $x^2$ 在 $S$ 中是不可约元。
		
		同理,因为 $\deg(xy)=2,\ \deg(y^2)=2$ 且 $S$ 中所有非常数元的次数为偶数,
		故 $xy$ 与 $y^2$ 在 $S$ 中也都是不可约元。
		
		\medskip
		\noindent\textbf{(3) $x^2,xy,y^2$ 彼此不相伴}
		
		$S$ 的单位与 $R=k[x,y]$ 的单位相同,都是 $k^*=\mathbb{Q}^*$。
		若 $x^2$ 与 $xy$ 相伴,则存在非零常数 $\lambda\in k^*$ 使
		\[
		x^2=\lambda xy,
		\]
		即
		\[
		x^2-\lambda xy=0
		\]
		作为多项式恒等成立。但左边看作 $k[x,y]$ 中的多项式,其 $x^2$ 项系数为 $1$,
		而 $\lambda xy$ 中 $x^2$ 项的系数为 $0$,矛盾。
		故 $x^2$ 与 $xy$ 不相伴。同理可证 $x^2$ 与 $y^2$、$xy$ 与 $y^2$ 也都不相伴。
		
		\medskip
		\noindent\textbf{(4) 在 $S$ 中存在非唯一分解}
		
		考虑元
		\[
		x^2y^2 \in S.
		\]
		在 $S$ 中有两种分解:
		\[
		x^2y^2 = (x^2)(y^2),\qquad x^2y^2 = (xy)(xy).
		\]
		其中
		\[
		x^2,\ y^2,\ xy
		\]
		均为不可约元,且三者两两不相伴。因此,上述两种分解都是“分解成不可约元的分解”,
		但它们并非只是在排列和相伴意义下等价:$(x^2,y^2)$ 与 $(xy,xy)$ 中的因子既不对应、
		也不相伴。
		
		于是,$x^2y^2$ 在 $S$ 中的分解不唯一,说明 $S$ \emph{不是}唯一分解整环。
		
		\medskip
		\noindent\textbf{结论.}
		$R=\mathbb{Q}[x,y]$ 是唯一分解整环,而其子环
		\[
		S=\mathbb{Q}[x^2,xy,y^2]
		\]
		却不是唯一分解整环。由此说明:\emph{一个唯一分解整环的子环未必是唯一分解整环。}
		\qed
		
		
	\end{enumerate}
	\newpage
	\section{素理想与极大理想}
	\subsection{思考题2.6.4}
	\textbf{命题.}
	设 $m$ 为大于 $1$ 的正整数,考虑剩余类环
	\[
	\mathbb{Z}_m = \mathbb{Z}/m\mathbb{Z}.
	\]
	试确定 $\mathbb{Z}_m$ 的所有素理想。
	
	\textbf{解.}
	$\mathbb{Z}_m$ 的所有素理想恰为
	\[
	\boxed{\ \{\, (p) \subseteq \mathbb{Z}_m \mid p \text{ 为整除 } m \text{ 的素数}\,\}\ }.
	\]
	即所有由整除 $m$ 的素数所生成的主理想。
	
	\medskip
	\textbf{证明.}
	
	我们记自然投影为
	\[
	\pi:\mathbb{Z} \longrightarrow \mathbb{Z}_m,
	\qquad
	a\mapsto \bar{a}.
	\]
	由环同态基本性质,$\mathbb{Z}_m$ 中的任意理想 $I$ 均可写成
	\[
	I=\pi(J) = J/m\mathbb{Z},
	\]
	其中 $J$ 是 $\mathbb{Z}$ 中的某个理想,且满足
	\[
	m\mathbb{Z} \subseteq J \subseteq \mathbb{Z}.
	\]
	
	\medskip
	\textbf{第一步:确定所有可能的 $J$.}
	
	由于 $\mathbb{Z}$ 是主理想整环,每个理想 $J$ 均形如 $J = d\mathbb{Z}$,其中 $d$ 为正整数。结合 $m\mathbb{Z} \subseteq d\mathbb{Z}$,得到
	\[
	d \mid m.
	\]
	因此 $\mathbb{Z}_m$ 中的所有理想全部为
	\[
	I_d = d\mathbb{Z}/m\mathbb{Z}
	= \{\ \bar{0},\; \bar{d},\; \bar{2d}, \dots \ \},
	\qquad d\mid m.
	\]
	
	\medskip
	\textbf{第二步:判定何时 $I_d$ 为素理想。}
	
	回忆:理想 $I_d$ 在 $\mathbb{Z}_m$ 中是素理想,当且仅当商环
	\[
	\mathbb{Z}_m / I_d
	\]
	是整环。
	
	利用同构定理,
	\[
	\mathbb{Z}_m / I_d
	\cong 
	\mathbb{Z}/d\mathbb{Z}.
	\]
	因此 $I_d$ 为素理想 $\iff \mathbb{Z}/d\mathbb{Z}$ 为整环。
	
	\medskip
	但 $\mathbb{Z}/d\mathbb{Z}$ 是整环 $\iff d$ 为素数。
	
	于是得到判别结论:
	\[
	I_d \text{ 为素理想 }
	\iff d \mid m \text{ 且 } d \text{ 为素数}.
	\]
	
	\medskip
	\textbf{第三步:将 $I_d$ 表示为 $\mathbb{Z}_m$ 中的主理想。}
	
	在 $\mathbb{Z}_m$ 中,
	\[
	I_d = (\bar{d}).
	\]
	并且当 $d=p$ 为素数时,$(\bar{p})$ 即为所有素理想。
	
	由此可得全部素理想为
	\[
	\boxed{
		\text{$\mathbb{Z}_m$ 的素理想为 $(\bar{p})$,其中 $p$ 为整除 $m$ 的素数。}
	}
	\]
	
	\medskip
	综上,$\mathbb{Z}_m$ 的所有素理想正是由整除 $m$ 的素数所生成的主理想,命题得证。
	
	\subsection{思考题2.6.13}
	\textbf{命题.}
	设 $R$ 为交换幺环. 证明下列两个条件等价:
	\begin{itemize}
		\item[(1)] $R$ 是域;
		\item[(2)] $\{0\}$ 是 $R$ 的极大理想.
	\end{itemize}
	
	\textbf{解:}
	
	\textbf{(1)$\Rightarrow$(2).}  
	设 $R$ 是域. 取 $I$ 为 $R$ 的任一非零理想, 则存在 $0\neq a\in I$.
	由于 $R$ 是域, 每个非零元都可逆, 故 $a$ 在 $R$ 中存在逆元 $a^{-1}$, 于是
	\[
	1 = aa^{-1}\in I.
	\]
	由理想的定义, $1\in I$ 即推出 $I=R$.  
	因此 $R$ 只有两个理想:$\{0\}$ 与 $R$ 本身, 其中 $\{0\}$ 是真理想且没有更大的真理想包含它, 故 $\{0\}$ 为极大理想.
	
	\textbf{(2)$\Rightarrow$(1).}  
	设 $R$ 为交换幺环, 并且 $\{0\}$ 是 $R$ 的极大理想.  
	取任意非零元 $a\in R$, 考虑由 $a$ 生成的主理想
	\[
	<a>=Ra=\{ra\mid r\in R\}.
	\]
	显然 $a\in (a)$, 故 $(a)\neq \{0\}$.  
	又因为 $\{0\}$ 是极大理想, 所以任一非零理想必为整个环, 因此
	\[
	<a>=R.
	\]
	于是存在某个 $r\in R$ 使得
	\[
	1\in (a) \quad\Longrightarrow\quad 1=ra.
	\]
	由于 $R$ 交换, 亦有 $1=ar$, 故 $r$ 是 $a$ 的逆元.  
	因此 $R$ 中每个非零元都可逆, 即 $R$ 是域.
	
	综上可得:含幺交换环 $R$ 是域当且仅当 $\{0\}$ 是 $R$ 的极大理想. \qed
	
	\subsection{ $R$ 为整环,若主理想 $\langle p\rangle$ 为素理想,则 $p$ 是不可约元}
	\textbf{命题.}
	设 $R$ 为整环,$p\in R$ 为非零非单位元素。
	若主理想 $\langle p\rangle$ 为素理想,则 $p$ 是不可约元。
	
	\textbf{证明.}
	我们要证明:若 $p = ab$,则 $a$ 或 $b$ 至少有一个是单位。
	
	因为 $\langle p\rangle$ 为素理想,按定义有:
	\[
	\langle p\rangle \neq R,\quad
	\forall a,b\in R,\ ab\in \langle p\rangle \;\Rightarrow\; a\in \langle p\rangle \ \text{或}\ b\in \langle p\rangle.
	\]
	
	现在取任意分解
	\[
	p = ab,\quad a,b\in R.
	\]
	显然 $p\in \langle p\rangle$,于是
	\[
	ab = p \in \langle p\rangle.
	\]
	由 $\langle p\rangle$ 为素理想,得到
	\[
	a \in \langle p\rangle \quad\text{或}\quad b \in \langle p\rangle.
	\]
	
	不妨设 $a\in \langle p\rangle$,则存在 $r\in R$,使得
	\[
	a = rp.
	\]
	将 $p = ab$ 代入,得到
	\[
	p = ab = (rp)b = rpb.
	\]
	整理得
	\[
	p - rpb = 0 \quad\Rightarrow\quad (1 - rb)p = 0.
	\]
	
	由于 $R$ 是整环且 $p\neq 0$,$R$ 中无零因子,所以由
	\[
	(1 - rb)p = 0
	\]
	可推出
	\[
	1 - rb = 0 \quad\Rightarrow\quad rb = 1.
	\]
	因此 $b$ 在 $R$ 中存在逆元 $r$,即 $b$ 是单位。
	
	同理,若一开始是 $b\in \langle p\rangle$,则可以得到 $a$ 是单位。
	
	综上所述,对任意分解 $p = ab$,必有 $a$ 或 $b$ 为单位。按不可约元的定义,这说明
	\[
	p \text{ 是不可约元}.
	\]\qed
	\subsection{设 $R$ 为交换幺环,$p\in R$。
		则$p \text{ 为素元}\ \Longleftrightarrow\ \langle p\rangle \text{ 为素理想}$}
	\textbf{命题.}\;
	设 $R$ 为交换幺环,$p\in R$。
	则
	\[
	p \text{ 为素元}\ \Longleftrightarrow\ \langle p\rangle \text{ 为素理想}.
	\]
	
	\textbf{证明.}
	\textbf{($\Rightarrow$)}\;
	设 $p$ 为素元。
	先证 $\langle p\rangle\neq R$。
	若 $\langle p\rangle=R$,则 $1\in\langle p\rangle$,存在 $r\in R$ 使 $rp=1$,
	这意味着 $p$ 是单位,矛盾。
	故 $\langle p\rangle\neq R$。
	
	再证素性:任取 $a,b\in R$,若 $ab\in\langle p\rangle$,
	则存在 $r\in R$ 使
	\[
	ab=rp,
	\]
	即 $p\mid ab$。
	由 $p$ 为素元,得 $p\mid a$ 或 $p\mid b$,
	等价于 $a\in\langle p\rangle$ 或 $b\in\langle p\rangle$。
	因此 $\langle p\rangle$ 为素理想。
	
	\textbf{($\Leftarrow$)}\;
	设 $\langle p\rangle$ 为素理想。
	则 $\langle p\rangle\neq R$,从而 $1\notin\langle p\rangle$,
	即不存在 $r\in R$ 使 $rp=1$,故 $p$ 不是单位。
	
	又由于素理想一般约定为真理想,故 $p\notin R^\times$ 且 $p\neq 0$(在整环中自动成立;一般交换环中常将 ``素元'' 定义为非零非单位)。
	现取任意 $a,b\in R$,若 $p\mid ab$,则存在 $r\in R$ 使
	\[
	ab=rp,
	\]
	即 $ab\in\langle p\rangle$。
	由 $\langle p\rangle$ 为素理想,得
	\[
	a\in\langle p\rangle\ \text{或}\ b\in\langle p\rangle,
	\]
	即 $p\mid a$ 或 $p\mid b$。
	因此 $p$ 为素元.\qed
	
	
	\subsection{在交换幺环 $R$ 中,每个极大理想都是素理想}
	\label{subsec:maximal_is_prime}
	\textbf{命题.}在交换幺环 $R$ 中,每个极大理想都是素理想。
	
	\textbf{证明.}
	因为$M$极大,所以$R/M $为域,所以$R/M $为整环,而$R \ne M $,所以$M$为素理想。\qed
	
	\clearpage 
	\subsection*{课后习题答案}
	\addcontentsline{toc}{subsection}{\textcolor{red}{课后习题答案}}
	\begin{enumerate}[label=\textcolor{blue}{\textbf{\large\arabic*.}}]	
		\item 2.
		在多项式环 $\mathbb{Z}[x]$ 中,理想
		\[
		\langle x\rangle =\{f(x)x \mid f(x)\in\mathbb{Z}[x]\}
		\]
		是一个素理想,但不是极大理想。
		
		\textbf{证明.}
		我们分两步进行:先证明 $\langle x\rangle$ 为素理想,再证明它不是极大理想。
		
		\medskip
		\noindent\textbf{一步:构造商环并证明 $\langle x\rangle$ 为素理想。}
		
		记 $R=\mathbb{Z}[x]$,考虑“在 $x=0$ 处取值”的映射
		\[
		\varphi : R \longrightarrow \mathbb{Z},\qquad
		\varphi(f(x)) := f(0).
		\]
		
		\medskip
		\noindent\emph{(1) $\varphi$ 是环同态。}
		
		任意 $f(x),g(x)\in R$,有
		\[
		\varphi(f(x)+g(x))
		= (f+g)(0)
		= f(0)+g(0)
		= \varphi(f(x))+\varphi(g(x)),
		\]
		\[
		\varphi(f(x)g(x))
		= (fg)(0)
		= f(0)g(0)
		= \varphi(f(x))\varphi(g(x)),
		\]
		同时
		\[
		\varphi(1)=1.
		\]
		故 $\varphi$ 保持加法、乘法与单位元,是一个环同态。
		
		\medskip
		\noindent\emph{(2) 计算 $\ker\varphi$。}
		
		设 $f(x)=a_0+a_1x+\cdots+a_nx^n\in R$($a_i\in\mathbb{Z}$),则
		\[
		\varphi(f(x))=f(0)=a_0.
		\]
		因此
		\[
		f(x)\in\ker\varphi
		\iff f(0)=0
		\iff a_0=0.
		\]
		当 $a_0=0$ 时,可以把 $f(x)$ 中的每一项都提取一个 $x$:
		\[
		f(x)=a_1x+a_2x^2+\cdots+a_nx^n
		=(a_1+a_2x+\cdots+a_nx^{n-1})x
		=g(x)x,
		\]
		其中 $g(x)\in\mathbb{Z}[x]$。于是
		\[
		\ker\varphi
		=\{f(x)\in\mathbb{Z}[x]\mid f(0)=0\}
		=\{g(x)x\mid g(x)\in\mathbb{Z}[x]\}
		=\langle x\rangle.
		\]
		即
		\[
		\ker\varphi=\langle x\rangle.
		\]
		
		\medskip
		\noindent\emph{(3) 计算 $\operatorname{Im}\varphi$。}
		
		任取 $a\in\mathbb{Z}$,把它看作常数多项式 $f(x)\equiv a$,
		则 $\varphi(f(x))=f(0)=a$。因此任意整数都是某个多项式在 $0$ 处的取值,
		故
		\[
		\operatorname{Im}\varphi=\mathbb{Z}.
		\]
		
		\medskip
		\noindent\emph{(4) 用同态基本定理得到商环结构。}
		
		由环同态基本定理(第一同构定理),有
		\[
		\mathbb{Z}[x]/\ker\varphi
		\;\cong\;
		\operatorname{Im}\varphi.
		\]
		结合 (2)、(3) 得
		\[
		\mathbb{Z}[x]/\langle x\rangle\;\cong\;\mathbb{Z}.
		\]
		
		\medskip
		\noindent\emph{(5) 由商环为整环得到素理想。}
		显然,$\ker\varphi=\langle x\rangle \ne \mathbb{Z}[x]$,则由定理2.6.5得$\langle x\rangle$为$\mathbb{Z}[x]$的素理想。
		
		\medskip
		\noindent\textbf{二步:证明 $\langle x\rangle$ 不是极大理想。}
		
		在交换幺环中,一个真理想 $M$ 为\emph{极大理想}的充要条件是
		商环 $R/M$ 为\emph{域}(每个非零元都有乘法逆元)。
		
		由上一步我们已经知道
		\[
		\mathbb{Z}[x]/\langle x\rangle\cong\mathbb{Z}.
		\]
		然而,$\mathbb{Z}$ 并不是域,所以 $\langle x\rangle$ 是素理想但不是极大理想。\qed
		··
		\item 5.
		\textbf{题目.}\;
		设 $R=\{2n\mid n\in\mathbb Z\}=2\mathbb Z$,证明:$(4)$ 是 $R$ 的极大理想;并判断 $R/(4)$ 是否为域。
		\textbf{答案:}\;
		\textbf{解.}
		这里 $R=2\mathbb Z$ 取通常的加法、乘法;$(4)$ 指 $R$ 中由 $4$ 生成的理想,即
		\[
		(4)=4\mathbb Z=\{4n\mid n\in\mathbb Z\}\subseteq 2\mathbb Z.
		\]
		
		\textbf{(1) 先刻画 $R$ 的所有理想.}
		注意到 $R$ 的加法群 $(R,+)$ 是由 $2$ 生成的无限循环群,因此它的任意加法子群必形如
		\[
		2m\mathbb Z\qquad (m\in\mathbb Z_{\ge 0}).
		\]
		下面说明这些子群全都是 $R$ 的理想:若 $I=2m\mathbb Z$,取任意 $r\in R$ 与 $x\in I$,
		可写 $r=2a,\ x=2mb$($a,b\in\mathbb Z$),则
		\[
		rx=(2a)(2mb)=4amb=2m(2ab)\in 2m\mathbb Z=I,
		\]
		因此 $I$ 对 $R$ 的乘法吸收,故 $I$ 是理想。
		
		反过来,任意理想必是加法子群,所以 $R$ 的理想恰为
		\[
		\{\,2m\mathbb Z\mid m\in\mathbb Z_{\ge 0}\,\}.
		\]
		
		\textbf{(2) 证明 $(4)$ 是 $R$ 的极大理想.}
		设 $I$ 是 $R$ 的理想,且满足
		\[
		(4)\subseteq I\subseteq R.
		\]
		由上一步,存在 $m\ge 0$ 使得 $I=2m\mathbb Z$。把包含关系翻译成整除关系:
		\[
		4\mathbb Z\subseteq 2m\mathbb Z\quad\Longleftrightarrow\quad 2m\mid 4,
		\]
		而 $2m\mathbb Z\subseteq 2\mathbb Z$ 恒成立(因为 $2m$ 一定是 $2$ 的倍数)。
		因此只可能有
		\[
		2m=4 \ \text{或}\ 2m=2,
		\]
		即
		\[
		I=4\mathbb Z=(4)\quad\text{或}\quad I=2\mathbb Z=R.
		\]
		不存在严格夹在 $(4)$ 与 $R$ 之间的理想,所以 $(4)$ 是 $R$ 的极大理想。
		
		\textbf{(3) 判断 $R/(4)$ 是否为域.}
		因为 $R$ 中元素都是偶数,模 $4$ 只可能同余于 $0$ 或 $2$,所以商环只有两个陪集:
		\[
		R/(4)=\{\, (4),\ 2+(4)\,\}.
		\]
		并且在商环中
		\[
		(2+(4))\cdot(2+(4))=4+(4)=(4),
		\]
		也就是说,商环里唯一的非零元 $2+(4)$ 的平方等于 $0$,它是零因子,因而不可能是域。
		
		另外,从“域必须有乘法单位元 $1$”这一常用定义看:$R$ 本身不含 $1$,所以 $R/(4)$ 也不含 $1$,
		也不可能是域。
		
		\textbf{结论.}\;
		$(4)$ 是 $R=2\mathbb Z$ 的极大理想;但 $R/(4)$ 不是域(它含零因子,且无乘法单位元)。
		
		
		
		
		\item 13.
		\textbf{命题.}
		设 $R$ 为唯一分解整环(UFD),$p\in R$ 为非零非单位。
		则
		\[
		p \text{ 为不可约元}
		\quad\Longleftrightarrow\quad
		\langle p\rangle \text{ 为素理想}.
		\]
		
		\textbf{证明.}
		在整个证明中,约定 $R$ 是整环,因而具有消去律。
		
		\medskip
		\noindent\textbf{方向一:若 $p$ 不可约,则 $\langle p\rangle$ 为素理想。}
		
		回顾两个定义:
		
		\begin{itemize}
			\item 元素 $p$ 称为\emph{素元},若 $p\neq 0$、$p$ 不是单位,
			且对任意 $a,b\in R$,
			\[
			p\mid ab \;\Rightarrow\; p\mid a \text{ 或 } p\mid b.
			\]
			\item 理想 $P$ 称为\emph{素理想},若 $P\neq R$,且对任意 $a,b\in R$,
			\[
			ab\in P \;\Rightarrow\; a\in P \text{ 或 } b\in P.
			\]
		\end{itemize}
		
		很容易验证:对非零非单位 $p$,下面两件事等价:
		\[
		p \text{ 为素元}
		\quad\Longleftrightarrow\quad
		\langle p\rangle \text{ 为素理想},
		\]
		因为
		\[
		ab\in \langle p\rangle \iff p\mid ab,
		\quad
		a\in\langle p\rangle \iff p\mid a,
		\quad
		b\in\langle p\rangle \iff p\mid b.
		\]\qed
		\item 14.
		\textbf{题目.}\;
		在 $\mathbb Z[x]$ 中证明:$\langle x,n\rangle$ 是极大理想当且仅当 $n$ 是素数。
		
		\textbf{证明.}\;
		
		记
		\[
		I:=\langle x,n\rangle=\{\,xf(x)+ng(x)\mid f(x),g(x)\in\mathbb Z[x]\,\}.
		\]
		考虑自然环同态
		\[
		\varphi:\mathbb Z[x]\longrightarrow \mathbb Z_n,\qquad
		\varphi\!\left(\sum_{k=0}^m a_k x^k\right)=\overline{a_0}\in\mathbb Z_n,
		\]
		即 ``取常数项并模 $n$''。
		
		\textbf{(0) 证明 $\varphi$ 是环同态 }
		
		\textbf{(i) 保加法.}\;
		\[
		f(x)+g(x)=\sum_{k\ge 0}(a_k+b_k)x^k,
		\]
		其常数项为 $a_0+b_0$,故
		\[
		\varphi(f+g)=\overline{a_0+b_0}
		=\overline{a_0}+\overline{b_0}
		=\varphi(f)+\varphi(g).
		\]
		
		\textbf{(ii) 保乘法.}\;
		设 $f(x)g(x)=\sum_{t\ge 0} c_t x^t$。
		根据多项式乘法,$c_0=a_0b_0$(因为常数项只由常数项相乘产生),故
		\[
		\varphi(fg)=\overline{c_0}=\overline{a_0b_0}
		=\overline{a_0}\,\overline{b_0}
		=\varphi(f)\,\varphi(g).
		\]
		
		\textbf{(iii) 保幺元(若要求环同态保幺元).}\;
		$\mathbb Z[x]$ 的幺元为 $1$(常数多项式 $1$),且
		\[
		\varphi(1)=\overline{1}=1_{\mathbb Z_n}.
		\]
		
		由 (i)(ii)以及 (iii)可知 $\varphi$ 保加法与乘法(并保幺元),因此 $\varphi$ 是环同态。
		
		\bigskip
		\textbf{(1) $\varphi$ 为满射且 $\ker\varphi=\langle x,n\rangle$.}\;
		
		满射:对任意 $\overline{c}\in\mathbb Z_n$,取常多项式 $c\in\mathbb Z[x]$,
		则 $\varphi(c)=\overline{c}$。
		
		求核:设 $f(x)=a_0+a_1x+\cdots+a_mx^m\in\mathbb Z[x]$。
		则
		\[
		f(x)\in\ker\varphi
		\ \Longleftrightarrow\
		\overline{a_0}=\overline{0}\ \text{于 }\mathbb Z_n
		\ \Longleftrightarrow\
		n\mid a_0.
		\]
		若 $n\mid a_0$,写 $a_0=nb$,则
		\[
		f(x)=nb+x(a_1+a_2x+\cdots+a_mx^{m-1})\in \langle n,x\rangle.
		\]
		反之,若 $f(x)\in\langle n,x\rangle$,则 $f(x)=ng(x)+xh(x)$,
		其常数项必被 $n$ 整除,故 $f(x)\in\ker\varphi$。
		因此
		\[
		\ker\varphi=\langle x,n\rangle.
		\]
		
		由第一同构定理得
		\[
		\mathbb Z[x]/\langle x,n\rangle\ \cong\ \mathbb Z_n.
		\]
		
		\textbf{(2) 证明“$\Rightarrow$”.}\;
		若 $\langle x,n\rangle$ 是极大理想,则商环 $\mathbb Z[x]/\langle x,n\rangle$ 为域。
		由上同构可知 $\mathbb Z_n$ 为域,从而 $n$ 必为素数。
		
		\textbf{(3) 证明“$\Leftarrow$”.}\;
		若 $n$ 为素数,则 $\mathbb Z_n$ 为域。
		由同构
		\[
		\mathbb Z[x]/\langle x,n\rangle\cong \mathbb Z_n
		\]
		可知商环为域,于是 $\langle x,n\rangle$ 为极大理想。
		
		综上,$\langle x,n\rangle$ 是极大理想当且仅当 $n$ 是素数。 \qed
		\item 15.设 $\mathbb{P}$ 为一个数域,则在多项式环
		\[
		\mathbb{P}[x]
		\]
		中,由 $x$ 生成的主理想
		\[
		\langle x\rangle
		=\{\,x f(x)\mid f(x)\in\mathbb{P}[x]\,\}
		\]
		是 $\mathbb{P}[x]$ 的极大理想。
		
		\textbf{证明.}
		在交换含幺环中,一个真理想 $M$ 为极大理想当且仅当
		商环 $R/M$ 为域。因此我们只需证明
		\[
		\mathbb{P}[x]/(x)\cong\mathbb{P},
		\]
		而 $\mathbb{P}$ 本身是域,即可推出 $(x)$ 为极大理想。
		
		\medskip
		\noindent\textbf{一、构造环同态}
		
		定义映射
		\[
		\varphi:\mathbb{P}[x]\longrightarrow\mathbb{P},\qquad
		\varphi(f(x)):=f(0),
		\]
		即把多项式在 $x=0$ 处的取值作为像。
		
		\medskip
		\noindent\emph{(1) $\varphi$ 是环同态。}
		
		任取 $f(x),g(x)\in\mathbb{P}[x]$,有
		\[
		\varphi(f(x)+g(x))
		=(f+g)(0)
		=f(0)+g(0)
		=\varphi(f(x))+\varphi(g(x)),
		\]
		\[
		\varphi(f(x)g(x))
		=(fg)(0)
		=f(0)g(0)
		=\varphi(f(x))\varphi(g(x)),
		\]
		并且
		\[
		\varphi(1)=1.
		\]
		所以 $\varphi$ 保持加法、乘法与单位元,是一个环同态。
		
		\medskip
		\noindent\emph{(2) $\varphi$ 是满射。}
		
		对任意 $a\in\mathbb{P}$,考虑常数多项式 $f(x)\equiv a$,
		则
		\[
		\varphi(f(x))=f(0)=a.
		\]
		故 $\operatorname{Im}\varphi=\mathbb{P}$,即 $\varphi$ 为满射。
		
		\medskip
		\noindent\textbf{二、求 $\ker\varphi$ 并与 $(x)$ 比较}
		
		设
		\[
		f(x)=a_0+a_1x+\cdots+a_nx^n\in\mathbb{P}[x].
		\]
		则
		\[
		\varphi(f(x))=f(0)=a_0.
		\]
		于是
		\[
		f(x)\in\ker\varphi
		\iff f(0)=0
		\iff a_0=0.
		\]
		
		当 $a_0=0$ 时,可以把 $f(x)$ 中每一项都提取一个 $x$:
		\[
		f(x)=a_1x+a_2x^2+\cdots+a_nx^n
		=x(a_1+a_2x+\cdots+a_nx^{n-1})
		= xg(x),
		\]
		其中 $g(x)\in\mathbb{P}[x]$。
		因此
		\[
		\ker\varphi
		=\{f(x)\in\mathbb{P}[x]\mid f(0)=0\}
		=\{xg(x)\mid g(x)\in\mathbb{P}[x]\}
		=(x).
		\]
		
		\medskip
		\noindent\textbf{三、应用同构定理描述商环}
		
		由环同态基本定理(第一同构定理),有
		\[
		\mathbb{P}[x]/\ker\varphi
		\;\cong\;
		\operatorname{Im}\varphi.
		\]
		结合上面的计算,得到
		\[
		\mathbb{P}[x]/(x)\cong\mathbb{P}.
		\]
		
		\medskip
		\noindent\textbf{四、由商环为域推出极大理想}
		
		因为 $\mathbb{P}$ 是数域,所以它是一个域。
		同构不改变“是否为域”的性质,
		故商环 $\mathbb{P}[x]/(x)$ 也是域。
		
		在交换含幺环中,真理想 $M$ 为极大理想
		当且仅当 $R/M$ 为域。
		现在 $(x)$ 显然是真理想($x$ 不是单位,故 $(x)\neq\mathbb{P}[x]$),
		且其商环为域,
		于是 $(x)$ 为 $\mathbb{P}[x]$ 的极大理想。
		
		\medskip
		\noindent\textbf{结论.}
		在多项式环 $\mathbb{P}[x]$ 中,由不定元 $x$ 生成的主理想 $(x)$ 的商环
		与数域 $\mathbb{P}$ 同构,是一个域,因此 $(x)$ 是 $\mathbb{P}[x]$ 的极大理想。\qed
		
	\end{enumerate}
	
	\clearpage
	\section{主理想整环与欧几里得环}
	\subsection{思考题2.7.4}
	
	\subsection{例2.7.13}
	\textbf{例 2.7.13.}
	最后,作为一个例子,让我们来证明 Gauss 整数环是欧几里得环。
	记
	\[
	\mathbb{Z}[\sqrt{-1}]
	=\{\,a+b\sqrt{-1}\mid a,b\in\mathbb{Z}\,\}
	\]
	为 Gauss 整数环。对
	\[
	\beta=a+b\sqrt{-1}\in\mathbb{Z}[\sqrt{-1}]
	\]
	定义
	\[
	\delta(\beta)=|\beta|^2=a^2+b^2.
	\]
	并约定 $\delta(0)=0$。证明:$\mathbb{Z}[\sqrt{-1}]$ 关于该函数 $\delta$ 是一个欧几里得环。
	
	\textbf{证明.}
	
	\textbf{第一步:说明 $\delta(\beta)$ 取值在非负整数中。}
	
	任意 $\beta=a+b\sqrt{-1}\in\mathbb{Z}[\sqrt{-1}]$,其中 $a,b\in\mathbb{Z}$,则
	\[
	\delta(\beta)=a^2+b^2\in\mathbb{N}\cup\{0\},
	\]
	并且
	\[
	\beta\neq 0 \iff (a,b)\neq(0,0)\iff a^2+b^2>0\iff \delta(\beta)>0.
	\]
	故 $\delta$ 满足欧几里得函数要求的“非零元取正整数值、零元取 $0$”。
	
	\medskip
	\textbf{第二步:给定 $\beta\neq0$ 与任意元素,构造商 $q$ 和余数 $r$。}
	
	令
	\[
	\beta=a_1+b_1\sqrt{-1}\in\mathbb{Z}[\sqrt{-1}],\qquad \beta\neq0,
	\]
	再取任意
	\[
	\gamma=c_1+d_1\sqrt{-1}\in\mathbb{Z}[\sqrt{-1}].
	\]
	我们要证明:存在 $q,r\in\mathbb{Z}[\sqrt{-1}]$,使得
	\[
	\gamma=q\beta+r,\qquad \text{且}\quad r=0\ \text{或}\ \delta(r)<\delta(\beta).
	\]
	
	先在复数域中考虑分式
	\[
	\frac{\gamma}{\beta}
	=\frac{c_1+d_1\sqrt{-1}}{a_1+b_1\sqrt{-1}}.
	\]
	为把其化为“实部 + 虚部”的形式,乘以共轭:
	\[
	\frac{c_1+d_1\sqrt{-1}}{a_1+b_1\sqrt{-1}}
	=\frac{(c_1+d_1\sqrt{-1})(a_1-b_1\sqrt{-1})}{(a_1+b_1\sqrt{-1})(a_1-b_1\sqrt{-1})}
	=\frac{(c_1a_1+d_1b_1)+(d_1a_1-c_1b_1)\sqrt{-1}}{a_1^2+b_1^2}.
	\]
	记
	\[
	s=\frac{c_1a_1+d_1b_1}{a_1^2+b_1^2},\qquad
	t=\frac{d_1a_1-c_1b_1}{a_1^2+b_1^2},
	\]
	则
	\[
	\frac{\gamma}{\beta}=s+t\sqrt{-1}.
	\]
	此处 $a_1,b_1,c_1,d_1\in\mathbb{Z}$,故 $s,t\in\mathbb{Q}$。
	
	\medskip
	\textbf{第三步:把 $s,t$ 近似为“最近的整数”。}
	
	因为 $s,t\in\mathbb{R}$,可各自取“最接近它的整数”。更精确地讲,存在整数
	\[
	c_2,d_2\in\mathbb{Z},
	\]
	使得
	\[
	|c_2-s|\le\frac12,\qquad |d_2-t|\le\frac12.
	\]
	(例如,可令 $c_2$ 为 $s$ 的四舍五入,$d_2$ 为 $t$ 的四舍五入。)
	
	令
	\[
	q=c_2+d_2\sqrt{-1}\in\mathbb{Z}[\sqrt{-1}].
	\]
	那么
	\[
	s+t\sqrt{-1}-q=(s-c_2)+(t-d_2)\sqrt{-1}.
	\]
	
	\medskip
	\textbf{第四步:定义余数 $r$ 并写成合适的形式。}
	
	由
	\[
	\gamma=\beta(s+t\sqrt{-1})
	\]
	可得
	\[
	\gamma-q\beta=\beta\bigl((s+t\sqrt{-1})-q\bigr).
	\]
	记
	\[
	r:=\gamma-q\beta.
	\]
	则
	\[
	\gamma=q\beta+r,
	\]
	且
	\[
	r=\beta\bigl((s+t\sqrt{-1})-q\bigr).
	\]
	
	显然 $q\in\mathbb{Z}[\sqrt{-1}]$,且 $\gamma,q,\beta\in\mathbb{Z}[\sqrt{-1}]$,故 $r=\gamma-q\beta$ 也在 $\mathbb{Z}[\sqrt{-1}]$ 中。
	
	\medskip
	\textbf{第五步:估计 $\delta(r)$,得到 $\delta(r)<\delta(\beta)$。}
	
	先计算
	\[
	\delta(r)=|r|^2.
	\]
	由上式 $r=\beta((s+t\sqrt{-1})-q)$,在复数的绝对值下有
	\[
	|r|=|\beta|\cdot|(s+t\sqrt{-1})-q|,
	\]
	于是
	\[
	\delta(r)=|r|^2
	=|\beta|^2\cdot|(s+t\sqrt{-1})-q|^2
	=\delta(\beta)\,\delta\bigl((s+t\sqrt{-1})-q\bigr).
	\]
	
	再对后者作具体估计:
	\[
	(s+t\sqrt{-1})-q=(s-c_2)+(t-d_2)\sqrt{-1},
	\]
	故
	\[
	\delta\bigl((s+t\sqrt{-1})-q\bigr)
	=\bigl(s-c_2\bigr)^2+\bigl(t-d_2\bigr)^2.
	\]
	利用
	\[
	|s-c_2|\le\frac12,\qquad |t-d_2|\le\frac12,
	\]
	得
	\[
	\bigl(s-c_2\bigr)^2\le\left(\frac12\right)^2=\frac14,\qquad
	\bigl(t-d_2\bigr)^2\le\frac14.
	\]
	于是
	\[
	\delta\bigl((s+t\sqrt{-1})-q\bigr)
	=\bigl(s-c_2\bigr)^2+\bigl(t-d_2\bigr)^2
	\le \frac14+\frac14=\frac12 < 1.
	\]
	
	将其代入 $\delta(r)$ 的表达式中:
	\[
	\delta(r)=\delta(\beta)\,\delta\bigl((s+t\sqrt{-1})-q\bigr)
	\le \delta(\beta)\cdot\frac12
	<\delta(\beta).
	\]
	
	若恰好 $s=c_2$ 且 $t=d_2$,则 $(s+t\sqrt{-1})-q=0$,于是 $r=0$,更有
	\[
	\delta(r)=\delta(0)=0<\delta(\beta),
	\]
	也符合欧几里得条件。
	
	\medskip
	\textbf{第六步:结论。}
	
	我们已经对任意非零 $\beta\in\mathbb{Z}[\sqrt{-1}]$ 和任意
	\[
	\gamma\in\mathbb{Z}[\sqrt{-1}]
	\]
	构造出了 $q,r\in\mathbb{Z}[\sqrt{-1}]$,使得
	\[
	\gamma=q\beta+r,\qquad r=0\ \text{或}\ \delta(r)<\delta(\beta).
	\]
	因此函数
	\[
	\delta:\mathbb{Z}[\sqrt{-1}]\longrightarrow\mathbb{N}\cup\{0\},\qquad
	\delta(a+b\sqrt{-1})=a^2+b^2
	\]
	是一个欧几里得函数,故 Gauss 整数环 $\mathbb{Z}[\sqrt{-1}]$ 是欧几里得环。\qed
	
	\subsection{在主理想整环中,$p\in R$ 为\textbf{非零非单位},则$\langle p\rangle \text{ 极大}\ \Longleftrightarrow\ \langle p\rangle \text{ 素}\ \Longleftrightarrow\ p \text{ 不可约}\Longleftrightarrow
		p \text{ 为素元}$}
	\textbf{命题.}
	设 $R$ 为主理想整环(PID),$p\in R$ 为\textbf{非零非单位}。
	证明在 PID 中有等价关系
	\[
	\langle p\rangle \text{ 极大}\ \Longleftrightarrow\ \langle p\rangle \text{ 素}\ \Longleftrightarrow\ p \text{ 不可约}.
	\]
	
	\textbf{解:}
	\textbf{证明.}
	由于 $p$ 非零非单位,$\langle p\rangle\neq R$ 是真理想。
	
	\medskip
	\textbf{(A) 证明:$\langle p\rangle$ 极大 $\Longrightarrow$ $\langle p\rangle$ 素.}
	
	这是交换幺环中的一般事实:极大理想必为素理想。(见\ref{subsec:maximal_is_prime})
	
	\medskip
	\textbf{(B) 证明:$\langle p\rangle$ 素 $\Longrightarrow$ $p$ 不可约.}
	
	反证。若 $p$ 可约,则存在非单位 $a,b\in R$ 使
	\[
	p=ab.
	\]
	由于 $p\in \langle p\rangle$,所以 $ab=p\in \langle p\rangle$。
	又 $\langle p\rangle$ 为素理想,故
	\[
	a\in \langle p\rangle\ \text{或}\ b\in \langle p\rangle.
	\]
	不妨设 $a\in \langle p\rangle$,则存在 $c\in R$ 使
	\[
	a=pc.
	\]
	代回 $p=ab$ 得
	\[
	p=(pc)b.
	\]
	由于 $R$ 是整环且 $p\neq 0$,可消去 $p$,得
	\[
	cb=1.
	\]
	于是 $b$ 为单位,矛盾(因假设 $b$ 非单位)。
	故 $p$ 不可约。
	
	因此
	\[
	\langle p\rangle \text{ 素}\ \Longrightarrow\ p \text{ 不可约}.
	\]
	
	\medskip
	\textbf{(C) 证明:$p$ 不可约 $\Longrightarrow$ $\langle p\rangle$ 极大.}
	
	设 $J$ 为满足
	\[
	\langle p\rangle \subseteq J \subseteq R
	\]
	的任意理想。由于 $R$ 为 PID,存在 $a\in R$ 使 $J=\langle a\rangle$。
	由 $\langle p\rangle\subseteq \langle a\rangle$ 得 $p\in\langle a\rangle$,故存在 $b\in R$ 使
	\[
	p=ab.
	\]
	由于 $p$ 不可约,因此 $a$ 或 $b$ 必为单位:
	
	\begin{itemize}
		\item 若 $a$ 为单位,则 $\langle a\rangle=R$,从而 $J=R$;
		\item 若 $b$ 为单位,则 $a$ 与 $p$ 相伴,从而 $\langle a\rangle=\langle p\rangle$,即 $J=\langle p\rangle$。
	\end{itemize}
	
	因此在 $\langle p\rangle$ 与 $R$ 之间没有第三个理想,
	故 $\langle p\rangle$ 是极大理想。
	
	综上得到链式推出:
	\[
	\langle p\rangle \text{ 极大}\ \Longleftrightarrow\ \langle p\rangle \text{ 素}\ \Longleftrightarrow\ p \text{ 不可约}.
	\]
	
	\clearpage 
	\subsection*{课后习题答案}
	\addcontentsline{toc}{subsection}{\textcolor{red}{课后习题答案}}
	\begin{enumerate}[label=\textcolor{blue}{\textbf{\large\arabic*.}}]	
		\item \textbf{题目.}
		设 $R$ 为整环,$\langle a\rangle,\langle b\rangle$ 是 $R$ 的主理想。证明:
		\[
		\langle a\rangle = \langle b\rangle \quad\Longleftrightarrow\quad a \text{ 与 } b \text{ 相伴}.
		\]
		\medskip
		\textbf{($\Rightarrow$) 若 $\langle a\rangle = \langle b\rangle$,则 $a$ 与 $b$ 相伴。}
		
		首先分两种情形讨论。
		
		\medskip
		\noindent\emph{情形 1:$a=0$。}
		
		由 $\langle a\rangle = \langle 0\rangle = \{0\}$ 可知
		\[
		\langle b\rangle = \langle a\rangle = \{0\},
		\]
		于是 $b=0$。此时
		\[
		b = 1\cdot a = 1\cdot 0,
		\]
		其中 $1$ 是单位,故 $a$ 与 $b$ 相伴($0$ 与自身相伴)。这一情形成立。
		
		\medskip
		\noindent\emph{情形 2:$a\neq 0$。}
		
		由
		\[
		\langle a\rangle = \langle b\rangle
		\]
		可得两条包含关系
		\[
		\langle a\rangle \subseteq \langle b\rangle,\qquad
		\langle b\rangle \subseteq \langle a\rangle.
		\]
		
		\begin{itemize}
			\item 因 $a\in\langle a\rangle = \langle b\rangle$,于是存在某个 $c\in R$,使得
			\[
			a = cb. \tag{1}
			\]
			\item 同理,$b\in\langle b\rangle = \langle a\rangle$,存在某个 $d\in R$,使得
			\[
			b = da. \tag{2}
			\]
		\end{itemize}
		
		将 \((2)\) 式代入 \((1)\) 式:
		\[
		a = c b = c(da) = (cd)a.
		\]
		即
		\[
		a = (cd)a. \tag{3}
		\]
		
		把 \((3)\) 式移项得
		\[
		a - (cd)a = 0
		\quad\Longrightarrow\quad
		(1-cd)a = 0. \tag{4}
		\]
		
		因为 $R$ 是整环,所以没有零因子。又假设此处 $a\neq 0$,故由 \((4)\) 式可在整环中“消去” $a$,得到
		\[
		1 - cd = 0 \quad\Longrightarrow\quad cd = 1.
		\]
		
		这说明 $c$ 与 $d$ 互为逆元,都是单位。记 $u:=d\in R^*$,则由 \((2)\) 式得到
		\[
		b = da = ua,
		\]
		也就是说,$b$ 是 $a$ 乘以一个单位,因此 $a$ 与 $b$ 相伴。
		
		\medskip
		综上两种情形,若 $\langle a\rangle = \langle b\rangle$,则必有 $a$ 与 $b$ 相伴。
		
		\medskip
		\textbf{($\Leftarrow$) 若 $a$ 与 $b$ 相伴,则 $\langle a\rangle = \langle b\rangle$。}
		
		假设 $a$ 与 $b$ 相伴,则存在单位 $u\in R^*$ 使
		\[
		b = ua.
		\]
		由于 $u$ 可逆,记其逆为 $u^{-1}\in R^*$,则
		\[
		a = u^{-1}b.
		\]
		
		\medskip
		\noindent\emph{证明 $\langle a\rangle \subseteq \langle b\rangle$.}
		
		任取 $\langle a\rangle$ 中的元素,它必形如
		\[
		x = ra,\qquad r\in R.
		\]
		利用 $a = u^{-1}b$,有
		\[
		x = ra = r(u^{-1}b) = (ru^{-1})b.
		\]
		因为 $r\in R$ 且 $u^{-1}\in R$,所以 $ru^{-1}\in R$。于是 $x$ 可以写成某个环元乘以 $b$ 的形式,即
		\[
		x\in\{\,sb\mid s\in R\,\} = \langle b\rangle.
		\]
		因此
		\[
		\langle a\rangle \subseteq \langle b\rangle.
		\]
		
		\medskip
		\noindent\emph{证明 $\langle b\rangle \subseteq \langle a\rangle$.}
		
		任取 $\langle b\rangle$ 中的元素,它必形如
		\[
		y = rb,\qquad r\in R.
		\]
		利用 $b = ua$,有
		\[
		y = rb = r(ua) = (ru)a.
		\]
		因 $r\in R,\,u\in R$,故 $ru\in R$,于是
		\[
		y\in\{\,ta\mid t\in R\,\} = \langle a\rangle,
		\]
		故
		\[
		\langle b\rangle \subseteq \langle a\rangle.
		\]
		
		\medskip
		两边包含关系同时成立,于是
		\[
		\langle a\rangle = \langle b\rangle.
		\]
		
		\medskip
		\textbf{结论.}
		在整环 $R$ 中,主理想 $\langle a\rangle$ 与 $\langle b\rangle$ 相等,当且仅当 $a$ 与 $b$ 相伴,即存在单位 $u\in R^*$ 使 $b=ua$。 \qed
		
		\item 3.
		\textbf{题目.}
		设 $R$ 为主理想整环,若 $a\in R,\ a\neq 0$ 且主理想 $\langle a\rangle$ 为一个极大理想,证明:$a$ 为不可约元素。
		
		\textbf{回顾定义.}
		在整环 $R$ 中,称 $a\in R$ 为\emph{不可约元素},是指
		\begin{itemize}
			\item $a\neq 0$;
			\item $a$ 不是单位;
			\item 若 $a=bc$,则 $b$ 或 $c$ 至少有一个是单位。
		\end{itemize}
		
		\textbf{证明.}
		
		\medskip
		\noindent\textbf{第一步:$a$ 非单位。}
		
		因为 $\langle a\rangle$ 是极大理想,所以它必为真理想,即
		\[
		\langle a\rangle\neq R.
		\]
		若 $a$ 是单位,则 $1\in\langle a\rangle$(因为 $1=a^{-1}a$),从而
		\[
		\langle a\rangle = R,
		\]
		与其为极大\emph{真}理想矛盾。故 $a$ 不是单位。
		
		题设又给出 $a\neq 0$,因此 $a$ 已满足“不可约元素”的前两条条件。
		
		\medskip
		\noindent\textbf{第二步:任意分解 $a=bc$,证明 $b$ 或 $c$ 为单位。}
		
		设有分解
		\[
		a = bc,\qquad b,c\in R.
		\]
		因为 $R$ 为整环而 $a\neq 0$,故
		\[
		bc=a\neq 0 \Longrightarrow b\neq 0,\ c\neq 0.
		\]
		
		由 $a=bc$ 得到主理想之间的包含关系:
		\[
		\langle a\rangle \subseteq \langle b\rangle.
		\]
		事实上,对任意 $r\in R$,有
		\[
		ra\in\langle a\rangle,\quad
		ra = r(bc) = (rc)b\in\langle b\rangle,
		\]
		故 $\langle a\rangle\subseteq\langle b\rangle$。
		
		又因为 $b\neq 0$,若 $\langle b\rangle=R$,则存在 $r\in R$ 使 $1=rb$,从而 $b$ 为单位;这种情形已经满足我们要证的结论。于是接下来只需讨论
		\[
		\langle b\rangle\neq R
		\]
		这一情形。
		
		此时我们得到链
		\[
		\langle a\rangle \subseteq \langle b\rangle \subseteq R,
		\]
		其中 $\langle b\rangle$ 也是一个真理想。由于 $\langle a\rangle$ 是极大真理想,只要有
		\[
		\langle a\rangle \subseteq I \subseteq R,
		\]
		就必有 $I=\langle a\rangle$ 或 $I=R$。对 $I=\langle b\rangle$,且已知 $\langle b\rangle\neq R$,只能是
		\[
		\langle b\rangle = \langle a\rangle. \tag{1}
		\]
		
		\medskip
		\noindent\textbf{第三步:由 $\langle a\rangle=\langle b\rangle$ 推出 $a$ 与 $b$ 相伴。}
		
		用到一个整环中的基本事实:
		
		\emph{引理:}设 $R$ 为整环,若 $\langle x\rangle=\langle y\rangle$,则 $x$ 与 $y$ 相伴,即存在单位 $u\in R^*$ 使 $y=ux$。
		
		\emph{证明:}
		由 $\langle x\rangle=\langle y\rangle$,有 $x\in\langle y\rangle$,故存在 $r\in R$ 使
		\[
		x=ry.
		\]
		同理 $y\in\langle x\rangle$,存在 $s\in R$ 使
		\[
		y=sx.
		\]
		代入得
		\[
		x = ry = r(sx) = (rs)x.
		\]
		移项得
		\[
		(1-rs)x=0.
		\]
		由于 $R$ 是整环且 $x\neq 0$,可消去 $x$,从而 $1-rs=0$,即 $rs=1$。于是 $r,s$ 互为逆元,$r$ 为单位。记 $u:=r^{-1}=s$,则
		\[
		y = sx = ux.
		\]
		故 $y$ 与 $x$ 相伴。引理得证。$\square$
		
		\smallskip
		回到本题,由 \((1)\) 式可知 $\langle a\rangle=\langle b\rangle$,应用引理可得存在单位 $u\in R^*$ 使
		\[
		b = ua. \tag{2}
		\]
		
		\medskip
		\noindent\textbf{第四步:从 $a=bc$ 和 $b=ua$ 推出 $c$ 为单位。}
		
		将 \((2)\) 代入 $a=bc$:
		\[
		a = bc = (ua)c = u(ac).
		\]
		于是
		\[
		a - u(ac) = 0
		\quad\Longrightarrow\quad
		(1-uc)a = 0.
		\]
		因为 $R$ 为整环且 $a\neq 0$,故可消去 $a$,得到
		\[
		1-uc=0 \quad\Longrightarrow\quad uc=1.
		\]
		这表明 $c$ 有左逆 $u$。在整环(实际上在任意含幺环)中,若某元素有左逆,则该左逆亦为右逆,从而
		\[
		cu=1.
		\]
		因此 $c$ 为单位。
		
		\medskip
		\noindent\textbf{第五步:总结。}
		
		我们对任意分解 $a=bc$ 证明了:要么
		\[
		\langle b\rangle = R \Longrightarrow b \text{ 为单位},
		\]
		要么由极大性推出 $\langle b\rangle=\langle a\rangle$,进而得到 $c$ 为单位。无论哪种情况,分解 $a=bc$ 中必有一因子为单位。
		
		结合第一步中 $a\neq 0$ 且 $a$ 非单位,$a$ 满足不可约元素的全部条件,故 $a$ 为不可约元素。\qed
		
		\item 5.
		\textbf{题目.} 在 $\mathbb{Q}[x]$ 中求多项式 $f(x)$,使得
		\[
		\langle x^2+1,\;x^5+x^3+1\rangle=\langle f(x)\rangle.
		\]
		
		\textbf{解.}
		设
		\[
		I=\langle x^2+1,\;x^5+x^3+1\rangle\subset\mathbb{Q}[x].
		\]
		因为 $\mathbb{Q}$ 是域,故 $\mathbb{Q}[x]$ 为欧几里得环,从而是主理想整环。
		因此理想 $I$ 必为某个多项式生成的主理想,即存在 $f(x)$ 使
		\[
		I=\langle f(x)\rangle,
		\]
		且 $f(x)$ 可取为 $x^2+1$ 与 $x^5+x^3+1$ 在 $\mathbb{Q}[x]$ 中的一个最大公因子。
		
		下面计算这两个多项式的最大公因子。注意到
		\[
		x^5+x^3+1 = x^3(x^2+1) + 1.
		\]
		于是
		\[
		x^5+x^3+1 - x^3(x^2+1) = 1.
		\]
		这表明
		\[
		1 \in I
		\]
		(因为 $1$ 是 $x^5+x^3+1$ 与 $x^3(x^2+1)$ 的线性组合,而 $x^3(x^2+1)\in I$),
		从而
		\[
		\langle 1\rangle\subseteq I.
		\]
		反过来,显然
		\[
		I\subseteq \mathbb{Q}[x]=\langle 1\rangle.
		\]
		因此
		\[
		I=\langle 1\rangle.
		\]
		
		所以在 $\mathbb{Q}[x]$ 中,满足
		\[
		\langle x^2+1,\;x^5+x^3+1\rangle=\langle f(x)\rangle
		\]
		的多项式 $f(x)$ 在相伴意义下唯一,可以取为
		\[
		f(x)=1.
		\]
		\textbf{答:} $f(x)=1$。
		\item 8.
		\textbf{题目.}
		设 $p$ 为一个素数,试证明下列环在指定的映射 $\delta$ 下均为欧几里得环:
		\[
		(1)\ \{\mathbb{Z}[\sqrt{-2}];+,\cdot\},\quad \delta(a+b\sqrt{-2})=a^{2}+2b^{2};
		\]
		\[
		(2)\ \{\mathbb{Z}[\sqrt2];+,\cdot\},\quad \delta(a+b\sqrt2)=\lvert a^{2}-2b^{2}\rvert;
		\]
		\[
		(3)\ \{R;+,\cdot\},\quad
		R=\mathbb{Z}\Bigl[\frac{1-\sqrt{-7}}2\Bigr]
		=\Bigl\{\,a+b\frac{1-\sqrt{-7}}2\ \bigm|\ a,b\in\mathbb{Z}\Bigr\},
		\]
		\[
		\delta\!\left(a+b\frac{1-\sqrt{-7}}2\right)=a^{2}+ab+2b^{2}.
		\]
		
		\textbf{解.}
		
		\medskip
		\textbf{(1) 环 $\mathbb{Z}[\sqrt{-2}]$,$\delta(a+b\sqrt{-2})=a^{2}+2b^{2}$.}
		
		\textbf{(a) $\delta$ 的基本性质.}
		记共轭映射
		\[
		\overline{a+b\sqrt{-2}}=a-b\sqrt{-2},
		\]
		定义数域 $K=\mathbb{Q}(\sqrt{-2})$ 上的范数
		\[
		N(\alpha)=\alpha\overline{\alpha}.
		\]
		易算得
		\[
		N(a+b\sqrt{-2})=(a+b\sqrt{-2})(a-b\sqrt{-2})=a^{2}+2b^{2}.
		\]
		于是对 $\alpha\in\mathbb{Z}[\sqrt{-2}]$,有
		\[
		\delta(\alpha)=N(\alpha)\in\mathbb{N}_0.
		\]
		且 $a^{2}+2b^{2}=0$ 只可能 $a=b=0$,故 $\delta(\alpha)=0 \iff \alpha=0$。
		又由共轭的乘法性 $(\alpha\beta)\,\overline{\ }=\overline{\alpha}\,\overline{\beta}$ 得
		\[
		\delta(\alpha\beta)=N(\alpha\beta)
		=\alpha\beta\overline{\alpha\beta}
		=(\alpha\overline{\alpha})(\beta\overline{\beta})
		=N(\alpha)N(\beta)=\delta(\alpha)\delta(\beta).
		\]
		
		\textbf{(b) 欧几里得性质.}
		取任意 $\alpha,\beta\in\mathbb{Z}[\sqrt{-2}]$,$\beta\ne0$。
		在数域 $K$ 中考虑
		\[
		z=\frac{\alpha}{\beta}\in K.
		\]
		将 $z$ 写成
		\[
		z=x+y\sqrt{-2},\qquad x,y\in\mathbb{R}.
		\]
		取整数 $m,n\in\mathbb{Z}$ 使得
		\[
		|x-m|\le\frac12,\qquad |y-n|\le\frac12,
		\]
		例如取 $m$、$n$ 为最近整数(四舍五入)。
		
		令
		\[
		q=m+n\sqrt{-2}\in\mathbb{Z}[\sqrt{-2}],\qquad
		z-q=(x-m)+(y-n)\sqrt{-2}=:s+t\sqrt{-2},
		\]
		其中 $|s|,|t|\le\frac12$。
		于是
		\[
		N(z-q)=s^{2}+2t^{2}\le \frac14+2\cdot\frac14=\frac34<1.
		\]
		
		现在令
		\[
		r:=\alpha-\beta q=\beta(z-q)\in\mathbb{Z}[\sqrt{-2}],
		\]
		则
		\[
		\delta(r)=N(r)=N(\beta)N(z-q)
		< N(\beta)=\delta(\beta),
		\]
		只要 $r\ne0$;若 $r=0$ 则更加满足定义。
		又有 $\alpha=\beta q+r$,故 $\mathbb{Z}[\sqrt{-2}]$ 在 $\delta$ 下为欧几里得环。 \qed		
		
		\medskip
		\textbf{(2) 环 $\mathbb{Z}[\sqrt2]$,$\delta(a+b\sqrt2)=|a^{2}-2b^{2}|$.}
		
		\textbf{(a) 范数与 $\delta$.}
		在数域 $K=\mathbb{Q}(\sqrt2)$ 上定义共轭
		\[
		\overline{a+b\sqrt2}=a-b\sqrt2,
		\]
		范数
		\[
		N(\alpha)=\alpha\overline{\alpha}=(a+b\sqrt2)(a-b\sqrt2)=a^{2}-2b^{2}.
		\]
		于是对 $\alpha\in\mathbb{Z}[\sqrt2]$,
		\[
		\delta(\alpha)=|N(\alpha)|\in\mathbb{N}_0.
		\]
		$N(\alpha)=0$ 时有 $a^{2}=2b^{2}$。若 $b\ne0$,则 $2\mid a$,写 $a=2a_1$,得
		\[
		4a_1^{2}=2b^{2}\Rightarrow 2a_1^{2}=b^{2},
		\]
		故 $2\mid b$。不断下去得到无穷下降,只能 $a=b=0$。
		故 $\delta(\alpha)=0\iff \alpha=0$。
		由范数乘法性得
		\[
		\delta(\alpha\beta)=|N(\alpha\beta)|=|N(\alpha)N(\beta)|
		=\delta(\alpha)\delta(\beta).
		\]
		
		\textbf{(b) 欧几里得性质.}
		同样取任意 $\alpha,\beta\in\mathbb{Z}[\sqrt2]$,$\beta\ne0$,
		考虑
		\[
		z=\frac{\alpha}{\beta}=x+y\sqrt2\in K\subset\mathbb{R},
		\quad x,y\in\mathbb{R}.
		\]
		取整数 $m,n$,使
		\[
		|x-m|\le\frac12,\qquad |y-n|\le\frac12.
		\]
		令
		\[
		q=m+n\sqrt2\in\mathbb{Z}[\sqrt2],\qquad
		z-q=(x-m)+(y-n)\sqrt2=:s+t\sqrt2,
		\]
		则 $|s|,|t|\le\frac12$,于是
		\[
		|N(z-q)|=\bigl|s^{2}-2t^{2}\bigr|
		\le s^{2}+2t^{2}\le\frac14+2\cdot\frac14=\frac34<1.
		\]
		
		设
		\[
		r=\alpha-\beta q=\beta(z-q)\in\mathbb{Z}[\sqrt2],
		\]
		则
		\[
		\delta(r)=|N(r)|
		=|N(\beta)||N(z-q)|
		<|N(\beta)|=\delta(\beta).
		\]
		若 $r\ne0$,则满足欧几里得条件;若 $r=0$ 当然也可。
		所以 $\mathbb{Z}[\sqrt2]$ 在该 $\delta$ 下是欧几里得环。\qed
		
		%--------------------------------------------------
		\medskip
		\textbf{(3) 环 $R=\mathbb{Z}\!\left[\dfrac{1-\sqrt{-7}}2\right]$,}
		\[
		\delta\!\left(a+b\frac{1-\sqrt{-7}}2\right)=a^{2}+ab+2b^{2}.
		\]
		
		记
		\[
		\omega=\frac{1-\sqrt{-7}}2,\qquad
		\overline{\omega}=\frac{1+\sqrt{-7}}2.
		\]
		易算出
		\[
		\omega+\overline{\omega}=1,\qquad
		\omega\overline{\omega}=2,
		\]
		即 $\omega$ 满足最小多项式 $x^{2}-x+2=0$。
		数域为 $K=\mathbb{Q}(\sqrt{-7})=\mathbb{Q}(\omega)$,且
		\[
		R=\mathbb{Z}[\omega]=\{a+b\omega\mid a,b\in\mathbb{Z}\}.
		\]
		
		\textbf{(a) 范数与 $\delta$.}
		定义共轭
		\[
		\overline{a+b\omega}=a+b\,\overline{\omega},
		\]
		范数
		\[
		N(\alpha)=\alpha\overline{\alpha}.
		\]
		于是
		\[
		N(a+b\omega)
		=(a+b\omega)(a+b\overline{\omega})
		=a^{2}+ab(\omega+\overline{\omega})+b^{2}\omega\overline{\omega}
		=a^{2}+ab+2b^{2}.
		\]
		这正是题中给出的 $\delta$:
		\[
		\delta(a+b\omega)=N(a+b\omega).
		\]
		因为二次型 $a^{2}+ab+2b^{2}$ 的判别式
		\[
		\Delta=1^{2}-4\cdot1\cdot2=-7<0,
		\]
		故该二次型正定,$N(\alpha)\ge0$ 且 $N(\alpha)=0$ 仅在 $a=b=0$ 时成立。
		显然 $N$ 具乘法性,因此
		\[
		\delta(\alpha\beta)=N(\alpha\beta)=N(\alpha)N(\beta)=\delta(\alpha)\delta(\beta).
		\]
		
		\textbf{(b) 欧几里得性质.}
		取任意 $\alpha,\beta\in R$,$\beta\ne0$,在 $K$ 中
		\[
		z=\frac{\alpha}{\beta}=x+y\omega,\qquad x,y\in\mathbb{R}.
		\]
		
		\emph{第一步:选择合适的整数 $a,b$.}
		令 $a_0=\lfloor x+\tfrac12\rfloor$,$b_0=\lfloor y+\tfrac12\rfloor$,
		则
		\[
		s_0:=x-a_0,\quad t_0:=y-b_0
		\]
		满足
		\[
		|s_0|\le\frac12,\qquad |t_0|\le\frac12.
		\]
		若 $(s_0,t_0)\neq\bigl(\tfrac12,\tfrac12\bigr)$ 且
		$(s_0,t_0)\neq\bigl(-\tfrac12,-\tfrac12\bigr)$,
		我们就取 $a=a_0$, $b=b_0$。
		否则分两种极端情形调整一坐标即可,例如:
		
		\begin{itemize}
			\item 若 $(s_0,t_0)=\bigl(\tfrac12,\tfrac12\bigr)$,则改取
			\[
			a=a_0+1,\quad b=b_0,
			\]
			此时
			\[
			s=x-a=-\tfrac12,\quad t=y-b=\tfrac12;
			\]
			\item 若 $(s_0,t_0)=\bigl(-\tfrac12,-\tfrac12\bigr)$,则取
			\[
			a=a_0-1,\quad b=b_0,
			\]
			得
			\[
			s=x-a=\tfrac12,\quad t=y-b=-\tfrac12.
			\]
		\end{itemize}
		综上总能选到整数 $a,b$ 使得
		\[
		s:=x-a,\quad t:=y-b \quad\text{满足}\quad
		|s|\le\frac12,\ |t|\le\frac12,
		\]
		且不出现 $(s,t)=(\tfrac12,\tfrac12)$ 或 $(-\tfrac12,-\tfrac12)$。
		
		令
		\[
		q=a+b\omega\in R,\qquad
		z-q=(x-a)+(y-b)\omega=s+t\omega.
		\]
		
		\emph{第二步:估计 $N(z-q)$.}
		有
		\[
		N(z-q)=N(s+t\omega)=s^{2}+st+2t^{2}.
		\]
		对任意 $|s|,|t|\le\frac12$,有
		\[
		s^{2}+st+2t^{2}\le s^{2}+|st|+2t^{2}
		\le \frac14+\frac14+2\cdot\frac14=1.
		\]
		并且等号 $s^{2}+st+2t^{2}=1$ 仅在
		\[
		(s,t)=\Bigl(\frac12,\frac12\Bigr),\quad
		\Bigl(-\frac12,-\frac12\Bigr)
		\]
		两点处成立(直接代入可验证)。
		由于我们在选取 $a,b$ 时排除了这两种情况,
		故实际得到的是
		\[
		N(z-q)=s^{2}+st+2t^{2}<1.
		\]
		
		\emph{第三步:构造带余除法并比较 $\delta$.}
		令
		\[
		r:=\alpha-\beta q=\beta(z-q)\in R.
		\]
		则
		\[
		\delta(r)=N(r)=N(\beta)N(z-q)<N(\beta)=\delta(\beta),
		\]
		若 $r\ne0$,则满足 Euclid 条件;若 $r=0$ 则更好。
		同时 $\alpha=\beta q+r$,故欧几里得条件完全满足。
		
		因此 $R$ 在给定的 $\delta$ 下是欧几里得环。 \qed
		
		\medskip
		综上,(1) $\mathbb{Z}[\sqrt{-2}]$,(2) $\mathbb{Z}[\sqrt2]$,
		(3) $R=\mathbb{Z}\!\left[\dfrac{1-\sqrt{-7}}2\right]$
		在题中给定的映射 $\delta$ 下均为欧几里得环。
		
		\item 9.
		\textbf{题目.}
		试在 Gauss 整环 $\mathbb{Z}[\sqrt{-1}]=\mathbb{Z}[i]$ 中举例说明:
		在欧几里得环中作带余除法时,商和余式可以\textbf{不唯一}。
		
		\medskip
		\textbf{解.}
		
		回忆:在高斯整数环 $\mathbb{Z}[i]$ 中的欧几里得函数取为
		\[
		N(a+bi)=a^2+b^2,\qquad a,b\in\mathbb{Z},
		\]
		称为范数。$\mathbb{Z}[i]$ 是欧几里得环,是因为对任意
		$a,b\in\mathbb{Z}[i]$ 且 $b\neq 0$,总可找到 $q,r\in\mathbb{Z}[i]$,
		使得
		\[
		a = bq + r,\qquad r=0\ \text{或}\ N(r)<N(b).
		\]
		
		下面我们就展示一个\emph{同一对} $(a,b)$ 有\emph{两种不同}的商、余式表示。
		
		\medskip
		取
		\[
		a=3,\qquad b=1+i\in\mathbb{Z}[i].
		\]
		
		\textbf{第一种分解:}
		
		令 $q_1 = 1 - i$。则
		\[
		(1+i)(1-i) = 1 - i^2 = 1 - (-1) = 2.
		\]
		于是
		\[
		3 = (1+i)(1-i) + 1.
		\]
		也就是说,以 $b=1+i$ 除 $a=3$,得到一组
		\[
		q_1 = 1-i,\qquad r_1 = 1.
		\]
		检查欧几里得条件:
		\[
		N(b) = N(1+i) = 1^2+1^2 = 2,\qquad N(r_1) = N(1)=1<2=N(b),
		\]
		所以这是一组合法的商、余式。
		
		\medskip
		\textbf{第二种分解:}
		
		现在令
		\[
		q_2 = 2 - i.
		\]
		计算
		\[
		(1+i)(2-i)
		= 2 - i + 2i - i^2
		= 2 + i + 1
		= 3 + i.
		\]
		于是
		\[
		3 = (1+i)(2-i) + (-i).
		\]
		因此得到另一组
		\[
		q_2 = 2 - i,\qquad r_2 = -i.
		\]
		同样检查欧几里得条件:
		\[
		N(r_2) = N(-i) = 0^2+(-1)^2 = 1 < 2 = N(1+i)=N(b),
		\]
		因此这也是一组\emph{合法}的商、余式。
		
		\medskip
		\textbf{比较两组结果:}
		
		显然
		\[
		q_1 = 1-i \neq 2 - i = q_2,\qquad
		r_1 = 1 \neq -i = r_2.
		\]
		但它们都满足
		\[
		3 = (1+i)q_1 + r_1 = (1+i)q_2 + r_2,
		\qquad N(r_1),N(r_2)<N(1+i).
		\]
		
		这就说明:在欧几里得环 $\mathbb{Z}[i]$ 中,对同一对元素 $a=3$、
		$b=1+i$ 作带余除法时,\emph{商与余式并不唯一}。
		
		\medskip
		因此,通过 $\mathbb{Z}[i]$ 中的此例,表明一般的欧几里得环中
		“存在 $q,r$ 使 $a=bq+r$ 且 $N(r)<N(b)$”并不保证 $(q,r)$ 的唯一性。
		\qed
		
		
		\item 10.
		\textbf{题目.}
		证明任何一个域都是欧几里得环。
		
		\textbf{证明.}
		在整环 $F$ 上定义映射
		\[
		\delta:F\setminus\{0\}\longrightarrow\mathbb{N},\qquad \delta(x):=0\ \text{(或任意固定自然数)}.
		\]
		即对一切非零元 $x\in F$,令 $\delta(x)$ 都等于同一个自然数 $0$(也可以取 $1$ 等)。
		
		现在任取 $a,b\in F$ 且 $b\neq 0$。由于 $F$ 是域,$b$ 一定可逆,记其逆元为 $b^{-1}\in F$。
		令
		\[
		q := ab^{-1},\qquad r := 0.
		\]
		则
		\[
		a = bq + r = b(ab^{-1}) + 0.
		\]
		此时余数 $r=0$,于是自动满足
		\[
		\text{“要么 } r=0,\ \text{要么 } \delta(r)<\delta(b)\text{”}
		\]
		中的第一种情形,而无需检查 $\delta(r)<\delta(b)$。
		
		这就说明:对任意 $a,b\in F$ 且 $b\neq 0$,我们都可以找到 $q,r\in F$ 使
		\[
		a=bq+r,
		\]
		并满足欧几里得环定义中的要求。
		
		\medskip
		
		综上,域 $F$ 是整环,且存在上述映射 $\delta$ 使得欧几里得除法条件成立,
		因此 \emph{任何一个域都是欧几里得环}。\qed
		
		
		
		
	\end{enumerate}
	
	\clearpage
	\section{环上的多项式}
	
	\subsection{命题2.8.1}
	设 $R$ 为一环,$S=R[x]$ 为其多项式环。若 $f,g\in S$,
	$\deg f=m,\ \deg g=n$,则
	
	\begin{enumerate}
		\item[(1)] $\deg(f+g)\le \max(m,n)$;
		\item[(2)] $\deg(fg)\le \deg f+\deg g$,且等号成立当且仅当
		$f$ 的首项系数 $a_m$ 与 $g$ 的首项系数 $b_n$ 的乘积 $a_mb_n\neq 0$。
	\end{enumerate}
	
	特别地,若 $R$ 是整环,则 $S$ 也是整环。
	
	\textbf{解:}
	
	\textbf{证明.}
	为避免对零多项式的讨论分支,约定 $\deg 0=-\infty$(于是 $\max(m,-\infty)=m$ 等仍成立)。
	写
	\[
	f(x)=a_mx^m+a_{m-1}x^{m-1}+\cdots+a_0,\quad a_m\neq 0,
	\]
	\[
	g(x)=b_nx^n+b_{n-1}x^{n-1}+\cdots+b_0,\quad b_n\neq 0.
	\]
	
	\medskip
	\textbf{(1) 证明 $\deg(f+g)\le \max(m,n)$.}
	
	不妨设 $m\ge n$(否则交换 $f,g$ 即可)。则 $g$ 的最高次数不超过 $m$,
	可将 $g$ 视为
	\[
	g(x)=b_mx^m+b_{m-1}x^{m-1}+\cdots+b_0,
	\]
	其中当 $k>n$ 时约定 $b_k=0$。于是
	\[
	(f+g)(x)=(a_m+b_m)x^m+(a_{m-1}+b_{m-1})x^{m-1}+\cdots+(a_0+b_0).
	\]
	因此 $f+g$ 的所有项的次数都不超过 $m$,从而
	\[
	\deg(f+g)\le m=\max(m,n).
	\]
	(若 $a_m+b_m=0$,则 $x^m$ 项消去,次数可能严格下降,但仍不超过 $m$。)
	
	\medskip
	\textbf{(2) 证明 $\deg(fg)\le \deg f+\deg g$,且刻画等号条件.}
	
	先看 $fg$ 的系数结构。乘积
	\[
	fg=\left(\sum_{i=0}^{m}a_ix^i\right)\left(\sum_{j=0}^{n}b_jx^j\right)
	=\sum_{k=0}^{m+n}\left(\sum_{i+j=k}a_ib_j\right)x^k.
	\]
	因此 $fg$ 中出现的最高次数至多为 $m+n$,故
	\[
	\deg(fg)\le m+n=\deg f+\deg g.
	\]
	
	下面讨论何时取等号。注意到 $x^{m+n}$ 项的系数只能来自 $a_mx^m\cdot b_nx^n$,
	所以 $x^{m+n}$ 的系数恰为
	\[
	a_mb_n.
	\]
	\begin{itemize}
		\item 若 $a_mb_n\neq 0$,则 $fg$ 的 $x^{m+n}$ 项系数非零,说明 $fg$ 确有次数 $m+n$ 的项,
		于是
		\[
		\deg(fg)=m+n.
		\]
		\item 若 $a_mb_n=0$,则 $x^{m+n}$ 项系数为零,$fg$ 的最高次数项(若存在)只能出现在
		$x^{m+n-1},x^{m+n-2},\dots$ 中,因此
		\[
		\deg(fg)<m+n,
		\]
		从而不可能有 $\deg(fg)=m+n$。
	\end{itemize}
	综上,
	\[
	\deg(fg)=\deg f+\deg g \quad\Longleftrightarrow\quad a_mb_n\neq 0.
	\]
	
	\medskip
	\textbf{特别地:若 $R$ 是整环,则 $S$ 也是整环.}
	
	设 $R$ 为整环,取任意 $0\neq f,g\in R[x]$。则它们的首项系数 $a_m\neq 0,\ b_n\neq 0$。
	由于 $R$ 无零因子,必有
	\[
	a_mb_n\neq 0.
	\]
	由上面(2)知
	\[
	\deg(fg)=m+n\ge 0,
	\]
	从而 $fg\neq 0$。这说明 $R[x]$ 中任意两个非零元素的乘积仍非零,
	即 $R[x]$ 无零因子,因此 $R[x]$ 是整环。
	
	
	\subsection{ $R$ 为整环,$f,g\in R[x]$, $f\neq 0$ 且 $g\neq 0$ ,则$\deg(fg)=\deg f+\deg g$}
	\textbf{命题.}
	若 $R$ 为整环,$f,g\in R[x]$, $f\neq 0$ 且 $g\neq 0$ 时,恒有 $\deg(fg)=\deg f+\deg g$。若 $f=0$ 或 $g=0$,则 $fg=0$,若约定 $\deg 0=-\infty$,则公式仍成立。
	
	\textbf{证明.}
	设 $f\neq 0,\ g\neq 0$,令
	\[
	\deg f=m,\qquad \deg g=n.
	\]
	则可写
	\[
	f(x)=a_m x^m+a_{m-1}x^{m-1}+\cdots+a_0,\quad a_m\neq 0,
	\]
	\[
	g(x)=b_n x^n+b_{n-1}x^{n-1}+\cdots+b_0,\quad b_n\neq 0.
	\]
	考虑乘积 $f(x)g(x)$。其中最高次数项来自于 $a_m x^m$ 与 $b_n x^n$ 的相乘,
	因此 $x^{m+n}$ 的系数为
	\[
	a_m b_n.
	\]
	由于 $R$ 是整环(无零因子),且 $a_m\neq 0,\ b_n\neq 0$,必有
	\[
	a_m b_n\neq 0.
	\]
	于是 $f(x)g(x)$ 的 $x^{m+n}$ 项系数非零,故
	\[
	\deg(fg)=m+n=\deg f+\deg g.
	\]
	
	若 $f=0$ 或 $g=0$,则
	\[
	\deg(fg)=\deg 0=-\infty=\deg f+\deg g
	\]
	(因为 $-\infty + n = -\infty$),公式仍保持一致。
	
	
	\subsection{思考题2.8.2}
	\begin{tabularx}{\textwidth}{|c|X|X|}
		\hline
		编号 & 命题 & 结论与说明 \\
		\hline
		(1) 
		& 若 $R$ 为主理想整环,$R[x]$ 是否为主理想整环? 
		& 一般\textbf{不是}。反例:取 $R=\mathbb{Z}$,则 $\mathbb{Z}$ 是主理想整环,但
		$\mathbb{Z}[x]$ 不是主理想整环。  
		考虑理想 $I=(2,x)\subset\mathbb{Z}[x]$。若 $I$ 为主理想,设 $I=(f(x))$。则有
		$2,x\in(f(x))$,所以 $f(x)\mid 2$ 且 $f(x)\mid x$。在整环 $\mathbb{Z}[x]$ 中,
		$\gcd(2,x)=1$,于是 $f(x)$ 为单位 $\pm1$,从而 $(f(x))=\mathbb{Z}[x]$。
		但 $1\notin(2,x)$(否则 $1=2a(x)+xb(x)$,令 $x=0$ 得 $1=2a(0)$,矛盾),
		故 $(2,x)$ 不是主理想,所以 $\mathbb{Z}[x]$ 不是主理想整环。 \\
		\hline
		(2)
		& 若 $R$ 为欧几里得环,$R[x]$ 是否为欧几里得环?
		& 一般\textbf{不是}。仍取 $R=\mathbb{Z}$。$\mathbb{Z}$ 是欧几里得环,但上面已知
		$\mathbb{Z}[x]$ 不是主理想整环。  
		然而任何欧几里得环必为主理想整环(欧几里得环 $\Rightarrow$ PID)。  
		若 $\mathbb{Z}[x]$ 是欧几里得环,则必是主理想整环,与 $I=(2,x)$ 不是主理想矛盾。
		故 $R$ 为欧几里得环时,$R[x]$ 也未必为欧几里得环。 \\
		\hline
	\end{tabularx}
	
	
	\subsection{思考题2.8.3}
	
	\subsection{$R$ 为整环,则多项式环 $R[x]$ 亦为整环}
	\textbf{命题.} 若 $R$ 为整环,则多项式环 $R[x]$ 亦为整环。
	
	\textbf{证明.}
	因为 $R$ 为整环,故 $R$ 中无零因子,即
	\[
	ab = 0 \ \Longrightarrow\ a=0 \text{ 或 } b=0.
	\]
	
	现取 $f(x),g(x)\in R[x]$,并假设
	\[
	f(x)\neq 0,\qquad g(x)\neq 0.
	\]
	设
	\[
	f(x)=a_m x^m + a_{m-1}x^{m-1} +\cdots + a_0,\qquad a_m\neq 0,
	\]
	\[
	g(x)=b_n x^n + b_{n-1}x^{n-1} +\cdots + b_0,\qquad b_n\neq 0,
	\]
	其中 $a_m$ 与 $b_n$ 为它们的首项系数。
	
	考虑乘积 $f(x)g(x)$ 的最高次项。逐项相乘可知其最高次项等于
	\[
	a_m b_n x^{m+n}.
	\]
	由于 $R$ 为整环且 $a_m\neq 0,\ b_n\neq 0$,可得
	\[
	a_m b_n \neq 0.
	\]
	因此 $f(x)g(x)$ 的最高次项系数非零,从而
	\[
	f(x)g(x)\neq 0.
	\]
	
	这说明 $R[x]$ 中两个非零多项式的乘积不可能为零,因此 $R[x]$ 中无零因子。
	
	综上,$R[x]$ 为整环。
	\qed
	
	\subsection{$R[a]$ 是 $R_1$ 中\textbf{包含 $R$ 及 $a$ 的最小子环}}
	\textbf{命题.}
	$R_1$是交换幺环,$R$是$R_1$的子环,且包含$R_1$的幺元,取定 $a\in R_1$。
	令同态
	\[
	\varphi: R[x]\to R_1,\qquad \varphi(f)=f(a),
	\]
	并记其像为
	\[
	R[a]:=\varphi(R[x])=\{f(a)\mid f(x)\in R[x]\}.
	\]
	证明\ $R[a]$ 是 $R_1$ 中\textbf{包含 $R$ 及 $a$ 的最小子环}。
	
	\textbf{证明.}
	我们证明两点:
	
	\medskip
	\textbf{(1) $R[a]$ 是 $R_1$ 的子环,且包含 $R$ 与 $a$.}
	
	由于 $\varphi$ 是环同态,环同态的像总是子环,因此 $R[a]=\varphi(R[x])$ 是 $R_1$ 的子环。
	
	并且对任意 $r\in R$,取常数多项式 $f(x)=r\in R[x]$,则
	\[
	r=\varphi(f)=f(a)\in R[a],
	\]
	故 $R\subseteq R[a]$。
	
	再取多项式 $f(x)=x\in R[x]$,则
	\[
	a=f(x)\in R[a].
	\]
	因此 $R[a]$ 的确是 $R_1$ 中一个包含 $R$ 与 $a$ 的子环。
	
	\medskip
	\textbf{(2) $R[a]$ 在所有包含 $R$ 与 $a$ 的子环中最小.}
	
	设 $S$ 是 $R_1$ 的任意子环,满足 $R\subseteq S$ 且 $a\in S$。
	任取 $R[a]$ 中元素 $y$,则存在 $f(x)\in R[x]$ 使得 $y=f(a)$。
	把 $f$ 写成
	\[
	f(x)=a_0+a_1x+\cdots+a_m x^m,\qquad a_i\in R.
	\]
	于是
	\[
	y=f(a)=a_0+a_1a+\cdots+a_m a^m.
	\]
	因为 $a_i\in R\subseteq S$ 且 $a\in S$,又 $S$ 是子环,故对每个 $i$ 有
	\[
	a^i\in S,\qquad a_i a^i\in S,
	\]
	并且 $S$ 对加法封闭,所以有限和仍在 $S$ 中,得到 $y\in S$。
	因此
	\[
	R[a]\subseteq S.
	\]
	
	由于对\textbf{任意}包含 $R$ 与 $a$ 的子环 $S$ 都有 $R[a]\subseteq S$,
	这说明 $R[a]$ 被包含在所有这类子环之中,从而它就是它们的交:
	\[
	R[a]=\bigcap_{\substack{S\le R_1\\ R\subseteq S,\ a\in S}} S,
	\]
	因此 $R[a]$ 正是 $R_1$ 中包含 $R$ 与 $a$ 的\textbf{最小子环}。
	\qed
	
	
	\subsection{title}
	\textbf{说明:多项式环的两个定义及其联系}
	
	在环论里,“$R$ 上的一元多项式环 $R[x]$”通常有 \emph{两个等价的定义}:
	
	\medskip
	%%%%%%%%%%%%%%%%%%%%%%%%%%%%%%%%%%%%%%%%%%%%%%%%%%%%%%%%
	\noindent\textbf{定义 1(序列 / 形式和的具体构造).}
	
	设 $R$ 为一个环。记
	\[
	S:=\Bigl\{\,f=(a_0,a_1,a_2,\dots)\ \Bigm|\ a_i\in R,\ \text{只有有限多个 }a_i\neq 0 \Bigr\}.
	\]
	把 $f=(a_0,a_1,\dots)$ 形式地写成
	\[
	f(x)=a_0+a_1x+a_2x^2+\cdots+a_nx^n
	\quad(a_n\neq 0),
	\]
	此时 $f$ 的\emph{次数}定义为 $\deg f=n$,零元 $(0,0,\dots)$ 记为 $0$,并约定 $\deg 0=-\infty$。
	
	在 $S$ 上定义加法、乘法如下:若
	\[
	f=(a_0,a_1,\dots),\qquad g=(b_0,b_1,\dots),
	\]
	则
	\[
	f+g=(a_0+b_0,\ a_1+b_1,\ a_2+b_2,\dots),
	\]
	\[
	fg=(c_0,c_1,c_2,\dots),\qquad
	c_k=\sum_{i+j=k}a_i b_j,\quad k=0,1,2,\dots.
	\]
	因为 $f,g$ 只有有限个非零项,上式中每个 $c_k$ 都是有限和,且当 $k$ 足够大时 $c_k=0$,故 $fg\in S$。
	
	容易验证:$S$ 在上述加法和乘法下构成一个环;若 $R$ 是交换环,则 $S$ 也是交换环;若 $R$ 含幺,则 $S$ 含有相同的幺元。我们把这个环记作
	\[
	R[x]:=S,
	\]
	称为 \emph{$R$ 上的一元多项式环},其中元素称为 $R$ 上的(系数在 $R$ 中的)一元多项式。
	
	这是教科书中“把多项式看成有限非零系数序列”得到的\emph{具体构造性定义}。
	
	\medskip
	%%%%%%%%%%%%%%%%%%%%%%%%%%%%%%%%%%%%%%%%%%%%%%%%%%%%%%%%
	\noindent\textbf{定义 2(泛性质 / 抽象定义).}
	
	设 $R$ 是交换环。一个环 $S$ 连同一个元素 $x\in S$,若满足下列性质:
	
	\begin{itemize}
		\item $R$ 可嵌入 $S$ 中,视为一个子环(记该嵌入为 $R\hookrightarrow S$);
		\item 对任意交换环 $T$、任意环同态
		\[
		\varphi:R\longrightarrow T
		\]
		以及任意元素 $t\in T$,存在唯一的环同态
		\[
		\Phi:S\longrightarrow T
		\]
		使得
		\[
		\Phi|_R=\varphi,\qquad \Phi(x)=t,
		\]
	\end{itemize}
	
	则称 $(S,x)$ 为 \emph{$R$ 上的一元多项式环},记作 $S=R[x]$,而 $x$ 称为 \emph{不定元(超越元)}。
	
	这一定义表述的是:$S$ 是在 $R$ 上加入一个“形式元” $x$ 所得到的“最通用”的环——
	任何把 $R$ 映入某个环 $T$ 并且选定 $T$ 中一个元素 $t$ 的情形,都可以唯一地由一个
	同态 $R[x]\to T$ 表达出来(把 $x$ 送到 $t$,把多项式 $a_0+a_1x+\cdots+a_nx^n$
	送到 $a_0+a_1t+\cdots+a_nt^n$)。这叫做 $R[x]$ 的\emph{泛性质}或\emph{普遍性质}。
	
	\medskip
	%%%%%%%%%%%%%%%%%%%%%%%%%%%%%%%%%%%%%%%%%%%%%%%%%%%%%%%%
	\noindent\textbf{命题.} 上述“序列构造”得到的环 $S$,连同元素
	\[
	x:=(0,1,0,0,\dots)\in S,
	\]
	满足定义 2 的泛性质;因此 $(S,x)$ 就是 $R$ 上的一元多项式环。进一步地,
	任何满足定义 2 的多项式环在环同构意义下是唯一的。
	
	\textbf{证明大意.}
	
	设 $S$ 为定义 1 中的多项式环,$x=(0,1,0,\dots)$。给定交换环 $T$、环同态
	$\varphi:R\to T$ 及元素 $t\in T$。我们定义
	\[
	\Phi:S\to T,\qquad
	\Phi\!\left(\sum_{i=0}^n a_i x^i\right)
	:=\sum_{i=0}^n \varphi(a_i)\,t^i.
	\]
	需要验证:
	
	\begin{itemize}
		\item 该定义与多项式的写法无关(即同一多项式只有一种表示),故 $\Phi$ 良好定义;
		\item $\Phi$ 保持加法与乘法,是环同态;
		\item $\Phi$ 在 $R$ 上与 $\varphi$ 一致:把 $a\in R$ 看成常数多项式 $a_0=a$、其余系数 $0$,则
		\[
		\Phi(a)=\varphi(a);
		\]
		\item $\Phi(x)=t$ 由定义立即得到;
		\item 若 $\Psi:S\to T$ 也是环同态并满足 $\Psi|_R=\varphi$、$\Psi(x)=t$,则对任意
		$f(x)=\sum a_i x^i$ 有
		\[
		\Psi(f)=\sum \Psi(a_i)\Psi(x)^i
		=\sum \varphi(a_i)t^i=\Phi(f),
		\]
		故 $\Psi=\Phi$,唯一性成立。
	\end{itemize}
	
	于是 $(S,x)$ 满足定义 2 的条件,即序列构造出来的 $R[x]$ 正是具有该泛性质的多项式环。
	
	另一方面,若 $(S_1,x_1)$、$(S_2,x_2)$ 都满足定义 2,则可分别对
	\[
	\varphi=\mathrm{id}_R,\ t=x_2
	\quad\text{和}\quad
	\varphi=\mathrm{id}_R,\ t=x_1
	\]
	应用泛性质,得到唯一同态
	\[
	\Phi_{12}:S_1\to S_2,\qquad \Phi_{21}:S_2\to S_1,
	\]
	并可验证 $\Phi_{21}\circ\Phi_{12}=\mathrm{id}_{S_1}$、
	$\Phi_{12}\circ\Phi_{21}=\mathrm{id}_{S_2}$,从而 $S_1\cong S_2$。
	这说明满足定义 2 的多项式环在同构意义下是唯一的。
	
	\medskip
	%%%%%%%%%%%%%%%%%%%%%%%%%%%%%%%%%%%%%%%%%%%%%%%%%%%%%%%%
	\noindent\textbf{总结.}
	
	\begin{itemize}
		\item \emph{定义 1} 给出了一个\textbf{具体的构造}:用“有限非零系数序列 + 卷积乘法”建出一个环。
		它用于证明“多项式环确实存在”。
		\item \emph{定义 2} 用\textbf{泛性质}刻画了多项式环:$R[x]$ 是“在 $R$ 上自由 adjoin 一个不定元”的最通用的环。
		它说明“只要满足这个性质的环都和我们用序列构造出的 $R[x]$ 同构”,从而给出了一个本质上不依赖具体实现的抽象定义。
	\end{itemize}
	
	因此,书上说“多项式环有两个定义”,其实是:先用\emph{序列}方式做出一个具体的 $R[x]$,
	再证明它满足\emph{泛性质},从而与任何其它“长得像 $R[x]$ 的环”同构。
	从此以后,常常直接使用第二种抽象定义来处理代入、延拓同态等问题。
	
	
	\textbf{定理.} 设 $R$ 为任意交换含幺环,则存在一个环 $R[x]$,
	称为 $R$ 上的一元多项式环,使得其元素可表示为形式
	\[
	a_0 + a_1 x + a_2 x^2 + \cdots + a_n x^n,\qquad a_i\in R,
	\]
	且只有有限多个 $a_i$ 非零。
	
	\textbf{证明(存在性).}
	
	\textbf{第一步:构造候选集合。}
	
	定义
	\[
	S := \Bigl\{\, (a_0,a_1,a_2,\dots)\ \Bigm|\ a_i\in R,\ 
	\text{且仅有有限个 } a_i \neq 0 \Bigr\}.
	\]
	这是一个集合,其元素是 $R$ 中的“有限支撑序列”。
	
	对 $f=(a_0,a_1,\dots)$,用形式和记为
	\[
	f(x)=a_0+a_1 x + a_2 x^2+\cdots.
	\]
	
	\textbf{第二步:定义加法。}
	
	对 $f=(a_0,a_1,\dots)$、$g=(b_0,b_1,\dots)$,定义
	\[
	f+g := (a_0+b_0,\ a_1+b_1,\ a_2+b_2,\dots).
	\]
	因为 $a_i,b_i$ 各自只有有限个非零项,所以 $a_i+b_i$ 只有有限个非零项,
	故 $f+g\in S$。
	
	\textbf{第三步:定义乘法(卷积乘法)。}
	
	定义
	\[
	fg := (c_0,c_1,c_2,\dots),
	\qquad c_k := \sum_{i+j=k}a_i b_j.
	\]
	注意:
	
	(1) 由于 $a_i$ 和 $b_j$ 只有有限多个非零项,所以满足 $i+j=k$ 的 $(i,j)$ 也只有有限对,  
	因此每个 $c_k$ 是有限和,落在 $R$ 中;
	
	(2) 当 $k$ 足够大时,所有满足 $i+j=k$ 的 $i,j$ 均越过了非零区域,因此 $c_k=0$,  
	故 $fg$ 也是有限非零序列,所以 $fg\in S$。
	
	\textbf{第四步:验证 $S$ 构成一个交换含幺环。}
	
	\begin{itemize}
		\item 加法显然满足交换律、结合律,并由零序列 $(0,0,\dots)$ 作为加法单位;
		\item 加法逆元存在:$-(a_0,a_1,\dots)=(-a_0,-a_1,\dots)$;
		\item 乘法满足结合律(这是卷积的标准性质,可直接由有限和的展开检查);
		\item 乘法单位存在:定义
		\[
		1_S := (1,0,0,\dots),
		\]
		对任意 $f$,计算卷积可得 $1_S f = f$;
		\item 若 $R$ 是交换的,则每项 $a_i b_j = b_j a_i$,从而卷积保持交换性,因此 $S$ 是交换环。
	\end{itemize}
	
	\textbf{第五步:因此 $S$ 是一个环,我们把它记为 $R[x]$。}
	
	定义的 $x:=(0,1,0,0,\dots)$,并可验证 $x^n=(0,0,\dots,1,0,\dots)$ 使得
	\[
	(a_0,a_1,\dots,a_n,0,\dots)=a_0 + a_1 x + \cdots + a_n x^n.
	\]
	
	于是 $S$ 与我们熟悉的“形式多项式”完全一致。
	
	\textbf{第六步(结论):} 环 $S$ 满足一元多项式的所有结构,故可定义为
	\[
	R[x] := S.
	\]
	
	\hfill\qed
	
	\medskip
	
	\textbf{说明.} 上述构造完全基于集合论与环运算的演算,不用任何其它附加结构,
	因此适用于\emph{任意}交换含幺环 $R$。
	这就证明了一元多项式环 $R[x]$ 的确在任何交换含幺环上都\emph{可以构造出来},因而必定存在。
	
	
	\clearpage 
	\subsection*{课后习题答案}
	\addcontentsline{toc}{subsection}{\textcolor{red}{课后习题答案}}
	\begin{enumerate}[label=\textcolor{blue}{\textbf{\large\arabic*.}}]	
		\item 
		
	\end{enumerate}
	
	\clearpage
	\section{整环上的多项式环}
	\subsection{定理2.9.4的证明}
	\textbf{定理 2.9.4.}
	设 $R$ 为整环,$R[x]$ 为 $R$ 上的多项式环。
	设 $f(x),g(x)\in R[x]$ 的次数分别为 $m$ 与 $n$,$g(x)\neq0$。
	记 $k=\max (m-n+1,0)$,设 $a$ 为 $g(x)$ 的首项系数,则存在多项式
	$q(x),r(x)\in R[x]$ 使得
	\[
	a^{k}f(x)=q(x)g(x)+r(x),
	\]
	其中 $\deg r(x)<n$,并且这样的 $q(x),r(x)$ 是唯一的。
	
	\textbf{证明.}
	分两步:先证存在性,再证唯一性。
	
	\medskip
	\textbf{一、存在性:对 $m=\deg f$ 作归纳。}
	
	\textbf{(1) 基本情况:$m<n$。}
	
	此时 $k=\max(m-n+1,0)=0$,命题变为:
	\[
	f(x)=q(x)g(x)+r(x),\qquad \deg r(x)<n.
	\]
	取 $q(x)=0,\ r(x)=f(x)$ 即可,显然 $\deg r(x)=m<n$,故命题成立。
	
	\medskip
	\textbf{(2)} 数学归纳法证明$m \ge n-1$时成立(相当于把n当作“定值”)
	
	
	\textbf{(i)}第一步:基例 :由(1)知$m=n-1$时已经成立 
	
	
	\textbf{(ii)}第二步:归纳假设:
	
	假设$m \le l\quad (l \ge n-1)$时原命题成立。下证明$m=l+1$时成立
	
	
	\textbf{(iii)}第三步:归纳推理:\\
	
	令
	\[
	h(x)=a f(x)-b x^{l+1-n}g(x).
	\]
	注意到 $a f(x)$ 与 $b x^{l+1-n}g(x)$ 的最高次项同为 $ab x^{l+1}$,且系数相同,
	所以相减后最高次项被抵消,从而
	\[
	\deg h(x)\le l.
	\]
	
	设 $\deg h(x)=l'\le l$。根据归纳假设,对多项式 $h(x)$ 可以应用定理。
	记
	\[
	k'=\max(l'-n+1,0),
	\]
	则存在 $q_1(x),r(x)\in R[x]$ 使得
	\[
	a^{k'}h(x)=q_1(x)g(x)+r(x),\qquad \deg r(x)<n.
	\]
	
	由 $l'\le l$ 得
	\[
	k'=\max(l'-n+1,0)\le \max(l-n+1,0)
	\]
	而$l \ge n-1$,所以
	\[
	k'=\max(l'-n+1,0)\le \max(l-n+1,0) = l-n+1
	\]
	所以
	\[
	l-n+1-k' \ge 0
	\]
	于是在$a^{k'}h(x)=q_1(x)g(x)+r(x)$左右同乘 $a^{l-n+1-k'}$,得到
	\[
	a^{l-n+1}h(x)=a^{l-n+1-k'}q_1(x)g(x)+a^{l-n+1-k'}r(x).
	\]
	现在利用 $h(x)=a f(x)-b x^{l+1-n}g(x)$,有
	\[
	a^{l-n+1}\bigl(a f(x)-b x^{l+1-n}g(x)\bigr)
	= a^{l-n+1}q_1(x)g(x)+a^{l-n+1}r(x),
	\]
	即
	\[
	a^{l+1-n+1}f(x)
	=\Bigl(a^{l+1-n+1}bx^{l+1-n}+a^{l-n+1}q_1(x)\Bigr)g(x)+a^{l-n+1}r(x).
	\]
	
	令
	\[
	q(x)=a^{l+1-n+1}bx^{l+1-n}+a^{l-n+1}q_1(x),\qquad
	\tilde r(x)=a^{l-n+1}r(x),
	\]
	则上式即为
	\[
	a^{k}f(x)=q(x)g(x)+\tilde r(x).
	\]
	
	因为 $\deg r(x)<n$ 且 $a\neq0$,在整环 $R$ 中,
	$a^{l-n+1}$ 为非零元素,从而乘以它不改变多项式的次数:
	\[
	\deg\tilde r(x)=\deg r(x)<n.
	\]
	因此对次数为 $l+1$ 的 $f(x)$,也得到了所需的分解。
	由数学归纳法,存在性得证。
	
	\medskip
	\medskip
	\textbf{二、唯一性:}
	
	假设存在两组多项式 $q_1(x), r_1(x)$ 与 $q_2(x), r_2(x)$,都满足定理中的分解式:
	\[
	a^k f(x)=q_1(x)g(x)+r_1(x)=q_2(x)g(x)+r_2(x),
	\]
	并且
	\[
	\deg r_1(x)<n,\qquad \deg r_2(x)<n.
	\]
	
	从两式相减得到
	\[
	\bigl(q_1(x)-q_2(x)\bigr)g(x)=r_2(x)-r_1(x). \tag{1}
	\]
	
	\textbf{(1) 若 $q_1(x)\ne q_2(x)$,则导出矛盾。}
	
	因为 $R$ 为整环,故 $R[x]$ 也是整环。
	若 $q_1-q_2\neq 0$,则其次数为某个非负整数,记为 $d\ge 0$。
	
	于是左边多项式的次数为
	\[
	\deg\bigl((q_1-q_2)g\bigr)=\deg(q_1-q_2)+\deg g = d+n \ge n. \tag{2}
	\]
	
	而右边的多项式 $r_2-r_1$ 的次数满足:
	\[
	\deg(r_2-r_1)\le \max\{\deg r_1,\deg r_2\}<n. \tag{3}
	\]
	
	式 (2) 表明左边次数 $\ge n$,而式 (3) 表明右边次数 $<n$。
	这是不可能的,因为两个多项式相等时必须次数相同。
	
	因此只能有
	\[
	q_1(x)-q_2(x)=0,
	\]
	即
	\[
	q_1(x)=q_2(x).
	\]
	
	\textbf{(2) 从而 $r_1(x)=r_2(x)$。}
	
	将 $q_1=q_2$ 代回等式 (1),得到
	\[
	0=r_2(x)-r_1(x),
	\]
	即
	\[
	r_1(x)=r_2(x).
	\]
	
	\medskip
	综上,满足 $\deg r(x)<n$ 的分解
	\[
	a^k f(x)=q(x)g(x)+r(x)
	\]
	在 $R[x]$ 中是唯一的。\qed
	
	\subsection{定理2.9.4的推论:域上的多项式除法}
	\textbf{(域上的多项式除法)}
	设 $F$ 为一域,$f(x),g(x)\in F[x]$ 且 $g(x)\neq 0$,记
	\[
	m:=\deg f(x),\quad n:=\deg g(x).
	\]
	则存在唯一的 $q(x),r(x)\in F[x]$,使得
	\[
	f(x) = q(x)g(x) + r(x),
	\]
	其中或者 $r(x)=0$,或 $\deg r(x) < \deg g(x)$。
	
	\textbf{证明.}
	将定理 2.9.4 应用于整环 $R=F$,于是 $F[x]$ 满足:
	设 $a$ 为 $g(x)$ 的首项系数,$k=\max(m-n+1,0)$,则存在唯一的 $q(x),r(x)\in F[x]$ 使得
	\[
	a^k f(x) = q(x) g(x) + r(x),\qquad \deg r(x) < n.
	\]
	
	因为 $F$ 是域,$a\neq 0$ 在 $F$ 中可逆,记 $a^{-1}$ 为其逆元。
	将上式两边同乘以 $a^{-k}$,得
	\[
	f(x) = a^{-k} q(x) g(x) + a^{-k} r(x).
	\]
	令
	\[
	\tilde{q}(x) := a^{-k} q(x),\qquad \tilde{r}(x) := a^{-k} r(x),
	\]
	可得
	\[
	f(x) = \tilde{q}(x) g(x) + \tilde{r}(x).
	\]
	
	注意到 $a^{-k}\neq 0$,所以若 $r(x)=0$ 则 $\tilde{r}(x)=0$;若 $r(x)\neq 0$,则
	\[
	\deg \tilde{r}(x) = \deg \bigl(a^{-k} r(x)\bigr) = \deg r(x) < n = \deg g(x),
	\]
	故 $\tilde{r}(x)$ 同样满足余式次数 $<\deg g$ 的条件.
	\qed 
	
	\subsection{$F$ 为域,则多项式环 $F[x]$ 是欧几里得环,从而是主理想整环和唯一分解整环}
	
	\textbf{命题 1.} 若 $F$ 为一域,则多项式环 $F[x]$ 是欧几里得环。
	
	\textbf{证明.}
	定义对任意非零多项式 $f(x)\in F[x]$:
	\[
	\delta\bigl(f(x)\bigr):=\deg f(x)\in\mathbb{N}\cup\{0\}.
	\]
	显然 $F[x]$ 是整环,且 $\delta$ 将 $F[x]^{\ast}$ 映到 $\mathbb{N}\cup\{0\}$。
	
	现取任意 $f(x),g(x)\in F[x]$,其中 $g(x)\neq 0$。由于 $F$ 为域,在 $F[x]$ 中可进行多项式带余除法:存在唯一的 $q(x),r(x)\in F[x]$ 使得
	\[
	f(x) = q(x)g(x) + r(x),\qquad r(x)=0 \ \text{或}\ \deg r(x)<\deg g(x).
	\]
	以 $\delta$ 表示即为
	\[
	r(x)=0 \ \text{或}\ \delta\bigl(r(x)\bigr)<\delta\bigl(g(x)\bigr).
	\]
	这正好满足欧几里得环定义中的条件。
	
	因此 $(F[x],\delta)$ 是欧几里得环。从而$F[x]$为主理想整环
	\qed
	
	\subsection{本原分解的相伴唯一性}
	\textbf{命题(内容–本原分解的相伴唯一性).}
	设 $R$ 为唯一分解整环(UFD)。对任意非零多项式 $f(x)\in R[x]$,设
	$c$ 为 $f(x)$ 的系数的一个最大公因,则可写成
	\[
	f(x)=c\,f_1(x),
	\]
	其中 $f_1(x)$ 为本原多项式。并且这种分解在相伴意义下是唯一的:若
	\[
	f(x)=c' f_1'(x)
	\]
	也是满足 $c'\in R$ 且 $f_1'(x)$ 为本原多项式的分解,则存在单位 $u$ 使
	\[
	f_1'(x)=u f_1(x),\qquad c'=u^{-1}c.
	\]
	
	\textbf{证明.}
	由于 $R$ 为 UFD,$R$ 中的元素具有唯一分解性质。
	
	设 $f(x)$ 为非零多项式,其全部系数的集合记为
	\[
	\mathrm{coeff}(f)=\{a_0,a_1,\dots,a_n\}\subset R.
	\]
	定义 $c=\gcd(a_0,\dots,a_n)$(在 UFD 中定义良好),并定义
	\[
	f_1(x)=\frac{1}{c}f(x),
	\]
	则 $f_1(x)$ 的系数两两互素(即 $\gcd$ 为单位),故 $f_1(x)$ 为本原多项式。于是
	\[
	f(x)=c f_1(x)
	\]
	给出了一个内容–本原的分解。
	
	现在设还有另一分解
	\[
	f(x)=c' f_1'(x),
	\]
	其中 $f_1'(x)$ 也是本原多项式。比较两种分解,得
	\[
	c f_1(x)=c' f_1'(x),
	\]
	从而
	\[
	f_1'(x)=\frac{c}{c'} f_1(x).
	\]
	设
	\[
	u:=\frac{c}{c'}\in \mathrm{Frac}(R),
	\]
	我们证明 $u$ 必为 $R$ 中的单位。
	
	由于 $f_1(x)$ 与 $f_1'(x)$ 都是本原多项式,其系数集合分别为
	\[
	\{b_0,b_1,\dots,b_m\},\qquad 
	\{b_0',b_1',\dots,b_m'\}.
	\]
	由 $f_1'(x)=u f_1(x)$ 可知
	\[
	b_i' = u b_i,\qquad (0\le i\le m).
	\]
	因为 $f_1'(x)$ 是本原多项式,所以 $\{b_0',\dots,b_m'\}$ 的最大公因子为单位。若 $u$ 不是 $R$ 的单位,则 $u$ 与 $b_i$ 至少含有一个非单位不可约因子,从而 $b_i'$ 将全部含有某个不可约因子作为公因子,这与 $f_1'(x)$ 为本原多项式矛盾。
	
	因此 $u$ 必是 $R$ 的单位,记 $u\in R^*$。由
	\[
	c f_1(x)=c' f_1'(x)=c'(u f_1(x))
	\]
	可得
	\[
	c = u^{-1} c'.
	\]
	
	综上所述:
	\[
	f_1'(x)=u f_1(x),\qquad c' = u^{-1}c.
	\]
	这说明内容–本原分解在相伴意义下唯一。
	
	\qed
	
	\subsection{域同态诱导的环同态}
	\textbf{命题.}\;
	设 $F,E$ 为域,$\phi:F\to E$ 为域同态。
	定义映射
	\[
	\bar\phi:F[x]\longrightarrow E[x],\qquad 
	\bar\phi\!\left(\sum_{i=0}^n a_i x^i\right):=\sum_{i=0}^n \phi(a_i)x^i.
	\]
	证明:$\bar\phi$ 是环同态(并且是带幺环同态),即对任意 $f,g\in F[x]$ 有
	\[
	\bar\phi(f+g)=\bar\phi(f)+\bar\phi(g),\qquad
	\bar\phi(fg)=\bar\phi(f)\bar\phi(g),\qquad
	\bar\phi(1)=1.
	\]
	
	\textbf{解:}
	
	\textbf{(0) 预备:$\bar\phi$ 良定义.}\;
	因为任意 $f(x)\in F[x]$ 都可唯一表示为有限和
	\[
	f(x)=\sum_{i=0}^n a_i x^i\quad (a_i\in F),
	\]
	故按系数逐项施加 $\phi$ 得到
	\[
	\bar\phi(f)=\sum_{i=0}^n \phi(a_i)x^i\in E[x],
	\]
	且由系数表示的唯一性,$\bar\phi(f)$ 不依赖于任何“表示方式”,因此 $\bar\phi$ 是良定义的函数。
	
	\textbf{(1) 加法保持:$\bar\phi(f+g)=\bar\phi(f)+\bar\phi(g)$.}\;
	取任意
	\[
	f(x)=\sum_{i=0}^n a_i x^i,\qquad g(x)=\sum_{i=0}^m b_i x^i,
	\]
	令 $N:=\max\{n,m\}$,并约定当 $i>n$ 时 $a_i=0$,当 $i>m$ 时 $b_i=0$,则
	\[
	f(x)=\sum_{i=0}^{N} a_i x^i,\qquad g(x)=\sum_{i=0}^{N} b_i x^i.
	\]
	于是
	\[
	(f+g)(x)=\sum_{i=0}^{N}(a_i+b_i)x^i.
	\]
	对其施加 $\bar\phi$:
	\[
	\bar\phi(f+g)
	=\bar\phi\!\left(\sum_{i=0}^{N}(a_i+b_i)x^i\right)
	=\sum_{i=0}^{N}\phi(a_i+b_i)x^i.
	\]
	因为 $\phi$ 是域同态,特别是环同态,所以对任意 $i$ 有 $\phi(a_i+b_i)=\phi(a_i)+\phi(b_i)$,故
	\[
	\bar\phi(f+g)
	=\sum_{i=0}^{N}\bigl(\phi(a_i)+\phi(b_i)\bigr)x^i
	=\sum_{i=0}^{N}\phi(a_i)x^i+\sum_{i=0}^{N}\phi(b_i)x^i
	=\bar\phi(f)+\bar\phi(g).
	\]
	加法保持得证。
	
	\textbf{(2) 乘法保持:$\bar\phi(fg)=\bar\phi(f)\bar\phi(g)$.}\;
	仍取
	\[
	f(x)=\sum_{i=0}^n a_i x^i,\qquad g(x)=\sum_{j=0}^m b_j x^j.
	\]
	则在 $F[x]$ 中乘积系数为卷积形式:
	\[
	(fg)(x)=\left(\sum_{i=0}^n a_i x^i\right)\left(\sum_{j=0}^m b_j x^j\right)
	=\sum_{k=0}^{n+m}\left(\sum_{i+j=k} a_i b_j\right)x^k.
	\]
	对其施加 $\bar\phi$:
	\[
	\bar\phi(fg)
	=\bar\phi\!\left(\sum_{k=0}^{n+m}\left(\sum_{i+j=k} a_i b_j\right)x^k\right)
	=\sum_{k=0}^{n+m}\phi\!\left(\sum_{i+j=k} a_i b_j\right)x^k.
	\]
	由于 $\phi$ 为环同态,保持有限和与乘法,因此对每个 $k$,
	\[
	\phi\!\left(\sum_{i+j=k} a_i b_j\right)
	=\sum_{i+j=k}\phi(a_i b_j)
	=\sum_{i+j=k}\phi(a_i)\phi(b_j).
	\]
	代回得
	\[
	\bar\phi(fg)
	=\sum_{k=0}^{n+m}\left(\sum_{i+j=k}\phi(a_i)\phi(b_j)\right)x^k.
	\]
	另一方面,在 $E[x]$ 中
	\[
	\bar\phi(f)\bar\phi(g)
	=\left(\sum_{i=0}^n \phi(a_i)x^i\right)\left(\sum_{j=0}^m \phi(b_j)x^j\right)
	=\sum_{k=0}^{n+m}\left(\sum_{i+j=k}\phi(a_i)\phi(b_j)\right)x^k.
	\]
	两式相同,故 $\bar\phi(fg)=\bar\phi(f)\bar\phi(g)$,乘法保持得证。
	
	\textbf{(3) 幺元保持:$\bar\phi(1)=1$.}\;
	在 $F[x]$ 中的幺元为常数多项式 $1_F$,于是
	\[
	\bar\phi(1)=\bar\phi(1_F)=\phi(1_F)\in E.
	\]
	因为 $\phi$ 是域同态(从而是带幺环同态),满足 $\phi(1_F)=1_E$,故
	\[
	\bar\phi(1)=1_E,
	\]
	即 $\bar\phi$ 保持幺元。
	
	\textbf{(4) 结论.}\;
	由 (1)(2)(3) 可知 $\bar\phi$ 同时保持加法、乘法与幺元,因此
	\[
	\bar\phi:F[x]\to E[x]
	\]
	是带幺环同态(特别地是环同态)。
	
	\textbf{补充(常用性质).}\;
	对任意 $a\in F$,视为常数多项式,有 $\bar\phi(a)=\phi(a)$;
	并且 $\bar\phi(x)=x$(因为 $x=0+1\cdot x$,故 $\bar\phi(x)=0+\phi(1)x=x$)。
	
	
	
	\clearpage 
	\subsection*{课后习题答案}
	\addcontentsline{toc}{subsection}{\textcolor{red}{课后习题答案}}
	\begin{enumerate}[label=\textcolor{blue}{\textbf{\large\arabic*.}}]	
		\item 3.
		\textbf{命题.}
		设 $R$ 为整环,$f(x)\in R[x]$,$a\in R$。则
		\[
		a \text{ 为 } f(x) \text{ 的根} \quad\Longleftrightarrow\quad (x-a)\mid f(x).
		\]
		
		\textbf{证明(利用定理 2.9.4).}\;
		
		取
		\[
		g(x)=x-a\in R[x],
		\]
		则 $\deg g=1$,且 $g(x)$ 的首项系数为 $1$。
		令 $m=\deg f,\ n=\deg g=1$,则
		\[
		k=\max(m-n+1,0)=\max(m,0)=m.
		\]
		由定理 2.9.4,设 $g(x)$ 的首项系数为 $\alpha$,这里 $\alpha=1$,故存在且唯一的
		$q(x),r(x)\in R[x]$ 使得
		\[
		\alpha^{\,k}f(x)=q(x)g(x)+r(x),
		\qquad \deg r< n=1.
		\]
		由于 $\alpha=1$,有 $\alpha^{\,k}=1$,于是上式化为
		\[
		f(x)=q(x)(x-a)+r(x),
		\qquad \deg r<1.
		\]
		因此 $r(x)$ 必为常数多项式,记作 $r(x)=r\in R$。
		
		将 $x=a$ 代入,得
		\[
		f(a)=q(a)(a-a)+r=r.
		\]
		于是
		\[
		f(a)=0 \ \Longleftrightarrow\ r=0.
		\]
		而 $r=0$ 当且仅当
		\[
		f(x)=q(x)(x-a),
		\]
		即 $(x-a)\mid f(x)$。\qed
		
		
		\item 4.
		\textbf{命题.}
		设 $R$ 是整环,$f(x)\in R[x]$,$a_1,a_2,\dots,a_s\in R$ 为 $f(x)$ 的不同的根。
		则有
		\[
		(x-a_1)(x-a_2)\cdots(x-a_s)\mid f(x).
		\]
		由此可知:在整环 $R$ 上,次数为 $m$ 的多项式的不同根的个数最多为 $m$。
		
		\medskip
		\textbf{证明.}
		\textbf{命题.}\;
		设 $R$ 是整环,$f(x)\in R[x]$,且 $a_1,a_2,\dots,a_s\in R$ 是 $f(x)$ 的两两不同的根。
		则
		\[
		(x-a_1)(x-a_2)\cdots(x-a_s)\mid f(x)\quad \text{于 }R[x].
		\]
		因此:在整环 $R$ 上,次数为 $m$ 的多项式的不同根的个数最多为 $m$。
		
		\textbf{证明.}\;
		用数学归纳法对 $s$ 证明。
		
		\textbf{(1) 基础步:$s=1$.}\;
		若 $a_1$ 为 $f(x)$ 的根,则由因式定理(上一命题)
		\[
		(x-a_1)\mid f(x),
		\]
		命题成立。
		
		\textbf{(2) 归纳假设.}\;
		假设对某个 $s-1\ge 1$,结论成立:若 $a_1,\dots,a_{s-1}$ 是 $f(x)$ 的两两不同的根,
		则存在 $g(x)\in R[x]$ 使
		\[
		f(x)=(x-a_1)(x-a_2)\cdots(x-a_{s-1})\,g(x).
		\]
		
		\textbf{(3) 归纳步:由 $s-1$ 推 $s$.}\;
		现设 $a_1,\dots,a_s$ 是 $f(x)$ 的两两不同的根。
		由归纳假设,存在 $g(x)\in R[x]$ 使
		\[
		f(x)=\Bigl(\prod_{i=1}^{s-1}(x-a_i)\Bigr)g(x).
		\]
		将 $x=a_s$ 代入上式,得
		\[
		0=f(a_s)=\Bigl(\prod_{i=1}^{s-1}(a_s-a_i)\Bigr)\,g(a_s).
		\]
		由于 $a_s\neq a_i$($1\le i\le s-1$),故每个 $a_s-a_i\neq 0$。
		又因 $R$ 是整环,无零因子,所以
		\[
		\prod_{i=1}^{s-1}(a_s-a_i)\neq 0.
		\]
		于是由
		\[
		\Bigl(\prod_{i=1}^{s-1}(a_s-a_i)\Bigr)\,g(a_s)=0
		\]
		推出
		\[
		g(a_s)=0,
		\]
		即 $a_s$ 是 $g(x)$ 的根。再由因式定理知
		\[
		(x-a_s)\mid g(x).
		\]
		因此存在 $h(x)\in R[x]$ 使 $g(x)=(x-a_s)h(x)$,代回得
		\[
		f(x)=\Bigl(\prod_{i=1}^{s-1}(x-a_i)\Bigr)(x-a_s)h(x)
		=\Bigl(\prod_{i=1}^{s}(x-a_i)\Bigr)h(x).
		\]
		即
		\[
		(x-a_1)(x-a_2)\cdots(x-a_s)\mid f(x).
		\]
		归纳完成,命题得证。
		\hfill$\square$
		
		\textbf{推论.}\;
		若 $\deg f=m$ 且 $f\neq 0$,则 $f$ 在 $R$ 中的不同根个数最多为 $m$。
		
		\textbf{证明.}\;
		若 $f$ 有 $s$ 个两两不同的根 $a_1,\dots,a_s$,则由命题
		\[
		\prod_{i=1}^s (x-a_i)\mid f(x),
		\]
		从而
		\[
		s=\deg\Bigl(\prod_{i=1}^s (x-a_i)\Bigr)\le \deg f=m.
		\]
		故 $s\le m$。\qed
		
		\item 6.\\
		\textbf{题目.}
		设 $p$ 为一个素数,证明在多项式环 $\mathbb{Z}_p[x]$ 中有分解
		\[
		x^p - x = x(x-1)(x-2)\cdots\bigl(x-(p-1)\bigr).
		\]
		
		\textbf{证明.}
		在环 $\mathbb{Z}_p$ 中,$p$ 为素数,因此对任意 $a\in\mathbb{Z}_p$,
		由费马小定理有
		\[
		a^p \equiv a \pmod p,
		\]
		即在 $\mathbb{Z}_p$ 中可写作
		\[
		a^p-a=0.
		\]
		于是对任意 $a\in\mathbb{Z}_p$,
		\[
		f(a):=a^p-a=0,
		\]
		也就是说,多项式
		\[
		f(x)=x^p-x\in\mathbb{Z}_p[x]
		\]
		在 $\mathbb{Z}_p$ 的每一个元素 $a\in\{0,1,2,\dots,p-1\}$ 处都为零,因此
		\[
		0,1,2,\dots,p-1
		\]
		都是 $f(x)$ 在 $\mathbb{Z}_p$ 中的根。
		
		令
		\[
		g(x)=x(x-1)(x-2)\cdots\bigl(x-(p-1)\bigr)\in\mathbb{Z}_p[x].
		\]
		显然,$g(x)$ 的根恰好也是 $0,1,2,\dots,p-1$,并且这些根两两不同。
		此外,$g(x)$ 为首一多项式,且
		\[
		\deg g(x)=p.
		\]
		而 $f(x)=x^p-x$ 也是首一多项式,且
		\[
		\deg f(x)=p.
		\]
		
		接下来使用如下基本结论:
		
		\quad\textbf{引理:}在域上的非零多项式,其不同根的个数不超过它的次数。
		
		\quad\textbf{说明:}由于 $\mathbb{Z}_p$ 是一个域,$\mathbb{Z}_p[x]$ 中的非零多项式
		至多有与其次数相同个数的不同根。
		
		因为 $f(x)$ 的次数为 $p$,而它在 $\mathbb{Z}_p$ 中已经有 $p$ 个不同的根
		$0,1,\dots,p-1$,所以这就是它的全部根。另一方面,多项式
		\[
		g(x)=x(x-1)\cdots\bigl(x-(p-1)\bigr)
		\]
		也是一个次数为 $p$ 的首一多项式,它的根集与 $f(x)$ 完全相同,且根的重数都为 $1$。
		因此,$f(x)$ 与 $g(x)$ 在 $\mathbb{Z}_p[x]$ 中只能相差一个非零常数因子,即存在
		$c\in\mathbb{Z}_p^*$ 使得
		\[
		f(x)=c\,g(x).
		\]
		
		比较两边的首项系数:$f(x)=x^p-x$ 的首项系数为 $1$,$g(x)$ 展开后也是首一多项式,
		其首项系数为 $1$,故必有 $c=1$。
		于是
		\[
		x^p-x = x(x-1)(x-2)\cdots\bigl(x-(p-1)\bigr)
		\]
		在 $\mathbb{Z}_p[x]$ 中成立。
		
		\qed。
		\item 6.
		\textbf{题目.}\;
		设 $p$ 为素数,证明在多项式环 $\mathbb Z_p[x]$ 中有分解
		\[
		x^p-x=x(x-1)(x-2)\cdots\bigl(x-(p-1)\bigr),
		\]
		并要求\textbf{不使用费马小定理}。
		
		\textbf{答案:}
		
		\textbf{证明.}\;
		在 $\mathbb Z_p[x]$ 中,利用二项式定理
		\[
		(x+1)^p=\sum_{k=0}^p \binom{p}{k}x^k
		= x^p+\binom{p}{1}x^{p-1}+\cdots+\binom{p}{p-1}x+1.
		\]
		由于 $p$ 为素数,故对 $1\le k\le p-1$ 有 $p\mid \binom{p}{k}$,
		从而在 $\mathbb Z_p[x]$ 中
		\[
		(x+1)^p\equiv x^p+1.
		\]
		据此对 $n\in\mathbb N$ 归纳可得
		\[
		(x+n)^p\equiv x^p+n \quad \text{于 }\mathbb Z_p[x].
		\]
		(归纳步:若 $(x+n)^p\equiv x^p+n$,则
		\[
		(x+n+1)^p=((x+n)+1)^p\equiv (x+n)^p+1\equiv x^p+n+1.)
		\]
		
		令 $x=0$,得对 $n=0,1,\dots,p-1$,
		\[
		n^p\equiv n \pmod p,
		\]
		即在 $\mathbb Z_p$ 中
		\[
		\overline n^{\,p}=\overline n.
		\]
		因此多项式
		\[
		f(x):=x^p-x\in\mathbb Z_p[x]
		\]
		在 $\mathbb Z_p$ 中至少有 $p$ 个两两不同的根
		\[
		\bar 0,\bar 1,\dots,\overline{p-1}.
		\]
		
		由于 $\mathbb Z_p$ 是域(从而是整环),而 $\deg f=p$,
		故 $f$ 的不同根个数至多为 $p$,于是 $f$ 的根\emph{恰好}是
		$\bar 0,\bar 1,\dots,\overline{p-1}$。
		由“不同根给出线性因子相乘整除”的命题可知
		\[
		x(x-1)\cdots\bigl(x-(p-1)\bigr)\mid x^p-x
		\quad \text{于 }\mathbb Z_p[x].
		\]
		两边次数同为 $p$,故存在常数 $c\in\mathbb Z_p$ 使
		\[
		x^p-x=c\,x(x-1)\cdots\bigl(x-(p-1)\bigr).
		\]
		比较首项系数:左边首项系数为 $1$,右边乘积的首项系数也为 $1$,
		故 $c=1$。
		
		因此在 $\mathbb Z_p[x]$ 中
		\[
		x^p-x=x(x-1)(x-2)\cdots\bigl(x-(p-1)\bigr).\]
		
		\item 10.
		\textbf{题目.}
		设 $R$ 为整环,$f(x_{1},\dots,x_{n})$ 是 $R$ 上非零 $n$ 元多项式,又 $S$ 是 $R$ 中的一个无限子集。
		证明:必存在 $a_{1},\dots,a_{n}\in S$ 使得
		\[
		f(a_{1},\dots,a_{n})\neq 0.
		\]
		
		\textbf{解:}
		
		\textbf{引理.}
		设 $R$ 为整环,$0\neq g(x)\in R[x]$ 为一元多项式,则 $g(x)$ 在 $R$ 中只有有限个根。
		
		\textbf{证明(引理).}
		设 $\deg g=m$。若 $g$ 有 $m+1$ 个两两不同的根 $a_0,\dots,a_m\in R$,则按因式定理,
		\[
		g(x)=(x-a_0)h_1(x),\quad h_1(x)\in R[x],\ \deg h_1=m-1.
		\]
		同理 $a_1$ 也是 $h_1$ 的根,如此反复,可得到
		\[
		g(x)=(x-a_0)(x-a_1)\cdots(x-a_m)h_{m+1}(x),
		\]
		其中 $h_{m+1}(x)\in R[x]$。于是 $\deg g\ge m+1$,与 $\deg g=m$ 矛盾。
		故 $g$ 不可能有 $m+1$ 个不同的根,从而根的个数有限。
		
		\medskip
		
		下面对元数 $n$ 作数学归纳。
		
		\textbf{(1) 基本情形 $n=1$.}
		
		此时 $f(x)$ 是 $R[x]$ 中的非零一元多项式。
		由上面引理知 $f(x)$ 在 $R$ 中只有有限个根。
		而 $S\subseteq R$ 是无限集合,因此 $S$ 中必有元素 $a_1$ 不是 $f$ 的根,
		即 $f(a_1)\neq 0$。
		命题在 $n=1$ 时成立。
		
		\medskip
		
		\textbf{(2) 归纳步骤.}
		
		设对某个 $n-1\ge1$,命题对所有 $n-1$ 元非零多项式成立:
		对任意整环 $R$、任意非零 $(n-1)$ 元多项式 $g(x_1,\dots,x_{n-1})\in R[x_1,\dots,x_{n-1}]$
		和 $R$ 的任一无限子集 $S$,都能在 $S$ 中找到 $b_1,\dots,b_{n-1}$ 使
		\[
		g(b_1,\dots,b_{n-1})\neq 0.
		\]
		
		现取 $n$ 元情形:设 $0\neq f(x_1,\dots,x_n)\in R[x_1,\dots,x_n]$,
		$S\subseteq R$ 为无限集。把 $f$ 看成关于 $x_n$ 的一元多项式,其系数在环
		$R[x_1,\dots,x_{n-1}]$ 中,即可写成
		\[
		f(x_1,\dots,x_n)
		= a_0(x_1,\dots,x_{n-1})+a_1(x_1,\dots,x_{n-1})x_n+\cdots
		+a_m(x_1,\dots,x_{n-1})x_n^{m},
		\]
		其中 $m\ge0$,且 $a_m(x_1,\dots,x_{n-1})\neq 0$ 是一个 $(n-1)$ 元非零多项式。
		
		由归纳假设,存在 $a_1,\dots,a_{n-1}\in S$,使得
		\[
		a_m(a_1,\dots,a_{n-1})\neq 0.
		\]
		于是固定 $x_1=a_1,\dots,x_{n-1}=a_{n-1}$,得到关于 $x_n$ 的一元多项式
		\[
		h(x_n):=f(a_1,\dots,a_{n-1},x_n)\in R[x_n].
		\]
		其最高次项系数正是 $a_m(a_1,\dots,a_{n-1})\neq 0$,故 $h(x_n)\neq 0$。
		
		由引理可知,$h(x_n)$ 在 $R$ 中只有有限个根。
		记这些根为 $c_1,\dots,c_k$(若没有根则 $k=0$)。
		因为 $S$ 是无限集,而 $\{c_1,\dots,c_k\}$ 只有有限个元素,
		故可在 $S$ 中选取 $a_n$ 使得 $a_n\notin\{c_1,\dots,c_k\}$。
		于是
		\[
		h(a_n)=f(a_1,\dots,a_{n-1},a_n)\neq 0.
		\]
		
		这就找到了 $a_1,\dots,a_n\in S$,满足 $f(a_1,\dots,a_n)\neq 0$。
		因此命题对 $n$ 元情形也成立。
		
		由数学归纳法,命题对一切正整数 $n$ 均成立。
		
		\item 14.
		\textbf{题目.}
		(Eisenstein 判别法)设 $R$ 为唯一分解整环,$F$ 为 $R$ 的分式域,
		\[
		f(x)=a_nx^n+a_{n-1}x^{n-1}+\cdots+a_1x+a_0\in R[x],
		\]
		且存在不可约元素 $p\in R$,使得
		\[
		p\mid a_i\ (0\le i\le n-1),\qquad p\nmid a_n,\qquad p^2\nmid a_0.
		\]
		证明:$f(x)$ 在 $F[x]$ 中为不可约元素。
		
		\textbf{解:}
		
		\begin{proof}
			\textbf{第一步:化为本原多项式情形。}
			
			令 $d=\gcd(a_0,a_1,\dots,a_n)\in R$ 为 $f(x)$ 的内容,则 $f(x)=d\,f_0(x)$,
			其中 $f_0(x)\in R[x]$ 为本原多项式(各系数的最大公因子为单位元)。
			
			因为 $p\nmid a_n$,所以 $p$ 不是所有系数的公因子,从而 $p\nmid d$。
			于是对 $f_0(x)$ 的系数 $b_i:=a_i/d$ 有
			\[
			p\mid b_i\ (0\le i\le n-1),\qquad p\nmid b_n,\qquad p^2\nmid b_0.
			\]
			也就是说,$f_0(x)$ 仍满足题设中关于 $p$ 的整除条件。
			
			注意到 $R$ 为唯一分解整环,则高斯引理成立:本原多项式
			在 $F[x]$ 中可约当且仅当在 $R[x]$ 中可约。
			而 $f(x)=d\,f_0(x)$ 在 $F[x]$ 中可约当且仅当 $f_0(x)$ 在 $F[x]$ 中可约。
			因此我们只需证明:\emph{在上述条件下,本原多项式 $f_0(x)$ 在 $F[x]$ 中不可约}。
			为简单起见,下面仍记 $f_0(x)$ 为 $f(x)$,即假设:
			\[
			\gcd(a_0,\dots,a_n)=1,\quad
			p\mid a_i\ (0\le i\le n-1),\ p\nmid a_n,\ p^2\nmid a_0.
			\]
			
			\medskip
			\textbf{第二步:反设 $f(x)$ 在 $F[x]$ 中可约并降到 $R[x]$。}
			
			反设 $f(x)$ 在 $F[x]$ 中可约,则存在次数 $\ge1$ 的多项式
			$g(x),h(x)\in F[x]$,使得
			\[
			f(x)=g(x)h(x).
			\]
			因为 $f(x)\in R[x]$ 是本原多项式,由高斯引理可知:
			存在本原多项式 $G(x),H(x)\in R[x]$,次数皆 $\ge1$,满足
			\[
			f(x)=G(x)H(x) \qquad\text{(在 } R[x]\text{ 中)}.
			\]
			
			\medskip
			\textbf{第三步:模 $(p)$ 化并利用次数比较得到矛盾。}
			
			由于 $p$ 在唯一分解整环 $R$ 中不可约,因此 $(p)$ 为素理想,
			商环 $\overline{R}:=R/(p)$ 是整环。记自然同态
			\[
			\varphi:R[x]\longrightarrow \overline{R}[x]
			\]
			为系数模 $(p)$ 的映射。
			
			由题设 $p\mid a_i\ (0\le i\le n-1)$ 且 $p\nmid a_n$ 知,
			在 $\overline{R}[x]$ 中有
			\[
			\varphi(f(x))=\overline{a_n}x^n,\qquad \overline{a_n}\neq 0,
			\]
			即 $\varphi(f)$ 只是一个首项不为零的单项式。
			
			另一方面,由 $f(x)=G(x)H(x)$ 得
			\[
			\varphi(f(x))=\varphi(G(x))\,\varphi(H(x))\quad\text{在 }\overline{R}[x]\text{ 中}.
			\]
			因为 $G(x),H(x)$ 在 $R[x]$ 中本原,若 $\varphi(G)=0$,
			则 $G$ 的每个系数都被 $p$ 整除,与 $G$ 本原矛盾;对 $H$ 同理。
			故 $\varphi(G),\varphi(H)$ 在 $\overline{R}[x]$ 中均非零多项式。
			
			设
			\[
			G(x)=b_sx^s+\cdots+b_0,\qquad H(x)=c_tx^t+\cdots+c_0,
			\]
			则 $s,t\ge1$,且 $s+t=n$。常数项满足 $a_0=b_0c_0$。
			因为 $p\mid a_0$ 且 $p^2\nmid a_0$,于是在 $b_0,c_0$ 中
			恰有一个被 $p$ 整除(若两个都被 $p$ 整除则 $p^2\mid a_0$,
			若都不被 $p$ 整除则 $p\nmid a_0$)。
			
			不妨设
			\[
			p\mid b_0,\qquad p\nmid c_0.
			\]
			于是
			\[
			\varphi(b_0)=0,\qquad \varphi(c_0)\neq 0.
			\]
			因此 $\varphi(G(x))$ 的常数项为 $0$,说明 $\varphi(G(x))$ 被 $x$ 整除;
			而 $\varphi(H(x))$ 的常数项非零,说明 $\varphi(H(x))$ 不被 $x$ 整除。
			
			在整环 $\overline{R}[x]$ 中,将
			\[
			\varphi(G(x))=x^r\widetilde{G}(x),\qquad
			\varphi(H(x))=x^s\widetilde{H}(x),
			\]
			其中 $\widetilde{G}(0),\widetilde{H}(0)\neq 0$,于是
			\[
			\varphi(f(x))=\varphi(G(x))\varphi(H(x))
			=x^{r+s}\,\widetilde{G}(x)\widetilde{H}(x).
			\]
			但另一方面 $\varphi(f(x))=\overline{a_n}x^n$,
			故 $x$ 在 $\varphi(f(x))$ 中的指数为 $n$,即
			\[
			r+s=n.
			\]
			由于 $\varphi(H)$ 不被 $x$ 整除,上式中 $s=0$,
			故 $r=n$。于是
			\[
			\deg\varphi(G)=r+\deg\widetilde{G}\ge r=n.
			\]
			而 $\deg\varphi(G)\le\deg G=s<n$(因为 $s+t=n$ 且 $t\ge1$),
			这就产生了矛盾。
			
			因此,假设 $f(x)$ 在 $F[x]$ 中可约不成立。
			故 $f(x)$ 在 $F[x]$ 中不可约。
		\end{proof}
		
		
	\end{enumerate}
	
	
	\clearpage
	\section{对称多项式}
	
	\subsection{$\prod_{i=1}^n (x-\alpha_i)$的系数为对称多项式}
	\textbf{命题.}\;
	设 $R$ 为交换环(例如域),$\alpha_1,\dots,\alpha_n\in R$,证明
	\[
	\prod_{i=1}^n (x-\alpha_i)
	=
	x^n-\sigma_1(\alpha_1,\dots,\alpha_n)x^{n-1}
	+\sigma_2(\alpha_1,\dots,\alpha_n)x^{n-2}
	-\cdots+(-1)^n\sigma_n(\alpha_1,\dots,\alpha_n),
	\]
	其中 $\sigma_k(\alpha_1,\dots,\alpha_n)$ 为第 $k$ 个基本对称多项式。
	
	\textbf{证明:}
	
	\textbf{定义(基本对称多项式).}\;
	对 $k=0,1,\dots,n$,定义
	\[
	\sigma_k(\alpha_1,\dots,\alpha_n)
	:=\sum_{1\le i_1<i_2<\cdots<i_k\le n}\alpha_{i_1}\alpha_{i_2}\cdots\alpha_{i_k},
	\]
	并约定 $\sigma_0(\alpha_1,\dots,\alpha_n):=1$。
	特别地,
	\[
	\sigma_1=\sum_{i=1}^n\alpha_i,\qquad
	\sigma_2=\sum_{1\le i<j\le n}\alpha_i\alpha_j,\qquad
	\sigma_n=\alpha_1\alpha_2\cdots\alpha_n.
	\]
	(之所以称为“对称”,是因为任意置换 $\alpha_i$ 的顺序,上式的值不变。)
	
	\textbf{命题.}\;
	在多项式环 $R[x]$ 中有恒等式
	\[
	\prod_{i=1}^n (x-\alpha_i)=\sum_{k=0}^n (-1)^k\,\sigma_k(\alpha_1,\dots,\alpha_n)\,x^{\,n-k}.
	\]
	因此,$\prod_{i=1}^n (x-\alpha_i)$ 的 $x^{n-k}$ 项系数恰为 $(-1)^k\sigma_k(\alpha_1,\dots,\alpha_n)$,
	从而各个系数都是(基本)对称多项式。
	
	\textbf{证明.}\;
	记
	\[
	P(x):=\prod_{i=1}^n (x-\alpha_i)\in R[x].
	\]
	我们要严格说明:对每个 $k=0,1,\dots,n$,$x^{n-k}$ 的系数等于 $(-1)^k\sigma_k$。
	
	\medskip
	\textbf{第 1 步:把“展开”写成“对所有选择求和”的形式.}\;
	把每个因子 $(x-\alpha_i)$ 看成“两项之和”:
	\[
	(x-\alpha_i)=x+(-\alpha_i).
	\]
	在乘积
	\[
	(x-\alpha_1)(x-\alpha_2)\cdots(x-\alpha_n)
	\]
	中进行分配律展开时,本质操作是:\emph{对每个 $i$,从第 $i$ 个括号里选取一项(选 $x$ 或选 $-\alpha_i$),然后把所有被选出的项相乘;最后把所有可能的选择得到的乘积相加。}
	
	形式化地:令 $S\subseteq\{1,2,\dots,n\}$ 表示“哪些位置选择 $-\alpha_i$”。那么对应的一个乘积项为
	\[
	\Bigl(\prod_{i\in S}(-\alpha_i)\Bigr)\Bigl(\prod_{i\notin S}x\Bigr).
	\]
	由于 $R$ 交换,且 $x$ 与 $R$ 中元素可交换,上式可整理为
	\[
	\Bigl(\prod_{i\in S}(-\alpha_i)\Bigr) x^{\,n-|S|}.
	\]
	因此,整体展开可以写成对所有子集 $S$ 求和:
	\[
	P(x)=\sum_{S\subseteq\{1,\dots,n\}}
	\Bigl(\prod_{i\in S}(-\alpha_i)\Bigr) x^{\,n-|S|}.
	\]
	
	\medskip
	\textbf{第 2 步:按 $|S|$(即选了多少个 $-\alpha$)分组.}\;
	对固定的 $k$,考虑所有满足 $|S|=k$ 的子集 $S$。
	对这样的 $S$,上式中对应项的 $x$ 的次数都是 $n-k$,并且
	\[
	\prod_{i\in S}(-\alpha_i)=(-1)^k\prod_{i\in S}\alpha_i.
	\]
	所以所有 $|S|=k$ 的项加起来恰好给出 $x^{n-k}$ 项的系数:
	\[
	[x^{n-k}]\,P(x)=\sum_{\substack{S\subseteq\{1,\dots,n\}\\ |S|=k}}
	\prod_{i\in S}(-\alpha_i)
	=
	(-1)^k\sum_{\substack{S\subseteq\{1,\dots,n\}\\ |S|=k}}
	\prod_{i\in S}\alpha_i.
	\]
	
	\medskip
	\textbf{第 3 步:识别出基本对称多项式.}\;
	注意
	\[
	\sum_{\substack{S\subseteq\{1,\dots,n\}\\ |S|=k}}
	\prod_{i\in S}\alpha_i
	\]
	正是“从 $\alpha_1,\dots,\alpha_n$ 中任取 $k$ 个相乘再求和”的定义,
	与
	\[
	\sigma_k(\alpha_1,\dots,\alpha_n)
	=
	\sum_{1\le i_1<\cdots<i_k\le n}\alpha_{i_1}\cdots\alpha_{i_k}
	\]
	完全一致(两者只是用“子集”与“严格递增指标组”两种等价方式枚举同一批 $k$ 元选择)。
	
	因此
	\[
	[x^{n-k}]\,P(x)=(-1)^k\sigma_k(\alpha_1,\dots,\alpha_n).
	\]
	
	\medskip
	\textbf{第 4 步:写出整个多项式.}\;
	由于对每个 $k=0,1,\dots,n$ 都成立,
	\[
	P(x)=\sum_{k=0}^n \bigl([x^{n-k}]\,P(x)\bigr)\,x^{n-k}
	=\sum_{k=0}^n (-1)^k\sigma_k(\alpha_1,\dots,\alpha_n)\,x^{n-k}.
	\]
	把 $k=0,1,2,\dots,n$ 依次展开即得到题目中的交错符号形式:
	\[
	\prod_{i=1}^n (x-\alpha_i)
	=
	x^n-\sigma_1x^{n-1}+\sigma_2x^{n-2}-\cdots+(-1)^n\sigma_n.
	\]
	从而证明了展开系数就是(基本)对称多项式(差一个 $(-1)^k$ 的符号因子)。
	
	
	\clearpage 
	\subsection*{课后习题答案}
	\addcontentsline{toc}{subsection}{\textcolor{red}{课后习题答案}}
	\begin{enumerate}[label=\textcolor{blue}{\textbf{\large\arabic*.}}]	
		\item 
		
	\end{enumerate}
		\chapter{模}
	\chapter{域}
\section{域的基本概念}
\subsection{引理 4.1.2}
\textbf{引理 4.1.2.}\;
设 $F$ 是一个非零整环,则 $F$ 是域当且仅当 $F$ 只有平凡理想(即仅有 $\langle 0 \rangle$ 与 $F$)。

\textbf{证明.}\;

\textbf{($\Rightarrow$)}\;
设 $F$ 为域。令 $I\lhd F$ 为任一理想,且 $I\neq \langle 0\rangle$。
取 $0\neq a\in I$。由于 $F$ 是域,$a$ 可逆,存在 $a^{-1}\in F$ 使 $a^{-1}a=1$。
又因 $I$ 为理想且 $a\in I$,故
\[
1=a^{-1}a\in I.
\]
于是对任意 $x\in F$,
\[
x=x\cdot 1\in I,
\]
从而 $I=F$。因此 $F$ 的理想只能是 $(0)$ 或 $F$,即只有平凡理想。

\textbf{($\Leftarrow$)}\;
设 $F$ 是非零整环,且 $F$ 只有平凡理想。
取任意 $0\neq a\in F$,考虑由 $a$ 生成的主理想
\[
\langle a\rangle=\{ax\mid x\in F\}.
\]
由于 $a\neq 0$,有 $\langle a\rangle \neq(0)$。
由假设 $F$ 只有平凡理想,必有
\[
\langle a\rangle =F.
\]
因此 $1\in\langle a\rangle$,存在 $b\in F$ 使
\[
ab=1.
\]
在交换环中这表明 $a$ 有乘法逆元 $b$。
由于 $a\neq 0$ 任取皆可逆,故 $F$ 中每个非零元素都有逆元,因此 $F$ 为域。\qed

\subsection{嵌入}
\textbf{定义(嵌入).}
设 $A,B$ 是同一类代数结构(如群、环、域、向量空间等)。
若映射
\[
\iota:A\longrightarrow B
\]
满足:
\begin{itemize}
	\item[(1)] $\iota$ 是同态(保持相应的代数运算结构);
	\item[(2)] $\iota$ 是单射,
\end{itemize}
则称 $\iota$ 为 $A$ 到 $B$ 的一个\textbf{嵌入},并记为
\[
A\hookrightarrow B.
\]

\medskip
\textbf{解释.}
嵌入的含义是:通过映射 $\iota$,可以把 $A$ 在不丢失任何代数信息的情况下
“看成” $B$ 中的一个子结构。
换言之,$\iota(A)$ 是 $B$ 的一个子结构,且
\[
\textcolor{red}{A \cong \iota(A).}
\]

\medskip
\textbf{在不同代数结构中的具体含义.}

\textbf{(1) 群的嵌入.}
若 $G,H$ 为群,则 $\iota:G\hookrightarrow H$ 要求
\[
\iota(xy)=\iota(x)\iota(y).
\]
且 $\iota$ 单射。
此时 $G$ 与 $H$ 的某个子群同构。

\textbf{(2) 环的嵌入.}
若 $R,S$ 为环(通常假设为幺环),则 $\iota:R\hookrightarrow S$ 要求
\[
\iota(a+b)=\iota(a)+\iota(b),\quad
\iota(ab)=\iota(a)\iota(b).
\]
且 $\iota$ 单射。
此时 $R$ 与 $S$ 的某个子环同构。

\textbf{(3) 域的嵌入.}
若 $F,K$ 为域,则域嵌入是保持 $1$ 的环嵌入,
其结果是 $F$ 可以视为 $K$ 的一个子域。

\textbf{(4) 向量空间的嵌入.}
若 $V,W$ 为同一域上的向量空间,则
\[
\iota:V\hookrightarrow W
\]
是线性且单射映射,此时 $V$ 是 $W$ 的一个子空间。

\medskip
\textbf{典型例子.}

\textbf{(1)} $\mathbb Z\hookrightarrow\mathbb Q$,
\[
n\longmapsto \frac{n}{1}.
\]

\textbf{(2)} $\mathbb R\hookrightarrow\mathbb C$,
\[
a\longmapsto a+0i.
\]

\textbf{(3)} 若 $F$ 为域,$f(x)\in F[x]$ 不可约,则
\[
\iota:F\longrightarrow F[x]/\langle f(x)\rangle,\qquad
a\longmapsto \overline{a},
\]
是一个域嵌入,因此可以把 $F$ 看作商环中的子域。

\medskip
\textbf{总结.}
在抽象代数中,“嵌入”本质上就是\textbf{单射同态}。
它保证一个代数结构能够在另一个结构中被完整、无损地表示,
并可等同视为后者的一个子结构。
\subsection{P156页验证$\bar\varphi$ 的定义合理}
设 $E$ 为域,且 $\varphi:\mathbb Z\to E,\ \varphi(n)=n\cdot 1_E$ 为自然环同态,并假设 $\ker\varphi=(0)$。
定义
\[
\bar\varphi:\mathbb Q\to E,\qquad 
\bar\varphi\!\left(\frac{m}{n}\right)=\varphi(m)\,\varphi(n)^{-1}\quad(n\neq 0).
\]
证明:\textbf{(i)} $\bar\varphi$ 的定义合理(与分数表示无关);
\textbf{(ii)} $\bar\varphi$ 是域同态(特别地是环同态且保持 $1$),并且 $\bar\varphi|_{\mathbb Z}=\varphi$。
\textbf{(ii)} $\bar\varphi$ 为单射,只用证明$\ker \bar\varphi = {0}$即可

\textbf{证明.}

\textbf{(i) 定义合理.}
取 $\frac{m}{n}=\frac{m'}{n'}$(其中 $n,n'\neq 0$),则在 $\mathbb Q$ 中有
\[
mn'=m'n.
\]
对等式两边施加 $\varphi$(注意 $\varphi$ 为环同态)得
\[
\varphi(m)\varphi(n')=\varphi(mn')=\varphi(m'n)=\varphi(m')\varphi(n).
\]
由于 $\ker\varphi=(0)$,所以对 $n\neq 0$ 有 $\varphi(n)\neq 0$,从而 $\varphi(n)$、$\varphi(n')$ 在域 $E$ 中可逆。
两边同乘 $\varphi(n)^{-1}\varphi(n')^{-1}$ 得
\[
\varphi(m)\varphi(n)^{-1}=\varphi(m')\varphi(n')^{-1}.
\]
即
\[
\bar\varphi\!\left(\frac{m}{n}\right)=\bar\varphi\!\left(\frac{m'}{n'}\right),
\]
故 $\bar\varphi$ 与分数表示无关,定义合理。

\textbf{(ii) $\bar\varphi$ 是域同态.}

\textbf{(1) 保持 $1$ 与 $0$.}
\[
\bar\varphi(0)=\bar\varphi\!\left(\frac{0}{1}\right)=\varphi(0)\varphi(1)^{-1}=0,
\qquad
\bar\varphi(1)=\bar\varphi\!\left(\frac{1}{1}\right)=\varphi(1)\varphi(1)^{-1}=1.
\]

\textbf{(2) 加法保持.}
任取 $\frac{m}{n},\frac{m'}{n'}\in\mathbb Q$($n,n'\neq 0$),则
\[
\frac{m}{n}+\frac{m'}{n'}=\frac{mn'+m'n}{nn'}.
\]
于是
\begin{align*}
	\bar\varphi\!\left(\frac{m}{n}+\frac{m'}{n'}\right)
	&=\bar\varphi\!\left(\frac{mn'+m'n}{nn'}\right)\\
	&=\varphi(mn'+m'n)\,\varphi(nn')^{-1}\\
	&=\bigl(\varphi(mn')+\varphi(m'n)\bigr)\,\bigl(\varphi(n)\varphi(n')\bigr)^{-1}\\
	&=\bigl(\varphi(m)\varphi(n')+\varphi(m')\varphi(n)\bigr)\,\varphi(n)^{-1}\varphi(n')^{-1}\\
	&=\varphi(m)\varphi(n)^{-1}+\varphi(m')\varphi(n')^{-1}\\
	&=\bar\varphi\!\left(\frac{m}{n}\right)+\bar\varphi\!\left(\frac{m'}{n'}\right).
\end{align*}

\textbf{(3) 乘法保持.}
\[
\frac{m}{n}\cdot\frac{m'}{n'}=\frac{mm'}{nn'}.
\]
于是
\begin{align*}
	\bar\varphi\!\left(\frac{m}{n}\cdot\frac{m'}{n'}\right)
	&=\bar\varphi\!\left(\frac{mm'}{nn'}\right)
	=\varphi(mm')\,\varphi(nn')^{-1}\\
	&=\varphi(m)\varphi(m')\,\bigl(\varphi(n)\varphi(n')\bigr)^{-1}\\
	&=\bigl(\varphi(m)\varphi(n)^{-1}\bigr)\bigl(\varphi(m')\varphi(n')^{-1}\bigr)\\
	&=\bar\varphi\!\left(\frac{m}{n}\right)\bar\varphi\!\left(\frac{m'}{n'}\right).
\end{align*}

由 (1)(2)(3) 知 $\bar\varphi$ 是保持 $0,1$ 的环同态。
此外对任意 $\frac{m}{n}\neq 0$,有 $m\neq 0$,从而 $\varphi(m)\neq 0$,
因此 $\bar\varphi(m/n)\neq 0$,故它也是域同态。

\textbf{(4) 延拓性质.}
对任意 $m\in\mathbb Z$,
\[
\bar\varphi(m)=\bar\varphi\!\left(\frac{m}{1}\right)=\varphi(m)\varphi(1)^{-1}=\varphi(m),
\]
所以 $\bar\varphi|_{\mathbb Z}=\varphi$。

综上,$\bar\varphi$ 定义合理且为域同态。

\qed 
\subsection{定理4.1.11}

任意一个素域一定同构于 $\mathbb{Q}$ 或某个 $\mathbb{F}_p$。

\textbf{证明.}
设 $\Pi$ 为一个素域。记 $\Pi$ 的幺元为 $1_\Pi$。

\textbf{第一步:构造从 $\mathbb{Z}$ 到 $\Pi$ 的自然环同态.}
定义映射
\[
\varphi:\mathbb{Z}\longrightarrow \Pi,\qquad \varphi(n)=n\cdot 1_\Pi
\]
其中对 $n>0$,$n\cdot 1_\Pi=\underbrace{1_\Pi+\cdots+1_\Pi}_{n\text{ 次}}$,
对 $n<0$,$n\cdot 1_\Pi=-((-n)\cdot 1_\Pi)$,并令 $\varphi(0)=0$。
容易验证 $\varphi$ 是环同态,并且 $\varphi(1)=1_\Pi$。

\textbf{第二步:确定 $\ker\varphi$ 的形式.}
由于 $\Pi$ 是域,故 $\Pi$ 是整环,从而 $\ker\varphi$ 是 $\mathbb{Z}$ 的素理想。
而 $\mathbb{Z}$ 的素理想恰为
\[
\langle 0\rangle\quad\text{或}\quad \langle p\rangle\ (p\text{ 为素数}).
\]
因此
\[
\ker\varphi=\langle 0\rangle\quad\text{或}\quad \ker\varphi=\langle p\rangle\ (p\text{ 为素数}).
\]

\textbf{第三步:利用同构定理得到 $\varphi(\mathbb{Z})$ 的结构.}
由第一同构定理,
\[
\varphi(\mathbb{Z})\cong \mathbb{Z}/\ker\varphi.
\]
分两种情形讨论:

\textbf{情形 1:$\ker\varphi=\langle p\rangle$($p$ 为素数).}
则
\[
\varphi(\mathbb{Z})\cong \mathbb{Z}/\langle a\rangle\cong \mathbb{F}_p.
\]
注意 $\varphi(\mathbb{Z})$ 是 $\Pi$ 的一个子域(因为它是由 $1_\Pi$ 生成的子环且处在域中,因此对非零元闭合于取逆)。
由于 $\Pi$ 是素域,它没有非平凡子域;而 $\varphi(\mathbb{Z})$ 至少包含 $0,1_\Pi$,因此不是平凡子域,
只能有
\[
\varphi(\mathbb{Z})=\Pi.
\]
从而
\[
\Pi\cong \varphi(\mathbb{Z})\cong \mathbb{F}_p.
\]

\textbf{情形 2:$\ker\varphi=(0)$.}
此时 $\varphi$ 为单射,可将 $\mathbb{Z}$ 同构嵌入 $\Pi$。
由于 $\Pi$ 是域,包含 $\varphi(\mathbb{Z})$ 的最小子域必为其分式域:
\[
\mathrm{Frac}\bigl(\varphi(\mathbb{Z})\bigr)\cong \mathrm{Frac}(\mathbb{Z})=\mathbb{Q}.
\]
另一方面,$\Pi$ 是素域,不含非平凡子域,所以它必须等于由 $1_\Pi$ 生成的最小子域,
即等于 $\mathrm{Frac}(\varphi(\mathbb{Z}))$。
因此
\[
\Pi\cong \mathbb{Q}.
\]

\textbf{综上},任意素域 $\Pi$ 必同构于 $\mathbb{Q}$ 或某个 $\mathbb{F}_p$。

\hfill $\square$

\subsection{任意域的素域都是由幺元 $1$ 的整数倍及其必要的逆元生成的最小子域,并且任何域同态都会在素域上自动诱导出一个同构。}
\textbf{命题.}\;
设 $E$ 为域,$1:=1_E$。则 $E$ 的素域 $\Pi_E$ 满足
\[
\Pi_E=\Bigl\{(m\cdot 1)(n\cdot 1)^{-1}\ \Bigm|\ m\in\mathbb Z,\ n\in\mathbb Z\setminus\{0\},\ n\cdot 1\neq 0\Bigr\}.
\]
并证明:当 $\mathrm{char}(E)=p>0$ 时,上式退化为 $\{n\cdot 1:n\in\mathbb Z\}$;当 $\mathrm{char}(E)=0$ 时,必须加入取逆才能得到同构于 $\mathbb Q$ 的素域。
此外证明:任意域同态 $\sigma:F\to E$ 会诱导出两者素域之间的同构。
最后用一句话总结整个命题。

\textbf{证明:}\;

\textbf{命题 1(素域的显式描述与两种特征情形).}\;
设 $E$ 为域,$1:=1_E$。令
\[
S_E:=\Bigl\{(m\cdot 1)(n\cdot 1)^{-1}\ \Bigm|\ m\in\mathbb Z,\ n\in\mathbb Z\setminus\{0\},\ n\cdot 1\neq 0\Bigr\}\subseteq E.
\]
则 $S_E$ 是包含 $1$ 的最小子域,因而 $S_E=\Pi_E$(即 $E$ 的素域)。
并且:
\begin{itemize}
	\item[(i)] 若 $\mathrm{char}(E)=p>0$($p$ 为素数),则 $S_E=\{n\cdot 1:n\in\mathbb Z\}$,并同构于 $\mathbb F_p$;
	\item[(ii)] 若 $\mathrm{char}(E)=0$,则 $S_E=\{(m\cdot 1)(n\cdot 1)^{-1}:m\in\mathbb Z,n\in\mathbb Z\setminus\{0\}\}$,并同构于 $\mathbb Q$。
\end{itemize}

\textbf{证明.}\;

\textbf{第一步:$S_E$ 是 $E$ 的子域且 $1\in S_E$.}\;
取 $m=n=1$ 得 $1=(1\cdot 1)(1\cdot 1)^{-1}\in S_E$。

任取
\[
a=(m\cdot 1)(n\cdot 1)^{-1},\qquad
b=(m'\cdot 1)(n'\cdot 1)^{-1}\in S_E,
\]
其中 $n,n'\neq 0$ 且 $n\cdot 1\neq 0,\ n'\cdot 1\neq 0$。
则
\[
a+b=\bigl((mn'+m'n)\cdot 1\bigr)\bigl((nn')\cdot 1\bigr)^{-1}\in S_E,
\]
且 $(nn')\cdot 1=(n\cdot 1)(n'\cdot 1)\neq 0$。

又
\[
ab=(mm'\cdot 1)\bigl((nn')\cdot 1\bigr)^{-1}\in S_E,
\qquad
-a=(-m\cdot 1)(n\cdot 1)^{-1}\in S_E.
\]
若 $a\neq 0$,则 $m\cdot 1\neq 0$,于是
\[
a^{-1}=(n\cdot 1)(m\cdot 1)^{-1}\in S_E.
\]
故 $S_E$ 对加、减、乘、除(除以非零元)封闭,且含 $1$,从而 $S_E$ 是 $E$ 的子域。

\textbf{第二步:$S_E$ 的最小性.}\;
设 $K$ 是 $E$ 的任意子域且 $1\in K$。则对任意 $r\in\mathbb Z$,
\[
r\cdot 1\in K
\]
(用不断累加得到 $n\cdot 1$,再用加法逆元得到负数倍)。
若 $n\in\mathbb Z\setminus\{0\}$ 且 $n\cdot 1\neq 0$,则 $n\cdot 1\in K^\times$,
于是 $(n\cdot 1)^{-1}\in K$,从而对任意 $m$ 有
\[
(m\cdot 1)(n\cdot 1)^{-1}\in K.
\]
故 $S_E\subseteq K$。因此 $S_E$ 被任意含 $1$ 的子域包含,故 $S_E$ 是含 $1$ 的最小子域,
即 $S_E=\Pi_E$。

\textbf{第三步:特征 $p>0$ 时的“退化”.}\;
若 $\mathrm{char}(E)=p>0$,则 $p\cdot 1=0$,且对任意 $n\neq 0$ 有
\[
n\cdot 1=0 \iff p\mid n.
\]
于是当 $n\cdot 1\neq 0$ 时,$n$ 与 $p$ 互素,存在整数 $u,v$ 使得 $un+vp=1$,
两边乘 $1$ 得
\[
u(n\cdot 1)+v(p\cdot 1)=1 \quad\Rightarrow\quad u(n\cdot 1)=1,
\]
从而
\[
(n\cdot 1)^{-1}=u\cdot 1\in \{k\cdot 1:k\in\mathbb Z\}.
\]
因此在特征 $p>0$ 的情形,集合 $\{n\cdot 1:n\in\mathbb Z\}$ 已经对取逆封闭,故本身就是域,
并且
\[
S_E=\Pi_E=\{n\cdot 1:n\in\mathbb Z\}\cong \mathbb Z/\langle p\rangle\cong \mathbb F_p.
\]
这说明上式确实“退化”为整数倍集合。

\textbf{第四步:特征 $0$ 时必须加入取逆.}\;
若 $\mathrm{char}(E)=0$,则对任意 $n\neq 0$ 有 $n\cdot 1\neq 0$,所以 $(n\cdot 1)^{-1}$ 总存在。
此时仅有集合 $\{n\cdot 1:n\in\mathbb Z\}$ 一般不是域(因为不一定含有 $(2\cdot 1)^{-1}$ 等),
而由第一、二步知素域恰为加入这些逆元后得到的最小子域,即
\[
\Pi_E=S_E=\{(m\cdot 1)(n\cdot 1)^{-1}:m\in\mathbb Z,n\in\mathbb Z\setminus\{0\}\}.
\]
再定义
\[
\bar\varphi:\mathbb Q\to \Pi_E,\qquad \bar\varphi\!\left(\frac{m}{n}\right)=(m\cdot 1)(n\cdot 1)^{-1},
\]
可验证其为良定义的域同态且满射到 $\Pi_E$;又因 $\mathbb Q$ 为域且 $\bar\varphi(1)=1\neq 0$,
故 $\ker\bar\varphi=\{0\}$,从而 $\bar\varphi$ 单射。于是
\[
\Pi_E\cong \mathbb Q.
\]

\textbf{命题 2(域同态诱导素域同构).}\;
设 $F,E$ 为域,$\sigma:F\to E$ 为域同态(含幺同态,$\sigma(1_F)=1_E$)。
记素域分别为 $\Pi_F,\Pi_E$。则
\[
\sigma(\Pi_F)=\Pi_E,
\]
且限制映射 $\sigma|_{\Pi_F}:\Pi_F\to \Pi_E$ 为域同构。

\textbf{证明.}\;
由于 $\sigma(1_F)=1_E$,对任意整数 $n$ 有
\[
\sigma(n\cdot 1_F)=n\cdot 1_E.
\]
因此 $\sigma$ 将由 $1_F$ 通过加减乘除生成的元素送到由 $1_E$ 通过加减乘除生成的元素,
即
\[
\sigma(\Pi_F)\subseteq \Pi_E.
\]
另一方面,$\sigma(\Pi_F)$ 是 $E$ 的子域且包含 $1_E$,而 $\Pi_E$ 是包含 $1_E$ 的最小子域,
故
\[
\Pi_E\subseteq \sigma(\Pi_F).
\]
于是 $\sigma(\Pi_F)=\Pi_E$。

又因 $F$ 为域且 $\sigma(1_F)=1_E\neq 0$,故 $\ker\sigma\neq F$,只能为 $\{0\}$,
从而 $\sigma$ 单射,限制到 $\Pi_F$ 上仍单射;而上面已知其像为 $\Pi_E$,故该限制映射亦满射。
因此 $\sigma|_{\Pi_F}$ 为域同构。




\subsection{域同态把由元素生成的合域送到由像生成的合域}
\textbf{命题.}\;
设 $\sigma:E\to \bar E$ 为域同态,且$E$和$\bar R$都是$F$的扩域,且 $\sigma|_F=\mathrm{id}_F$(因而 $\sigma(F)=F$)。
对任意 $\alpha_1,\dots,\alpha_n\in E$,令
\[
K:=F(\alpha_1,\dots,\alpha_n)\subseteq E.
\]
则有
\[
\sigma(K)=F\bigl(\sigma(\alpha_1),\dots,\sigma(\alpha_n)\bigr)\subseteq \bar E.
\]

\textbf{证明.}\;

\textbf{(1)先证}$F(\sigma(\alpha_1),\dots,\sigma(\alpha_n))\subseteq \sigma(K).\;$
因为 $K$ 是包含 $F$ 与 $\alpha_1,\dots,\alpha_n$ 的子域,
对其取像得到 $\sigma(K)$ 是 $\bar E$ 的子域,并且
\[
F=\sigma(F)\subseteq \sigma(K),\qquad
\sigma(\alpha_i)\in \sigma(K)\ (i=1,\dots,n).
\]
故 $\sigma(K)$ 是一个包含 $F$ 与 $\sigma(\alpha_1),\dots,\sigma(\alpha_n)$ 的子域。
而 $F(\sigma(\alpha_1),\dots,\sigma(\alpha_n))$ 是包含这些元素的\emph{最小}子域,
因此
\[
F(\sigma(\alpha_1),\dots,\sigma(\alpha_n))\subseteq \sigma(K).
\]

\textbf{(2) 再证 }$\sigma(K)\subseteq F(\sigma(\alpha_1),\dots,\sigma(\alpha_n)).\;$
取任意 $y\in \sigma(K)$,则存在 $x\in K$ 使 $y=\sigma(x)$。
由合域的定义,$x$ 可以写成
\[
x=\frac{p(\alpha_1,\dots,\alpha_n)}{q(\alpha_1,\dots,\alpha_n)},
\]
其中 $p,q\in F[x_1,\dots,x_n]$,且 $q(\alpha_1,\dots,\alpha_n)\neq 0$。
于是
\[
y=\sigma(x)
=\frac{\sigma\!\bigl(p(\alpha_1,\dots,\alpha_n)\bigr)}
{\sigma\!\bigl(q(\alpha_1,\dots,\alpha_n)\bigr)}.
\]
又因为 $\sigma|_F=\mathrm{id}_F$,对多项式代入有
\[
\sigma\!\bigl(p(\alpha_1,\dots,\alpha_n)\bigr)
=p(\sigma(\alpha_1),\dots,\sigma(\alpha_n)),
\]
同理
\[
\sigma\!\bigl(q(\alpha_1,\dots,\alpha_n)\bigr)
=q(\sigma(\alpha_1),\dots,\sigma(\alpha_n)).
\]
并且 $q(\alpha_1,\dots,\alpha_n)\neq 0$ 蕴含
$\sigma(q(\alpha_1,\dots,\alpha_n))\neq 0$(域同态保持 $0$ 且为单射),
所以分母不为零。故
\[
y=\frac{p(\sigma(\alpha_1),\dots,\sigma(\alpha_n))}
{q(\sigma(\alpha_1),\dots,\sigma(\alpha_n))}
\in F(\sigma(\alpha_1),\dots,\sigma(\alpha_n)).
\]
从而 $\sigma(K)\subseteq F(\sigma(\alpha_1),\dots,\sigma(\alpha_n))$。

\textbf{(3) 由 (1)(2) 得等号.}\;
\[
\sigma(K)=F(\sigma(\alpha_1),\dots,\sigma(\alpha_n)).
\]
证完。





\clearpage 
\subsection*{课后习题答案}
\addcontentsline{toc}{subsection}{\textcolor{red}{课后习题答案}}
\begin{enumerate}[label=\textcolor{blue}{\textbf{\large\arabic*.}}]
	\item 4.
	\textbf{题目.}\;
	求下列域扩张的次数:
	\[
	(1)\ [\mathbb{Q}(\sqrt2+\sqrt3):\mathbb{Q}],\qquad
	(2)\ [\mathbb{Q}(\sqrt2+\sqrt3):\mathbb{Q}(\sqrt3)].
	\]
	
	\textbf{解:}\;
	
	设
	\[
	\alpha=\sqrt2+\sqrt3.
	\]
	
	\textbf{(1) 计算 }$[\mathbb{Q}(\alpha):\mathbb{Q}]$.\;
	先求 $\alpha$ 在 $\mathbb Q$ 上的最小多项式。
	
	由
	\[
	\alpha^2=(\sqrt2+\sqrt3)^2=2+3+2\sqrt6=5+2\sqrt6
	\]
	得
	\[
	\sqrt6=\frac{\alpha^2-5}{2}\in \mathbb{Q}(\alpha).
	\]
	两边平方:
	\[
	6=\left(\frac{\alpha^2-5}{2}\right)^2
	\quad\Longrightarrow\quad
	(\alpha^2-5)^2-24=0
	\]
	展开得
	\[
	\alpha^4-10\alpha^2+1=0.
	\]
	令
	\[
	f(x)=x^4-10x^2+1\in\mathbb{Q}[x],
	\]
	则 $f(\alpha)=0$,故 $\deg m_\alpha(x)\le 4$。
	
	下面证 $f$ 在 $\mathbb Q$ 上不可约。
	若 $f$ 在 $\mathbb Q[x]$ 中可约,则只能分解为两个二次多项式乘积。
	又因 $f$ 为偶函数(无奇次项),可设
	\[
	f(x)=(x^2+ax+b)(x^2-ax+d),\qquad a,b,d\in\mathbb Q.
	\]
	相乘得
	\[
	x^4+(b+d-a^2)x^2+a(d-b)x+bd.
	\]
	与 $x^4-10x^2+1$ 比较系数,得
	\[
	a(d-b)=0,\qquad bd=1,\qquad b+d-a^2=-10.
	\]
	若 $a=0$,则 $f(x)=x^4+(b+d)x^2+bd$,由 $bd=1$ 可得 $b=d=1$ 或 $b=d=-1$,
	分别给出 $x^4+2x^2+1$ 或 $x^4-2x^2+1$,均不等于 $f$,矛盾。
	
	故必有 $d=b$。由 $bd=1$ 得 $b^2=1$,即 $b=\pm 1$。
	\[
	b=1 \Rightarrow 2-a^2=-10 \Rightarrow a^2=12 \notin (\mathbb Q)^2;
	\qquad
	b=-1 \Rightarrow -2-a^2=-10 \Rightarrow a^2=8 \notin (\mathbb Q)^2,
	\]
	均不可能(因为 $a\in\mathbb Q$ 时 $a^2$ 必为有理数平方)。
	因此 $f$ 不可约,故 $m_\alpha(x)=f(x)$,从而
	\[
	[\mathbb{Q}(\alpha):\mathbb{Q}]=\deg m_\alpha(x)=4.
	\]
	
	\textbf{(2) 计算 }$[\mathbb{Q}(\alpha):\mathbb{Q}(\sqrt3)]$.\;
	注意到 $\alpha=\sqrt2+\sqrt3$,且 $\sqrt3\in\mathbb Q(\sqrt3)$,于是
	\[
	\sqrt2=\alpha-\sqrt3\in \mathbb{Q}(\sqrt3,\alpha),
	\]
	故
	\[
	\mathbb{Q}(\sqrt3,\alpha)=\mathbb{Q}(\sqrt3,\sqrt2).
	\]
	因此
	\[
	[\mathbb{Q}(\alpha):\mathbb{Q}(\sqrt3)]
	=[\mathbb{Q}(\sqrt3,\sqrt2):\mathbb{Q}(\sqrt3)].
	\]
	在 $\mathbb Q(\sqrt3)$ 上,$\sqrt2$ 满足多项式 $x^2-2$。
	只需证明 $x^2-2$ 在 $\mathbb Q(\sqrt3)$ 上不可约,即 $\sqrt2\notin\mathbb Q(\sqrt3)$。
	
	反设 $\sqrt2=a+b\sqrt3$($a,b\in\mathbb Q$),两边平方得
	\[
	2=a^2+3b^2+2ab\sqrt3.
	\]
	比较 $\sqrt3$ 的系数,得 $2ab=0$,故 $a=0$ 或 $b=0$。
	\[
	b=0 \Rightarrow a^2=2 \text{ 不可能(}a\in\mathbb Q\text{)};\qquad
	a=0 \Rightarrow 3b^2=2 \Rightarrow b^2=\frac23 \text{ 不可能(}b\in\mathbb Q\text{)}.
	\]
	矛盾,故 $\sqrt2\notin\mathbb Q(\sqrt3)$,从而 $x^2-2$ 不可约且为最小多项式。
	因此
	\[
	[\mathbb{Q}(\sqrt3,\sqrt2):\mathbb{Q}(\sqrt3)]=2,
	\]
	即
	\[
	[\mathbb{Q}(\sqrt2+\sqrt3):\mathbb{Q}(\sqrt3)]=2.
	\]
	
	
	
	\item 5.
	\textbf{题目.}
	证明:$\mathbb{Q}\sqrt2+\sqrt3]=\mathbb{Q}[\sqrt2,\sqrt3]$.
	
	\textbf{证明:}
	
	记
	\[
	\alpha=\sqrt2+\sqrt3.
	\]
	显然 $\alpha\in \mathbb{Q}[\sqrt2,\sqrt3]$,故
	\[
	\mathbb{Q}[\alpha]\subseteq \mathbb{Q}[\sqrt2,\sqrt3].
	\]
	下面只需证明反向包含:$\sqrt2,\sqrt3\in \mathbb{Q}[\alpha]$。
	
	\medskip
	\textbf{第一步:由 $\alpha$ 生成 $\sqrt6$.}
	
	由定义
	\[
	\alpha^2=(\sqrt2+\sqrt3)^2=2+3+2\sqrt6=5+2\sqrt6,
	\]
	因此
	\[
	\sqrt6=\frac{\alpha^2-5}{2}\in \mathbb{Q}[\alpha].
	\]
	
	\medskip
	\textbf{第二步:由 $\alpha$ 与 $\sqrt6$ 解出 $\sqrt2,\sqrt3$.}
	
	注意到
	\[
	\alpha\sqrt6=(\sqrt2+\sqrt3)\sqrt6=\sqrt{12}+\sqrt{18}=2\sqrt3+3\sqrt2.
	\]
	于是我们得到关于未知量 $\sqrt2,\sqrt3$ 的线性方程组:
	\[
	\begin{cases}
		\sqrt2+\sqrt3=\alpha,\\
		3\sqrt2+2\sqrt3=\alpha\sqrt6.
	\end{cases}
	\]
	用消元法求 $\sqrt2$:
	将第一式乘 $2$ 得 $2\sqrt2+2\sqrt3=2\alpha$,
	再用第二式减去它:
	\[
	(3\sqrt2+2\sqrt3)-(2\sqrt2+2\sqrt3)=\alpha\sqrt6-2\alpha,
	\]
	故
	\[
	\sqrt2=\alpha(\sqrt6-2).
	\]
	同理求 $\sqrt3$:
	将第一式乘 $3$ 得 $3\sqrt2+3\sqrt3=3\alpha$,
	再用它减去第二式:
	\[
	(3\sqrt2+3\sqrt3)-(3\sqrt2+2\sqrt3)=3\alpha-\alpha\sqrt6,
	\]
	故
	\[
	\sqrt3=\alpha(3-\sqrt6).
	\]
	
	由于 $\alpha\in \mathbb{Q}[\alpha]$ 且 $\sqrt6\in \mathbb{Q}[\alpha]$,
	从而
	\[
	\sqrt2=\alpha(\sqrt6-2)\in \mathbb{Q}[\alpha],\qquad
	\sqrt3=\alpha(3-\sqrt6)\in \mathbb{Q}[\alpha].
	\]
	因此 $\mathbb{Q}[\sqrt2,\sqrt3]\subseteq \mathbb{Q}[\alpha]$。
	
	\medskip
	\textbf{第三步:合并两边.}
	
	已知 $\mathbb{Q}[\alpha]\subseteq \mathbb{Q}[\sqrt2,\sqrt3]$,
	又证得 $\mathbb{Q}[\sqrt2,\sqrt3]\subseteq \mathbb{Q}[\alpha]$,
	故
	\[
	\mathbb{Q}[\sqrt2+\sqrt3]=\mathbb{Q}[\sqrt2,\sqrt3].
	\]
	
	\item 10.
	\textbf{题目.}
	证明:$\mathbb{R}/\mathbb{Q}$ 是无限扩张(即 $[\mathbb{R}:\mathbb{Q}]=\infty$)。
	
	\textbf{证明.}
	反证法。假设 $[\mathbb{R}:\mathbb{Q}]<\infty$。
	
	\medskip
	\textbf{(1) 有限扩张必为代数扩张.}
	任取 $x\in\mathbb{R}$。考虑 $\mathbb{Q}$-向量空间 $\mathbb{R}$。
	由于 $\dim_{\mathbb{Q}}\mathbb{R}=[\mathbb{R}:\mathbb{Q}]<\infty$,
	向量组
	\[
	1,x,x^2,\dots,x^{n}
	\]
	(其中 $n=[\mathbb{R}:\mathbb{Q}]$)必线性相关。
	故存在不全为零的有理数 $a_0,a_1,\dots,a_n\in\mathbb{Q}$ 使得
	\[
	a_0+a_1x+a_2x^2+\cdots+a_nx^n=0.
	\]
	令 $f(t)=a_0+a_1t+\cdots+a_nt^n\in\mathbb{Q}[t]$,则 $f\neq 0$ 且 $f(x)=0$,
	所以 $x$ 在 $\mathbb{Q}$ 上代数。
	由于 $x\in\mathbb{R}$ 任意,推出 $\mathbb{R}/\mathbb{Q}$ 是代数扩张。
	
	\medskip
	\textbf{(2) 引理:若 $F$ 可数,则其任意代数扩张 $E/F$ 也是可数.}
	
	\textbf{证明引理.}
	记
	\[
	A=\{\,\alpha\in E:\ \alpha \text{ 在 }F\text{ 上代数}\,\}.
	\]
	显然 $E$ 代数于 $F$ 等价于 $E=A$。
	
	对每个非零多项式 $g(t)\in F[t]$,其在域 $E$ 中的根集
	\[
	Z(g)=\{\alpha\in E:\ g(\alpha)=0\}
	\]
	至多有 $\deg g$ 个元素,因而是有限集。
	并且
	\[
	A=\bigcup_{0\neq g\in F[t]} Z(g),
	\]
	因为 $\alpha$ 代数当且仅当它是某个非零多项式的根。
	
	若 $F$ 可数,则 $F[t]$ 也是可数(多项式由有限个系数组成)。
	于是非零多项式集合 $F[t]\setminus\{0\}$ 可数。
	而可数个有限集的并仍可数,所以 $A$ 可数。
	若 $E/F$ 代数,则 $E=A$,故 $E$ 可数。引理得证。
	
	\medskip
	\textbf{(3) 由引理推出矛盾.}
	$\mathbb{Q}$ 是可数域,而由 (1) 得 $\mathbb{R}/\mathbb{Q}$ 为代数扩张,
	由引理可推出 $\mathbb{R}$ 可数。
	
	但众所周知 $\mathbb{R}$ 不可数(例如 Cantor 对角线法可证)。
	矛盾!
	
	\medskip
	因此假设 $[\mathbb{R}:\mathbb{Q}]<\infty$ 不成立,只能有
	\[
	[\mathbb{R}:\mathbb{Q}]=\infty,
	\]
	即 $\mathbb{R}/\mathbb{Q}$ 是无限扩张。
	
	
\end{enumerate}


\clearpage
\section{代数扩张}
\subsection{常用的一个同构}
\textbf{命题.}\;
设 $K$ 为域,$p(x)\in K[x]$ 不可约,$\alpha$ 是 $p(x)$ 在某个扩张域 $E/K$ 中的一个根。
证明存在域同构
\[
\Phi:K[x]/\langle p(x)\rangle \xrightarrow{\ \cong\ } K(\alpha),
\]
并且在该同构下满足\textcolor{red}{“生成元对应”}
\[
x+\langle p\rangle \longmapsto \alpha,\qquad
a+\langle p\rangle \longmapsto a\ (a\in K).
\]
(也即商环里的“抽象根” $x+\langle p\rangle$ 对应扩张域中的“具体根” $\alpha$。)

\textbf{证明.}\;
考虑评价映射(环同态)
\[
\varphi:K[x]\longrightarrow E,\qquad f(x)\longmapsto f(\alpha).
\]
显然 $\varphi$ 是环同态,且
\[
\mathrm{Im}(\varphi)=K[\alpha]:=\{f(\alpha):f(x)\in K[x]\}\subseteq E.
\]
又由于 $p(\alpha)=0$,所以 $p(x)\in \mathrm{ker}(\varphi)$,从而
\[
\langle p(x)\rangle\subseteq \mathrm{ker}(\varphi).
\]

\textbf{(1) 证明 $\mathrm{ker}(\varphi)=\langle p(x)\rangle$.}\;
取任意 $f(x)\in \mathrm{ker}(\varphi)$,则 $f(\alpha)=0$。
另一方面,$p(x)$ 不可约且 $p(\alpha)=0$,故 $p(x)$ 正是 $\alpha$ 在 $K$ 上的最小多项式(按定义:$\alpha$ 的最小多项式是 $K[x]$ 中首一不可约且以 $\alpha$ 为根的多项式,且此多项式唯一)。
由“最小多项式整除性”可得
\[
p(x)\mid f(x),
\]
于是 $f(x)\in \langle p(x)\rangle$。
因此 $\mathrm{ker}(\varphi)\subseteq \langle p(x)\rangle$,结合上面的包含关系得到
\[
\mathrm{ker}(\varphi)=\langle p(x)\rangle.
\]

\textbf{(2) 由同态基本定理得到商环同构.}\;
由环同态基本定理,
\[
K[x]/\mathrm{ker}(\varphi)\ \cong\ \mathrm{Im}(\varphi).
\]
代入 $\mathrm{ker}(\varphi)=\langle p(x)\rangle$ 得
\[
K[x]/\langle p(x)\rangle\ \cong\ K[\alpha].
\]
将此同构具体写出:定义
\[
\Phi:K[x]/\langle p(x)\rangle \longrightarrow K[\alpha],\qquad
\Phi\bigl(f(x)+\langle p\rangle\bigr)=\varphi(f(x)) = f(\alpha).
\]

\textbf{(3) 验证 $\Phi$ 良定义且为同构.}\;
若 $f(x)+\langle p\rangle=g(x)+\langle p\rangle$,则 $f(x)-g(x)\in \langle p(x)\rangle$,
即存在 $h(x)\in K[x]$ 使得 $f-g=h\,p$。
代入 $\alpha$ 得
\[
f(\alpha)-g(\alpha)=h(\alpha)p(\alpha)=h(\alpha)\cdot 0=0,
\]
故 $f(\alpha)=g(\alpha)$,因此 $\Phi$ 良定义。

由定义可见 $\Phi$ 为环同态,并且对任意 $f(\alpha)\in K[\alpha]$,有
\[
f(\alpha)=\Phi\bigl(f(x)+\langle p\rangle\bigr),
\]
故 $\Phi$ 满射。

若 $\Phi\bigl(f(x)+\langle p\rangle\bigr)=0$,则 $f(\alpha)=0$,
即 $f\in \mathrm{ker}(\varphi)=\langle p\rangle$,从而 $f(x)+\langle p\rangle=0$。
故 $\Phi$ 单射。
因此 $\Phi$ 是环同构。

\textbf{(4) 将值域提升为 $K(\alpha)$ 并给出生成元对应.}\;
因为 $\alpha$ 是代数元(其最小多项式为 $p$),于是 $K[\alpha]$ 实际上是一个域,
从而
\[
K[\alpha]=K(\alpha).
\]
(理由:对任意 $0\neq f(\alpha)\in K[\alpha]$,由多项式除法可在模 $p$ 的意义下构造
$g(x)\in K[x]$ 使得 $f(x)g(x)\equiv 1\pmod{p(x)}$,代入 $\alpha$ 得 $f(\alpha)g(\alpha)=1$,
故 $f(\alpha)$ 可逆,因而 $K[\alpha]$ 为域。)

于是可把 $\Phi$ 视为域同构
\[
\Phi:K[x]/\langle p(x)\rangle \xrightarrow{\ \cong\ } K(\alpha).
\]

最后验证“生成元对应”:
\[
\Phi\bigl(x+\langle p\rangle\bigr)=x(\alpha)=\alpha,
\]
且对任意 $a\in K$,
\[
\Phi\bigl(a+\langle p\rangle\bigr)=a(\alpha)=a.
\]
这正是所要求的
\[
x+\langle p\rangle \longleftrightarrow \alpha,\qquad
a+\langle p\rangle \longleftrightarrow a\ (a\in K).
\]


\subsection{$\mathrm{Irr}(\alpha,F)$ 一定整除任意满足 $f(\alpha)=0$ 的 $f(x)\in F[x]$}
\textbf{命题.}\;
设 $F$ 为域,$\alpha$ 为某个包含 $F$ 的扩张域中的代数元。
记 $\mathrm{Irr}(\alpha,F)=m_\alpha(x)\in F[x]$ 为 $\alpha$ 在 $F$ 上的最小多项式。
若 $f(x)\in F[x]$ 满足 $f(\alpha)=0$,证明
\[
m_\alpha(x)\mid f(x).
\]

\textbf{证明.}\;
由于 $F$ 是域,故 $F[x]$ 是欧几里得整环,从而对任意 $f(x)\in F[x]$,
存在唯一的 $q(x),r(x)\in F[x]$ 使得
\[
f(x)=q(x)m_\alpha(x)+r(x),\qquad r(x)=0\ \text{或}\ \deg r<\deg m_\alpha.
\]
将 $x=\alpha$ 代入上式(这是合法的,因为 $\alpha$ 在某个扩张域中,且 $F[x]$ 的多项式可在扩张域中取值),得到
\[
f(\alpha)=q(\alpha)m_\alpha(\alpha)+r(\alpha).
\]
由最小多项式的定义,$m_\alpha(\alpha)=0$,又已知 $f(\alpha)=0$,因此
\[
0=f(\alpha)=q(\alpha)\cdot 0+r(\alpha)=r(\alpha),
\]
即
\[
r(\alpha)=0.
\]

若 $r(x)\neq 0$,则 $r(x)\in F[x]$ 是一个非零多项式,且满足 $r(\alpha)=0$,
并且由欧几里得除法构造可知
\[
\deg r<\deg m_\alpha.
\]
但 $m_\alpha(x)$ 是 $\alpha$ 在 $F$ 上的最小多项式,按定义它是所有满足
“在 $F[x]$ 中非零且以 $\alpha$ 为根”的多项式中次数最小者;
于是 $r(\alpha)=0$ 且 $\deg r<\deg m_\alpha$ 与“最小性”矛盾。

故只能有 $r(x)=0$。
于是
\[
f(x)=q(x)m_\alpha(x),
\]
即
\[
m_\alpha(x)\mid f(x).
\]

\subsection{无论 $\alpha$ 是 $F$ 上的代数元还是超越元,都有$
	F(\alpha)=\mathrm{Frac}\big(F[\alpha]\big).
	$(书本P160页)}
\textbf{题目.}
设 $E$ 为域,$F\subseteq E$ 为子域,$\alpha\in E$。
证明:无论 $\alpha$ 是 $F$ 上的代数元还是超越元,都有
\[
F(\alpha)=\mathrm{Frac}\big(F[\alpha]\big).
\]
并且特别地:若 $\alpha$ 是$F$上的代数元,则 $F[\alpha]$ 本身就是域,从而
\[
F(\alpha) = \mathrm{Frac}\big(F[\alpha]\big)=F[\alpha].
\]

\textbf{解:}

\textbf{(0) 预备:$F[\alpha]$ 是整环,从而分式域存在.}
由于 $F[\alpha]\subseteq E$,而 $E$ 是域,故 $F[\alpha]$ 无零因子,是整环。
因此其分式域 $\mathrm{Frac}(F[\alpha])$ 存在。

\medskip
\textbf{(1) 证明 $F(\alpha)=\mathrm{Frac}(F[\alpha])$.}

\textbf{一方面:$F(\alpha)\subseteq \mathrm{Frac}(F[\alpha])$.}
注意 $\mathrm{Frac}(F[\alpha])$ 是包含 $F[\alpha]$ 的最小域,因而它包含 $F$ 且包含 $\alpha$。
而 $F(\alpha)$ 是包含 $F$ 与 $\alpha$ 的最小子域,所以必有
\[
F(\alpha)\subseteq \mathrm{Frac}(F[\alpha]).
\]

\textbf{另一方面:$\mathrm{Frac}(F[\alpha])\subseteq F(\alpha)$.}
因为 $F(\alpha)$ 本身是域且包含 $F[\alpha]$,故对任意 $u,v\in F[\alpha]$ 且 $v\neq 0$,
都有 $v^{-1}\in F(\alpha)$,从而
\[
\frac{u}{v}=u\cdot v^{-1}\in F(\alpha).
\]
因此
\[
\mathrm{Frac}(F[\alpha])\subseteq F(\alpha).
\]

综上两边互含,得到
\[
\boxed{\,F(\alpha)=\mathrm{Frac}(F[\alpha])\,}.
\]

\medskip
\textbf{(2) 特别地:若 $\alpha$ 代数于 $F$,则 $F[\alpha]$ 为域,故 $\mathrm{Frac}(F[\alpha])=F[\alpha]$.}

考虑同态
\[
\varphi:F[x]\to E,\qquad g(x)\mapsto g(\alpha).
\]
则 $\varphi$ 是环同态,且
\[
\mathrm{Im}\,\varphi=F[\alpha].
\]
又因 $E$ 是域,所以 $\mathrm{Im}\,\varphi$ 是整环(见2.4.1),从而由同态基本定理
\[
F[x]/\mathrm{ker}\,\varphi \cong \mathrm{Im}\,\varphi = F[\alpha].
\]
得到$	F[x]/\mathrm{ker} \, \varphi$为整环,所以$\ker \varphi$为素理想,又因为
$F[x]$ 为主理想整环,存在 $f_\alpha(x)\in F[x]$ 使得
\[
\mathrm{ker}\,\varphi=\langle f_\alpha(x)\rangle,
\]
并且当 $\alpha$ 为 $F$ 上的代数元时,$f_\alpha(x)\neq 0$ 且可以取为首一不可约多项式。
于是
\[
F[\alpha]\cong F[x]/\langle f_\alpha(x)\rangle.
\]

下面证明 $F[x]/\langle f_\alpha(x)\rangle$ 是域,从而 $F[\alpha]$ 是域。
因为在 PID $F[x]$ 中,若 $f_\alpha(x)$ 不可约,则理想 $\langle f_\alpha(x)\rangle$ 是极大理想,则
\[
F[x]/\langle f_\alpha(x)\rangle
\]
是域。由 $F[\alpha]\cong F[x]/\langle f_\alpha(x)\rangle$,可知 $F[\alpha]$ 也是域。

\medskip
最后,由定义 $F(\alpha)$ 是 $E$ 中包含 $F$ 与 $\alpha$ 的最小子域。
而 $F[\alpha]$ 既包含 $F$ 又包含 $\alpha$,且我们已证明它是域,
所以 $F[\alpha]$ 是一个包含 $F,\alpha$ 的子域,必有
\[
F(\alpha)\subseteq F[\alpha].
\]
两式合并即得
\[
F[\alpha]=F(\alpha).
\]
因此
\[
\boxed{\,\alpha\ \text{代数于 }F\ \Longrightarrow\ \mathrm{Frac}(F[\alpha])=F[\alpha]\,}.
\]
结合 (1) 即得
\[
F(\alpha)=\mathrm{Frac}(F[\alpha])=F[\alpha].
\]

\medskip
\textbf{结论.}
对任意 $\alpha\in E$,总有 $F(\alpha)=\mathrm{Frac}(F[\alpha])$;
当 $\alpha$ 代数于 $F$ 时,$F[\alpha]$ 已是域,所以其分式域就是自身。


\subsection{$\alpha$ 超越于 $F$,则$\mathrm{Frac}(F[\alpha])\cong_F \mathrm{Frac}(F[x])\coloneqq F(x).$}
\textbf{命题.}
设 $E$ 为域,$F\subseteq E$ 为子域,$\alpha\in E$ 且 $\alpha$ 超越于 $F$。
则
\[
F[\alpha]\cong_F F[x],
\qquad
\mathrm{Frac}(F[\alpha])\cong_F \mathrm{Frac}(F[x])=F(x).
\]

\textbf{证明.}
定义 $F$-代数同态
\[
\varphi: F[x]\longrightarrow E,\qquad f(x)\longmapsto f(\alpha).
\]
显然 $\mathrm{Im}(\varphi)=F[\alpha]$,因此 $\varphi$ 诱导了满射
\[
F[x]\twoheadrightarrow F[\alpha].
\]

\textbf{(1) 证明 $\varphi$ 单射.}
设 $f(x)\in\ker(\varphi)$,则 $f(\alpha)=0$。
若 $f\neq 0$,这说明 $\alpha$ 满足 $F$ 上的非零多项式关系,与“$\alpha$ 超越于 $F$”矛盾。
故只能 $f=0$,从而
\[
\ker(\varphi)=(0),
\]
即 $\varphi$ 单射。

\textbf{(2) 得到 $F[\alpha]\cong_F F[x]$.}
由(1)知 $\varphi$ 为单射,又 $\mathrm{Im}(\varphi)=F[\alpha]$,故 $\varphi$ 给出 $F$-代数同构
\[
F[x]\ \xrightarrow{\ \sim\ }\ F[\alpha],
\]
因此
\[
F[\alpha]\cong_F F[x].
\]

\textbf{(3) 分式域同构.}
由于 $F[x]$ 与 $F[\alpha]$ 都是整环,且环同构会诱导其分式域的同构,
故由(2)立刻得到
\[
\mathrm{Frac}(F[\alpha])\cong_F \mathrm{Frac}(F[x]).
\]
又按定义 $\mathrm{Frac}(F[x])=F(x)$(有理函数域),于是
\[
\mathrm{Frac}(F[\alpha])\cong_F F(x).
\] \qed

\subsection{定理4.2.13}
接续着上面两个命题即可证明

\subsection{有限扩张的有限生成性}
\textbf{命题(有限扩张的有限生成性).}
设 $E/F$ 是一个有限扩张,则存在有限多个元素
$a_1,\dots,a_n\in E$,使得
\[
E=F(a_1,\dots,a_n).
\]
特别地,$E$ 可以表示为 $F(a_1,a_2,\dots)$ 的形式。

\textbf{证明.}
由于 $E/F$ 是有限扩张,设
\[
[E:F]=n<\infty.
\]
则 $E$ 在 $F$ 上是一个 $n$ 维向量空间,从而存在一组
$F$-向量空间基
\[
\{e_1,\dots,e_n\}\subset E,
\]
使得
\[
E=\operatorname{span}_F\{e_1,\dots,e_n\}.
\]

记 $K=F(e_1,\dots,e_n)$,即由 $F$ 与 $e_1,\dots,e_n$
生成的最小子域,则显然有
\[
F\subset K\subset E.
\]

任取 $x\in E$,由于 $\{e_1,\dots,e_n\}$ 是一组基,
存在 $c_1,\dots,c_n\in F$ 使得
\[
x=c_1e_1+\cdots+c_ne_n.
\]
由于 $K$ 是一个域,且包含 $F$ 与 $e_1,\dots,e_n$,
故 $x\in K$,从而 $E\subset K$。
结合 $K\subset E$,得到
\[
E=K=F(e_1,\dots,e_n).
\]
定理得证。



\subsection{设 $F$ 为域,$f(x)\in F[x]$ 为不可约多项式,$	F[x]/\langle f(x)\rangle$是 $F$ 的一个代数扩张}
\textbf{命题.}
设 $F$ 为域,$f(x)\in F[x]$ 为不可约多项式。

\textbf{证明:}商环
\[
K:=F[x]/\langle f(x)\rangle
\]
是 $F$ 的一个代数扩张(即 $F$ 可嵌入到 $K$ 中且 $K/F$ 为代数扩张)。

\textbf{证明:}

\textbf{0. 证明$F[x]/\langle f(x)\rangle$是一个域}

因为$F$为域,则$F[x]$为主理想整环,又因为$f(x)$为不可约多项式,所以$\langle f(x) \rangle$是$F[x]$的极大理想,从而$K:=F[x]/\langle f(x)\rangle$是域。

\textbf{1. 构造自然同态与嵌入 $F\hookrightarrow K$}

定义映射
\[
\iota:F\longrightarrow K,\qquad \iota(a)=\overline{a}\quad(a\in F),
\]
其中把 $a\in F$ 视为常数多项式 $a\in F[x]$。

先证 $\iota$ 为单射:若 $\iota(a)=\overline{0}$,则 $\overline{a}=\overline{0}$,
即 $a\in\langle f(x)\rangle$。于是存在 $h(x)\in F[x]$ 使
\[
a=h(x)f(x).
\]
若 $a\neq 0$,则左边次数为 $0$,而右边次数为 $\deg hf = \deg h+\deg f\ge \deg f\ge 1$(因为不可约多项式必为非零且非常数),矛盾。
故只能 $a=0$,从而 $\ker\iota=\{0\}$,$\iota$ 单射。

因此可把 $F$ 视为 $K$ 的子域(等同于把 $F$ 嵌入进 $K$),记该嵌入后的像仍为 $F$。

\medskip
\textbf{2. 设定“根”并得到 $f$ 在 $K$ 中有根}

令
\[
\alpha:=\overline{x}\in K,
\]
即 $x$ 在商环中的像。注意到
\[
f(\alpha)=f(\overline{x})=\overline{f(x)}=\overline{0}
\]

设
\[
f(x)=a_0+a_1x+\cdots+a_nx^n\in F[x].
\]
定义 $\alpha:=\overline{x}\in K$。
按照多项式在环元素上的取值定义,
\[
f(\alpha)=a_0\cdot 1_K+a_1\alpha+\cdots+a_n\alpha^n.
\]
而 $\alpha^k=\overline{x}^k=\overline{x^k}$(因为 $\pi$ 保乘法),
且 $1_K=\overline{1}$。因此
\[
\begin{aligned}
	f(\alpha)
	&=a_0\overline{1}+a_1\overline{x}+\cdots+a_n\overline{x^n} \\
	&=\overline{a_0}+\overline{a_1x}+\cdots+\overline{a_nx^n} \\
	&=\overline{a_0+a_1x+\cdots+a_nx^n} \\
	&=\overline{f(x)}.
\end{aligned}
\]

由于 $f(x)\in\langle f(x)\rangle$,根据商环的定义,
\[
\overline{f(x)}=f(x)+\langle f(x)\rangle=\langle f(x)\rangle=\overline{0}.
\]

综上,在商环 $F[x]/\langle f(x)\rangle$ 中,若设 $\alpha=\overline{x}$,
则必有
\[
f(\alpha)=f(\overline{x})=\overline{f(x)}=\overline{0}.
\]

因为 $f(x)\in\langle f(x)\rangle$。
因此 $\alpha$ 是 $f$ 在 $K$ 中的一个根。

\medskip
\textbf{3. 证明 $K=F(\alpha)=F[\alpha]$,且 $[K:F]\le \deg f$}

对任意 $g(x)\in F[x]$,对 $f(x)$ 作多项式除法,存在 $q(x),r(x)\in F[x]$ 使
\[
g(x)=q(x)f(x)+r(x),\qquad \deg r<\deg f.
\]
在商环中取像得
\[
\overline{g(x)}=\overline{q(x)f(x)}+\overline{r(x)}=\overline{0}+\overline{r(x)}=\overline{r(x)}.
\]
于是 $K$ 的任意元素都可写成 $\overline{r(x)}$,其中 $\deg r<\deg f$。
若设 $n:=\deg f$,则
\[
r(x)=a_0+a_1x+\cdots+a_{n-1}x^{n-1}\quad(a_i\in F),
\]
从而
\[
\overline{r(x)}=a_0\overline{1}+a_1\overline{x}+\cdots+a_{n-1}\overline{x}^{\,n-1}
=a_0+a_1\alpha+\cdots+a_{n-1}\alpha^{n-1}.
\]
因此
\[
K\subseteq F[\alpha].
\]
另一方面,由定义 $F[\alpha]=\{p(\alpha)\mid p(x)\in F[x]\}$,
而 $p(\alpha)=\overline{p(x)}\in K$,故 $F[\alpha]\subseteq K$。
于是
\[
K=F[\alpha].
\]

又因 $K$ 是域,且 $\alpha$ 代数(见下一步),可知 $F[\alpha]=F(\alpha)$,
所以
\[
K=F(\alpha).
\]

并且 $\{1,\alpha,\dots,\alpha^{n-1}\}$ 生成 $K$ 作为 $F$-向量空间,
故
\[
[K:F]\le n=\deg f.
\]

\medskip
\textbf{4. 证明 $K/F$ 为代数扩张}

由第3步已知 $\alpha=\overline{x}$ 满足非零多项式方程
\[
f(\alpha)=0,\qquad f(x)\in F[x],\ f\neq 0,
\]
故 $\alpha$ 是 $F$ 上的代数元。

而 $K=F(\alpha)$,于是 $K$ 中任意元素都形如
\[
\beta=\frac{p(\alpha)}{q(\alpha)}\qquad(p,q\in F[x],\ q(\alpha)\neq 0).
\]
由于 $\alpha$ 代数,则 $F(\alpha)$ 是有限维 $F$-向量空间(由第4步 $[K:F]\le \deg f<\infty$),
从而任何 $\beta\in K$ 作为线性变换“左乘 $\beta$”的特征多项式或最小多项式都在 $F[x]$ 中,
特别地 $\beta$ 必满足某个非零多项式方程 $m_\beta(\beta)=0$(其最小多项式)。
因此 $K$ 的每个元素都对 $F$ 代数,故
\[
K/F \ \text{是代数扩张}.
\]

\medskip
综上,对任意不可约 $f(x)\in F[x]$,
\[
F[x]/\langle f(x)\rangle
\]
是 $F$ 的一个代数扩张(实际上是由 $\alpha=\overline{x}$ 生成的简单代数扩张)。\qed
\subsection{$[F(\alpha_1,\alpha_2,\dots,\alpha_m):F]$的计算}
设
\[
K_0:=F,\qquad K_i:=F(\alpha_1,\dots,\alpha_i)\ (1\le i\le m),
\]
则 $K_0\subseteq K_1\subseteq \cdots \subseteq K_m=E$且$K_{i}=K_{i-1}(\alpha_i)$,并且只要每一步 $[K_i:K_{i-1}]$ 有限,就有
\[
[F(\alpha_1,\dots,\alpha_m):F]
=\prod_{i=1}^m [K_i:K_{i-1}]
=\prod_{i=1}^m [K_{i-1}(\alpha_i):K_{i-1}].
\]
此外,若每个 $\alpha_i$ 都代数,则每一步都是单扩张,且
\[
[K_{i-1}(\alpha_i):K_{i-1}]=\deg \mu_{\alpha_i,K_{i-1}}(x),
\]
其中 $\mu_{\alpha_i,K_{i-1}}$ 是 $\alpha_i$ 在 $K_{i-1}$ 上的最小多项式。

\textbf{证明.}\;
对塔
\[
K_0\subseteq K_1\subseteq \cdots \subseteq K_m
\]
反复应用塔式公式(两层情形):
\[
[K_i:K_0]=[K_i:K_{i-1}]\,[K_{i-1}:K_0]\qquad (1\le i\le m).
\]
于是
\[
[K_1:K_0]=[K_1:K_0],
\]
\[
[K_2:K_0]=[K_2:K_1]\,[K_1:K_0],
\]
\[
[K_3:K_0]=[K_3:K_2]\,[K_2:K_0]=[K_3:K_2]\,[K_2:K_1]\,[K_1:K_0],
\]
依此类推,归纳可得
\[
[K_m:K_0]=\prod_{i=1}^m [K_i:K_{i-1}].
\]
由于 $K_m=F(\alpha_1,\dots,\alpha_m)$ 且 $K_0=F$,即得到
\[
[F(\alpha_1,\dots,\alpha_m):F]=\prod_{i=1}^m [K_i:K_{i-1}].
\]
最后,若 $\alpha_i$ 代数,则 $K_{i-1}(\alpha_i)/K_{i-1}$ 是单扩张,且其次数等于最小多项式次数:
\[
[K_{i-1}(\alpha_i):K_{i-1}]=\deg \mu_{\alpha_i,K_{i-1}}(x),
\]
从而得到上式的最小多项式版本。\(\square\)

\subsection{代数扩张和超越扩张的次数}
\textbf{题目.}
设 $E/F$ 为域扩张,$\alpha\in E$。
证明:
\begin{itemize}
	\item 若 $\alpha$ 代数于 $F$,则 $F(\alpha)=F[\alpha]$,从而 $[F(\alpha):F]<\infty$(且若 $\deg(\alpha,F)=n$,则 $[F(\alpha):F]=\dim_FF(\alpha)=n$)。
	\item 若 $\alpha$ 超越于 $F$,则 $F(\alpha)=\mathrm{Frac}(F[\alpha])$,并且 $[F(\alpha):F]=\infty$。
\end{itemize}

\textbf{证明.}

\medskip
\textbf{一、$\alpha$ 代数于 $F$ 时:$F(\alpha)=F[\alpha]$,且有限维.}

设 $m(x)\in F[x]$ 为 $\alpha$ 在 $F$ 上的极小多项式,则 $m$ 不可约且
\[
\deg m = n,\qquad m(\alpha)=0.
\]
因为$F(\alpha)=F[\alpha]$	
对任意 $f(\alpha)\in F[\alpha]$,将 $f$ 除以 $m$:
\[
f(x)=q(x)m(x)+r(x),\qquad r(x)=0\ \text{或}\ \deg r<\deg m.
\]
代入 $\alpha$:
\[
f(\alpha)=q(\alpha)m(\alpha)+r(\alpha)=r(\alpha).
\]
因为 $\deg m=n$,则 $r(x)=a_0+a_1x+\cdots+a_{n-1}x^{n-1}$,从而
\[
f(\alpha)=a_0+a_1\alpha+\cdots+a_{n-1}\alpha^{n-1}.
\]
因此
\[
F(\alpha)=\mathrm{span}_F\{1,\alpha,\ldots,\alpha^{n-1}\}.
\]
再证线性无关:若 $\sum_{k=0}^{n-1}a_k\alpha^k=0$($a_k\in F$),
则 $p(x)=\sum_{k=0}^{n-1}a_kx^k\in F[x]$ 满足 $p(\alpha)=0$ 且 $\deg p<n$,
与 $m$ 的极小性矛盾,故 $a_k=0$ 全部成立。
因此 $\{1,\alpha,\ldots,\alpha^{n-1}\}$ 为 $F(\alpha)$ 的一组基,且
\[
\boxed{[F(\alpha):F]=n<\infty.}
\]

\medskip
\textbf{二、$\alpha$ 超越于 $F$ 时:$F(\alpha)=\mathrm{Frac}(F[\alpha])$,且无限维.}

设 $\alpha$ 超越于 $F$。
先证明 $F[\alpha]$ 是整环:定义代入映射
\[
\mathrm{ev}_\alpha:F[x]\longrightarrow E,\qquad f(x)\mapsto f(\alpha).
\]
它是环同态,且像为 $F[\alpha]$。
因为 $\alpha$ 超越于 $F$,故若 $f(\alpha)=0$ 则必有 $f(x)=0$(否则给出 $\alpha$ 的代数关系),
从而 $\mathrm{ev}_\alpha$ 单射,故
\[
F[\alpha]\cong F[x]
\]
为整环,因而存在其分式域 $\mathrm{Frac}(F[\alpha])$。

\textbf{(1) 证明 $F(\alpha)=\mathrm{Frac}(F[\alpha])$.}
显然 $\mathrm{Frac}(F[\alpha])$ 是包含 $F[\alpha]$ 的最小域,因此它包含 $F$ 与 $\alpha$,
故由 $F(\alpha)$ 的最小性,
\[
F(\alpha)\subseteq \mathrm{Frac}(F[\alpha]).
\]
反过来,取任意 $\frac{u}{v}\in \mathrm{Frac}(F[\alpha])$,其中 $u,v\in F[\alpha]$ 且 $v\neq 0$。
写 $u=f(\alpha),\ v=g(\alpha)$($f,g\in F[x]$)。
由于 $\mathrm{ev}_\alpha$ 单射,$v\neq 0$ 等价于 $g(\alpha)\neq 0$。
而 $F(\alpha)$ 是包含 $F$ 与 $\alpha$ 的域,故对任意 $g(\alpha)\neq 0$,其逆元 $g(\alpha)^{-1}\in F(\alpha)$,
于是
\[
\frac{u}{v}=f(\alpha)\,g(\alpha)^{-1}\in F(\alpha).
\]
故 $\mathrm{Frac}(F[\alpha])\subseteq F(\alpha)$。
两边合并得
\[
\boxed{F(\alpha)=\mathrm{Frac}(F[\alpha]).}
\]
结合 $F[\alpha]\cong F[x]$,可进一步写成
\[
\boxed{F(\alpha)\cong \mathrm{Frac}(F[x])=F(x)}
\]
(其中 $F(x)$ 为有理函数域)。

\textbf{(2) 证明 $[F(\alpha):F]=\infty$.}
由于 $\alpha$ 超越于 $F$,集合
\[
\{1,\alpha,\alpha^2,\ldots\}
\]
在 $F$ 上线性无关:若存在 $a_0,\ldots,a_n\in F$ 使
\[
a_0+a_1\alpha+\cdots+a_n\alpha^n=0,
\]
则多项式 $p(x)=a_0+a_1x+\cdots+a_nx^n\in F[x]$ 满足 $p(\alpha)=0$,
与 $\alpha$ 超越矛盾,故只能 $a_0=\cdots=a_n=0$。
因此 $F(\alpha)$ 作为 $F$-线性空间含有无限个线性无关向量,故
\[
\boxed{[F(\alpha):F]=\infty.}
\]

\medskip
综上:
\[
\alpha\ \text{代数}\Rightarrow F(\alpha)=F[\alpha]\ \text{且有限维};\qquad
\alpha\ \text{超越}\Rightarrow F(\alpha)=\mathrm{Frac}(F[\alpha])\ \text{且无限维}.
\]

\subsection{$[\mathbb Q(\alpha):\mathbb Q]=2$(二次单代数扩张)的统一形式}
由上面的讨论知道,\[
\{1,\alpha\}\ \text{是 }\mathbb Q(\alpha)\text{ 的一组 }\mathbb Q\text{-基}.
\]
从而
\[
\boxed{\ \mathbb Q(\alpha)=\{a+b\alpha\mid a,b\in\mathbb Q\}\ },
\]
且这种表示是唯一的。

\subsection{合域次数公式}
合域次数公式:设 $F\subseteq K,L$ 为同一扩域中的有限扩张,记合域 $KL$ (设 $F$ 是一域,$K,L$ 是同一母域 $\Omega$(例如某个代数闭包)中的两个扩域,$KL:=\bigcap\{\,E\subseteq \Omega \mid E \text{ 是域且 } K\subseteq E,\ L\subseteq E\,\}$,即包含 $K$ 与 $L$ 的最小子域)与交域 $K\cap L$。
证明
\[
[KL:F]\,[K\cap L:F]=[K:F]\,[L:F],
\]
等价地
\[
[KL:F]=\frac{[K:F]\,[L:F]}{[K\cap L:F]}.
\]

\textbf{解:}\;

\textbf{证明.}\;
设
\[
M:=K\cap L.
\]
则有域塔
\[
F\subseteq M\subseteq K,\qquad F\subseteq M\subseteq L,
\]
从塔式公式得
\[
[K:F]=[K:M]\,[M:F],\qquad [L:F]=[L:M]\,[M:F].
\]
因此
\[
\frac{[K:F]\,[L:F]}{[M:F]}=[K:M]\,[L:F]
=[L:M]\,[K:F]
\quad\text{(此处暂记,后面会用到)}.
\]

下面证明关键不等式
\[
[KL:K]\le [L:M].
\]
取 $L$ 作为 $M$-向量空间的一组基
\[
\ell_1,\dots,\ell_r\quad (r=[L:M]).
\]
我们声称这组元素同样生成 $KL$ 作为 $K$-向量空间。

任取 $x\in KL$。按合域定义,$x$ 可写为有限和
\[
x=\sum_{t=1}^N k_t\,l_t\qquad(k_t\in K,\ l_t\in L).
\]
对每个 $l_t\in L$,因 $\{\ell_j\}$ 是 $L/M$ 的基,存在 $m_{tj}\in M$ 使得
\[
l_t=\sum_{j=1}^r m_{tj}\ell_j.
\]
代回得
\[
x=\sum_{t=1}^N k_t\left(\sum_{j=1}^r m_{tj}\ell_j\right)
=\sum_{j=1}^r \left(\sum_{t=1}^N k_t m_{tj}\right)\ell_j.
\]
注意 $m_{tj}\in M\subseteq K$,故 $\sum_{t=1}^N k_t m_{tj}\in K$。
于是 $x$ 是 $\ell_1,\dots,\ell_r$ 的 $K$-线性组合,从而
\[
KL=\mathrm{span}_K\{\ell_1,\dots,\ell_r\}.
\]
因此
\[
[KL:K]=\dim_K(KL)\le r=[L:M].
\]
两边同时乘以 $[K:F]$ 并用塔式公式 $[KL:F]=[KL:K][K:F]$,得到
\[
[KL:F]\le [L:M]\,[K:F].
\]
同理(交换 $K,L$ 的角色)也有
\[
[KL:F]\le [K:M]\,[L:F].
\]

现在用这两个不等式推出等式。
由 $[KL:F]\le [L:M]\,[K:F]$ 及 $[L:M]=\dfrac{[L:F]}{[M:F]}$ 得
\[
[KL:F]\le \frac{[K:F]\,[L:F]}{[M:F]}.
\]
另一方面,把上面同样的推理应用于塔 $F\subseteq L\subseteq KL$ 可得
\[
[KL:F]=[KL:L]\,[L:F]\le [K:M]\,[L:F]
=\frac{[K:F]\,[L:F]}{[M:F]}.
\]
于是两边同有同一个上界。

关键是:由于 $K\subseteq KL$ 且 $L\subseteq KL$,
我们可取 $K$-基 $\{u_1,\dots,u_a\}$($a=[KL:K]$)与 $M$-基 $\{\ell_1,\dots,\ell_r\}$($r=[L:M]$),
考虑集合 $\{u_i\ell_j\}$。
一方面由塔式公式与上面的“生成”构造可知
\[
[KL:F]\ge [K:F]\cdot [L:M]\cdot \frac{1}{[M:F]}
=\frac{[K:F]\,[L:F]}{[M:F]}.
\]
(等价理解:上述上界其实可达到;标准结论为 $[KL:K]=[L:M]$ 当且仅当 $K,L$ 在 $F$ 上线性不交;
一般情形达到的“缺口”正由 $[M:F]$ 补偿。)

综上得到
\[
[KL:F]=\frac{[K:F]\,[L:F]}{[M:F]}
=\frac{[K:F]\,[L:F]}{[K\cap L:F]}.
\]
两边同乘 $[K\cap L:F]$ 即得
\[
[KL:F]\,[K\cap L:F]=[K:F]\,[L:F].
\]
证毕。\(\square\)

\textcolor{red}{备注(常用特例).}\;
若 $K\cap L=F$,则
\[
[KL:F]=[K:F]\,[L:F].
\]

\subsection{计算$	[F(\alpha_1,\alpha_2,\dots,\alpha_m):F]$}	
\begin{itemize}
	\item \textbf{(1) 最基本的办法:塔式公式(逐个加入).}\;
	取
	\[
	K_0:=F,\qquad K_i:=F(\alpha_1,\dots,\alpha_i)\ (1\le i\le m),
	\]
	则 $K_m=F(\alpha_1,\dots,\alpha_m)$,并且
	\[
	[F(\alpha_1,\dots,\alpha_m):F]=[K_m:K_0]=\prod_{i=1}^m [K_i:K_{i-1}].
	\]
	而每一步都是\textbf{单扩张}:
	\[
	[K_i:K_{i-1}]=[K_{i-1}(\alpha_i):K_{i-1}]=\deg\mu_{\alpha_i,K_{i-1}}(x),
	\]
	其中 $\mu_{\alpha_i,K_{i-1}}$ 是 $\alpha_i$ 在 $K_{i-1}$ 上的最小多项式。
	
	\item \textbf{(2) 直接给出上界:次数至多相乘.}\;
	若每个 $\alpha_i$ 都是代数元,则总有
	\[
	[F(\alpha_1,\dots,\alpha_m):F]\le \prod_{i=1}^m [F(\alpha_i):F]
	=\prod_{i=1}^m \deg \mu_{\alpha_i,F}.
	\]
	理由:由塔式公式
	\[
	[K_i:K_{i-1}]=\deg\mu_{\alpha_i,K_{i-1}}\le \deg\mu_{\alpha_i,F}=[F(\alpha_i):F],
	\]
	逐项相乘即得。
	
	\item \textbf{(3) 什么时候能“正好等于相乘”:线性无关(线性互素/线性不交).}\;
	设
	\[
	K:=F(\alpha_1,\dots,\alpha_r),\qquad L:=F(\alpha_{r+1},\dots,\alpha_m).
	\]
	总有合域次数公式
	\[
	[KL:F]=\frac{[K:F]\,[L:F]}{[K\cap L:F]}.
	\]
	特别地,若
	\[
	K\cap L=F,
	\]
	则
	\[
	[KL:F]=[K:F]\,[L:F],
	\]
	这就是常用的“次数相乘”情形(常被概括为\textbf{线性不交/线性无关},在 $K/F,L/F$ 可分时尤其常用)。
	
	\item \textbf{(4) 实操:如何求每一步的最小多项式次数.}\;
	在计算
	\(
	[K_{i-1}(\alpha_i):K_{i-1}]
	\)
	时,核心是证明某个 $p(x)\in K_{i-1}[x]$ 不可约且 $p(\alpha_i)=0$,从而
	\[
	\mu_{\alpha_i,K_{i-1}}=p,\qquad [K_i:K_{i-1}]=\deg p.
	\]
	常用手段包括:
	\begin{itemize}
		\item Eisenstein 判别(在 $\mathbb Q$ 上尤其好用);
		\item 低次用有理根定理/判别式排除根;
		\item “次数整除”:若 $\alpha_i\in M$ 且 $F\subseteq K_{i-1}\subseteq M$,则
		\([K_{i-1}(\alpha_i):K_{i-1}] \mid [M:K_{i-1}]\);
		\item 若已知某个分裂域 $E/F$ 的伽罗瓦群规模,则可用
		\([E:F]=|\mathrm{Gal}(E/F)|\)
		以及中间域对应子群来反推交域与次数(用来判定 $K\cap L$)。
	\end{itemize}
	
	\item \textbf{(5) 一个最常用的“模板式”算例(说明套路).}\;
	例如
	\[
	E=\mathbb Q(\sqrt[3]{2},\zeta_3,\sqrt2).
	\]
	先算
	\[
	[\mathbb Q(\sqrt[3]{2}):\mathbb Q]=3,\qquad [\mathbb Q(\zeta_3):\mathbb Q]=2,
	\]
	且 $\mathbb Q(\sqrt[3]{2})\subset\mathbb R$ 与 $\mathbb Q(\zeta_3)\not\subset\mathbb R$ 交为 $\mathbb Q$,
	故
	\[
	[\mathbb Q(\sqrt[3]{2},\zeta_3):\mathbb Q]=3\cdot 2=6.
	\]
	再加 $\sqrt2$:
	\[
	[\mathbb Q(\sqrt2):\mathbb Q]=2,
	\]
	若能判定 $\sqrt2\notin \mathbb Q(\sqrt[3]{2},\zeta_3)$(等价于交为 $\mathbb Q$),则
	\[
	[E:\mathbb Q]=6\cdot 2=12.
	\]
\end{itemize}



\subsection{单代数扩张 $F(\alpha)$ 的三种等价形式}  

设 $E/F$ 为域扩张,$\alpha\in E$ 且 $\alpha$ 代数于 $F$。


\subsubsection{(1) 生成式(最抽象的定义)}
\[
F(\alpha)=\bigcap\{\,K\subseteq E \mid K\text{ 是子域且 }F\subseteq K,\ \alpha\in K\,\}.
\]
即:$F(\alpha)$ 是包含 $F$ 与 $\alpha$ 的最小子域。

\subsubsection{(2) 多项式值形式(代数元时\textcolor{red}{$F(\alpha)=F[\alpha]$})} 
\[
F[\alpha]=\{\,f(\alpha)\mid f(x)\in F[x]\,\},\qquad
\boxed{\,\alpha\ \text{代数}\ \Rightarrow\ F(\alpha)=F[\alpha]\,}.
\]

\subsubsection{(3) 基(线性空间)形式(最常用的标准表示)}
\textbf{命题(单代数扩张的维数与幂基).}

设 $E/F$ 为域扩张,$\alpha\in E$ 为 $F$ 上的代数元,且
\[
\deg(\alpha,F)=n.
\]
则
\[
F(\alpha)=F[\alpha],
\]
并且作为 $F$-线性空间有
\[
F(\alpha)=\mathrm{span}_F\{1,\alpha,\ldots,\alpha^{n-1}\},\qquad \dim_F F(\alpha)=n,
\]
从而
\[
\{1,\alpha,\ldots,\alpha^{n-1}\}
\ \text{是}\ F(\alpha)\ \text{的一组}\ F\text{-基}.
\]
\textbf{证明.}
设 $m(x)=\mathrm{Irr}(\alpha,F)\in F[x]$ 为 $\alpha$ 在 $F$ 上的极小多项式,则
$m$ 不可约、首一且
\[
\deg m = n,\qquad m(\alpha)=0.
\]

\medskip
\textbf{(1) 证明 $F(\alpha)=F[\alpha]$.}
显然 $F[\alpha]\subseteq F(\alpha)$。
反过来取任意 $\dfrac{f(\alpha)}{g(\alpha)}\in F(\alpha)$(其中 $f,g\in F[x],\ g(\alpha)\neq 0$)。
因 $m$ 是极小多项式,故 $m\nmid g$,从而 $\gcd(m,g)=1$。
由贝祖等式,存在 $u(x),v(x)\in F[x]$ 使
\[
u(x)g(x)+v(x)m(x)=1.
\]
代入 $x=\alpha$ 得
\[
u(\alpha)g(\alpha)+v(\alpha)m(\alpha)=1
\quad\Rightarrow\quad
u(\alpha)g(\alpha)=1,
\]
故 $g(\alpha)^{-1}=u(\alpha)\in F[\alpha]$。
于是
\[
\frac{f(\alpha)}{g(\alpha)}=f(\alpha)\,g(\alpha)^{-1}=f(\alpha)u(\alpha)\in F[\alpha],
\]
从而 $F(\alpha)\subseteq F[\alpha]$,故 $F(\alpha)=F[\alpha]$。

\medskip
\textbf{(2) 证明 $\{1,\alpha,\ldots,\alpha^{n-1}\}$ 张成 $F(\alpha)$.}
任取 $y\in F(\alpha)=F[\alpha]$,则 $y=f(\alpha)$,其中 $f(x)\in F[x]$。
用多项式除法将 $f$ 除以 $m$:
\[
f(x)=q(x)m(x)+r(x),\qquad r(x)=0\ \text{或}\ \deg r<n.
\]
代入 $x=\alpha$ 得
\[
f(\alpha)=q(\alpha)m(\alpha)+r(\alpha)=r(\alpha).
\]
写 $r(x)=a_0+a_1x+\cdots+a_{n-1}x^{n-1}$($a_i\in F$),则
\[
y=f(\alpha)=r(\alpha)=a_0+a_1\alpha+\cdots+a_{n-1}\alpha^{n-1},
\]
故 $\{1,\alpha,\ldots,\alpha^{n-1}\}$ 张成 $F(\alpha)$。

\medskip
\textbf{(3) 证明线性无关.}
若存在 $a_0,\ldots,a_{n-1}\in F$ 使
\[
a_0+a_1\alpha+\cdots+a_{n-1}\alpha^{n-1}=0,
\]
令 $p(x)=a_0+a_1x+\cdots+a_{n-1}x^{n-1}\in F[x]$,
则 $p(\alpha)=0$ 且 $\deg p\le n-1$。
由极小多项式的定义,非零多项式 $p$ 若满足 $p(\alpha)=0$,则必有 $m\mid p$,
从而
\[
n=\deg m\le \deg p\le n-1,
\]
矛盾。故只能 $p=0$,即 $a_0=\cdots=a_{n-1}=0$,线性无关成立。

\medskip
由 (2) 与 (3),$\{1,\alpha,\ldots,\alpha^{n-1}\}$ 是 $F(\alpha)$ 的一组 $F$-基,
因此 $\dim_F F(\alpha)=n$	\qed

\subsection{判断$F(a_1,a_2,\dots)$是否为代数扩张}
\textbf{命题.}\;
设 $F\subseteq E$ 为域扩张,$a_1,a_2,\dots\in E$,令
\[
K:=F(a_1,a_2,\dots)
\]
为由 $\{a_i\}_{i\ge 1}$ 生成的合域。证明:
\[
K/F \text{ 是代数扩张}\ \Longleftrightarrow\ \forall i,\ a_i \text{ 都是 }F\text{-代数元}.
\]

\textbf{证明.}

\textbf{($\Rightarrow$) 若 $K/F$ 代数,则每个 $a_i$ 都是 $F$-代数元.}\;
若 $K/F$ 为代数扩张,按定义 $K$ 的每个元素都对 $F$ 代数。
而 $a_i\in K$(因为 $K$ 是由 $a_1,a_2,\dots$ 生成的最小包含它们的子域),
故每个 $a_i$ 均对 $F$ 代数。

\textbf{($\Leftarrow$) 若每个 $a_i$ 都对 $F$ 代数,则 $K/F$ 代数.}\;
假设对所有 $i$,$a_i$ 都是 $F$-代数元。
欲证 $K/F$ 为代数扩张,即证任取 $x\in K$,都有 $x$ 对 $F$ 代数。

\textbf{步骤 1:任意 $x\in K$ 实际只依赖有限多个生成元.}\;
记
\[
K=\bigcap\{\,L\subseteq E\mid L \text{ 为域},\ F\subseteq L,\ a_i\in L\ \forall i\,\},
\]
即 $K$ 是包含 $F$ 与全部 $a_i$ 的最小子域。
另一方面,令
\[
K_n:=F(a_1,\dots,a_n)\quad(n\ge 1),
\]
则显然 $K_1\subseteq K_2\subseteq\cdots$ 且每个 $K_n$ 都包含 $F$ 与 $a_1,\dots,a_n$。
因此并集
\[
U:=\bigcup_{n\ge 1} K_n
\]
包含 $F$ 与全部 $a_i$。

\textbf{断言:$U$ 是一个域,且 $U=K$.}
证明如下:任取 $x,y\in U$,则存在 $m,n$ 使 $x\in K_m$、$y\in K_n$。
令 $N:=\max\{m,n\}$,则 $K_m,K_n\subseteq K_N$,从而 $x,y\in K_N$。
由于 $K_N$ 是域,故
\[
x\pm y,\ xy\in K_N,\qquad x\neq 0\Rightarrow x^{-1}\in K_N.
\]
于是 $x\pm y,xy,x^{-1}\in U$,说明 $U$ 对域运算封闭,故 $U$ 为域。
又 $U$ 含 $F$ 与全部 $a_i$,而 $K$ 是含这些元素的最小子域,故 $K\subseteq U$。
反过来每个 $K_n$ 都是含 $F$ 与全部 $a_i$ 的子域之一(当然含全部 $a_i$ 不成立,但 $K$ 含所有 $a_i$,
而 $K_n\subseteq K$ 显然成立),更直接地:由 $K$ 含 $F$ 与全部 $a_i$ 可知 $K$ 也含 $a_1,\dots,a_n$,
因此 $K_n=F(a_1,\dots,a_n)\subseteq K$,从而 $U=\bigcup K_n\subseteq K$。
综上 $U=K$。
于是得到关键结论:
\[
\forall x\in K,\ \exists n\ \text{使得}\ x\in K_n=F(a_1,\dots,a_n).
\]

\textbf{步骤 2:每个有限生成子域 $K_n/F$ 都是代数扩张.}\;
我们对 $n$ 归纳证明 $K_n/F$ 为代数扩张。

当 $n=1$ 时,$K_1=F(a_1)$,而 $a_1$ 对 $F$ 代数,
故任意 $x\in F(a_1)$ 都对 $F$ 代数,从而 $K_1/F$ 代数(这是单生成代数扩张的基本性质)。

设对某个 $n-1$ 已知 $K_{n-1}/F$ 为代数扩张。
因为 $a_n$ 对 $F$ 代数,而 $F\subseteq K_{n-1}$,
所以 $a_n$ 也对 $K_{n-1}$ 代数:确有 $p_n(x)\in F[x]\subseteq K_{n-1}[x]$ 使 $p_n(a_n)=0$。
于是
\[
K_n=K_{n-1}(a_n)
\]
是由向 $K_{n-1}$ 添加一个代数元得到的扩张,因此 $K_n/K_{n-1}$ 代数。
而“代数扩张的传递性”(若 $L/M$ 与 $M/F$ 代数,则 $L/F$ 代数)给出 $K_n/F$ 代数。

归纳完成,故对一切 $n$,$K_n/F$ 代数。

\textbf{步骤 3:推出 $K/F$ 代数.}\;
任取 $x\in K$。由步骤 1 存在 $n$ 使 $x\in K_n$。
而步骤 2 已知 $K_n/F$ 代数,故 $x$ 对 $F$ 代数。
因为 $x$ 任意,得 $K/F$ 为代数扩张。

综上,($\Leftarrow$) 成立。

由 ($\Rightarrow$) 与 ($\Leftarrow$) 两方向,等价性证毕。



\clearpage 
\subsection*{课后习题答案}
\addcontentsline{toc}{subsection}{\textcolor{red}{课后习题答案}}
\begin{enumerate}[label=\textcolor{blue}{\textbf{\large\arabic*.}}]
	\item \[
	\textbf{题目.}\quad \text{对下列每个 }\alpha\in\mathbb C,\ \text{求 }\mathrm{Irr}(\alpha,\mathbb Q)\ (\text{即 }\alpha\text{ 在 }\mathbb Q\text{ 上的首一极小多项式}): 
	\]
	\[
	(1)\ 1+\sqrt{-1};\qquad
	(2)\ \sqrt2+\sqrt{-1};\qquad
	(3)\ \sqrt{\,1+\sqrt[3]{2}\,};\qquad
	(4)\ \sqrt{\,\sqrt[3]{2}-\sqrt{-1}\,}.
	\]
	
	
	\textbf{解:}
	
	
	\[
	\textbf{(1)}\ \alpha=1+i\ (i=\sqrt{-1}).
	\]
	\[
	(\alpha-1)^2=i^2=-1\ \Longrightarrow\ \alpha^2-2\alpha+2=0.
	\]
	故 \(\alpha\) 满足 \(f(x)=x^2-2x+2\in\mathbb Q[x]\)。其判别式
	\(\Delta= (-2)^2-4\cdot 1\cdot 2=4-8=-4<0\),在 \(\mathbb Q\) 上无根,从而不可约。
	\[
	\boxed{\ \mathrm{Irr}(1+i,\mathbb Q)=x^2-2x+2\ }.
	\]
	
	\[
	\textbf{(2)}\ \alpha=\sqrt2+i.
	\]
	记 \(\beta=\sqrt2\)。则 \(\alpha-\beta=i\),平方得
	\[
	(\alpha-\beta)^2=-1\ \Longrightarrow\ \alpha^2-2\alpha\beta+\beta^2+1=0
	\ \Longrightarrow\ \alpha^2-2\alpha\beta+3=0.
	\]
	从而
	\[
	2\alpha\beta=\alpha^2+3\ \Longrightarrow\ \beta=\frac{\alpha^2+3}{2\alpha}.
	\]
	再用 \(\beta^2=2\) 消去 \(\beta\):
	\[
	\left(\frac{\alpha^2+3}{2\alpha}\right)^2=2
	\ \Longrightarrow\ (\alpha^2+3)^2=8\alpha^2
	\ \Longrightarrow\ \alpha^4-2\alpha^2+9=0.
	\]
	故 \(\alpha\) 满足 \(f(x)=x^4-2x^2+9\in\mathbb Q[x]\)。
	
	下面证不可约:令 \(f(x)=(x^2+ax+b)(x^2-ax+d)\)(若在 \(\mathbb Q\) 上可约,只能分解为两二次或含一次因子;而一次因子意味着有有理根,但 \(f(\pm1)=8, f(\pm3)=72\neq0\),且由有理根定理可排除有理根)。
	展开得
	\[
	f(x)=x^4+(b+d-a^2)x^2+a(d-b)x+bd.
	\]
	与 \(x^4-2x^2+9\) 对比得
	\[
	a(d-b)=0,\quad bd=9,\quad b+d-a^2=-2.
	\]
	若 \(a=0\),则 \(b+d=-2,\ bd=9\),从而 \(t^2+2t+9=0\) 有根 \(t=-1\pm2\sqrt2\,i\notin\mathbb Q\),矛盾。
	若 \(d=b\),则 \(b^2=9\Rightarrow b=\pm3\)。代入 \(2b-a^2=-2\) 得
	\[
	b=3\Rightarrow a^2=8\notin\mathbb Q^2;\qquad
	b=-3\Rightarrow a^2=-4\ (\text{无有理 }a).
	\]
	均矛盾,故不可约。
	\[
	\boxed{\ \mathrm{Irr}(\sqrt2+i,\mathbb Q)=x^4-2x^2+9\ }.
	\]
	
	\[
	\textbf{(3)}\ \alpha=\sqrt{\,1+\sqrt[3]{2}\,}.
	\]
	设 \(\gamma=\sqrt[3]{2}\)。则
	\[
	\alpha^2=1+\gamma\ \Longrightarrow\ \gamma=\alpha^2-1.
	\]
	又 \(\gamma^3=2\),故
	\[
	(\alpha^2-1)^3=2.
	\]
	展开并化简:
	\[
	(\alpha^2-1)^3=\alpha^6-3\alpha^4+3\alpha^2-1,
	\]
	于是
	\[
	\alpha^6-3\alpha^4+3\alpha^2-3=0.
	\]
	故 \(\alpha\) 满足
	\[
	f(x)=x^6-3x^4+3x^2-3\in\mathbb Q[x].
	\]
	再证 \(f\) 为极小多项式:注意 \(\mathbb Q(\gamma)\) 的次数为 \(3\)(\(\gamma\) 的极小多项式为 \(t^3-2\))。
	且 \(\alpha\notin\mathbb Q(\gamma)\),否则 \(\alpha^2\in\mathbb Q(\gamma)\) 但 \(\alpha\in\mathbb Q(\gamma)\) 将使得
	\(\mathbb Q(\alpha)\subseteq\mathbb Q(\gamma)\),从而 \([\mathbb Q(\alpha):\mathbb Q]\mid 3\)。
	然而 \(\alpha\) 显然不是有理数,故度只能为 \(3\),但若为 \(3\) 则 \(\alpha\in\mathbb Q(\gamma)\)(因为 \(\mathbb Q(\gamma)\) 是 \(\mathbb Q\) 的唯一三次子扩张之一且已包含 \(\alpha^2\) 的信息),与 \(\alpha^2=1+\gamma\) 作为“非平方”元素的平方根相矛盾;更直接地:
	若 \(\alpha\in\mathbb Q(\gamma)\),则存在 \(a,b,c\in\mathbb Q\) 使 \(\alpha=a+b\gamma+c\gamma^2\)。
	平方后与 \(1+\gamma\) 比较会得到关于 \(\gamma\) 的 \(\{1,\gamma,\gamma^2\}\) 线性组合恒等式,导出 \(b=c=0\) 且 \(a^2=1\) 矛盾(因为右端含 \(\gamma\) 项)。
	因此 \([\mathbb Q(\alpha):\mathbb Q(\gamma)]=2\),从而
	\[
	[\mathbb Q(\alpha):\mathbb Q]=2\cdot 3=6.
	\]
	而 \(f\) 为首一且 \(\deg f=6\),故 \(f\) 即为极小多项式。
	\[
	\boxed{\ \mathrm{Irr}\!\left(\sqrt{1+\sqrt[3]{2}},\mathbb Q\right)=x^6-3x^4+3x^2-3\ }.
	\]
	
	\[
	\textbf{(4)}\ \alpha=\sqrt{\,\sqrt[3]{2}-i\,}.
	\]
	设 \(\gamma=\sqrt[3]{2}\)。则
	\[
	\alpha^2=\gamma-i\ \Longrightarrow\ \gamma=\alpha^2+i,\quad i=\gamma-\alpha^2.
	\]
	利用 \(i^2=-1\) 得
	\[
	(\gamma-\alpha^2)^2=-1\ \Longrightarrow\ \gamma^2-2\gamma\alpha^2+\alpha^4+1=0.
	\]
	从而
	\[
	2\gamma\alpha^2=\gamma^2+\alpha^4+1\ \Longrightarrow\ 
	\gamma=\frac{\gamma^2+\alpha^4+1}{2\alpha^2}.
	\]
	为消去 \(\gamma\),更方便的做法是直接写出
	\[
	\gamma=\alpha^2+i\ \Rightarrow\ \gamma^3=(\alpha^2+i)^3=2.
	\]
	展开:
	\[
	(\alpha^2+i)^3=\alpha^6+3\alpha^4 i+3\alpha^2 i^2+i^3
	=\alpha^6+3\alpha^4 i-3\alpha^2-i.
	\]
	令其实部与虚部分别等于 \(2+0i\),得方程组
	\[
	\alpha^6-3\alpha^2=2,\qquad 3\alpha^4-1=0\cdot? \ \text{(不对,因为 }\alpha\text{ 非实,不能按实虚分离)}.
	\]
	因此改用“消去 \(i\)”的方法:由 \(\alpha^2=\gamma-i\) 得 \(i=\gamma-\alpha^2\)。
	代入 \(i^2=-1\) 得
	\[
	(\gamma-\alpha^2)^2=-1\ \Longrightarrow\ \gamma^2-2\gamma\alpha^2+\alpha^4+1=0.
	\]
	把它视为关于 \(\gamma\) 的二次方程:
	\[
	\gamma=\alpha^2\pm i,
	\]
	与 \(\gamma=\alpha^2+i\) 一致(取 \(+\))。接下来用 \(\gamma^3=2\) 消去 \(\gamma\):
	\[
	(\alpha^2+i)^3-2=0.
	\]
	对该式取共轭并相乘,得到一个系数在 \(\mathbb Q\) 中的多项式(消去 \(i\)):
	\[
	\bigl((x^2+i)^3-2\bigr)\bigl((x^2-i)^3-2\bigr)\in\mathbb Q[x]
	\]
	并令 \(x=\alpha\)。计算:
	\[
	(x^2+i)^3-2=(x^6-3x^2-2)+i(3x^4-1),
	\]
	\[
	(x^2-i)^3-2=(x^6-3x^2-2)-i(3x^4-1).
	\]
	相乘得
	\[
	\left((x^6-3x^2-2)+i(3x^4-1)\right)\left((x^6-3x^2-2)-i(3x^4-1)\right)
	=(x^6-3x^2-2)^2+(3x^4-1)^2.
	\]
	展开并合并:
	\[
	(x^6-3x^2-2)^2=x^{12}-6x^8-4x^6+9x^4+12x^2+4,
	\]
	\[
	(3x^4-1)^2=9x^8-6x^4+1.
	\]
	相加得到
	\[
	x^{12}+3x^8-4x^6+3x^4+12x^2+5.
	\]
	因此 \(\alpha\) 满足
	\[
	f(x)=x^{12}+3x^8-4x^6+3x^4+12x^2+5\in\mathbb Q[x].
	\]
	再说明它是极小多项式:注意 \(\alpha^2=\gamma-i\in \mathbb Q(\gamma,i)\),且 \(\mathbb Q(\gamma,i)\) 的次数为
	\[
	[\mathbb Q(\gamma,i):\mathbb Q]=[\mathbb Q(\gamma):\mathbb Q]\cdot[\mathbb Q(i):\mathbb Q]=3\cdot 2=6
	\]
	(因 \(\mathbb Q(\gamma)\subset\mathbb R\) 与 \(\mathbb Q(i)\) 交为 \(\mathbb Q\))。
	又 \(\alpha\) 是 \(\alpha^2\) 的平方根,通常将次数再乘 \(2\),因此
	\[
	[\mathbb Q(\alpha):\mathbb Q]\in\{6,12\}.
	\]
	若为 \(6\),则 \(\alpha\in \mathbb Q(\gamma,i)\);但在该域中 \(\alpha^2=\gamma-i\) 不是平方(可用范数或基 \(\{1,\gamma,\gamma^2\}\) 上比较系数验证),从而平方根不在域内,故次数应为 \(12\)。
	而上面得到的 \(f\) 是首一且次数 \(12\),因此 \(f\) 即为极小多项式。
	\[
	\boxed{\ \mathrm{Irr}\!\left(\sqrt{\sqrt[3]{2}-i},\mathbb Q\right)
		=x^{12}+3x^8-4x^6+3x^4+12x^2+5\ }.
	\]
	
	
	\item 5.
	\textbf{题目.}
	设 $F(\alpha)$ 是 $F$ 的单代数扩张,且 $\deg(\alpha,F)=n$(即 $\alpha$ 在 $F$ 上的极小多项式次数为 $n$)。
	证明:$F(\alpha)$ 是 $n$ 维 $F$-线性空间,且
	\[
	1,\alpha,\ldots,\alpha^{n-1}
	\]
	是 $F(\alpha)$ 的一组基。
	
	\textbf{解:}
	
	设 $m(x)\in F[x]$ 为 $\alpha$ 在 $F$ 上的极小多项式,则 $m$ 不可约且
	\[
	\deg m = n,\qquad m(\alpha)=0.
	\]
	
	\textbf{(1) 先证张成:$F(\alpha)=\mathrm{span}_F\{1,\alpha,\ldots,\alpha^{n-1}\}$.}
	
	先证明 $F(\alpha)=F[\alpha]$(代数元情形)。
	任取 $0\neq g(\alpha)\in F[\alpha]$,其中 $g(x)\in F[x]$。
	由于 $m(x)$ 是极小多项式,故 $m\nmid g$,从而 $\gcd(g,m)=1$。
	由贝祖等式,存在 $u(x),v(x)\in F[x]$ 使得
	\[
	u(x)g(x)+v(x)m(x)=1.
	\]
	令 $x=\alpha$,得
	\[
	u(\alpha)g(\alpha)+v(\alpha)m(\alpha)=1 \;\Rightarrow\; u(\alpha)g(\alpha)=1.
	\]
	故 $g(\alpha)$ 在 $F[\alpha]$ 中可逆,且 $g(\alpha)^{-1}=u(\alpha)\in F[\alpha]$。
	于是任取 $F(\alpha)$ 中元素(按定义为有理式)可写为
	\[
	\frac{f(\alpha)}{g(\alpha)} = f(\alpha)\,g(\alpha)^{-1}\in F[\alpha],
	\]
	从而 $F(\alpha)\subseteq F[\alpha]$;反向包含显然,故 $F(\alpha)=F[\alpha]$。
	
	接着证明任意 $F[\alpha]$ 中元素都可用 $\{1,\alpha,\ldots,\alpha^{n-1}\}$ 表示。
	任取 $f(\alpha)\in F[\alpha]$,其中 $f(x)\in F[x]$。
	用多项式除法将 $f$ 除以 $m$:
	\[
	f(x)=q(x)m(x)+r(x),\qquad r(x)=0\ \text{或}\ \deg r<\deg m=n.
	\]
	代入 $x=\alpha$ 得
	\[
	f(\alpha)=q(\alpha)m(\alpha)+r(\alpha)=r(\alpha).
	\]
	因此每个 $f(\alpha)$ 都等于某个次数 $<n$ 的多项式在 $\alpha$ 处的值,即
	\[
	f(\alpha)=c_0+c_1\alpha+\cdots+c_{n-1}\alpha^{n-1}\quad (c_i\in F).
	\]
	结合 $F(\alpha)=F[\alpha]$,得到 $F(\alpha)$ 被 $1,\alpha,\ldots,\alpha^{n-1}$ 张成。
	
	\textbf{(2) 再证线性无关:$1,\alpha,\ldots,\alpha^{n-1}$ 在 $F$ 上线性无关。}
	
	若存在不全为零的 $c_0,\ldots,c_{n-1}\in F$ 使
	\[
	c_0+c_1\alpha+\cdots+c_{n-1}\alpha^{n-1}=0,
	\]
	则令
	\[
	p(x)=c_0+c_1x+\cdots+c_{n-1}x^{n-1}\in F[x],
	\]
	便有 $p(\alpha)=0$ 且 $\deg p\le n-1$。
	但 $m(x)$ 是 $\alpha$ 的极小多项式,意味着任何满足 $p(\alpha)=0$ 的非零多项式都必须被 $m(x)$ 整除,因而
	\[
	\deg m \le \deg p \le n-1,
	\]
	与 $\deg m=n$ 矛盾。
	故上述线性关系只能是平凡的,即 $c_0=\cdots=c_{n-1}=0$,从而线性无关成立。
	
	\textbf{(3) 结论.}
	
	由 (1) 得它们张成 $F(\alpha)$,由 (2) 得它们线性无关,所以
	\[
	\{1,\alpha,\ldots,\alpha^{n-1}\}
	\]
	是 $F(\alpha)$ 的一组基,因此
	\[
	\dim_F F(\alpha)=n.
	\]
	
	\item 6.
	\textbf{题目.}\;
	证明:对任意正整数 $n$,存在 $\mathbb Q$ 的代数扩张 $K$,使得
	\[
	[K:\mathbb Q]=n.
	\]
	
	\textbf{证明.}\;
	
	我们分情况构造这样的代数扩张。
	
	\textbf{(1) $n=1$ 的情形.}\;
	取 $K=\mathbb Q$,则
	\[
	[K:\mathbb Q]=1,
	\]
	结论成立。
	
	\textbf{(2) $n\ge 2$ 的情形.}\;
	考虑多项式
	\[
	f(x)=x^n-2\in\mathbb Q[x].
	\]
	
	\textbf{断言:}$f(x)$ 在 $\mathbb Q[x]$ 中不可约。
	
	事实上,取素数 $p=2$,则
	\[
	f(x)=x^n-2
	\]
	满足 Eisenstein 判别法的条件:
	\begin{itemize}
		\item $2\mid 0$(所有中间系数为 $0$);
		\item $2\nmid 1$(首项系数);
		\item $2^2\nmid 2$(常数项)。
	\end{itemize}
	因此 $f(x)$ 在 $\mathbb Q[x]$ 中不可约。
	
	设
	\[
	\alpha=\sqrt[n]{2},
	\]
	即 $\alpha^n=2$。
	由于 $f(\alpha)=0$,$\alpha$ 是代数数,且 $f(x)$ 是 $\alpha$ 在 $\mathbb Q$ 上的最小多项式。
	于是
	\[
	[\mathbb Q(\alpha):\mathbb Q]=\deg f=n.
	\]
	
	令
	\[
	K=\mathbb Q(\sqrt[n]{2}),
	\]
	则 $K/\mathbb Q$ 是代数扩张,且
	\[
	[K:\mathbb Q]=n.
	\]
	
	\textbf{结论.}\;
	对任意正整数 $n$,都存在 $\mathbb Q$ 的代数扩张 $K$ 使得 $[K:\mathbb Q]=n$。\qed
	
	
	\item 7.
	\textbf{题目.}\;
	设 $E/F$ 为域扩张,$\alpha,\beta\in E$ 都是 $F$ 上的代数元,且
	\[
	\deg(\alpha,F)\ \text{与}\ \deg(\beta,F)\ \text{互素}.
	\]
	证明:$\mathrm{Irr}(\alpha,F)$ 在 $F(\beta)[x]$ 中不可约,从而
	\[
	[F(\alpha,\beta):F]=\deg(\alpha,F)\deg(\beta,F).
	\]
	
	\textbf{答案:}
	
	记
	\[
	m:=\deg(\alpha,F)=[F(\alpha):F],\qquad n:=\deg(\beta,F)=[F(\beta):F],
	\]
	并设 $\gcd(m,n)=1$。
	
	\textbf{(1) 先比较 $\alpha$ 在 $F(\beta)$ 上的次数.}\;
	令
	\[
	d:=\deg(\alpha,F(\beta))=[F(\alpha,\beta):F(\beta)].
	\]
	由于 $F\subseteq F(\beta)$,$\alpha$ 在更大的基域上最小多项式次数只会不增,
	等价地说
	\[
	\mathrm{Irr}(\alpha,F(\beta))\ \mid\ \mathrm{Irr}(\alpha,F)\quad\text{于}\ F(\beta)[x],
	\]
	因此
	\[
	d=\deg(\alpha,F(\beta))\ \mid\ \deg(\alpha,F)=m.
	\tag{1}
	\]
	
	另一方面,由塔式公式,
	\[
	[F(\alpha,\beta):F]=[F(\alpha,\beta):F(\beta)]\,[F(\beta):F]=d\,n.
	\tag{2}
	\]
	同样用塔式公式(换一条塔):
	\[
	[F(\alpha,\beta):F]=[F(\alpha,\beta):F(\alpha)]\,[F(\alpha):F]
	=\deg(\beta,F(\alpha))\cdot m,
	\tag{3}
	\]
	其中 $\deg(\beta,F(\alpha))=[F(\alpha,\beta):F(\alpha)]$。
	
	由 (2)(3) 得
	\[
	d\,n=m\cdot \deg(\beta,F(\alpha)).
	\tag{4}
	\]
	从 (4) 可见 $m\mid d\,n$。又因为 $\gcd(m,n)=1$,于是必有
	\[
	m\mid d.
	\tag{5}
	\]
	结合 (1) 中 $d\mid m$,推出
	\[
	d=m.
	\tag{6}
	\]
	
	\textbf{(2) 推出 $\mathrm{Irr}(\alpha,F)$ 在 $F(\beta)[x]$ 不可约.}\;
	若 $\mathrm{Irr}(\alpha,F)$ 在 $F(\beta)[x]$ 可约,则 $\alpha$ 在 $F(\beta)$ 上的最小多项式
	$\mathrm{Irr}(\alpha,F(\beta))$ 会是它的真因子,从而
	\[
	\deg(\alpha,F(\beta))<\deg(\alpha,F),
	\]
	即 $d<m$,这与 (6) 矛盾。
	故 $\mathrm{Irr}(\alpha,F)$ 在 $F(\beta)[x]$ 中不可约。
	
	\textbf{(3) 计算合成扩张次数.}\;
	由 (6) 与塔式公式 (2),得到
	\[
	[F(\alpha,\beta):F]=[F(\alpha,\beta):F(\beta)]\,[F(\beta):F]
	=d\,n=m\,n
	=\deg(\alpha,F)\deg(\beta,F).
	\]
	
	\item 9.
	\textbf{题目.}\;
	若 $a,b\in\mathbb Q$ 且 $\sqrt a+\sqrt b\neq 0$,证明:
	\[
	\mathbb Q(\sqrt a,\sqrt b)=\mathbb Q(\sqrt a+\sqrt b).
	\]
	
	\textbf{证明.}\;
	令
	\[
	\alpha:=\sqrt a+\sqrt b.
	\]
	显然 $\alpha\in\mathbb Q(\sqrt a,\sqrt b)$,故
	\[
	\mathbb Q(\alpha)\subseteq \mathbb Q(\sqrt a,\sqrt b).
	\]
	下面证明反向包含。
	
	由于 $\alpha\neq 0$,在域 $\mathbb Q(\alpha)$ 中有 $\alpha^{-1}\in\mathbb Q(\alpha)$。
	注意到
	\[
	(\sqrt a+\sqrt b)(\sqrt a-\sqrt b)=a-b,
	\]
	于是
	\[
	\sqrt a-\sqrt b=\frac{a-b}{\sqrt a+\sqrt b}=\frac{a-b}{\alpha}\in\mathbb Q(\alpha),
	\]
	其中用到 $a-b\in\mathbb Q\subseteq\mathbb Q(\alpha)$ 且 $\alpha\neq 0$。
	
	因此
	\[
	2\sqrt a=(\sqrt a+\sqrt b)+(\sqrt a-\sqrt b)=\alpha+\frac{a-b}{\alpha}\in\mathbb Q(\alpha),
	\]
	\[
	2\sqrt b=(\sqrt a+\sqrt b)-(\sqrt a-\sqrt b)=\alpha-\frac{a-b}{\alpha}\in\mathbb Q(\alpha).
	\]
	从而 $\sqrt a,\sqrt b\in\mathbb Q(\alpha)$,即
	\[
	\mathbb Q(\sqrt a,\sqrt b)\subseteq \mathbb Q(\alpha)=\mathbb Q(\sqrt a+\sqrt b).
	\]
	
	结合两侧包含关系,得到
	\[
	\mathbb Q(\sqrt a,\sqrt b)=\mathbb Q(\sqrt a+\sqrt b).
	\] \qed
	\item 12.
	\textbf{题目.}\;
	设 $E/F$ 为代数扩张,$D$ 为 $E$ 的子环且 $F\subseteq D$,求证 $D$ 为域。
	特别地,如果 $\alpha_1,\alpha_2,\dots,\alpha_n$ 都是 $F$ 上代数元,则
	\[
	F[\alpha_1,\alpha_2,\dots,\alpha_n]=F(\alpha_1,\alpha_2,\dots,\alpha_n).
	\]
	
	\textbf{答案:}
	
	\textbf{(1) 证明:$D$ 是域.}\;
	要证 $D$ 是域,只需证明:对任意 $0\neq d\in D$,有 $d^{-1}\in D$。
	
	取 $0\neq d\in D$。因 $E/F$ 为代数扩张且 $F\subseteq D\subseteq E$,
	则 $d\in E$ 也必为 $F$ 上代数元。
	令 $m_d(x)\in F[x]$ 为 $d$ 在 $F$ 上的最小多项式,则
	\[
	m_d(x)=x^k+a_{k-1}x^{k-1}+\cdots+a_1x+a_0,\qquad a_i\in F,
	\]
	且 $m_d(d)=0$。
	
	\underline{关键:常数项 $a_0\neq 0$.}
	若 $a_0=0$,则 $m_d(x)=x\cdot h(x)$(其中 $h(x)\in F[x]$),于是
	\[
	0=m_d(d)=d\cdot h(d).
	\]
	由于 $E$ 是域且 $d\neq 0$,可消去 $d$ 得 $h(d)=0$,
	这与 $m_d$ 的最小性矛盾(因为 $\deg h<\deg m_d$ 且 $h(d)=0$)。
	故 $a_0\neq 0$。
	
	由 $m_d(d)=0$ 得
	\[
	d^k+a_{k-1}d^{k-1}+\cdots+a_1d+a_0=0,
	\]
	移项并两边除以 $d$(在域 $E$ 中合法):
	\[
	d^{k-1}+a_{k-1}d^{k-2}+\cdots+a_2d+a_1+a_0d^{-1}=0,
	\]
	从而
	\[
	d^{-1}=-a_0^{-1}\Bigl(d^{k-1}+a_{k-1}d^{k-2}+\cdots+a_2d+a_1\Bigr).
	\tag{$\ast$}
	\]
	注意右端是 $d$ 的多项式(系数在 $F$ 中),并且 $a_0^{-1}\in F$。
	
	因为 $F\subseteq D$ 且 $d\in D$,所以 $D$ 在加法、乘法下封闭,
	推出 $d^j\in D$($j\ge 0$),以及右端括号内元素属于 $D$;
	再乘以 $-a_0^{-1}\in F\subseteq D$ 仍在 $D$ 中。
	因此由 $(\ast)$ 得 $d^{-1}\in D$。
	
	于是 $D$ 中任一非零元都有逆元,故 $D$ 为域。
	
	\textbf{(2) 推论:$F[\alpha_1,\dots,\alpha_n]=F(\alpha_1,\dots,\alpha_n)$.}\;
	令
	\[
	D:=F[\alpha_1,\alpha_2,\dots,\alpha_n]\subseteq E.
	\]
	显然 $D$ 是 $E$ 的子环且 $F\subseteq D$。
	又因 $E/F$ 为代数扩张而每个 $\alpha_i$ 都是 $F$ 上代数元,
	从而 $F(\alpha_1,\dots,\alpha_n)/F$ 仍是代数扩张,
	并且 $D\subseteq F(\alpha_1,\dots,\alpha_n)$。
	对任意 $x\in D$,都有 $x\in E$ 且 $x$ 为 $F$ 上代数元,
	于是可对 $D$ 应用 (1) 的结论,得到 $D$ 是域。
	
	但 $F(\alpha_1,\dots,\alpha_n)$ 是包含 $F$ 与所有 $\alpha_i$ 的最小\emph{域},
	而 $D$ 已经是一个包含 $F$ 与所有 $\alpha_i$ 的\emph{域},所以必有
	\[
	F(\alpha_1,\dots,\alpha_n)\subseteq D.
	\]
	另一方面总有 $F[\alpha_1,\dots,\alpha_n]\subseteq F(\alpha_1,\dots,\alpha_n)$,
	因此二者相等:
	\[
	F[\alpha_1,\alpha_2,\dots,\alpha_n]=F(\alpha_1,\alpha_2,\dots,\alpha_n).
	\] \qed
	
\end{enumerate}


\clearpage
\section{尺规作图}

\clearpage 
\subsection*{课后习题答案}
\addcontentsline{toc}{subsection}{\textcolor{red}{课后习题答案}}
\begin{enumerate}[label=\textcolor{blue}{\textbf{\large\arabic*.}}]
	\item 
	\[
	\textbf{题目.}\quad \text{证明:正五边形可以用尺规作图作出,并探讨作图方法。}
	\]
	
	\[
	\textbf{证明.}
	\]
	\textbf{一、判定思想:化为构造 } $\cos\frac{2\pi}{5}.$
	\[
	\text{设单位圆上正五边形的相邻顶点所对圆心角为 } \frac{2\pi}{5}=72^\circ.
	\]
	因此只要能尺规作出角 \(72^\circ\)(或等价地作出弦长),就能在给定圆上依次截取弦得到正五边形。
	
	令
	\[
	x=\cos 72^\circ=\cos\frac{2\pi}{5}.
	\]
	利用恒等式
	\[
	\cos 5\theta = 16\cos^5\theta-20\cos^3\theta+5\cos\theta,
	\]
	取 \(\theta=72^\circ\),则 \(5\theta=360^\circ\),从而
	\[
	\cos 360^\circ = 1 = 16x^5-20x^3+5x.
	\]
	故 \(x\) 满足多项式
	\[
	16x^5-20x^3+5x-1=0.
	\]
	但对 \(\theta=72^\circ\) 可用更简洁的方式降到二次:注意
	\[
	\cos 72^\circ = \sin 18^\circ,\qquad
	\cos 36^\circ = 2\cos^2 18^\circ-1.
	\]
	并且经典结果(可由上式或用 \(\cos 5\theta\) 分解得到)给出
	\[
	\cos 36^\circ = \frac{1+\sqrt5}{4}.
	\]
	于是
	\[
	\cos 72^\circ = 2\cos^2 36^\circ -1
	=2\left(\frac{1+\sqrt5}{4}\right)^2-1
	=\frac{\sqrt5-1}{4}.
	\]
	因此
	\[
	\cos 72^\circ = \frac{\sqrt5-1}{4}\in \mathbb Q(\sqrt5),
	\]
	而 \(\sqrt5\) 是尺规可作的(作线段 5 的平方根即可),所以 \(\cos 72^\circ\) 可尺规作出,进而角 \(72^\circ\) 可作出,从而正五边形可尺规作图。
	
	\[
	\textbf{结论:正五边形是尺规可作的。}
	\]
	
	
	\item 
	\textbf{题目.}
	正 $9$ 边形是否可以通过尺规作图得到?
	
	\textbf{解:}
	不能尺规作出正 $9$ 边形。
	
	\textbf{证明.}
	尺规作图能作出的实数(可作数)满足一个基本性质:若实数 $x$ 可作,则其生成的域扩张次数
	\[
	[\mathbb{Q}(x):\mathbb{Q}]
	\]
	必须是 $2$ 的幂。
	
	而能否作出正 $n$ 边形,等价于能否作出中心角 $\frac{2\pi}{n}$,也等价于能否作出
	\[
	\cos\frac{2\pi}{n}.
	\]
	因此,若正 $9$ 边形可尺规作出,则
	\[
	\cos\frac{2\pi}{9}=\cos 40^\circ
	\]
	必须是可作数,从而其次数 $[\mathbb{Q}(\cos\frac{2\pi}{9}):\mathbb{Q}]$ 必为 $2$ 的幂。
	
	下面计算这个次数。令
	\[
	c=\cos\frac{2\pi}{9}.
	\]
	利用三倍角公式
	\[
	\cos(3\theta)=4\cos^3\theta-3\cos\theta,
	\]
	取 $\theta=\frac{2\pi}{9}$,则 $3\theta=\frac{2\pi}{3}$,所以
	\[
	\cos\frac{2\pi}{3} = 4c^3-3c.
	\]
	又 $\cos\frac{2\pi}{3}=-\frac12$,故
	\[
	4c^3-3c=-\frac12
	\quad\Longleftrightarrow\quad
	8c^3-6c+1=0.
	\]
	因此 $c$ 是多项式 $8x^3-6x+1\in\mathbb{Q}[x]$ 的根。
	
	进一步可证明(或引用:$\mathbb{Q}(\zeta_9)$ 的实子域次数为 $\varphi(9)/2=3$),
	$8x^3-6x+1$ 在 $\mathbb{Q}[x]$ 中不可约,于是
	\[
	[\mathbb{Q}(c):\mathbb{Q}]=3.
	\]
	但 $3$ 不是 $2$ 的幂,这与“可作数的次数必须是 $2$ 的幂”矛盾。
	
	故 $\cos\frac{2\pi}{9}$ 不是可作数,从而正 $9$ 边形不能尺规作出。
	
	\item 7.
	\textbf{题目.}\;
	证明:正 $17$ 边形可以由尺规作图得到。
	
	\textbf{证明.}\;
	设 $\zeta_{17}=e^{2\pi i/17}$,令
	\[
	L:=\mathbb{Q}(\zeta_{17}),\qquad K:=\mathbb{Q}(\zeta_{17}+\zeta_{17}^{-1}).
	\]
	显然
	\[
	\cos\frac{2\pi}{17}=\frac{\zeta_{17}+\zeta_{17}^{-1}}{2}\in K,
	\]
	因此只需说明 $\cos\frac{2\pi}{17}$ 是可作图数(即可作出圆心角 $2\pi/17$)。
	
	因为 $17$ 为素数,第 $17$ 次分圆多项式
	\[
	\Phi_{17}(x)=x^{16}+x^{15}+\cdots+x+1
	\]
	在 $\mathbb{Q}[x]$ 中不可约,从而
	\[
	[L:\mathbb{Q}]=\deg\Phi_{17}=\varphi(17)=16.
	\]
	复共轭在 $L$ 上给出一个阶为 $2$ 的 $\mathbb{Q}$-自同构,其不动域恰为最大实子域 $K$,
	故
	\[
	[L:K]=2,\qquad [K:\mathbb{Q}]=\frac{[L:\mathbb{Q}]}{[L:K]}=\frac{16}{2}=8=2^3.
	\]
	并且
	\[
	\mathrm{Gal}(K/\mathbb{Q})
	\]
	是一个阶为 $8$ 的循环群(因为 $\mathrm{Gal}(L/\mathbb{Q})\cong(\mathbb{Z}/17\mathbb{Z})^\times$ 为阶 $16$ 循环群,
	而 $K$ 对应其包含复共轭的子群,商仍为循环群)。
	
	于是 $\mathrm{Gal}(K/\mathbb{Q})$ 存在子群列
	\[
	G=G_0\supset G_1\supset G_2\supset G_3\supset G_4=\{1\},
	\qquad [G_i:G_{i+1}]=2.
	\]
	由伽罗瓦对应,得到中间域列
	\[
	\mathbb{Q}=K_0\subset K_1\subset K_2\subset K_3\subset K_4=K,
	\qquad [K_{i+1}:K_i]=2.
	\]
	因此 $\cos\frac{2\pi}{17}\in K$ 落在从 $\mathbb{Q}$ 出发的逐次二次扩张中,故为尺规可作图数,
	从而可作出圆心角 $2\pi/17$。在单位圆上依次截取该圆心角得到 $17$ 个等分点,连接相邻点即得正 $17$ 边形。 \qed
\end{enumerate}

\clearpage
\section{分裂域}
\subsection{引理 4.4.2}
(引理 4.4.2)设 $f(x)\in F[x]$,且 $\deg f(x)>0$。证明:存在 $F$ 的有限扩张 $E$,使得 $f(x)$ 在 $E$ 中有一个根。

\textbf{解:}
设 $n=\deg f>0$。

\textbf{第一步:取一个不可约因子。}
由于 $F$ 是域,$F[x]$ 是一个 PID(从而也是 UFD),因此任意非常数多项式都能分解为不可约多项式的乘积。
于是可写
\[
f(x)=c\cdot p_1(x)^{e_1}p_2(x)^{e_2}\cdots p_k(x)^{e_k},
\]
其中 $c\in F^\times$,每个 $p_i(x)\in F[x]$ 都是不可约多项式且 $\deg p_i\ge 1$,$e_i\ge 1$。
取其中任意一个不可约因子,记为
\[
p(x):=p_1(x).
\]
则 $p(x)\mid f(x)$,并且 $\deg p=m\ge 1$。

\textbf{第二步:构造有限扩张 $E=F[x]/\langle p(x)\rangle$。}
考虑商环
\[
E:=F[x]/\langle p(x)\rangle,
\]
并记自然映射为
\[
\pi:F[x]\to E,\qquad g(x)\mapsto \overline{g(x)}.
\]

\textbf{(1) 说明 $E$ 是域。}
因为 $p(x)$ 在 $F[x]$ 中不可约,而 $F[x]$ 是 PID,
故由“PID 中不可约元生成极大理想”等价结论可知理想 $\langle p(x)\rangle$ 是极大理想。
于是商环 $F[x]/\langle p(x)\rangle$ 是域,即 $E$ 为域。

\textbf{(2) 说明 $E/F$ 是有限扩张,并计算其次数。}
将 $F$ 视为常数多项式组成的子环,定义嵌入
\[
\iota:F\hookrightarrow E,\qquad a\mapsto \overline{a}.
\]
该映射是单射(因为 $a\neq 0$ 时常数多项式 $a$ 不可能被非常数多项式 $p(x)$ 整除,从而 $\overline{a}\neq 0$),
故可把 $F$ 看作 $E$ 的子域,从而 $E/F$ 为域扩张。

对任意 $g(x)\in F[x]$,用带余除法写
\[
g(x)=q(x)p(x)+r(x),\qquad \deg r<m.
\]
在 $E$ 中取像得
\[
\overline{g(x)}=\overline{r(x)}.
\]
因此 $E$ 中每个元素都可表示为 $\overline{r(x)}$,其中
\[
r(x)=a_0+a_1x+\cdots+a_{m-1}x^{m-1}\quad(a_i\in F).
\]
令
\[
\alpha:=\overline{x}\in E,
\]
则任意元素都可写成
\[
a_0+a_1\alpha+\cdots+a_{m-1}\alpha^{m-1}.
\]
从而 $E$ 作为 $F$-向量空间由 $\{1,\alpha,\dots,\alpha^{m-1}\}$ 生成,故
\[
[E:F]\le m<\infty,
\]
即 $E/F$ 是有限扩张。

\textbf{第三步:证明 $f$ 在 $E$ 中有根。}
由 $\alpha=\overline{x}$ 的定义,立刻有
\[
p(\alpha)=\overline{p(x)}=0\quad\text{于 }E,
\]
因为 $p(x)\in\langle p(x)\rangle$,其在商环中的像为零元。

又因 $p(x)\mid f(x)$,存在 $g(x)\in F[x]$ 使
\[
f(x)=p(x)g(x).
\]
将 $x$ 替换为 $\alpha$(即在 $E$ 中取像)得到
\[
f(\alpha)=p(\alpha)\,g(\alpha)=0\cdot g(\alpha)=0.
\]
因此 $\alpha\in E$ 是 $f(x)$ 的一个根。

综上,取 $E=F[x]/\langle p(x)\rangle$ 即得到一个 $F$ 的有限扩张,使得 $f$ 在 $E$ 中有根。

\subsection{定理4.4.2嵌入的构造}
(问题 4.4.1)对域 $F$ 上的任意非常数多项式 $f(x)\in F[x]$,
是否一定存在扩张 $E/F$ 使得 $E$ 中包含 $f(x)$ 的一个根?

\textbf{答案:}

\textbf{(0) 先化到不可约情形.}\;
对任意非常数 $f(x)\in F[x]$,在 $F[x]$ 中取一个不可约因子 $p(x)\mid f(x)$。
若能在某个扩张中找到 $p(x)$ 的根 $\alpha$,则 $f(\alpha)=0$ 也成立(因为 $p\mid f$)。
因此只需证明:\emph{对任意首一不可约多项式 $p(x)\in F[x]$,存在扩张包含其一根。}
下文把 $p(x)$ 仍记为 $f(x)$。

\textbf{(1) 构造扩张域 $E=F[x]/(f(x))$(并说明它是域).}\;
设 $f(x)\in F[x]$ 首一不可约且 $\deg f\ge 1$。令
\[
E:=F[x]/(f(x)),
\]
其中 $(f(x))$ 为由 $f(x)$ 生成的理想。
因为 $F[x]$ 是 PID,且 $f(x)$ 不可约 $\Longrightarrow (f(x))$ 是极大理想,
从而商环 $E$ 是域。于是 $E/F$ 是一个域扩张。

记自然商映射为
\[
\pi: F[x]\longrightarrow E,\qquad g(x)\longmapsto g(x)+(f(x)).
\]
并令
\[
\alpha:=\pi(x)=x+(f(x))\in E.
\]

\textbf{(2) 证明 $\alpha$ 是 $f(x)$ 的一个根(“根的存在”).}\;
由商映射的定义,
\[
f(\alpha)=f\bigl(\pi(x)\bigr)=\pi\bigl(f(x)\bigr)=f(x)+(f(x))=0\in E.
\]
因此 $\alpha$ 是 $f(x)$ 在 $E$ 中的一个根。

\textbf{(3) 证明 $E=F(\alpha)$ 且 $\mathrm{Irr}(\alpha,F)=f$(“生成元与最小多项式”).}\;
对任意 $g(x)\in F[x]$,有 $g(\alpha)=\pi(g(x))$。
因此 $E$ 的任意元素都可写为 $g(\alpha)$ 的形式,故
\[
E=\{\pi(g(x)):\,g\in F[x]\}=\{g(\alpha):\,g\in F[x]\}=F(\alpha).
\]
再看 $\alpha$ 的最小多项式。由 $f(\alpha)=0$ 知 $\mathrm{Irr}(\alpha,F)\mid f$。
但 $f$ 在 $F[x]$ 中不可约,且首一,所以只能有
\[
\mathrm{Irr}(\alpha,F)=f.
\]

\textbf{(4) 完整嵌入过程(核心:从 $F[x]/(f)$ 嵌入到任何含根的扩张中).}\;
设 $K/F$ 是任意域扩张,且存在 $\beta\in K$ 使 $f(\beta)=0$。
定义\emph{代入同态}(evaluation homomorphism)
\[
\mathrm{ev}_\beta: F[x]\longrightarrow K,\qquad g(x)\longmapsto g(\beta).
\]
这是一个环同态,并且在 $F$ 上恒等:$\mathrm{ev}_\beta(c)=c$(对 $c\in F$)。

\textbf{(4.1) 核与理想包含:$(f)\subseteq \mathrm{ker}(\mathrm{ev}_\beta)$.}\;
由于 $f(\beta)=0$,
\[
\mathrm{ev}_\beta(f)=f(\beta)=0,
\]
因此对任意 $h(x)\in F[x]$,
\[
\mathrm{ev}_\beta\bigl(h(x)f(x)\bigr)=h(\beta)\,f(\beta)=h(\beta)\cdot 0=0,
\]
即 $(f(x))\subseteq \mathrm{ker}(\mathrm{ev}_\beta)$。

\textbf{(4.2) 由商的泛性质诱导同态 $\widetilde{\mathrm{ev}}_\beta: E\to K$.}\;
因为 $(f)\subseteq \mathrm{ker}(\mathrm{ev}_\beta)$,存在唯一环同态
\[
\widetilde{\mathrm{ev}}_\beta: F[x]/(f)\longrightarrow K
\]
使得
\[
\widetilde{\mathrm{ev}}_\beta\circ \pi = \mathrm{ev}_\beta.
\]
等价地,它由
\[
\widetilde{\mathrm{ev}}_\beta\bigl(g(x)+(f)\bigr)=g(\beta)
\]
定义(需检验良定义:若 $g_1-g_2\in (f)$,则 $g_1(\beta)=g_2(\beta)$,这正由 (4.1) 保证)。

用交换图表示就是:
\[
\begin{array}{ccc}
	F[x] & \xrightarrow{\mathrm{ev}_\beta} & K\\
	\downarrow{\pi} & \nearrow{\widetilde{\mathrm{ev}}_\beta} & \\
	F[x]/(f) & &
\end{array}
\]

\textbf{(4.3) 说明它是 $F$-同态,并给出“把 $\alpha$ 送到 $\beta$”的对应.}\;
对 $c\in F$,
\[
\widetilde{\mathrm{ev}}_\beta\bigl(\pi(c)\bigr)=\mathrm{ev}_\beta(c)=c,
\]
所以 $\widetilde{\mathrm{ev}}_\beta$ 限制在 $F$ 上为恒等,因此它是一个 $F$-同态。
并且
\[
\widetilde{\mathrm{ev}}_\beta(\alpha)=\widetilde{\mathrm{ev}}_\beta\bigl(\pi(x)\bigr)
=\mathrm{ev}_\beta(x)=\beta.
\]
也就是说:\emph{这条诱导同态把 $E=F[x]/(f)$ 中的“抽象根” $\alpha$ 送到 $K$ 中的实际根 $\beta$。}

\textbf{(4.4) 何时是“嵌入”(单射)?此处不可约保证必为单射.}\;
因为 $f$ 在 $F[x]$ 中不可约,$E=F[x]/(f)$ 是域。
而 $\widetilde{\mathrm{ev}}_\beta$ 是从域到域 $K$ 的非零环同态(注意 $1\mapsto 1$,所以不为零同态),
因此其核只能是 $\{0\}$,从而 $\widetilde{\mathrm{ev}}_\beta$ 必为单射。
故 $\widetilde{\mathrm{ev}}_\beta$ 实际给出一个 $F$-\emph{嵌入}
\[
F[x]/(f)\hookrightarrow K,
\]
其像是 $F(\beta)\subseteq K$,并且满足 $\alpha\mapsto \beta$。

\textbf{(5) 结论汇总.}\;
对任意非常数 $f(x)\in F[x]$,取其不可约因子 $p(x)$,
构造 $E=F[x]/\langle p\rangle$,令 $\alpha=x+\langle p\rangle$,则 $p(\alpha)=0$,因而 $f(\alpha)=0$。
并且对任何包含 $f$(或 $p$)某个根 $\beta$ 的扩张 $K/F$,
存在唯一 $F$-同态(且在不可约情形下为 $F$-嵌入)
\[
\widetilde{\mathrm{ev}}_\beta: F[x]/(f)\to K,\qquad x+(f)\mapsto \beta,
\]
这就是“完整的嵌入过程”。

\subsection{推论 4.4.5和推论 4.4.6的证明}

\textbf{推论 4.4.5.}
设 $E$ 是 $f(x)\in F[x]$ 的分裂域,且 $\deg f(x)=n$,证明:$[E:F]\le n!$。

\textbf{推论 4.4.6.}
设 $E$ 是 $f(x)\in F[x]$ 的分裂域,且 $K$ 是 $E/F$ 的中间域($F\subseteq K\subseteq E$),证明:$E$ 也是同一多项式 $f(x)\in K[x]$ 的分裂域。

\textbf{解:}

\textbf{一、推论 4.4.5 的证明($[E:F]\le n!$).}
由定理 4.4.4,$f$ 的分裂域 $E$ 存在。由于 $E$ 是分裂域,$f$ 在 $E[x]$ 中分解为一次因子的乘积。
在某个代数闭包中把 $f$ 的根(按重数计)记为
\[
f(x)=c\prod_{i=1}^{n}(x-\alpha_i)\qquad(c\in F^\times,\ \alpha_i\in E).
\]
令
\[
F_0:=F,\qquad F_i:=F_{i-1}(\alpha_i)\ (i=1,2,\dots,n).
\]
显然 $F_i\subseteq E$(因为 $\alpha_i\in E$),从而 $F_n\subseteq E$;反过来 $F_n$ 含有全部根 $\alpha_1,\dots,\alpha_n$,
而 $E$ 是包含这些根的最小扩张域,因此
\[
E=F(\alpha_1,\dots,\alpha_n)=F_n.
\]

下面估计每一步的扩张次数。对每个 $i$,在 $F_{i-1}[x]$ 中定义
\[
f_{i-1}(x):=\frac{f(x)}{\prod_{j=1}^{i-1}(x-\alpha_j)}.
\]
注意到在域 $F_{i-1}$ 中,因子 $(x-\alpha_j)$ 都属于 $F_{i-1}[x]$,所以 $f_{i-1}(x)\in F_{i-1}[x]$,
并且
\[
\deg f_{i-1}=n-(i-1)=n-i+1.
\]
又由于 $f(\alpha_i)=0$,且前 $i-1$ 个一次因子已被除去,可知 $\alpha_i$ 仍是 $f_{i-1}$ 的一个根,因此
\[
f_{i-1}(\alpha_i)=0.
\]
设 $\operatorname{Irr}(\alpha_i,F_{i-1})$ 为 $\alpha_i$ 在 $F_{i-1}$ 上的最小多项式,则
\[
\operatorname{Irr}(\alpha_i,F_{i-1})\mid f_{i-1}(x)\quad\text{于 }F_{i-1}[x],
\]
从而
\[
[F_i:F_{i-1}]=\deg\operatorname{Irr}(\alpha_i,F_{i-1})\le \deg f_{i-1}=n-i+1.
\]
于是由塔式公式
\[
[E:F]=[F_n:F_0]=\prod_{i=1}^{n}[F_i:F_{i-1}]
\le\prod_{i=1}^{n}(n-i+1)=n!.
\]
这就证明了 $[E:F]\le n!$。

\textbf{二、推论 4.4.6 的证明(中间域上仍是分裂域).}
设 $E$ 是 $f$ 在 $F$ 上的分裂域,且 $F\subseteq K\subseteq E$。

\textbf{(1) $f$ 在 $E$ 上对 $K$ 也分裂.}
因为 $f\in F[x]\subseteq K[x]$,且 $f$ 在 $E[x]$ 中分解为一次因子(根都在 $E$ 中),所以同一个分解当然也是
$K[x]$ 上的分解,即 $f$ 在 $E$ 中作为 $K$-系数多项式仍然分裂。

\textbf{(2) 证明最小性:$E$ 是包含 $K$ 且使 $f$ 分裂的最小域.}
固定一个代数闭包 $\Omega$,使得 $E\subseteq \Omega$。设 $L$ 是 $\Omega$ 的一个子域,满足
\[
K\subseteq L,\qquad\text{且 }f\text{ 在 }L\text{ 中分裂}.
\]
“$f$ 在 $L$ 中分裂”意味着 $f(x)$ 在 $L[x]$ 中分解为一次因子,因此 $f$ 的全部根(按重数计)都属于 $L$。
而 $F\subseteq K\subseteq L$,所以
\[
E=F(\text{所有根})\subseteq L.
\]
这说明:在同一代数闭包 $\Omega$ 中,任何包含 $K$ 且使 $f$ 分裂的域都必然包含 $E$。
结合 (1) 中 $E$ 本身就使 $f$ 分裂,可得 $E$ 正是 $f$ 在 $K$ 上的分裂域。

两条推论均得证。


\subsection{引理4.4.11}
(引理 4.4.11)设 $\varphi:K\to\bar K$ 为域同构。令对任意
$f(x)=\sum_{i=0}^n a_ix^i\in K[x]$ 定义
\[
\bar f(x):=\varphi(f)(x)=\sum_{i=0}^n \bar a_i x^i\in \bar K[x],\qquad \bar a_i:=\varphi(a_i).
\]
设 $p(x)\in K[x]$ 不可约,$\alpha$ 为 $p(x)$ 在 $K$ 的某个扩张中的一个根;$\bar\alpha$ 为 $\bar p(x)$
在 $\bar K$ 的某个扩张中的一个根。证明:

(1) 存在域同构 $\eta:K[x]/(p(x))\to \bar K[x]/(\bar p(x))$ 使得 $\eta|_K=\varphi$。

(2) 存在域同构 $\sigma:K(\alpha)\to \bar K(\bar\alpha)$ 使得 $\sigma|_K=\varphi$ 且 $\sigma(\alpha)=\bar\alpha$。

(3) 若 $\psi:K(\alpha)\to \bar K(\bar\alpha)$ 为域同态且 $\psi|_K=\varphi$,则 $\psi(\alpha)$ 是 $\bar p(x)$ 的根。

\bigskip
\textbf{解(深入讲解与证明).}

\textbf{0. 这条引理在干什么(直觉).}
它回答了“\emph{如何把一个域同构 $\varphi:K\to\bar K$ 延拓到单代数扩张}”:
\begin{itemize}
	\item 先把多项式 $p(x)$ 的系数通过 $\varphi$ 搬运到 $\bar K$,得到 $\bar p(x)$;
	\item 再规定“新元 $\alpha$”在延拓后应当送到 $\bar p$ 的某个根 $\bar\alpha$;
	\item 于是延拓同构就被唯一确定(本引理的 (2)(3) 正是这个机制的基础)。
\end{itemize}

\bigskip
\textbf{1. 预备:$\varphi$ 诱导多项式环同构.}
定义
\[
\widetilde\varphi:K[x]\longrightarrow \bar K[x],\qquad
\widetilde\varphi\!\left(\sum_{i=0}^n a_ix^i\right)=\sum_{i=0}^n \varphi(a_i)x^i.
\]
则 $\widetilde\varphi$ 是环同构(逐项验证加法与乘法即可),并且
\[
\widetilde\varphi|_K=\varphi,\qquad \widetilde\varphi(p(x))=\bar p(x).
\]
这一步的意义:\emph{把“系数域的同构”提升到“多项式环的同构”。}

\bigskip
\textbf{2. 证明 (1):商环(商域)之间的同构.}

\textbf{构造 $\eta$.}
对任意 $g(x)\in K[x]$,定义
\[
\eta\bigl(g(x)+(p(x))\bigr):=\widetilde\varphi(g(x))+(\bar p(x)).
\]
也就是把同余类的代表多项式先做系数映射,再取模 $(\bar p)$。

\textbf{(i) 良定义.}
若 $g_1(x)\equiv g_2(x)\pmod{\langle p\rangle}$,则 $g_1-g_2\in \langle p\rangle$,即存在 $h(x)\in K[x]$ 使
\[
g_1(x)-g_2(x)=h(x)p(x).
\]
对两边施加 $\widetilde\varphi$ 得
\[
\widetilde\varphi(g_1)-\widetilde\varphi(g_2)=\widetilde\varphi(h)\,\widetilde\varphi\langle p\rangle
=\widetilde\varphi(h)\,\bar p\in (\bar p),
\]
故 $\widetilde\varphi(g_1)\equiv \widetilde\varphi(g_2)\pmod{(\bar p)}$,从而 $\eta$ 与代表元无关。

\textbf{(ii) 同态性.}
由 $\widetilde\varphi$ 为环同构,立刻有
\[
\eta(\overline{g_1+g_2})=\eta(\overline{g_1})+\eta(\overline{g_2}),\qquad
\eta(\overline{g_1g_2})=\eta(\overline{g_1})\,\eta(\overline{g_2}),
\]
其中 $\overline{\cdot}$ 表示取模相应理想后的同余类。

\textbf{(iii) $\eta$ 是同构且 $\eta|_K=\varphi$.}
因为 $\widetilde\varphi$ 可逆,其逆为 $\widetilde{\varphi^{-1}}:\bar K[x]\to K[x]$。
同样方式可定义
\[
\eta^{-1}\bigl(\bar g(x)+(\bar p)\bigr):=\widetilde{\varphi^{-1}}(\bar g(x))+\langle p\rangle,
\]
并验证 $\eta^{-1}$ 为 $\eta$ 的逆映射,所以 $\eta$ 是环同构。

对任意 $a\in K$(视为常数多项式),有
\[
\eta\bigl(a+\langle p\rangle\bigr)=\widetilde\varphi(a)+(\bar p)=\varphi(a)+(\bar p),
\]
因此在自然嵌入 $K\hookrightarrow K[x]/\langle p\rangle$、$\bar K\hookrightarrow \bar K[x]/(\bar p)$ 的意义下,
$\eta|_K=\varphi$。

\textbf{补充:为何说“域同构”?}
因 $p$ 在 $K[x]$ 中不可约,$\langle p\rangle$ 为极大理想,故 $K[x]/\langle p\rangle$ 是域;
同理 $\bar K[x]/(\bar p)$ 也是域,所以这里的同构是域同构。

\bigskip
\textbf{3. 证明 (2):从 $K(\alpha)$ 到 $\bar K(\bar\alpha)$ 的同构(核心延拓).}

\textbf{3.1 先把 $K(\alpha)$ 认成商域 $K[x]/\langle p\rangle$.}
定义“代入同态”
\[
\mathrm{ev}_\alpha:K[x]\to K(\alpha),\qquad g(x)\mapsto g(\alpha).
\]
因为 $p(\alpha)=0$,所以 $\langle p\rangle\subseteq \ker(\mathrm{ev}_\alpha)$。
又因 $p$ 不可约且非零,理想 $\langle p\rangle$ 极大,从而 $\ker(\mathrm{ev}_\alpha)$ 只能是 $\langle p\rangle$ 或整个 $K[x]$。
但 $\mathrm{ev}_\alpha(1)=1\neq 0$,故核不可能是整个环,必有
\[
\ker(\mathrm{ev}_\alpha)=\langle p\rangle.
\]
由第一同构定理得到域同构
\[
\theta:K[x]/\langle p\rangle\xrightarrow{\ \sim\ } \mathrm{Im}(\mathrm{ev}_\alpha)=K[\alpha].
\]
由于 $\alpha$ 代数(满足 $p(\alpha)=0$),$K[\alpha]=K(\alpha)$,故
\[
K[x]/\langle p\rangle\cong K(\alpha).
\]
同理可得
\[
\bar K[x]/(\bar p)\cong \bar K(\bar\alpha),
\]
记该同构为
\[
\bar\theta:\bar K[x]/(\bar p)\xrightarrow{\ \sim\ } \bar K(\bar\alpha),\qquad
\bar g(x)+(\bar p)\mapsto \bar g(\bar\alpha).
\]

\textbf{3.2 用 (1) 拼出所需同构 $\sigma$.}
定义
\[
\sigma:=\bar\theta\circ \eta\circ \theta^{-1}:K(\alpha)\longrightarrow \bar K(\bar\alpha).
\]
这是同构的复合,因而是域同构。

\textbf{验证 $\sigma|_K=\varphi$.}
对 $a\in K$,
\[
\sigma(a)=\bar\theta\Bigl(\eta\bigl(\theta^{-1}(a)\bigr)\Bigr)
=\bar\theta\Bigl(\eta\bigl(a+\langle p\rangle\bigr)\Bigr)
=\bar\theta\bigl(\varphi(a)+(\bar p)\bigr)=\varphi(a).
\]

\textbf{验证 $\sigma(\alpha)=\bar\alpha$.}
注意 $\theta(x+\langle p\rangle)=\alpha$,且 $\eta(x+\langle p\rangle)=x+(\bar p)$,再由 $\bar\theta(x+(\bar p))=\bar\alpha$,
故
\[
\sigma(\alpha)=\bar\theta\bigl(\eta(x+\langle p\rangle)\bigr)=\bar\theta\bigl(x+(\bar p)\bigr)=\bar\alpha.
\]
这就完成 (2)。

\bigskip
\textbf{4. 证明 (3):任何此类延拓都会把 $\alpha$ 送到 $\bar p$ 的根.}

设 $\psi:K(\alpha)\to \bar K(\bar\alpha)$ 为域同态,且 $\psi|_K=\varphi$。
由于 $p(\alpha)=0$,对等式施加 $\psi$:
\[
0=\psi\bigl(p(\alpha)\bigr).
\]
把 $p(x)=a_mx^m+\cdots+a_0$ 展开,则
\[
\psi\bigl(p(\alpha)\bigr)
=\psi(a_m)\psi(\alpha)^m+\cdots+\psi(a_0).
\]
而 $\psi(a_i)=\varphi(a_i)=\bar a_i$,故
\[
0=\bar a_m\psi(\alpha)^m+\cdots+\bar a_0=\bar p\bigl(\psi(\alpha)\bigr).
\]
因此 $\psi(\alpha)$ 是 $\bar p(x)$ 的一个根。

\bigskip
\textbf{5. 这一引理的“可操作版本”}
由 (2) 的构造可以推出一个非常实用的表达:
对任意 $g,h\in K[x]$ 且 $h(\alpha)\neq 0$,
\[
\sigma\!\left(\frac{g(\alpha)}{h(\alpha)}\right)
=\frac{\bar g(\bar\alpha)}{\bar h(\bar\alpha)}.
\]
其中 $\bar g,\bar h$ 是把系数通过 $\varphi$ 映到 $\bar K$ 后得到的多项式。
这就是“\emph{先搬系数,再把 $\alpha$ 换成 $\bar\alpha$}”。

\bigskip
\textbf{小结.}
(1) 给出商域层面的自然同构;
(2) 借助 $K(\alpha)\cong K[x]/\langle p\rangle$ 把同构延拓到单代数扩张,并且由 $\alpha\mapsto\bar\alpha$ 决定;
(3) 说明任何延拓都必须把 $\alpha$ 送到对应多项式 $\bar p$ 的根——这为后面“延拓个数 $\le \deg p$”的计数奠基。
\subsection{分裂域的最小性}
\textbf{定理(分裂域的最小性).}\;
设 $E/F$ 是 $f$ 的分裂域,则满足:

\textbf{(1) 生成性.}\;
存在 $f$ 在 $E$ 中的全部根 $\alpha_1,\dots,\alpha_n\in E$ 使
\[
f(x)=c\prod_{i=1}^n (x-\alpha_i)\quad (c\in F^\times),
\qquad
E=F(\alpha_1,\dots,\alpha_n).
\]

\textbf{(2) 最小性(包含意义下).}\;
若 $K/F$ 为任意扩域且 $f$ 在 $K$ 中分裂(即 $f$ 在 $K[x]$ 中分解为一次因式的乘积),
则存在 $F$-嵌入(同态且恒等于 $F$)
\[
\iota: E \hookrightarrow K,
\]
从而(在把 $E$ 视为其在 $K$ 中的像后)有 $E\subseteq K$。
特别地:\emph{$E$ 是所有使 $f$ 分裂的扩域中“最小”的一个}。

\textbf{(3) 交性质(等价表述).}\;
在某个代数闭包 $\Omega\supseteq F$ 中固定 $f$ 的一个分裂域 $E\subseteq \Omega$。
则
\[
E=\bigcap\Bigl\{\,K\subseteq \Omega \ \Bigm|\ F\subseteq K,\ f \text{ 在 }K\text{ 中分裂}\Bigr\}.
\]

\textbf{证明.}

\textbf{证明 (1).}\;
因为 $E$ 是分裂域,按定义 $f$ 在 $E$ 中分裂,故存在 $\alpha_1,\dots,\alpha_n\in E$ 使
\[
f(x)=c\prod_{i=1}^n(x-\alpha_i)\quad (c\in F^\times).
\]
另一方面,分裂域的定义还要求 $E$ 由这些根生成,即 $E$ 是包含 $F$ 且包含全部 $\alpha_i$ 的最小域;
这恰等价于
\[
E=F(\alpha_1,\dots,\alpha_n).
\]
故 (1) 成立。

\textbf{证明 (2).}\;
设 $K/F$ 为扩域且 $f$ 在 $K$ 中分裂。取 $f$ 在 $E$ 中的一组全体根 $\alpha_1,\dots,\alpha_n$。
因为 $f$ 在 $K$ 中也分裂,故在 $K$ 中也存在 $f$ 的全部根(记作 $\beta_1,\dots,\beta_n$,允许重排)。

由分裂域同构构造定理(逐个对应根的“延拓同态”)可构造出一个 $F$-同态
\[
\iota:E=F(\alpha_1,\dots,\alpha_n)\longrightarrow K
\]
使得 $\iota(\alpha_i)$ 是 $K$ 中的某个根(可通过递推:先把 $F$ 恒等嵌入 $K$,再将 $F(\alpha_1)$
嵌入到 $K$ 中使 $\alpha_1\mapsto \beta_{j_1}$,然后逐步延拓到 $F(\alpha_1,\alpha_2)$、$\dots$、直到 $E$)。
由于域同态要么为单射要么为零映射,而这里 $\iota(1)=1$,故 $\iota$ 非零,从而必为单射,即为嵌入。

于是 $E$ 的像 $\iota(E)$ 是 $K$ 的子域,并且包含 $F$ 与全部 $\iota(\alpha_i)$。
在通常的识别下(把 $E$ 视为其在 $K$ 中的同构像),可写作 $E\subseteq K$。
因此任意使 $f$ 分裂的扩域 $K$ 都“包含” $E$(至少包含一个与 $E$ 同构的 $F$-子域),这正是最小性。

\textbf{证明 (3).}\;
在代数闭包 $\Omega$ 中考虑所有满足 $F\subseteq K\subseteq \Omega$ 且 $f$ 在 $K$ 中分裂的域 $K$。
每个这样的 $K$ 必然包含 $f$ 的全部根,因此也包含 $F$ 与这些根生成的域 $F(\alpha_1,\dots,\alpha_n)=E$,
故对所有此类 $K$ 有 $E\subseteq K$,从而
\[
E \subseteq \bigcap K.
\]
反过来,显然 $E$ 自己也在这族 $K$ 中(因为 $f$ 在 $E$ 中分裂),因此交集必包含于 $E$:
\[
\bigcap K \subseteq E.
\]
两边合并即得 $\displaystyle E=\bigcap K$。

综上,分裂域确为使 $f$ 分裂的最小扩域。

\subsection{分裂域同构的构造的完整详细过程}
设 $f(x)\in F[x]$,$E,\bar E$ 为 $f$ 在 $F$ 上的两个分裂域。
在 $E$ 中取根 $\alpha_1,\dots,\alpha_n$ 使
\[
f(x)=\prod_{j=1}^n (x-\alpha_j),\qquad E=F(\alpha_1,\dots,\alpha_n).
\]
更详细地写出构造满足 $\sigma|_F=\mathrm{id}_F$ 的分裂域同构 $\sigma:E\to\bar E$ 的每一步。

\textbf{完整过程:}

\textbf{(1) 造塔并固定归纳目标.}\;
令
\[
F_0:=F,\qquad F_i:=F(\alpha_1,\dots,\alpha_i)=F_{i-1}(\alpha_i)\quad (i=1,\dots,n),
\]
则 $F = F_0 \subset F_1\subset F_2 \subset \dots F_n =E$。
我们要归纳构造一列域同态
\[
\sigma_i:F_i\longrightarrow \bar E\quad (i=0,1,\dots,n)
\]
满足
\[
\sigma_0=\mathrm{id}_F,\qquad \sigma_i|_{F_{i-1}}=\sigma_{i-1}\ (i\ge 1).
\]
最终取 $\sigma:=\sigma_n:E\to\bar E$。

\textbf{(2) 第 $0$ 步:起始同态.}\;
定义
\[
\sigma_0:=\mathrm{id}_F:F_0=F\to\bar E.
\]
显然 $\sigma_0(1)=1\neq 0$,故 $\sigma_0$ 非零。

\textbf{(3) 归纳步:从 $\sigma_{i-1}$ 构造 $\sigma_i$}\;

固定 $i\in\{1,\dots,n\}$,假设已经构造了
\[
\sigma_{i-1}:F_{i-1}\to\bar E.
\]
下面构造 $\sigma_i:F_i=F_{i-1}(\alpha_i)\to\bar E$。

\textbf{(3.1) 写出 $\alpha_i$ 在 $F_{i-1}$ 上的最小多项式.}\;
因为$f(x)\in F[x]$,且$f(\alpha_i)=0$,所以$\mathrm{Irr}(\alpha_i,F_{i-1})$一定存在。
令
\[
p_i(x):=\mathrm{Irr}(\alpha_i,F_{i-1})\in F_{i-1}[x],
\]
则 $p_i$ 首一不可约且
\[
p_i(\alpha_i)=0.
\]

\textbf{(3.2) 把 $p_i$ 的系数经 $\sigma_{i-1}$ 推到 $\bar E$(得到“对应多项式”).}\;
对 $p_i(x)=\sum_{k=0}^d a_k x^k$($a_k\in F_{i-1}$)定义
\[
\bar p_i(x):=\sigma_{i-1}(p_i(x)):=\sum_{k=0}^d \sigma_{i-1}(a_k)\,x^k\in \bar E[x].
\]
这是把 $F_{i-1}$ 中的系数用 $\sigma_{i-1}$ 送入 $\bar E$ 后得到的多项式。

\textbf{(3.3) 解释“像能否随意选”:必须选 $\bar p_i$ 的根.}\;
若 $\sigma_i$ 是 $\sigma_{i-1}$ 的延拓,则由同态性
\[
0=\sigma_i\bigl(p_i(\alpha_i)\bigr)=\sigma_i(p_i)\bigl(\sigma_i(\alpha_i)\bigr)=\bar p_i(\sigma_i(\alpha_i)\bigr).
\]
因此 $\sigma_i(\alpha_i)$ \emph{不能随便选},必须满足
\[
\bar p_i(\sigma_i(\alpha_i))=0,
\]
即 $\sigma_i(\alpha_i)$ 必须是 $\bar p_i$ 在 $\bar E$ 中的一个根(在根集合内可任选)。

\textbf{(3.4) 解释“为何一定找得到这样的根”:$\bar p_i$ 在 $\bar E$ 中分裂.}\;
因为 $\alpha_i$ 是 $f$ 的根,所以 $f(\alpha_i)=0$。
又 $f(x)\in F[x]\subseteq F_{i-1}[x]$,由最小多项式的整除性质得到
\[
p_i(x)\mid f(x)\quad \text{在 }F_{i-1}[x]\text{ 中成立}.
\]
将系数经 $\sigma_{i-1}$ 推过去,得
\[
\bar p_i(x)=\sigma_{i-1}(p_i(x))\mid \sigma_{i-1}(f(x))\quad \text{在 }\bar E[x]\text{ 中成立}.
\]
由于 $\sigma_{i-1}$ 在 $F$ 上恒等(归纳保证),而 $f$ 的系数在 $F$ 中,故
\[
\sigma_{i-1}(f(x))=f(x).
\]
从而
\[
\bar p_i(x)\mid f(x)\quad \text{在 }\bar E[x]\text{ 中成立}.
\]
但 $\bar E$ 是 $f$ 的分裂域,$f$ 在 $\bar E[x]$ 完全分裂,
因此它的任一因子(特别是 $\bar p_i$)也在 $\bar E$ 中分裂,
于是可以取到 $\beta_i\in\bar E$ 使
\[
\bar p_i(\beta_i)=0.
\]

\textbf{(3.5) 用 $\beta_i$ 具体构造延拓:先构造多项式环上的代入同态.}\;
定义
\[
\Phi_i:F_{i-1}[x]\to \bar E,\qquad
\Phi_i\Bigl(\sum_{k=0}^m a_k x^k\Bigr):=\sum_{k=0}^m \sigma_{i-1}(a_k)\,\beta_i^k.
\]
则 $\Phi_i$ 是环同态(系数经 $\sigma_{i-1}$,再代入 $x=\beta_i$),并且对任意 $a\in F_{i-1}$,
\[
\Phi_i(a)=\sigma_{i-1}(a).
\]

\textbf{(3.6) 计算核:证明 $\ker\Phi_i=(p_i)$.}\;
由 $p_i(\alpha_i)=0$ 与 $\bar p_i(\beta_i)=0$ 得
\[
\Phi_i(p_i)=\bar p_i(\beta_i)=0,
\]
故 $(p_i)\subseteq\ker\Phi_i$。
又 $\Phi_i(1)=1\neq 0$,故 $\ker\Phi_i$ 是真理想。
由于 $p_i$ 在 $F_{i-1}[x]$ 中不可约,理想 $(p_i)$ 为极大理想,
因此夹在 $(p_i)\subseteq\ker\Phi_i\subsetneq F_{i-1}[x]$ 之间只能有
\[
\ker\Phi_i=(p_i).
\]

\textbf{(3.7) 下放到商环并得到单射.}\;
由 $(p_i)=\ker\Phi_i$,$\Phi_i$ 诱导出单射同态
\[
\widetilde\Phi_i:F_{i-1}[x]/(p_i)\hookrightarrow \bar E,\qquad
\widetilde\Phi_i\bigl(g(x)+(p_i)\bigr):=\Phi_i(g).
\]
并且
\[
\widetilde\Phi_i\bigl(x+(p_i)\bigr)=\Phi_i(x)=\beta_i,\qquad
\widetilde\Phi_i\bigl(a+(p_i)\bigr)=\Phi_i(a)=\sigma_{i-1}(a)\ (a\in F_{i-1}).
\]

\textbf{(3.8) 识别 $F_{i-1}[x]/(p_i)\cong F_{i-1}(\alpha_i)=F_i$(写出同构与对应).}\;
定义评价同态
\[
\theta_i:F_{i-1}[x]\to F_{i-1}(\alpha_i),\qquad \theta_i(g)=g(\alpha_i).
\]
则 $\theta_i$ 是环同态,且
\[
\ker\theta_i=(p_i)
\]
(因为 $g(\alpha_i)=0\iff p_i\mid g$)。
由同态基本定理得到同构
\[
\overline\theta_i:F_{i-1}[x]/(p_i)\stackrel{\cong}{\longrightarrow} F_{i-1}(\alpha_i)=F_i,
\qquad
\overline\theta_i\bigl(g(x)+(p_i)\bigr)=g(\alpha_i).
\]
特别地,
\[
\overline\theta_i\bigl(x+(p_i)\bigr)=\alpha_i,\qquad
\overline\theta_i\bigl(a+(p_i)\bigr)=a\ (a\in F_{i-1}).
\]

\textbf{(3.9) 合成得到所求延拓 $\sigma_i$(并验证两条要求).}\;
令
\[
\sigma_i:=\widetilde\Phi_i\circ \overline\theta_i^{-1}:F_i=F_{i-1}(\alpha_i)\to\bar E.
\]
则对任意 $a\in F_{i-1}$,
\[
\sigma_i(a)=\widetilde\Phi_i\bigl(\overline\theta_i^{-1}(a)\bigr)=\widetilde\Phi_i\bigl(a+(p_i)\bigr)=\sigma_{i-1}(a),
\]
故 $\sigma_i|_{F_{i-1}}=\sigma_{i-1}$。
并且
\[
\sigma_i(\alpha_i)=\widetilde\Phi_i\bigl(\overline\theta_i^{-1}(\alpha_i)\bigr)
=\widetilde\Phi_i\bigl(x+(p_i)\bigr)=\beta_i.
\]

\textbf{(3.10) 解释“延拓唯一”:为何一旦选定 $\beta_i$ 就只有这一种 $\sigma_i$?}\;

设 $\tau:F_i\to\bar E$ 也是域同态,满足
\[
\tau|_{F_{i-1}}=\sigma_{i-1},\qquad \tau(\alpha_i)=\beta_i.
\]
任取 $z\in F_i=F_{i-1}(\alpha_i)$,存在 $g,h\in F_{i-1}[x]$ 使
\[
z=\frac{g(\alpha_i)}{h(\alpha_i)},\qquad h(\alpha_i)\neq 0.
\]
则
\[
\tau(z)=\frac{\tau(g(\alpha_i))}{\tau(h(\alpha_i))}
=\frac{\sigma_{i-1}(g)(\beta_i)}{\sigma_{i-1}(h)(\beta_i)}
=\sigma_i(z),
\]
故 $\tau=\sigma_i$,唯一性得证。

\textbf{(4) 完成构造:定义最终同态.}\;
按 (3) 的归纳步骤对 $i=1,2,\dots,n$ 依次构造出
\[
\sigma_1,\sigma_2,\dots,\sigma_n,
\]
并令
\[
\sigma:=\sigma_n:E=F_n\to\bar E.
\]
由构造知 $\sigma|_F=\mathrm{id}_F$,且每一步都体现了“对应根”:在第 $i$ 步选取
\[
\beta_i\in\bar E\ \text{为 }\bar p_i(x)=\sigma_{i-1}(p_i(x))\ \text{的一个根,并规定 }\sigma_i(\alpha_i)=\beta_i.
\]

\textbf{(5) 证明 $\sigma$ 是同构(单射与满射分别写出).}\;

\textbf{(5.1) 单射.}\;
$\sigma$ 为域同态,所以为单射。

\textbf{(5.2) 满射.}\;
在 $E$ 中有 $f(x)=\prod_{j=1}^n (x-\alpha_j)$,且$f(x) \in F[x]$,
因 $\sigma|_F=\mathrm{id}_F$,将 $\sigma$ 作用到等式系数并保持乘法得
\[
f(x)=\sigma(f(x))=\prod_{j=1}^n (x-\sigma(\alpha_j))\quad \text{在 }\bar E[x]\text{ 中成立}.
\]
故 $f$ 在$\bar E$的子域 $\sigma(E)$ 中分裂,且 $F\subseteq \sigma(E)\subseteq \bar E$。
由分裂域的最小性($\bar E$ 是包含 $F$ 且使 $f$ 分裂的最小域)可得
\[
\bar E\subseteq \sigma(E).
\]
于是 $\sigma(E)=\bar E$,从而 $\sigma$ 满射。

综上,$\sigma$ 为双射域同态,故 $\sigma:E\to\bar E$ 为 $F$-同构。

\textbf{(6) 同构个数的上界(每一步的“可选根数”相乘).}\;
第 $i$ 步可选的 $\beta_i$ 个数等于 $\bar p_i$ 在 $\bar E$ 中的不同根数,故
\[
\#\{\text{第 $i$ 步的可选 }\beta_i\}\le \deg(\bar p_i)=\deg(p_i)=[F_i:F_{i-1}].
\]
于是
\[
\#\{\sigma:E\to\bar E\mid \sigma|_F=\mathrm{id}_F\}
\le \prod_{i=1}^n [F_i:F_{i-1}]
=[E:F].
\]
等号成立当且仅当每个 $\bar p_i$ 在 $\bar E$ 中无重根(从而根数达到次数上界),
等价地,当且仅当 $f$ 的不可约因子在$E$中无重根。



\subsection{$x^n-d$的分裂域}
\textbf{命题.}\;
设 $F$ 为一域,$n\in\mathbb{N}$,$d\in F$ 且 $d\neq 0$。
在某个代数闭包 $\overline F$ 中取 $x^n-d$ 的一根 $\alpha$(即 $\alpha^n=d$),再取一个本原 $n$ 次单位根 $\zeta_n$(即 $\zeta_n^n=1$ 且其阶为 $n$)。
证明:多项式
\[
f(x)=x^n-d\in F[x]
\]
在 $F$ 上的分裂域为
\[
\mathrm{Spl}_F(f)=F(\alpha,\zeta_n),
\]
并且该域与所选根 $\alpha$ 无关:若 $\alpha'$ 也是 $f$ 的一根,则
\[
F(\alpha',\zeta_n)=F(\alpha,\zeta_n).
\]

\textbf{答案:}

\textbf{(1) $x^n-d$ 的全部根形状.}\;
在 $\overline F$ 中,$f(x)=x^n-d$ 的根集记为 $\Omega$。
因 $d\neq 0$,故 $\alpha\neq 0$。
对任意整数 $k$,有
\[
(\alpha\zeta_n^k)^n=\alpha^n(\zeta_n^k)^n=d\cdot 1=d,
\]
所以 $\alpha\zeta_n^k\in\Omega$。
反过来,若 $\beta\in\Omega$,则 $\beta^n=d=\alpha^n$,于是
\[
\left(\frac{\beta}{\alpha}\right)^n=\frac{\beta^n}{\alpha^n}=1.
\]
因此 $u:=\beta/\alpha$ 是一个 $n$ 次单位根,记为 $u\in\mu_n:=\{u\in\overline F:\ u^n=1\}$。
又因为 $\zeta_n$ 为本原 $n$ 次单位根,$\mu_n=\{1,\zeta_n,\dots,\zeta_n^{n-1}\}$,
从而存在 $k\in\{0,1,\dots,n-1\}$ 使 $u=\zeta_n^k$,于是
\[
\beta=\alpha u=\alpha\zeta_n^k.
\]
综上,
\[
\Omega=\{\alpha\zeta_n^k:\ k=0,1,\dots,n-1\}.
\]

\textbf{(2) 证明 $f$ 在 $F(\alpha,\zeta_n)$ 中分裂(充分性).}\;
由 (1) 得 $f$ 的全部根均属于 $F(\alpha,\zeta_n)$,因为对每个 $k$,
\[
\alpha\zeta_n^k\in F(\alpha,\zeta_n).
\]
因此在 $F(\alpha,\zeta_n)[x]$ 中
\[
f(x)=\prod_{k=0}^{n-1}\bigl(x-\alpha\zeta_n^k\bigr),
\]
即 $f$ 在 $F(\alpha,\zeta_n)$ 中完全分解为一次因子。
故 $F(\alpha,\zeta_n)$ 是 $f$ 的一个分裂域的候选扩域。

\textbf{(3) 证明最小性:任何分裂域都包含 $F(\alpha,\zeta_n)$.}\;
设 $K/F$ 为任意扩域,使得 $f$ 在 $K$ 中分裂。
则 $K$ 包含 $f$ 的全部根,特别地包含某个根(我们选取的)$\alpha$,故 $\alpha\in K$。
同时,$K$ 还包含另一个根 $\alpha\zeta_n$(因为 (1) 表明它确实是根),从而
\[
\zeta_n=\frac{\alpha\zeta_n}{\alpha}\in K
\quad(\text{因 }\alpha\neq 0).
\]
因此 $K$ 同时包含 $\alpha$ 与 $\zeta_n$,于是
\[
F(\alpha,\zeta_n)\subseteq K.
\]
这说明在所有使 $f$ 分裂的扩域中,$F(\alpha,\zeta_n)$ 是最小的那个,
即
\[
\mathrm{Spl}_F(f)=F(\alpha,\zeta_n).
\]

\textbf{(4) 与所选根 $\alpha$ 无关.}\;
设 $\alpha'$ 也是 $f$ 的一根,则 $(\alpha')^n=d=\alpha^n$,
故
\[
\left(\frac{\alpha'}{\alpha}\right)^n=1.
\]
因此存在 $j\in\{0,1,\dots,n-1\}$ 使 $\alpha'/\alpha=\zeta_n^j$,即
\[
\alpha'=\alpha\zeta_n^j.
\]
于是
\[
F(\alpha',\zeta_n)=F(\alpha\zeta_n^j,\zeta_n)=F(\alpha,\zeta_n),
\]
因为 $F(\alpha,\zeta_n)$ 显然包含 $\alpha\zeta_n^j$,反过来它也包含 $\alpha'$
且又含 $\zeta_n$,两者互相包含即相等。

\textbf{(5) 结论.}\;
综上,$x^n-d$ 在 $F$ 上的分裂域为 $F(\alpha,\zeta_n)$,且该域与根的选取无关。

\clearpage 
\subsection*{课后习题答案}
\addcontentsline{toc}{subsection}{\textcolor{red}{课后习题答案}}
\begin{enumerate}[label=\textcolor{blue}{\textbf{\large\arabic*.}}]
	\item 
	\textbf{题目.}
	求下列 $\mathbb{Q}[x]$ 中多项式的分裂域及其分裂域的 $\mathbb{Q}$-自同构个数:
	
	(1)\;$x^2+3$;\quad
	(2)\;$x^5-1$;\quad
	(3)\;$x^3-2$;\quad
	(4)\;$(x^2-2)(x^3-2)$;\quad
	(5)\;$x^5-3.$
	
	\textbf{解:}
	说明:在特征 $0$ 上一切多项式可分,分裂域扩张 $E/\mathbb{Q}$ 为 Galois,
	故
	\[
	|\mathrm{Aut}_{\mathbb{Q}}(E)|=|\mathrm{Gal}(E/\mathbb{Q})|=[E:\mathbb{Q}].
	\]
	因此关键是:写出分裂域并计算其次数。
	
	\bigskip
	\textbf{(1) $f(x)=x^2+3$.}
	根为 $\pm\sqrt{-3}$。分裂域
	\[
	E=\mathbb{Q}(\sqrt{-3}).
	\]
	显然 $[\mathbb{Q}(\sqrt{-3}):\mathbb{Q}]=2$,故
	\[
	|\mathrm{Aut}_{\mathbb{Q}}(E)|=2.
	\]
	(两个自同构:$\sqrt{-3}\mapsto \sqrt{-3}$ 或 $\sqrt{-3}\mapsto -\sqrt{-3}$。)
	
	\bigskip
	\textbf{(2) $f(x)=x^5-1$.}
	全部根为 $1,\zeta_5,\zeta_5^2,\zeta_5^3,\zeta_5^4$,其中 $\zeta_5=e^{2\pi i/5}$ 为本原 $5$ 次单位根。
	分裂域
	\[
	E=\mathbb{Q}(\zeta_5).
	\]
	由圆分多项式理论,$[\mathbb{Q}(\zeta_5):\mathbb{Q}]=\varphi(5)=4$,故
	\[
	|\mathrm{Aut}_{\mathbb{Q}}(E)|=4.
	\]
	并且任一 $\mathbb{Q}$-自同构由
	\[
	\zeta_5\longmapsto \zeta_5^a\quad (a\in(\mathbb{Z}/5\mathbb{Z})^\times=\{1,2,3,4\})
	\]
	唯一确定。
	
	\bigskip
	\textbf{(3) $f(x)=x^3-2$.}
	令 $\alpha=\sqrt[3]{2}$,$\omega=\dfrac{-1+\sqrt{-3}}{2}$(本原三次单位根)。
	三个根为 $\alpha,\alpha\omega,\alpha\omega^2$,故分裂域
	\[
	E=\mathbb{Q}(\alpha,\omega).
	\]
	先由 Eisenstein 判别(素数 $2$)知 $x^3-2$ 在 $\mathbb{Q}[x]$ 不可约,故
	\[
	[\mathbb{Q}(\alpha):\mathbb{Q}]=3.
	\]
	又 $\mathbb{Q}(\omega)=\mathbb{Q}(\sqrt{-3})$ 为二次扩张,且 $\mathbb{Q}(\alpha)\subset\mathbb{R}$ 为实域,
	而 $\omega\notin\mathbb{R}$,因此 $\mathbb{Q}(\alpha)\cap\mathbb{Q}(\omega)=\mathbb{Q}$,
	从而(线性无交/次数相乘)
	\[
	[E:\mathbb{Q}]=[\mathbb{Q}(\alpha,\omega):\mathbb{Q}]
	=[\mathbb{Q}(\alpha):\mathbb{Q}]\,[\mathbb{Q}(\omega):\mathbb{Q}]=3\cdot 2=6.
	\]
	故
	\[
	|\mathrm{Aut}_{\mathbb{Q}}(E)|=6.
	\]
	(事实上 $\mathrm{Gal}(E/\mathbb{Q})\cong S_3$:可把 $\alpha$ 送到 $\alpha,\alpha\omega,\alpha\omega^2$,
	并把 $\omega$ 送到 $\omega$ 或 $\omega^2$,共 $3\cdot 2=6$ 个。)
	
	\bigskip
	\textbf{(4) $f(x)=(x^2-2)(x^3-2)$.}
	$x^2-2$ 的分裂域为 $\mathbb{Q}(\sqrt2)$;
	$x^3-2$ 的分裂域为 $L=\mathbb{Q}(\alpha,\omega)$(同 (3))。
	因此 $f$ 的分裂域是二者的合成域:
	\[
	E=\mathbb{Q}(\sqrt2,\alpha,\omega).
	\]
	计算次数:$[\mathbb{Q}(\sqrt2):\mathbb{Q}]=2$,$[L:\mathbb{Q}]=6$。
	注意到 $L$ 的二次子域唯一(对应 $A_3$ 的固定域)且为 $\mathbb{Q}(\sqrt{-3})$(因为 $L$ 含 $\omega$),
	而 $\mathbb{Q}(\sqrt2)$ 是实二次域,不可能等于 $\mathbb{Q}(\sqrt{-3})$,
	故
	\[
	\mathbb{Q}(\sqrt2)\cap L=\mathbb{Q}.
	\]
	于是
	\[
	[E:\mathbb{Q}]=[\mathbb{Q}(\sqrt2)L:\mathbb{Q}]
	=[\mathbb{Q}(\sqrt2):\mathbb{Q}]\,[L:\mathbb{Q}]=2\cdot 6=12,
	\]
	从而
	\[
	|\mathrm{Aut}_{\mathbb{Q}}(E)|=12.
	\]
	(并且 $\mathrm{Gal}(E/\mathbb{Q})\cong C_2\times S_3$,阶为 $12$。)
	
	\bigskip
	\textbf{(5) $f(x)=x^5-3$.}
	令 $\beta=\sqrt[5]{3}$,$\zeta_5$ 为本原 $5$ 次单位根。
	全部根为 $\beta\zeta_5^k\ (k=0,1,2,3,4)$,故分裂域
	\[
	E=\mathbb{Q}(\beta,\zeta_5).
	\]
	由 Eisenstein 判别(素数 $3$)知 $x^5-3$ 在 $\mathbb{Q}[x]$ 不可约,故
	\[
	[\mathbb{Q}(\beta):\mathbb{Q}]=5.
	\]
	又 $[\mathbb{Q}(\zeta_5):\mathbb{Q}]=\varphi(5)=4$。
	由于 $[\mathbb{Q}(\beta):\mathbb{Q}]=5$ 为素数,
	$\mathbb{Q}(\beta)$ 的子域只有 $\mathbb{Q}$ 与自身;
	而 $\mathbb{Q}(\zeta_5)$ 的次数是 $4$,故
	\[
	\mathbb{Q}(\beta)\cap\mathbb{Q}(\zeta_5)=\mathbb{Q}.
	\]
	于是
	\[
	[E:\mathbb{Q}]=[\mathbb{Q}(\beta,\zeta_5):\mathbb{Q}]
	=[\mathbb{Q}(\beta):\mathbb{Q}]\,[\mathbb{Q}(\zeta_5):\mathbb{Q}]
	=5\cdot 4=20,
	\]
	从而
	\[
	|\mathrm{Aut}_{\mathbb{Q}}(E)|=20.
	\]
	(并且 $\mathrm{Gal}(E/\mathbb{Q})\cong C_5\rtimes C_4$,阶为 $20$。)
	
	\bigskip
	\textbf{汇总:}
	\[
	\begin{array}{c|c|c}
		f(x) & \text{分裂域 }E & |\mathrm{Aut}_{\mathbb{Q}}(E)|=[E:\mathbb{Q}]\\
		\hline
		x^2+3 & \mathbb{Q}(\sqrt{-3}) & 2\\
		x^5-1 & \mathbb{Q}(\zeta_5) & 4\\
		x^3-2 & \mathbb{Q}(\sqrt[3]{2},\omega) & 6\\
		(x^2-2)(x^3-2) & \mathbb{Q}(\sqrt2,\sqrt[3]{2},\omega) & 12\\
		x^5-3 & \mathbb{Q}(\sqrt[5]{3},\zeta_5) & 20
	\end{array}
	\]
	
	\item 3.
	\textbf{题目.}\;
	设 $F$ 为有限域,试证明一定存在 $F$ 的一个代数扩张 $E$ 使得 $E\neq F$。
	
	\textbf{解:}\;
	设 $|F|=q$(其中 $q=p^n\ge 2$)。考虑所有首一二次多项式
	\[
	x^2+ax+b\qquad (a,b\in F).
	\]
	显然这样的多项式一共有 $q^2$ 个(由 $a,b$ 的取值决定)。
	
	下面估计其中\textbf{可约}的首一二次多项式数量。
	若 $x^2+ax+b$ 在 $F[x]$ 中可约,则它必须在 $F$ 中有根(因为次数为 $2$),于是存在 $c,d\in F$ 使得
	\[
	x^2+ax+b=(x-c)(x-d).
	\]
	反过来,任意 $c,d\in F$ 都给出一个首一可约二次多项式。注意到
	\[
	(x-c)(x-d)=(x-d)(x-c),
	\]
	因此不同的有序对 $(c,d)$ 可能对应同一个多项式;为了给出\emph{上界},我们只需数无序对 $\{c,d\}$ 的个数即可。
	无序对 $\{c,d\}$(允许 $c=d$)的数量为
	\[
	\binom{q}{2}+q=\frac{q(q-1)}{2}+q=\frac{q(q+1)}{2}.
	\]
	因此首一可约二次多项式的个数至多为 $\dfrac{q(q+1)}{2}$。
	
	于是首一\textbf{不可约}二次多项式的个数至少为
	\[
	q^2-\frac{q(q+1)}{2}=\frac{q(q-1)}{2}>0.
	\]
	所以在 $F[x]$ 中存在一个首一不可约二次多项式 $f(x)$。
	
	取
	\[
	E:=F[x]/(f(x)).
	\]
	因为 $f(x)$ 在 $F[x]$ 中不可约,故理想 $(f(x))$ 极大,从而商环 $E$ 是域;并且在 $E$ 中 $x+(f)$ 是 $f$ 的一个根,所以 $E/F$ 是代数扩张(由一个代数元生成,即$E = F(\alpha)$)。
	
	又因 $\deg f=2$,作为 $F$-向量空间,$E$ 的维数为 $2$,因此
	\[
	|E|=q^2\neq q=|F|.
	\]
	从而 $E\neq F$。
	
	综上,必存在 $F$ 的一个代数扩张 $E$ 使得 $E\ne F$。
	
	\item 5.
	\textbf{题目.}\;
	设 $F$ 是特征不为 $2$ 的域,证明:$F$ 的每个二次扩张都可写成 $F(\alpha)$,且 $\alpha^2\in F$。
	如果 $\mathrm{Ch}\,F=2$,结论是否成立?
	
	\textbf{解:}\;
	
	\textbf{(1) $\mathrm{Ch}\,F\neq 2$ 时结论成立.}
	
	设 $K/F$ 为二次扩张,即 $[K:F]=2$。取 $\alpha\in K\setminus F$,则 $K=F(\alpha)$ 且 $\alpha$ 在 $F$ 上的最小多项式 $m_\alpha(x)$ 的次数为 $2$。
	因此
	\[
	m_\alpha(x)=x^2-tx+u \quad (t,u\in F).
	\]
	令
	\[
	\beta=\alpha-\frac{t}{2}\in K.
	\]
	注意到 $\frac{1}{2}\in F$(因为 $\mathrm{Ch}\,F\neq 2$),并计算
	\[
	\beta^2
	=\left(\alpha-\frac{t}{2}\right)^2
	=\alpha^2-t\alpha+\frac{t^2}{4}.
	\]
	由 $m_\alpha(\alpha)=0$ 得 $\alpha^2=t\alpha-u$,代回上式:
	\[
	\beta^2=(t\alpha-u)-t\alpha+\frac{t^2}{4}=-u+\frac{t^2}{4}\in F.
	\]
	又因为 $\beta=\alpha-\frac{t}{2}$ 与 $\alpha$ 相差域内元素,所以
	\[
	F(\beta)=F(\alpha)=K.
	\]
	于是 $K=F(\beta)$ 且 $\beta^2\in F$,从而任意二次扩张都可写成 $F(\alpha)$ 且满足 $\alpha^2\in F$。
	
	\textbf{(2) $\mathrm{Ch}\,F=2$ 时结论不一定成立.}
	
	给出反例:令 $F=\mathbb{F}_2(t)$,取 $\alpha$ 为多项式
	\[
	x^2+x+t\in F[x]
	\]
	的一个根,令 $K=F(\alpha)$。因为
	\[
	(x^2+x+t)'=2x+1=1\neq 0,
	\]
	该多项式在 $\mathrm{Ch}\,F=2$ 下可分离,且它在 $F$ 中无根(否则应有 $b^2+b=t$,不可能在有理函数域中成立),故它不可约,从而 $[K:F]=2$。
	
	下面证明:在此 $K/F$ 中,不存在 $\beta\in K\setminus F$ 使得 $\beta^2\in F$。
	任取 $\beta\in K$,可唯一写成
	\[
	\beta=u+v\alpha,\quad u,v\in F.
	\]
	若 $\beta\notin F$,则 $v\neq 0$。又由 $\alpha^2+\alpha=t$ 得
	\[
	\alpha^2=t+\alpha.
	\]
	利用 $\mathrm{Ch}\,F=2$ 下的 $(a+b)^2=a^2+b^2$,计算
	\[
	\beta^2=(u+v\alpha)^2=u^2+v^2\alpha^2
	=u^2+v^2(t+\alpha)=(u^2+v^2t)+v^2\alpha.
	\]
	若 $\beta^2\in F$,则其 $\alpha$ 系数必须为 $0$,即 $v^2=0$,从而 $v=0$,与 $v\neq 0$ 矛盾。
	因此该二次扩张 $K/F$ 不能写成 $F(\gamma)$ 且 $\gamma^2\in F$ 的形式。
	
	\textbf{补充说明.}\;
	在 $\mathrm{Ch}\,F=2$ 时,若二次扩张是纯不可分的(例如由 $x^2-a$ 生成),则确实可以取生成元 $\gamma$ 满足 $\gamma^2\in F$;
	但存在上面的可分二次扩张(Artin--Schreier 型)不满足该结论。
	
	
\end{enumerate}
\clearpage
\section{Galois群}
\subsection{有限Galois扩张的Galois群 $\mathrm{Gal}(K/F)$同构于某个对称群 $S_n$ 的子群}
\textbf{命题.}\;
设 $f(x)\in F[x]$,$E$ 为 $f$ 在 $F$ 上的分裂域,
并设 $f$ 在 $E$ 中的互异根集合为
\[
\Omega=\{\alpha_1,\dots,\alpha_n\}\subset E.
\]
则存在自然的群同态
\[
\rho:\mathrm{Gal}(E/F)\longrightarrow \mathrm{Sym}(\Omega),
\qquad
\rho(\sigma)=\sigma|_{\Omega},
\]
并且满足:

\textbf{(1)(作用于根集)}\;
对任意 $\sigma\in\mathrm{Gal}(E/F)$ 与任意 $\alpha\in\Omega$,都有 $\sigma(\alpha)\in\Omega$;
因此 $\rho$ 定义良好,亦即 $\mathrm{Gal}(E/F)$ 作为一个置换群作用在根集 $\Omega$ 上。

\textbf{(2)(忠实性/单射性)}\;
$\rho$ 是单射。换言之,若 $\sigma,\tau\in\mathrm{Gal}(E/F)$ 满足
\[
\sigma(\alpha)=\tau(\alpha)\quad (\forall\,\alpha\in\Omega),
\]
则必有 $\sigma=\tau$。
特别地,不同的 Galois 元素诱导的根置换一定不同。

\textbf{证明.}\;
(1) 若 $\alpha\in\Omega$,则 $f(\alpha)=0$。由于 $\sigma|_F=\mathrm{id}_F$,
有
\[
f(\sigma(\alpha))=\sigma(f(\alpha))=\sigma(0)=0,
\]
故 $\sigma(\alpha)$ 仍为 $f$ 的根,即 $\sigma(\alpha)\in\Omega$,从而 $\rho$ 定义良好。

(2) 由分裂域的定义,$E=F(\alpha_1,\dots,\alpha_n)$。
若 $\sigma,\tau\in\mathrm{Gal}(E/F)$ 在 $\Omega$ 上一致,则在生成元 $\alpha_1,\dots,\alpha_n$ 上一致,
且二者在 $F$ 上恒等,于是对任意 $x\in E$(可表示为 $F$-系数有理式/多项式在这些根上的取值)
都有 $\sigma(x)=\tau(x)$,从而 $\sigma=\tau$。因此 $\rho$ 单射。

\subsection{Galois 群 $\mathrm{Gal}(K/F)$同构于某个对称群 $S_n$ 的子群}
\textbf{命题.}\;设 $K/F$ 为有限 Galois 扩张,则其 Galois 群 $\mathrm{Gal}(K/F)$
\emph{天然同构}于某个对称群 $S_n$ 的子群(即可看作一群置换)。

\textbf{定理(Galois 群作为根的置换群).}\;
设 $K/F$ 为有限 Galois 扩张。取 $K$ 中元素 $\alpha_1,\dots,\alpha_m$ 使
\[
K=F(\alpha_1,\dots,\alpha_m).
\]
令
\[
f_i(x):=\mathrm{Irr}(\alpha_i,F)\in F[x]\quad (i=1,\dots,m),
\]
并令 $R$ 为这些多项式在 $K$ 中的\emph{全部不同根}的集合:
\[
R:=\bigcup_{i=1}^m \{\beta\in K\mid f_i(\beta)=0\}.
\]
则 $R$ 为有限集,记 $|R|=n$。定义映射
\[
\Phi:\mathrm{Gal}(K/F)\longrightarrow \mathrm{Sym}(R)\cong S_n,\qquad
\Phi(\sigma):=\sigma|_{R}.
\]
则:
\textbf{(1)} $\Phi$ 是群同态,且 $\Phi(\mathrm{Gal}(K/F))\le \mathrm{Sym}(R)\cong S_n$;

\textbf{(2)} $\Phi$ 为单射,从而 $\mathrm{Gal}(K/F)\cong \Phi(\mathrm{Gal}(K/F))\le S_n$。

特别地,$\mathrm{Gal}(K/F)$ 可自然视为 $R$ 上的一个置换群。

\textbf{证明.}

\textbf{第一步:$R$ 在 Galois 群作用下封闭,从而 $\sigma|_R$ 是置换.}\;
任取 $\sigma\in \mathrm{Gal}(K/F)$。
对任意 $\beta\in R$,存在某个 $i$ 使 $f_i(\beta)=0$。
因为 $f_i\in F[x]$ 且 $\sigma|_F=\mathrm{id}_F$,故系数不变,从而 $\sigma(f_i)=f_i$。
于是
\[
0=\sigma\!\bigl(f_i(\beta)\bigr)=f_i\bigl(\sigma(\beta)\bigr),
\]
即 $\sigma(\beta)$ 仍是 $f_i$ 的根,故 $\sigma(\beta)\in R$。
因此 $\sigma$ 将有限集 $R$ 映到自身,$\sigma|_R$ 是 $R$ 上的一个双射,
即属于 $\mathrm{Sym}(R)$。

\textbf{第二步:$\Phi$ 是群同态.}\;
对 $\sigma,\tau\in \mathrm{Gal}(K/F)$,
有
\[
\Phi(\sigma\tau)=(\sigma\tau)|_R=\sigma|_R\circ \tau|_R=\Phi(\sigma)\circ \Phi(\tau),
\]
故 $\Phi$ 为群同态。

\textbf{第三步:$\Phi$ 单射.}\;
若 $\Phi(\sigma)=\mathrm{id}$,即 $\sigma$ 在 $R$ 上恒等,则特别地对每个生成元 $\alpha_i$ 有
$\sigma(\alpha_i)=\alpha_i$(因为 $\alpha_i\in R$:它是 $f_i$ 的根)。
因此 $\sigma$ 在 $F$ 与所有 $\alpha_i$ 上恒等,
而 $K=F(\alpha_1,\dots,\alpha_m)$ 表明 $K$ 中任意元素均可由 $F$ 与 $\alpha_i$ 经有限次域运算得到,
域同态保持这些运算,故 $\sigma$ 在整个 $K$ 上恒等。
于是
\[
\ker \Phi=\{\mathrm{id}\},
\]
故 $\Phi$ 单射。

\textbf{第四步:结论.}\;
由第三步知 $\mathrm{Gal}(K/F)\cong \Phi(\mathrm{Gal}(K/F))$,
而 $\Phi(\mathrm{Gal}(K/F))\le \mathrm{Sym}(R)\cong S_n$。
因此
\[
\mathrm{Gal}(K/F)\cong H\le S_n
\]
对某个 $n$ 成立,证毕。

\bigskip

\textbf{推论(多项式分裂域情形的常用表述).}\;
设 $f(x)\in F[x]$ 为可分多项式,$K$ 为其分裂域,
令 $R$ 为 $f$ 在 $K$ 中的全部不同根集合,$|R|=n$。
则存在自然单射
\[
\mathrm{Gal}(K/F)\hookrightarrow \mathrm{Sym}(R)\cong S_n,
\]
即 $\mathrm{Gal}(K/F)$ 可视为 $n$ 个根的置换群。
\subsection{$F$-域同态的唯一性}
\textbf{定理($F$-域同态的唯一性 / 生成元决定同态).}

设 $E/F$ 是域扩张,且
\[
E = F(a_1,\dots,a_n).
\]
若 $\sigma,\tau : E \to K$ 是两个 $F$-域同态,并且
\[
\sigma(a_i)=\tau(a_i)\qquad (i=1,\dots,n),
\]
则
\[
\sigma=\tau.
\]

\textbf{证明.}
任取 $x\in E$。由于 $E=F(a_1,\dots,a_n)$,存在多项式
\[
f,g\in F[X_1,\dots,X_n],\quad g(a_1,\dots,a_n)\neq 0,
\]
使得
\[
x=\frac{f(a_1,\dots,a_n)}{g(a_1,\dots,a_n)}.
\]
因为 $\sigma,\tau$ 均为 $F$-域同态,且在 $a_1,\dots,a_n$ 上取值相同,有
\[
\sigma(x)
=\frac{f(\sigma(a_1),\dots,\sigma(a_n))}
{g(\sigma(a_1),\dots,\sigma(a_n))}
=\frac{f(\tau(a_1),\dots,\tau(a_n))}
{g(\tau(a_1),\dots,\tau(a_n))}
=\tau(x).
\]
由于 $x$ 任意,故 $\sigma=\tau$。




\clearpage 
\subsection*{课后习题答案}
\addcontentsline{toc}{subsection}{\textcolor{red}{课后习题答案}}
\begin{enumerate}[label=\textcolor{blue}{\textbf{\large\arabic*.}}]
	\item \textbf{题目.}\;
	试求 $\mathrm{Gal}\bigl(\mathbb Q(\sqrt2+\sqrt3)/\mathbb Q\bigr)$。
	
	\textbf{答案:}
	
	设
	\[
	\alpha:=\sqrt2+\sqrt3,\qquad K:=\mathbb Q(\alpha).
	\]
	
	\textbf{(1) 先确定 $K=\mathbb Q(\sqrt2,\sqrt3)$,并计算次数.}\;
	计算
	\[
	\alpha^2=(\sqrt2+\sqrt3)^2=5+2\sqrt6
	\quad\Longrightarrow\quad
	\sqrt6=\frac{\alpha^2-5}{2}\in K.
	\]
	又因为 $\sqrt2,\sqrt3$ 满足
	\[
	t^2-\alpha\,t+\sqrt6=0,
	\]
	其判别式为
	\[
	\Delta=\alpha^2-4\sqrt6=(5+2\sqrt6)-4\sqrt6=5-2\sqrt6=(\sqrt3-\sqrt2)^2,
	\]
	故在 $K$ 中
	\[
	\sqrt{\Delta}=\sqrt3-\sqrt2\in K.
	\]
	于是
	\[
	\sqrt2=\frac{\alpha-(\sqrt3-\sqrt2)}{2}\in K,\qquad
	\sqrt3=\frac{\alpha+(\sqrt3-\sqrt2)}{2}\in K,
	\]
	从而
	\[
	\mathbb Q(\sqrt2,\sqrt3)\subseteq K.
	\]
	反过来显然 $\alpha=\sqrt2+\sqrt3\in \mathbb Q(\sqrt2,\sqrt3)$,故
	\[
	K=\mathbb Q(\sqrt2,\sqrt3).
	\]
	
	并且
	\[
	[\mathbb Q(\sqrt2):\mathbb Q]=2,\qquad \sqrt3\notin \mathbb Q(\sqrt2)
	\]
	(否则 $\mathbb Q(\sqrt3)\subseteq \mathbb Q(\sqrt2)$,二次扩张将相等,矛盾),
	因此
	\[
	[K:\mathbb Q]=[\mathbb Q(\sqrt2,\sqrt3):\mathbb Q]=4.
	\]
	
	\textbf{(2) $K/\mathbb Q$ 是 Galois 扩张.}\;
	因为 $K=\mathbb Q(\sqrt2,\sqrt3)$ 是多项式 $(x^2-2)(x^2-3)$ 在 $\mathbb Q$ 上的分裂域,
	且 $\mathrm{char}(\mathbb Q)=0$ 保证可分,
	故 $K/\mathbb Q$ 为有限 Galois 扩张。
	
	\textbf{(3) 列出全部 $\mathbb Q$-自同构并确定群结构.}\;
	任取 $\sigma\in \mathrm{Gal}(K/\mathbb Q)$,由“最小多项式根必须送到共轭根”可知
	\[
	\sigma(\sqrt2)\in\{\sqrt2,-\sqrt2\},\qquad
	\sigma(\sqrt3)\in\{\sqrt3,-\sqrt3\}.
	\]
	反之,任意给定符号选择
	\[
	\sqrt2\mapsto \pm\sqrt2,\qquad \sqrt3\mapsto \pm\sqrt3
	\]
	都唯一延拓为 $K$ 上的 $\mathbb Q$-自同构(因 $K=\mathbb Q(\sqrt2,\sqrt3)$)。
	因此共有 $4$ 个元素:
	\[
	\mathrm{id},
	\quad \sigma_2:\sqrt2\mapsto-\sqrt2,\ \sqrt3\mapsto\sqrt3,
	\quad \sigma_3:\sqrt2\mapsto\sqrt2,\ \sqrt3\mapsto-\sqrt3,
	\quad \sigma_{23}=\sigma_2\sigma_3.
	\]
	它们都满足 $\sigma^2=\mathrm{id}$,且两两可交换,所以
	\[
	\mathrm{Gal}(K/\mathbb Q)\cong (\mathbb Z/2\mathbb Z)\times(\mathbb Z/2\mathbb Z)
	\]
	即 Klein 四元群 $V_4$。
	
	\textbf{(4)(可选)给出 $\alpha$ 的最小多项式以佐证次数为 $4$.}\;
	由 $\alpha^2=5+2\sqrt6$ 得
	\[
	(\alpha^2-5)^2=24
	\ \Longrightarrow\
	\alpha^4-10\alpha^2+1=0.
	\]
	又因 $[K:\mathbb Q]=4$,故 $\alpha$ 的最小多项式次数为 $4$,于是
	\[
	\mathrm{Irr}(\alpha,\mathbb Q)=x^4-10x^2+1.
	\]
	其四个共轭为 $\pm\sqrt2\pm\sqrt3$,均在 $K$ 中,符合 $K/\mathbb Q$ 为 Galois 扩张。
	
	\item \textbf{题目.}\;
	试求 $x^3-2\in\mathbb Q[x]$ 的分裂域的 Galois 群。
	
	\textbf{答案:}
	
	\textbf{(1) 确定分裂域.}\;
	令
	\[
	\alpha:=\sqrt[3]{2}\in\mathbb R,\qquad \omega:=e^{2\pi i/3}=\frac{-1+\sqrt{-3}}2.
	\]
	则 $x^3-2$ 的三个根为
	\[
	\alpha,\ \omega\alpha,\ \omega^2\alpha,
	\]
	故其分裂域为
	\[
	E=\mathbb Q(\alpha,\omega).
	\]
	
	\textbf{(2) 计算扩张次数.}\;
	由 Eisenstein 判别法(素数 $p=2$)知 $x^3-2$ 在 $\mathbb Q[x]$ 上不可约,
	从而
	\[
	[\mathbb Q(\alpha):\mathbb Q]=3.
	\]
	又 $\omega\notin\mathbb Q(\alpha)$(因为 $\mathbb Q(\alpha)\subset\mathbb R$ 而 $\omega\notin\mathbb R$),
	且 $[\mathbb Q(\omega):\mathbb Q]=2$,
	故
	\[
	[E:\mathbb Q]=[\mathbb Q(\alpha,\omega):\mathbb Q(\alpha)]\,[\mathbb Q(\alpha):\mathbb Q]=2\cdot 3=6.
	\]
	
	\textbf{(3) 把 Galois 群嵌入 $S_3$ 并确定其阶.}\;
	任意 $\sigma\in\mathrm{Gal}(E/\mathbb Q)$ 必将根送到根,故在根集
	\[
	\{\alpha,\omega\alpha,\omega^2\alpha\}
	\]
	上诱导一个置换,得到单射
	\[
	\Phi:\mathrm{Gal}(E/\mathbb Q)\hookrightarrow S_3.
	\]
	又因 $E/\mathbb Q$ 是分裂域扩张且 $\mathrm{char}(\mathbb Q)=0$,故 $E/\mathbb Q$ 为 Galois 扩张,
	从而
	\[
	|\mathrm{Gal}(E/\mathbb Q)|=[E:\mathbb Q]=6.
	\]
	由于 $S_3$ 的阶亦为 $6$,单射必为同构,故
	\[
	\mathrm{Gal}(E/\mathbb Q)\cong S_3.
	\]
	
	\textbf{(4)(可选)给出两个生成元以直观呈现 $S_3$.}\;
	定义
	\[
	\sigma:\alpha\mapsto\omega\alpha,\ \omega\mapsto\omega
	\quad(\text{对应 3-循环}),
	\qquad
	\tau:\alpha\mapsto\alpha,\ \omega\mapsto\omega^2
	\quad(\text{复共轭,对应换位}),
	\]
	则 $\sigma$ 的阶为 $3$,$\tau$ 的阶为 $2$,且
	\[
	\tau\sigma\tau=\sigma^{-1},
	\]
	从而
	\[
	\mathrm{Gal}(E/\mathbb Q)=\langle\sigma,\tau\rangle\cong S_3.\]
	
	\item 
	\textbf{题目.}\;
	设 $E$ 为有理数域 $\mathbb{Q}$ 上多项式
	\[
	f(x)=x^{4}-10x^{2}+1
	\]
	的分裂域,求 $\mathrm{Gal}(E/\mathbb{Q})$。
	
	\textbf{答案:}
	
	\textbf{(1) 先把根写出来,锁定分裂域.}\;
	令 $y=x^{2}$,则
	\[
	f(x)=0 \iff y^{2}-10y+1=0.
	\]
	解得
	\[
	y=\frac{10\pm\sqrt{100-4}}{2}=\frac{10\pm\sqrt{96}}{2}=5\pm 2\sqrt{6}.
	\]
	注意到
	\[
	(\sqrt2\pm \sqrt3)^2=2+3\pm 2\sqrt6=5\pm 2\sqrt6,
	\]
	因此 $f(x)$ 的四个根为
	\[
	\pm(\sqrt2+\sqrt3),\qquad \pm(\sqrt3-\sqrt2).
	\]
	于是分裂域包含 $\sqrt2,\sqrt3$,并且
	\[
	E=\mathbb{Q}(\sqrt2,\sqrt3).
	\]
	
	\textbf{(2) 计算扩张次数 $[E:\mathbb{Q}]$.}\;
	由塔式公式
	\[
	[E:\mathbb{Q}]=[\mathbb{Q}(\sqrt2,\sqrt3):\mathbb{Q}(\sqrt2)]\,[\mathbb{Q}(\sqrt2):\mathbb{Q}].
	\]
	显然 $[\mathbb{Q}(\sqrt2):\mathbb{Q}]=2$。
	下面证 $\sqrt3\notin\mathbb{Q}(\sqrt2)$,从而
	$[\mathbb{Q}(\sqrt2,\sqrt3):\mathbb{Q}(\sqrt2)]=2$。
	
	若 $\sqrt3\in\mathbb{Q}(\sqrt2)$,则存在 $a,b\in\mathbb{Q}$ 使
	\[
	\sqrt3=a+b\sqrt2.
	\]
	两边平方得
	\[
	3=a^2+2b^2+2ab\sqrt2.
	\]
	比较有理部分与 $\sqrt2$ 系数可得 $2ab=0$,即 $a=0$ 或 $b=0$。
	\begin{itemize}
		\item 若 $b=0$,则 $\sqrt3=a\in\mathbb{Q}$,矛盾;
		\item 若 $a=0$,则 $\sqrt3=b\sqrt2$,平方得 $3=2b^2$,即 $b^2=\frac32$,
		这不可能在 $\mathbb{Q}$ 中成立(因为若 $b=\frac{m}{n}$ 约分,则
		$2m^2=3n^2$ 推出 $2\mid n$ 进而 $2\mid m$,与约分矛盾)。
	\end{itemize}
	故 $\sqrt3\notin\mathbb{Q}(\sqrt2)$,从而
	\[
	[E:\mathbb{Q}]=2\cdot 2=4.
	\]
	
	\textbf{(3) 构造并计数 $\mathbb{Q}$-自同构.}\;
	$E=\mathbb{Q}(\sqrt2,\sqrt3)$ 中任一 $\mathbb{Q}$-自同构由
	$\sqrt2,\sqrt3$ 的像完全决定,并且必须满足
	\[
	\sigma(\sqrt2)^2=\sigma(2)=2,\qquad \sigma(\sqrt3)^2=\sigma(3)=3,
	\]
	所以
	\[
	\sigma(\sqrt2)\in\{\sqrt2,-\sqrt2\},\qquad
	\sigma(\sqrt3)\in\{\sqrt3,-\sqrt3\}.
	\]
	因此至少有下面四个不同的自同构(它们都把 $\mathbb{Q}$ 固定):
	\[
	\begin{aligned}
		\mathrm{id}:&\ \sqrt2\mapsto\sqrt2,\ \sqrt3\mapsto\sqrt3;\\
		\tau_2:&\ \sqrt2\mapsto-\sqrt2,\ \sqrt3\mapsto\sqrt3;\\
		\tau_3:&\ \sqrt2\mapsto\sqrt2,\ \sqrt3\mapsto-\sqrt3;\\
		\tau_2\tau_3:&\ \sqrt2\mapsto-\sqrt2,\ \sqrt3\mapsto-\sqrt3.
	\end{aligned}
	\]
	它们互不相同,所以
	\[
	|\mathrm{Gal}(E/\mathbb{Q})|\ge 4.
	\]
	另一方面,$E/\mathbb{Q}$ 是特征 $0$ 上的分裂域扩张,故为 Galois 扩张,
	从而
	\[
	|\mathrm{Gal}(E/\mathbb{Q})|=[E:\mathbb{Q}]=4.
	\]
	于是上面四个自同构就是全部的 Galois 群元素。
	
	\textbf{(4) 群结构.}\;
	显然 $\tau_2^2=\tau_3^2=\mathrm{id}$ 且 $\tau_2\tau_3=\tau_3\tau_2$,
	因此
	\[
	\mathrm{Gal}(E/\mathbb{Q})
	=\{\mathrm{id},\tau_2,\tau_3,\tau_2\tau_3\}
	\cong C_2\times C_2,
	\]
	即 Klein 四元群 $V_4$。
	
	\textbf{(5)(可选)写成根集上的置换(嵌入 $S_4$).}\;
	记
	\[
	r_1=\sqrt2+\sqrt3,\quad r_2=-(\sqrt2+\sqrt3),\quad
	r_3=\sqrt3-\sqrt2,\quad r_4=-(\sqrt3-\sqrt2).
	\]
	则
	\[
	\tau_2:\ (r_1\,r_3)(r_2\,r_4),\qquad
	\tau_3:\ (r_1\,r_4)(r_2\,r_3),\qquad
	\tau_2\tau_3:\ (r_1\,r_2)(r_3\,r_4),
	\]
	从而该 Galois 群作为 $S_4$ 的子群同构于 $V_4$。
	
\end{enumerate}
\clearpage
\section{Galois扩张与Galois对应}
\subsection{Galois 群的置换表示}
\textbf{命题(Galois 群的置换表示).}\;
设 $f(x)\in F[x]$,$E$ 为 $f$ 在 $F$ 上的分裂域,
$\Omega=\{\alpha_1,\dots,\alpha_n\}$ 为 $f$ 在 $E$ 中的全部根,
从而 $E=F(\alpha_1,\dots,\alpha_n)$。
定义映射
\[
\Phi:\mathrm{Gal}(E/F)\longrightarrow \mathrm{Sym}(\Omega),\qquad
\Phi(\sigma)=\sigma|_{\Omega}.
\]
($\mathrm{Sym}(\Omega)$表示$\Omega$上的置换群),则:

\begin{itemize}
	\item $\Phi$ 为群同态;
	\item $\Phi$ 为单射;
	\item 因而 $\mathrm{Gal}(E/F)\cong \Phi(\mathrm{Gal}(E/F))\le \mathrm{Sym}(\Omega)\cong S_n$,
	即 $\mathrm{Gal}(E/F)$ 同构于一个置换群的子群。
\end{itemize}

\textbf{证明.}\;
对任意 $\sigma,\tau\in\mathrm{Gal}(E/F)$ 与任意 $\alpha\in\Omega$,
\[
\Phi(\sigma\tau)(\alpha)=(\sigma\tau)(\alpha)=\sigma(\tau(\alpha))
=\Phi(\sigma)\bigl(\Phi(\tau)(\alpha)\bigr),
\]
故 $\Phi(\sigma\tau)=\Phi(\sigma)\circ\Phi(\tau)$,从而 $\Phi$ 为群同态。

再证单射:若 $\sigma\in\ker(\Phi)$,则 $\sigma(\alpha_i)=\alpha_i\ (i=1,\dots,n)$,
且 $\sigma|_F=\mathrm{id}_F$。
由于 $E=F(\alpha_1,\dots,\alpha_n)$,
任取 $x\in E$,可写为
\[
x=R(\alpha_1,\dots,\alpha_n)\qquad (R\in F(X_1,\dots,X_n)),
\]
于是
\[
\sigma(x)=R(\sigma(\alpha_1),\dots,\sigma(\alpha_n))
=R(\alpha_1,\dots,\alpha_n)=x,
\]
故 $\sigma=\mathrm{id}_E$,从而 $\ker(\Phi)=\{\mathrm{id}\}$,$\Phi$ 单射。

最后由同态基本定理得
\[
\mathrm{Gal}(E/F)\cong \Phi(\mathrm{Gal}(E/F))\le \mathrm{Sym}(\Omega)\cong S_n.
\]

\clearpage 
\subsection*{课后习题答案}
\addcontentsline{toc}{subsection}{\textcolor{red}{课后习题答案}}
\begin{enumerate}[label=\textcolor{blue}{\textbf{\large\arabic*.}}]
	\item 3.
	\textbf{题目.}
	设
	\[
	f(x)=x^3-3\in \mathbb Q[x],
	\]
	令 $E$ 为 $f(x)$ 在 $\mathbb Q$ 上的分裂域。
	\begin{itemize}
		\item 求 $\mathrm{Gal}(E/\mathbb Q)$ 的所有子群及其对应的子域;
		\item 证明 $\mathrm{Gal}(E/\mathbb Q)\cong S_3$。
	\end{itemize}
	
	\textbf{解答.}
	
	\textbf{第一步:确定分裂域 $E$.}
	
	设
	\[
	\alpha=\sqrt[3]{3},\qquad \omega=\frac{-1+\sqrt{-3}}{2},
	\]
	其中 $\omega$ 是三次单位根,满足 $\omega^3=1,\ \omega\neq1$。
	
	则 $x^3-3$ 的三个根为
	\[
	\alpha,\ \omega\alpha,\ \omega^2\alpha.
	\]
	因此
	\[
	E=\mathbb Q(\alpha,\omega).
	\]
	
	注意:
	\[
	\omega\in \mathbb Q(\sqrt{-3}),\qquad [\mathbb Q(\omega):\mathbb Q]=2,
	\]
	而 $x^3-3$ 在 $\mathbb Q[x]$ 中由 Eisenstein 判别法(取素数 $3$)知不可约,
	故
	\[
	[\mathbb Q(\alpha):\mathbb Q]=3.
	\]
	
	又因为 $\mathbb Q(\alpha)\cap\mathbb Q(\omega)=\mathbb Q$,
	于是
	\[
	[E:\mathbb Q]
	=[\mathbb Q(\alpha,\omega):\mathbb Q]
	=3\times2=6.
	\]
	
	---
	
	\textbf{第二步:描述 $\mathrm{Gal}(E/\mathbb Q)$ 的作用.}
	
	任取
	\[
	\sigma\in\mathrm{Gal}(E/\mathbb Q),
	\]
	则 $\sigma$ 完全由它对 $\alpha,\omega$ 的作用决定,且必须满足:
	
	\begin{itemize}
		\item $\sigma(\alpha)$ 是 $x^3-3$ 的根;
		\item $\sigma(\omega)$ 是三次单位根。
	\end{itemize}
	
	因此:
	\[
	\sigma(\alpha)\in\{\alpha,\omega\alpha,\omega^2\alpha\},\qquad
	\sigma(\omega)\in\{\omega,\omega^2\}.
	\]
	
	任意这样的选择都唯一确定一个 $\mathbb Q$-自同构,
	故
	\[
	|\mathrm{Gal}(E/\mathbb Q)|=3\times2=6.
	\]
	
	---
	
	\textbf{第三步:证明 $\mathrm{Gal}(E/\mathbb Q)\cong S_3$.}
	
	考虑 $x^3-3$ 的根集
	\[
	\{\alpha,\omega\alpha,\omega^2\alpha\}.
	\]
	任意 $\sigma\in\mathrm{Gal}(E/\mathbb Q)$ 给出该根集上的一个置换。
	
	于是得到群单射
	\[
	\Phi:\mathrm{Gal}(E/\mathbb Q)\hookrightarrow S_3.
	\]
	
	由于
	\[
	|\mathrm{Gal}(E/\mathbb Q)|=6=|S_3|,
	\]
	$\Phi$ 为双射,从而
	\[
	\mathrm{Gal}(E/\mathbb Q)\cong S_3.
	\]
	
	---
	
	\textbf{第四步:列出所有子群及其对应子域.}
	
	$S_3$ 的子群结构如下:
	
	\begin{center}
		\begin{tabular}{c|c|c}
			子群 $H$ & 阶 $|H|$ & 对应子域 $E^H$ \\
			\hline
			$\{e\}$ & $1$ & $E=\mathbb Q(\alpha,\omega)$ \\
			$\langle(12)\rangle,\ \langle(13)\rangle,\ \langle(23)\rangle$ & $2$ & 三个不同的三次子域 \\
			$\langle(123)\rangle$ & $3$ & $\mathbb Q(\omega)=\mathbb Q(\sqrt{-3})$ \\
			$S_3$ & $6$ & $\mathbb Q$
		\end{tabular}
	\end{center}
	
	其中:
	
	\begin{itemize}
		\item 阶为 $3$ 的正规子群 $\langle(123)\rangle$ 固定 $\omega$,
		因此对应子域为 $\mathbb Q(\omega)$;
		\item 每个阶为 $2$ 的子群对应一个次数为 $3$ 的中间域,
		如 $\mathbb Q(\alpha)$、$\mathbb Q(\omega\alpha)$ 等;
		\item 极大子群与极小子域互相对应,完全符合 Galois 对应定理。
	\end{itemize}
	
	---
	
	\textbf{结论.}
	
	\begin{itemize}
		\item $x^3-3$ 的分裂域为 $E=\mathbb Q(\sqrt[3]{3},\omega)$;
		\item $[E:\mathbb Q]=6$;
		\item $\mathrm{Gal}(E/\mathbb Q)\cong S_3$;
		\item 其子群与中间域一一对应,结构完全明确。
	\end{itemize}
	
	\textbf{题目.}\;
	设 $E$ 是 $x^3-3\in\mathbb{Q}[x]$ 的分裂域,求 $\mathrm{Gal}(E/\mathbb{Q})$ 的所有子群及其对应的子域,并证明
	\[
	\mathrm{Gal}(E/\mathbb{Q})\cong S_3.
	\]
	\textbf{题目.}\;
	设 $E$ 为 $x^3-3\in\mathbb Q[x]$ 的分裂域,已知
	\[
	\mathrm{Gal}(E/\mathbb Q)\cong S_3.
	\]
	解释图中 $S_3$ 的子群表格如何得到;并说明:已知子群 $H\le \mathrm{Gal}(E/F)$ 后,如何计算不变子域 $E^H$。
	
	\textbf{答案.}
	
	\textbf{(1) 先说明 $S_3$ 的子群为什么只有这些.}\;
	由 $|S_3|=6$,Lagrange 定理知任意子群阶只能是 $1,2,3,6$。
	
	\begin{itemize}
		\item 阶为 $1$:只有平凡子群 $\{e\}$;
		\item 阶为 $6$:只有整体 $S_3$;
		\item 阶为 $3$:必为 Sylow-$3$ 子群。$S_3$ 中三循环只有两个:$(123),(132)$,
		它们生成同一个子群
		\[
		\langle(123)\rangle=\{e,(123),(132)\},
		\]
		故阶为 $3$ 的子群唯一;
		\item 阶为 $2$:由任意 2-子群必为循环群且由一个 2-阶元素生成。
		$S_3$ 中 2-阶元素正好是三个换位
		\[
		(12),(13),(23),
		\]
		故阶为 $2$ 的子群正好是
		\[
		\langle(12)\rangle,\ \langle(13)\rangle,\ \langle(23)\rangle.
		\]
	\end{itemize}
	
	因此图中“子群列表”部分就是由上述分类得到的。
	
	\medskip
	\textbf{(2) 为什么“对应子域”那一列是那样写的:用 Galois 对应 + 次数公式.}\;
	
	因为 $E/\mathbb Q$ 是 Galois 扩张,对每个子群 $H\le G:=\mathrm{Gal}(E/\mathbb Q)$,
	有不变子域
	\[
	E^H:=\{x\in E:\forall \sigma\in H,\ \sigma(x)=x\},
	\]
	并且满足次数关系
	\[
	[E:E^H]=|H|,\qquad [E^H:\mathbb Q]=\frac{|G|}{|H|}=\frac{6}{|H|}.
	\]
	据此立刻得到表格中的“次数信息”:
	\[
	|H|=1\Rightarrow [E^H:\mathbb Q]=6;\quad
	|H|=2\Rightarrow [E^H:\mathbb Q]=3;\quad
	|H|=3\Rightarrow [E^H:\mathbb Q]=2;\quad
	|H|=6\Rightarrow [E^H:\mathbb Q]=1.
	\]
	
	接下来要做的是:把这些次数为 $2$ 或 $3$ 的中间域在 $E$ 中\emph{具体识别出来}。
	
	\medskip
	\textbf{(3) 在本题中具体识别:$E=\mathbb Q(\alpha,\omega)$ 的情形.}\;
	取
	\[
	\alpha=\sqrt[3]{3}\in\mathbb R,\qquad \omega=\frac{-1+\sqrt{-3}}2,\ \omega^3=1,\ \omega\ne 1,
	\]
	则 $x^3-3$ 的根为 $\alpha,\omega\alpha,\omega^2\alpha$,从而
	\[
	E=\mathbb Q(\alpha,\omega),\qquad |G|=[E:\mathbb Q]=6.
	\]
	
	\textbf{(3.1) 阶为 $3$ 的子群对应 $\mathbb Q(\omega)=\mathbb Q(\sqrt{-3})$.}\;
	令 $\sigma\in G$ 对根作 3-循环:
	\[
	\sigma:\alpha\mapsto\omega\alpha\mapsto\omega^2\alpha\mapsto\alpha,\qquad \sigma(\omega)=\omega.
	\]
	则 $\langle\sigma\rangle$ 的阶为 $3$,它\emph{逐点固定} $\omega$,故
	\[
	\mathbb Q(\omega)\subseteq E^{\langle\sigma\rangle}.
	\]
	而由次数公式
	\[
	[E^{\langle\sigma\rangle}:\mathbb Q]=\frac{6}{3}=2,
	\]
	又 $\mathbb Q(\omega)/\mathbb Q$ 已是 2 次扩张,因此只能是
	\[
	E^{\langle\sigma\rangle}=\mathbb Q(\omega)=\mathbb Q(\sqrt{-3}).
	\]
	这就解释了表格中 $|H|=3$ 时对应子域写成 $\mathbb Q(\omega)$。
	
	\textbf{(3.2) 阶为 $2$ 的三个子群对应三个不同的 3 次中间域.}\;
	取复共轭自同构(它属于 $G$)
	\[
	\tau:\omega\mapsto \omega^2,\qquad \alpha\mapsto \alpha.
	\]
	则 $\tau$ 在根集 $\{\alpha,\omega\alpha,\omega^2\alpha\}$ 上的作用是交换后两者、固定 $\alpha$,
	对应置换是一个换位(如 $(23)$),因此 $|\langle\tau\rangle|=2$。
	并且 $\tau(\alpha)=\alpha$,所以
	\[
	\mathbb Q(\alpha)\subseteq E^{\langle\tau\rangle}.
	\]
	由次数公式
	\[
	[E^{\langle\tau\rangle}:\mathbb Q]=\frac{6}{2}=3,
	\]
	而 $[\mathbb Q(\alpha):\mathbb Q]=3$,故
	\[
	E^{\langle\tau\rangle}=\mathbb Q(\alpha).
	\]
	
	另外两个阶为 $2$ 的子群是共轭的(在 $S_3$ 里三个换位共轭),其不变域也必是三个互不相同但同构的 3 次子域。
	更具体地,如果把三个换位理解为“固定某一个根、交换另外两个根”,则它们对应的不变子域正是
	\[
	\mathbb Q(\alpha),\qquad \mathbb Q(\omega\alpha),\qquad \mathbb Q(\omega^2\alpha),
	\]
	它们在 $E$ 中两两不同(因为例如 $\omega\notin \mathbb Q(\alpha)$,否则 $\mathbb Q(\omega)\subseteq \mathbb Q(\alpha)$ 将迫使 $2\mid 3$ 矛盾)。
	
	因此表格中 $|H|=2$ 时写“\;三个不同的三次子域\;”是由次数公式 + 三个换位子群得到的。
	
	\textbf{(3.3) 其余两行.}\;
	\[
	E^{\{e\}}=E,\qquad E^{S_3}=\mathbb Q
	\]
	分别由定义(全不动/全都要求不变)立得。
	
	\medskip
	\textbf{(4) 一般地:已知子群 $H$ 后,如何\;“算”\;不变子域 $E^H$?(实操方法)}
	
	设 $E/F$ 为有限 Galois 扩张,$G=\mathrm{Gal}(E/F)$,$H\le G$。
	
	\begin{itemize}
		\item \textbf{第一步(先算次数)}:
		用
		\[
		[E:E^H]=|H|,\qquad [E^H:F]=\frac{|G|}{|H|}
		\]
		先确定 $E^H/F$ 的次数,这一步往往把目标域“锁死到很小的候选范围”。
		
		\item \textbf{第二步(构造 $H$-不变元:求和/求积法)}:
		任取 $b\in E$,定义
		\[
		s(b):=\sum_{\sigma\in H}\sigma(b),\qquad p(b):=\prod_{\sigma\in H}\sigma(b).
		\]
		则对任意 $\tau\in H$ 都有
		\[
		\tau(s(b))=s(b),\qquad \tau(p(b))=p(b),
		\]
		故 $s(b),p(b)\in E^H$。这是一种“平均化(orbit sum/product)”构造不变元的通法。
		
		\item \textbf{第三步(用次数验证生成)}:
		通常取一个合适的 $b$,让 $F(s(b))$(或 $F(p(b))$)的次数恰好等于 $[E^H:F]$,
		就可推出
		\[
		E^H=F(s(b))\quad\text{或}\quad E^H=F(p(b)).
		\]
		(因为 $F(s(b))\subseteq E^H$ 且二者对 $F$ 的次数相同,只能相等。)
		
		\item \textbf{第四步(在“根置换”模型下的快捷法)}:
		若 $E$ 是某个不可约多项式的分裂域,且 $G$ 作用在根集上,
		则“稳定子 $\mathrm{Stab}(\alpha)$”的固定域往往就是 $F(\alpha)$;
		更一般地,稳定某个根的子群对应“把这个根加入后的中间域”。
		这就是本题中把阶 $2$ 的子群识别成 $\mathbb Q(\alpha)$、$\mathbb Q(\omega\alpha)$ 等的原因。
	\end{itemize}
	
	\textbf{解:}
	
	\textbf{(1) 分裂域的具体描述与次数.}\;
	令
	\[
	\alpha:=\sqrt[3]{3}\in\mathbb{R},\qquad \omega:=e^{2\pi i/3}=\frac{-1+\sqrt{-3}}2,
	\]
	则 $x^3-3$ 在 $\mathbb{C}$ 中的三根为
	\[
	\alpha,\ \alpha\omega,\ \alpha\omega^2.
	\]
	故分裂域
	\[
	E=\mathbb{Q}(\alpha,\omega).
	\]
	用 Eisenstein 判别法(素数 $3$)知 $x^3-3$ 在 $\mathbb{Q}[x]$ 中不可约,因此
	\[
	[\mathbb{Q}(\alpha):\mathbb{Q}]=3.
	\]
	又 $\omega$ 的最小多项式为 $x^2+x+1$,故 $[\mathbb{Q}(\omega):\mathbb{Q}]=2$。
	并且 $\mathbb{Q}(\alpha)\subset\mathbb{R}$,而 $\omega\notin\mathbb{R}$,所以 $\omega\notin\mathbb{Q}(\alpha)$,从而
	\[
	[E:\mathbb{Q}]=[\mathbb{Q}(\alpha,\omega):\mathbb{Q}(\alpha)]\,[\mathbb{Q}(\alpha):\mathbb{Q}]
	=2\cdot 3=6.
	\]
	由于 $E/\mathbb{Q}$ 是分裂域扩张(特征 $0$ 下可分),故 $E/\mathbb{Q}$ 为 Galois 扩张,因而
	\[
	|\mathrm{Gal}(E/\mathbb{Q})|=[E:\mathbb{Q}]=6.
	\]
	
	\textbf{(2) 构造 6 个自同构并嵌入到 $S_3$.}\;
	任取 $\sigma\in\mathrm{Gal}(E/\mathbb{Q})$,必有
	\[
	\sigma(\alpha)^3=\sigma(\alpha^3)=\sigma(3)=3,
	\]
	故 $\sigma(\alpha)$ 必为 $x^3-3$ 的某个根,即
	\[
	\sigma(\alpha)\in\{\alpha,\alpha\omega,\alpha\omega^2\}.
	\]
	同理 $\sigma(\omega)$ 必为 $x^2+x+1$ 的根,所以
	\[
	\sigma(\omega)\in\{\omega,\omega^2\}.
	\]
	因此至多有 $3\times 2=6$ 个 $\mathbb{Q}$-自同构;而我们已知群的阶为 $6$,所以这些可能性全部实现。
	
	将 $\mathrm{Gal}(E/\mathbb{Q})$ 作用在三根集合
	\[
	R:=\{\alpha,\alpha\omega,\alpha\omega^2\}
	\]
	上,得群同态
	\[
	\Phi:\mathrm{Gal}(E/\mathbb{Q})\longrightarrow S_R\cong S_3,
	\qquad \Phi(\sigma)=\sigma|_R.
	\]
	若 $\Phi(\sigma)=\mathrm{id}$,则 $\sigma$ 固定三根,因而固定由它们生成的分裂域 $E$,即 $\sigma=\mathrm{id}$,
	故 $\mathrm{ker}(\Phi)=\{1\}$,$\Phi$ 为单射。
	于是
	\[
	|\mathrm{Im}(\Phi)|=|\mathrm{Gal}(E/\mathbb{Q})|=6.
	\]
	而 $|S_3|=6$,故 $\mathrm{Im}(\Phi)=S_3$,$\Phi$ 为同构,从而
	\[
	\mathrm{Gal}(E/\mathbb{Q})\cong S_3.
	\]
	
	\textbf{(3) 给出生成元(显式对应到 $S_3$ 的 3-循环与换位).}\;
	定义两个 $\mathbb{Q}$-自同构:
	\[
	\sigma:\ \alpha\mapsto \alpha\omega,\ \omega\mapsto \omega;
	\qquad
	\tau:\ \alpha\mapsto \alpha,\ \omega\mapsto \omega^2\quad(\text{复共轭}).
	\]
	则 $\sigma$ 将根作循环
	\[
	\alpha\mapsto \alpha\omega\mapsto \alpha\omega^2\mapsto \alpha,
	\]
	故 $\sigma$ 的像对应 $3$-循环;而 $\tau$ 固定 $\alpha$、交换 $\alpha\omega$ 与 $\alpha\omega^2$,对应一个换位。
	因此 $\langle\sigma,\tau\rangle$ 的像包含一个 $3$-循环与一个换位,生成整个 $S_3$,
	这也与上面的“阶为 6”相吻合。
	
	\textbf{(4) 所有子群与对应不动域(Galois 对应).}\;
	记 $G:=\mathrm{Gal}(E/\mathbb{Q})\cong S_3$,则 $G$ 的子群只有以下几类:
	\[
	\{1\},\quad G,\quad A_3=\langle\sigma\rangle,\quad
	\langle\tau\rangle,\ \langle\sigma\tau\sigma^{-1}\rangle,\ \langle\sigma^2\tau\sigma^{-2}\rangle.
	\]
	它们对应的不动域(即 $E^H:=\{x\in E:\forall h\in H,\ h(x)=x\}$)为:
	
	\[
	\begin{tabularx}{\textwidth}{@{}lX X@{}}
		\toprule
		子群 $H\le G$ & 阶 $|H|$ 与 $[E^H:\mathbb{Q}]=|G|/|H|$ & 对应子域 $E^H$ \\
		\midrule
		$\{1\}$ & $|H|=1,\ [E^H:\mathbb{Q}]=6$ & $E^H=E=\mathbb{Q}(\alpha,\omega)$\\
		$G$ & $|H|=6,\ [E^H:\mathbb{Q}]=1$ & $E^H=\mathbb{Q}$\\
		$A_3=\langle\sigma\rangle$ & $|H|=3,\ [E^H:\mathbb{Q}]=2$ & $E^H=\mathbb{Q}(\omega)=\mathbb{Q}(\sqrt{-3})$\\
		$\langle\tau\rangle$ & $|H|=2,\ [E^H:\mathbb{Q}]=3$ & $E^H=\mathbb{Q}(\alpha)$\\
		$\langle\sigma\tau\sigma^{-1}\rangle$ & $|H|=2,\ [E^H:\mathbb{Q}]=3$ & $E^H=\mathbb{Q}(\alpha\omega)$\\
		$\langle\sigma^2\tau\sigma^{-2}\rangle$ & $|H|=2,\ [E^H:\mathbb{Q}]=3$ & $E^H=\mathbb{Q}(\alpha\omega^2)$\\
		\bottomrule
	\end{tabularx}
	\]
	
	\textbf{(5) 上表中关键等式的简证(用次数卡死).}\;
	\textbf{1) }$E^{A_3}=\mathbb{Q}(\omega)$:因为 $\sigma(\omega)=\omega$,故 $\mathbb{Q}(\omega)\subseteq E^{A_3}$。
	又 $|A_3|=3$,所以
	\[
	[E^{A_3}:\mathbb{Q}]=\frac{|G|}{|A_3|}=\frac{6}{3}=2.
	\]
	而 $[\mathbb{Q}(\omega):\mathbb{Q}]=2$,因此只能有
	\[
	E^{A_3}=\mathbb{Q}(\omega).
	\]
	
	\textbf{2) }$E^{\langle\tau\rangle}=\mathbb{Q}(\alpha)$:$\tau$ 为复共轭,固定所有实数,特别固定 $\alpha$,
	故 $\mathbb{Q}(\alpha)\subseteq E^{\langle\tau\rangle}$。
	又 $|\langle\tau\rangle|=2$,所以
	\[
	[E^{\langle\tau\rangle}:\mathbb{Q}]=\frac{6}{2}=3.
	\]
	而 $[\mathbb{Q}(\alpha):\mathbb{Q}]=3$,故
	\[
	E^{\langle\tau\rangle}=\mathbb{Q}(\alpha).
	\]
	同理,对共轭子群 $\langle\sigma\tau\sigma^{-1}\rangle$ 与 $\langle\sigma^2\tau\sigma^{-2}\rangle$,
	它们分别固定 $\alpha\omega$ 与 $\alpha\omega^2$,再用次数同样得到
	\[
	E^{\langle\sigma\tau\sigma^{-1}\rangle}=\mathbb{Q}(\alpha\omega),\qquad
	E^{\langle\sigma^2\tau\sigma^{-2}\rangle}=\mathbb{Q}(\alpha\omega^2).
	\]
	
	\textbf{结论.}\;
	$\mathrm{Gal}(E/\mathbb{Q})$ 的阶为 $6$,并通过对三根的置换同构于 $S_3$;
	其全部子群与对应子域如上表所示。
	\item 4.
	\textbf{题目.}\;
	设
	\[
	E=\mathbb Q(\sqrt2,\sqrt3,\sqrt5),
	\]
	求 $\mathrm{Gal}(E/\mathbb Q)$ 的所有子群及其对应的中间子域。
	
	\textbf{答案:}
	
	\textbf{(1) 先确定 $E/\mathbb Q$ 的次数与 Galois 群结构.}\;
	
	注意
	\[
	E \text{ 是多项式 } (x^2-2)(x^2-3)(x^2-5)\in\mathbb Q[x] \text{ 的分裂域}.
	\]
	故 $E/\mathbb Q$ 为有限 Galois 扩张($\mathrm{char}(\mathbb Q)=0$ 故可分,且为分裂域故正规)。
	
	\medskip
	下面计算 $[E:\mathbb Q]$。
	显然 $[\mathbb Q(\sqrt2):\mathbb Q]=2$,且 $\sqrt3\notin\mathbb Q(\sqrt2)$(否则 $\mathbb Q(\sqrt3)\subseteq\mathbb Q(\sqrt2)$,与二次扩张的不同判别类矛盾),
	从而
	\[
	[\mathbb Q(\sqrt2,\sqrt3):\mathbb Q]=4.
	\]
	再证 $\sqrt5\notin \mathbb Q(\sqrt2,\sqrt3)$:
	任取 $x\in\mathbb Q(\sqrt2,\sqrt3)$,可唯一写成
	\[
	x=a+b\sqrt2+c\sqrt3+d\sqrt6\qquad(a,b,c,d\in\mathbb Q).
	\]
	若 $x=\sqrt5$,对嵌入 $\sqrt2\mapsto-\sqrt2,\ \sqrt3\mapsto \sqrt3$ 取像得
	\[
	\sqrt5=a-b\sqrt2+c\sqrt3-d\sqrt6.
	\]
	两式相减得 $2b\sqrt2+2d\sqrt6=0$,故 $b=d=0$。
	同理对嵌入 $\sqrt2\mapsto\sqrt2,\ \sqrt3\mapsto-\sqrt3$ 取像可得 $c=0$。
	于是 $\sqrt5=a\in\mathbb Q$,矛盾。
	故 $\sqrt5\notin\mathbb Q(\sqrt2,\sqrt3)$,从而
	\[
	[E:\mathbb Q]=[\mathbb Q(\sqrt2,\sqrt3,\sqrt5):\mathbb Q(\sqrt2,\sqrt3)]\cdot[\mathbb Q(\sqrt2,\sqrt3):\mathbb Q]=2\cdot 4=8.
	\]
	
	\medskip
	定义三个 $\mathbb Q$-自同构(“改符号”):
	\[
	\sigma_2:\sqrt2\mapsto-\sqrt2,\ \sqrt3\mapsto\sqrt3,\ \sqrt5\mapsto\sqrt5,
	\]
	\[
	\sigma_3:\sqrt2\mapsto\sqrt2,\ \sqrt3\mapsto-\sqrt3,\ \sqrt5\mapsto\sqrt5,
	\]
	\[
	\sigma_5:\sqrt2\mapsto\sqrt2,\ \sqrt3\mapsto\sqrt3,\ \sqrt5\mapsto-\sqrt5.
	\]
	它们两两可交换且均满足 $\sigma_2^2=\sigma_3^2=\sigma_5^2=\mathrm{id}$。
	由生成元决定同态,三者唯一延拓为 $E$ 上的 $\mathbb Q$-自同构,
	并且
	\[
	\{\sigma_2^{i}\sigma_3^{j}\sigma_5^{k}: i,j,k\in\{0,1\}\}
	\]
	给出 $8$ 个互不相同的自同构(因为它们在 $(\sqrt2,\sqrt3,\sqrt5)$ 的符号模式不同)。
	因此
	\[
	|\mathrm{Gal}(E/\mathbb Q)|\ge 8.
	\]
	又因 $E/\mathbb Q$ 为 Galois 扩张,$|\mathrm{Gal}(E/\mathbb Q)|=[E:\mathbb Q]=8$,
	故上面的 $8$ 个自同构即为全部自同构。
	于是
	\[
	\mathrm{Gal}(E/\mathbb Q)=\langle \sigma_2,\sigma_3,\sigma_5\rangle\cong (\mathbb Z/2\mathbb Z)^3.
	\]
	下面记
	\[
	G:=\mathrm{Gal}(E/\mathbb Q)\cong (\mathbb Z/2\mathbb Z)^3.
	\]
	
	\medskip
	\textbf{(2) 中间域与子群的对应:只需用固定域与次数公式.}\;
	
	对任意子群 $H\le G$,其不变子域
	\[
	E^H:=\{x\in E:\forall \sigma\in H,\ \sigma(x)=x\}
	\]
	满足
	\[
	[E:E^H]=|H|,\qquad [E^H:\mathbb Q]=\frac{|G|}{|H|}=\frac{8}{|H|}.
	\]
	因为 $G\cong (\mathbb Z/2)^3$,其所有子群阶只能是 $1,2,4,8$。
	并且(把 $G$ 看成 $\mathbb F_2^3$ 的加法群)共有
	\[
	1\ (\text{阶 }1),\quad 7\ (\text{阶 }2),\quad 7\ (\text{阶 }4),\quad 1\ (\text{阶 }8)
	\]
	个子群。
	
	\medskip
	\textbf{(3) 列出所有子群及对应固定域(中间域).}\;
	
	为简洁起见,记
	\[
	\sqrt6=\sqrt2\sqrt3,\quad \sqrt{10}=\sqrt2\sqrt5,\quad \sqrt{15}=\sqrt3\sqrt5,\quad \sqrt{30}=\sqrt2\sqrt3\sqrt5.
	\]
	
	\textbf{(3.1) 阶为 $8$ 与阶为 $1$ 的两端.}
	\[
	E^{G}=\mathbb Q,\qquad E^{\{e\}}=E.
	\]
	
	\textbf{(3.2) 阶为 $4$ 的 $7$ 个子群 $\Longleftrightarrow$ $7$ 个二次中间域.}\;
	它们的固定域次数为 $[E^H:\mathbb Q]=8/4=2$,故必为某个 $\mathbb Q(\sqrt d)$。
	逐个给出一组“生成元”即可:
	
	\begin{itemize}
		\item $H_{\sqrt2}:=\langle \sigma_3,\sigma_5\rangle$,则 $H_{\sqrt2}$ 逐点固定 $\sqrt2$,
		故 $\mathbb Q(\sqrt2)\subseteq E^{H_{\sqrt2}}$,且二者次数同为 $2$,
		从而 $E^{H_{\sqrt2}}=\mathbb Q(\sqrt2)$。
		\item $H_{\sqrt3}:=\langle \sigma_2,\sigma_5\rangle$,有 $E^{H_{\sqrt3}}=\mathbb Q(\sqrt3)$。
		\item $H_{\sqrt5}:=\langle \sigma_2,\sigma_3\rangle$,有 $E^{H_{\sqrt5}}=\mathbb Q(\sqrt5)$。
		\item $H_{\sqrt6}:=\langle \sigma_2\sigma_3,\ \sigma_5\rangle$,注意 $(\sigma_2\sigma_3)(\sqrt6)=\sqrt6$ 且 $\sigma_5(\sqrt6)=\sqrt6$,
		故 $E^{H_{\sqrt6}}=\mathbb Q(\sqrt6)$。
		\item $H_{\sqrt{10}}:=\langle \sigma_2\sigma_5,\ \sigma_3\rangle$,有 $E^{H_{\sqrt{10}}}=\mathbb Q(\sqrt{10})$。
		\item $H_{\sqrt{15}}:=\langle \sigma_3\sigma_5,\ \sigma_2\rangle$,有 $E^{H_{\sqrt{15}}}=\mathbb Q(\sqrt{15})$。
		\item $H_{\sqrt{30}}:=\langle \sigma_2\sigma_3,\ \sigma_2\sigma_5\rangle$,这恰是“翻转偶数个根号”的子群,
		其逐点固定 $\sqrt{30}$,从而 $E^{H_{\sqrt{30}}}=\mathbb Q(\sqrt{30})$。
	\end{itemize}
	
	因此全部二次中间域正好是
	\[
	\mathbb Q(\sqrt2),\ \mathbb Q(\sqrt3),\ \mathbb Q(\sqrt5),\ \mathbb Q(\sqrt6),\ \mathbb Q(\sqrt{10}),\ \mathbb Q(\sqrt{15}),\ \mathbb Q(\sqrt{30}).
	\]
	
	\textbf{(3.3) 阶为 $2$ 的 $7$ 个子群 $\Longleftrightarrow$ $7$ 个四次中间域.}\;
	它们的固定域次数为 $[E^H:\mathbb Q]=8/2=4$。
	对每个非平凡元素(阶 $2$)生成的子群,直接写出其固定域(即“它不改符号的根号们”所生成的域):
	
	\begin{itemize}
		\item $H=\langle \sigma_2\rangle$:$\sigma_2$ 只改 $\sqrt2$ 的符号,逐点固定 $\sqrt3,\sqrt5$,
		故 $E^H=\mathbb Q(\sqrt3,\sqrt5)$。
		\item $H=\langle \sigma_3\rangle$:$E^H=\mathbb Q(\sqrt2,\sqrt5)$。
		\item $H=\langle \sigma_5\rangle$:$E^H=\mathbb Q(\sqrt2,\sqrt3)$。
		\item $H=\langle \sigma_2\sigma_3\rangle$:它同时改 $\sqrt2,\sqrt3$ 的符号,逐点固定 $\sqrt5$ 与 $\sqrt6=\sqrt2\sqrt3$,
		故 $E^H=\mathbb Q(\sqrt5,\sqrt6)$。
		\item $H=\langle \sigma_2\sigma_5\rangle$:逐点固定 $\sqrt3$ 与 $\sqrt{10}=\sqrt2\sqrt5$,
		故 $E^H=\mathbb Q(\sqrt3,\sqrt{10})$。
		\item $H=\langle \sigma_3\sigma_5\rangle$:逐点固定 $\sqrt2$ 与 $\sqrt{15}=\sqrt3\sqrt5$,
		故 $E^H=\mathbb Q(\sqrt2,\sqrt{15})$。
		\item $H=\langle \sigma_2\sigma_3\sigma_5\rangle$:它同时改三者符号,逐点固定
		\[
		\sqrt6=\sqrt2\sqrt3,\quad \sqrt{10}=\sqrt2\sqrt5,\quad \sqrt{15}=\sqrt3\sqrt5,
		\]
		故 $E^H=\mathbb Q(\sqrt6,\sqrt{10})$(亦等于 $\mathbb Q(\sqrt6,\sqrt{15})=\mathbb Q(\sqrt{10},\sqrt{15})$)。
	\end{itemize}
	
	以上 $7$ 个四次域两两不同,并与 $7$ 个阶为 $2$ 的子群一一对应。
	
	\medskip
	\textbf{(4) 汇总(子群阶 $\leftrightarrow$ 中间域次数).}\;
	
	\begin{itemize}
		\item $|H|=8 \Longleftrightarrow E^H=\mathbb Q$;
		\item $|H|=4 \Longleftrightarrow [E^H:\mathbb Q]=2$,共有 $7$ 个,分别为
		\[
		\mathbb Q(\sqrt2),\mathbb Q(\sqrt3),\mathbb Q(\sqrt5),\mathbb Q(\sqrt6),\mathbb Q(\sqrt{10}),\mathbb Q(\sqrt{15}),\mathbb Q(\sqrt{30});
		\]
		\item $|H|=2 \Longleftrightarrow [E^H:\mathbb Q]=4$,共有 $7$ 个,分别为
		\[
		\mathbb Q(\sqrt3,\sqrt5),\ \mathbb Q(\sqrt2,\sqrt5),\ \mathbb Q(\sqrt2,\sqrt3),\ \mathbb Q(\sqrt5,\sqrt6),\ \mathbb Q(\sqrt3,\sqrt{10}),\ \mathbb Q(\sqrt2,\sqrt{15}),\ \mathbb Q(\sqrt6,\sqrt{10});
		\]
		\item $|H|=1 \Longleftrightarrow E^H=E=\mathbb Q(\sqrt2,\sqrt3,\sqrt5)$。
	\end{itemize}
	
	\textbf{题目.}\;
	设 $E=\mathbb{Q}(\sqrt2,\sqrt3,\sqrt5)$,求 $\mathrm{Gal}(E/\mathbb{Q})$ 的所有子群及其对应的中间子域。
	
	\textbf{解:}
	
	\textbf{(1) 先求扩张次数与 Galois 群结构.}\;
	注意到
	\[
	E=\mathbb{Q}(\sqrt2,\sqrt3,\sqrt5)
	\]
	是三重二次(multiquadratic)扩张。并且
	\[
	\{1,\sqrt2,\sqrt3,\sqrt5,\sqrt6,\sqrt{10},\sqrt{15},\sqrt{30}\}
	\]
	在 $\mathbb{Q}$ 上线性无关,从而
	\[
	[E:\mathbb{Q}]=8.
	\]
	此外 $E$ 是多项式
	\[
	(x^2-2)(x^2-3)(x^2-5)
	\]
	在 $\mathbb{Q}$ 上的分裂域,因此 $E/\mathbb{Q}$ 为 Galois 扩张,故
	\[
	|\mathrm{Gal}(E/\mathbb{Q})|=[E:\mathbb{Q}]=8.
	\]
	
	令三个基本自同构为
	\[
	\sigma_2(\sqrt2)=-\sqrt2,\ \sigma_2(\sqrt3)=\sqrt3,\ \sigma_2(\sqrt5)=\sqrt5;
	\]
	\[
	\sigma_3(\sqrt3)=-\sqrt3,\ \sigma_3(\sqrt2)=\sqrt2,\ \sigma_3(\sqrt5)=\sqrt5;
	\]
	\[
	\sigma_5(\sqrt5)=-\sqrt5,\ \sigma_5(\sqrt2)=\sqrt2,\ \sigma_5(\sqrt3)=\sqrt3.
	\]
	则任意 $\sigma\in \mathrm{Gal}(E/\mathbb{Q})$ 都由对 $\sqrt2,\sqrt3,\sqrt5$ 的独立变号决定,故
	\[
	\mathrm{Gal}(E/\mathbb{Q})
	=\langle \sigma_2,\sigma_3,\sigma_5\rangle
	\cong (\mathbb{Z}/2\mathbb{Z})^3,
	\]
	并且群元素共有 $8$ 个:
	\[
	1,\ \sigma_2,\ \sigma_3,\ \sigma_5,\ \sigma_2\sigma_3,\ \sigma_2\sigma_5,\ \sigma_3\sigma_5,\ \sigma_2\sigma_3\sigma_5.
	\]
	
	\textbf{(2) G 的全部子群(共 $16$ 个)及对应子域.}\;
	由 Galois 对应:
	\[
	H\le G\quad \longleftrightarrow\quad E^H=\{x\in E:\forall \tau\in H,\ \tau(x)=x\},
	\]
	且
	\[
	[E^H:\mathbb{Q}]=\frac{|G|}{|H|}.
	\]
	因此:
	\[
	|H|=8 \Rightarrow [E^H:\mathbb{Q}]=1;\qquad
	|H|=4 \Rightarrow [E^H:\mathbb{Q}]=2;
	\]
	\[
	|H|=2 \Rightarrow [E^H:\mathbb{Q}]=4;\qquad
	|H|=1 \Rightarrow [E^H:\mathbb{Q}]=8.
	\]
	
	\textbf{A. 极端两项.}\;
	\[
	\{1\}\ \longleftrightarrow\ E,
	\qquad
	G\ \longleftrightarrow\ \mathbb{Q}.
	\]
	
	\textbf{B. $7$ 个阶为 $4$ 的子群 $\longleftrightarrow$ $7$ 个二次子域.}\;
	二次子域恰好是
	\[
	\mathbb{Q}(\sqrt d),\qquad d\in\{2,3,5,6,10,15,30\}.
	\]
	对应的阶 $4$ 子群(即“固定 $\sqrt d$ 的全部自同构”)分别为:
	\[
	\langle\sigma_3,\sigma_5\rangle=\{1,\sigma_3,\sigma_5,\sigma_3\sigma_5\}
	\longleftrightarrow \mathbb{Q}(\sqrt2),
	\]
	\[
	\langle\sigma_2,\sigma_5\rangle=\{1,\sigma_2,\sigma_5,\sigma_2\sigma_5\}
	\longleftrightarrow \mathbb{Q}(\sqrt3),
	\]
	\[
	\langle\sigma_2,\sigma_3\rangle=\{1,\sigma_2,\sigma_3,\sigma_2\sigma_3\}
	\longleftrightarrow \mathbb{Q}(\sqrt5),
	\]
	\[
	\langle\sigma_5,\sigma_2\sigma_3\rangle=\{1,\sigma_5,\sigma_2\sigma_3,\sigma_2\sigma_3\sigma_5\}
	\longleftrightarrow \mathbb{Q}(\sqrt6),
	\]
	\[
	\langle\sigma_3,\sigma_2\sigma_5\rangle=\{1,\sigma_3,\sigma_2\sigma_5,\sigma_2\sigma_3\sigma_5\}
	\longleftrightarrow \mathbb{Q}(\sqrt{10}),
	\]
	\[
	\langle\sigma_2,\sigma_3\sigma_5\rangle=\{1,\sigma_2,\sigma_3\sigma_5,\sigma_2\sigma_3\sigma_5\}
	\longleftrightarrow \mathbb{Q}(\sqrt{15}),
	\]
	\[
	\langle\sigma_2\sigma_3,\sigma_2\sigma_5\rangle=\{1,\sigma_2\sigma_3,\sigma_2\sigma_5,\sigma_3\sigma_5\}
	\longleftrightarrow \mathbb{Q}(\sqrt{30}).
	\]
	
	\textbf{C. $7$ 个阶为 $2$ 的子群 $\longleftrightarrow$ $7$ 个四次子域.}\;
	阶 $2$ 子群都形如 $\langle g\rangle=\{1,g\}$($g\neq 1$)。其不动域是次数 $4$ 的双二次子域(biquadratic field):
	\[
	\langle\sigma_2\rangle=\{1,\sigma_2\}\longleftrightarrow \mathbb{Q}(\sqrt3,\sqrt5),
	\]
	\[
	\langle\sigma_3\rangle=\{1,\sigma_3\}\longleftrightarrow \mathbb{Q}(\sqrt2,\sqrt5),
	\]
	\[
	\langle\sigma_5\rangle=\{1,\sigma_5\}\longleftrightarrow \mathbb{Q}(\sqrt2,\sqrt3),
	\]
	\[
	\langle\sigma_2\sigma_3\rangle=\{1,\sigma_2\sigma_3\}\longleftrightarrow \mathbb{Q}(\sqrt5,\sqrt6),
	\]
	\[
	\langle\sigma_2\sigma_5\rangle=\{1,\sigma_2\sigma_5\}\longleftrightarrow \mathbb{Q}(\sqrt3,\sqrt{10}),
	\]
	\[
	\langle\sigma_3\sigma_5\rangle=\{1,\sigma_3\sigma_5\}\longleftrightarrow \mathbb{Q}(\sqrt2,\sqrt{15}),
	\]
	\[
	\langle\sigma_2\sigma_3\sigma_5\rangle=\{1,\sigma_2\sigma_3\sigma_5\}\longleftrightarrow \mathbb{Q}(\sqrt6,\sqrt{10})
	\quad(\text{此时也包含 }\sqrt{15}).
	\]
	
	\textbf{结论汇总.}\;
	$G\cong(\mathbb{Z}/2\mathbb{Z})^3$,共有 $16$ 个子群:
	\[
	1\ (\text{个阶 }8),\quad 7\ (\text{个阶 }4),\quad 7\ (\text{个阶 }2),\quad 1\ (\text{个阶 }1),
	\]
	并分别与
	\[
	\mathbb{Q}\ (1\text{个}),\quad \mathbb{Q}(\sqrt d)\ (7\text{个}),\quad \mathbb{Q}(\sqrt a,\sqrt b)\ (7\text{个}),\quad E\ (1\text{个})
	\]
	一一对应(如上所列)。
	
	\item 5.
	\textbf{题目.}\;
	设 $\alpha$ 是
	\[
	f(x)=x^3+x^2-2x-1\in\mathbb Q[x]
	\]
	的一个根,证明 $\alpha^2-2$ 也是一个根,$\mathbb Q(\alpha)$ 是 $\mathbb Q$ 上的正规扩张,并求
	$\mathrm{Gal}(\mathbb Q(\alpha)/\mathbb Q)$。
	
	\textbf{解:}
	
	\textbf{(1) 先证:若 $f(\alpha)=0$,则 $f(\alpha^2-2)=0$.}\;
	直接计算并因式分解(可通过展开再合并同类项验证):
	\[
	\begin{aligned}
		f(x^2-2)
		&=(x^2-2)^3+(x^2-2)^2-2(x^2-2)-1 \\
		&=\bigl(x^3+x^2-2x-1\bigr)\bigl(x^3-x^2-2x+1\bigr).
	\end{aligned}
	\]
	于是若 $f(\alpha)=0$,代入 $x=\alpha$ 得
	\[
	0=f(\alpha^2-2)=f(\alpha)\,(\alpha^3-\alpha^2-2\alpha+1)=0,
	\]
	从而 $\alpha^2-2$ 也是 $f(x)$ 的一个根。
	
	\textbf{(2) 求出三个根都在 $\mathbb Q(\alpha)$ 中,从而 $\mathbb Q(\alpha)$ 为分裂域.}\;
	令
	\[
	\beta:=\alpha^2-2.
	\]
	由(1)知 $f(\beta)=0$。再对 $\beta$ 重复同样的构造,令
	\[
	\gamma:=\beta^2-2=(\alpha^2-2)^2-2=\alpha^4-4\alpha^2+2.
	\]
	利用 $f(\alpha)=0$ 给出的关系
	\[
	\alpha^3=-\alpha^2+2\alpha+1
	\]
	把 $\alpha^4=\alpha\alpha^3$ 化简:
	\[
	\alpha^4=\alpha(-\alpha^2+2\alpha+1)=-\alpha^3+2\alpha^2+\alpha
	= -(-\alpha^2+2\alpha+1)+2\alpha^2+\alpha
	=3\alpha^2-\alpha-1.
	\]
	代回 $\gamma$ 得
	\[
	\gamma=(3\alpha^2-\alpha-1)-4\alpha^2+2=-\alpha^2-\alpha+1\in\mathbb Q(\alpha).
	\]
	又因为 $\beta$ 是根,所以按(1)同理有 $f(\beta^2-2)=0$,即 $f(\gamma)=0$。
	因此 $f(x)$ 的三个根是
	\[
	\alpha,\qquad \beta=\alpha^2-2,\qquad \gamma=-\alpha^2-\alpha+1,
	\]
	并且都属于 $\mathbb Q(\alpha)$,所以 $f(x)$ 在 $\mathbb Q(\alpha)$ 中分裂,
	即 $\mathbb Q(\alpha)$ 正是 $f$ 在 $\mathbb Q$ 上的分裂域。
	
	\textbf{(3) 正规性与 Galois 群.}\;
	先证 $[\mathbb Q(\alpha):\mathbb Q]=3$:由有理根判别法,
	$f(1)=-1,f(-1)=1$,故 $f$ 在 $\mathbb Q$ 上无一次因子,从而不可约,
	所以 $\alpha$ 的最小多项式就是 $f$,故
	\[
	[\mathbb Q(\alpha):\mathbb Q]=\deg f=3.
	\]
	特征为 $0$ 时一切代数扩张可分,因此 $\mathbb Q(\alpha)/\mathbb Q$ 可分;
	而(2)说明它还是 $f$ 的分裂域,所以它是正规扩张。
	于是 $\mathbb Q(\alpha)/\mathbb Q$ 是 Galois 扩张,并且
	\[
	\bigl|\mathrm{Gal}(\mathbb Q(\alpha)/\mathbb Q)\bigr|
	=[\mathbb Q(\alpha):\mathbb Q]=3.
	\]
	
	任一 $\mathbb Q$-自同构由 $\alpha$ 的像唯一决定,而 $\alpha$ 只能被送到同一最小多项式
	$f$ 的根上,所以共有三种可能:
	\[
	\alpha\longmapsto \alpha,\quad
	\alpha\longmapsto \beta=\alpha^2-2,\quad
	\alpha\longmapsto \gamma=-\alpha^2-\alpha+1.
	\]
	定义
	\[
	\sigma(\alpha):=\alpha^2-2.
	\]
	则
	\[
	\sigma^2(\alpha)=\sigma(\alpha^2-2)=(\sigma(\alpha))^2-2=(\alpha^2-2)^2-2
	=-\alpha^2-\alpha+1=\gamma,
	\]
	并且
	\[
	\sigma^3(\alpha)=\sigma(\gamma)=\gamma^2-2=\alpha
	\]
	(最后一步可由同样的“代入并用 $f(\alpha)=0$ 化简”验证),故 $\sigma$ 的阶为 $3$。
	因此
	\[
	\mathrm{Gal}(\mathbb Q(\alpha)/\mathbb Q)=\langle \sigma\rangle
	=\{\,\mathrm{id},\sigma,\sigma^2\,\}\cong C_3\cong A_3,
	\]
	其在三根上的作用对应于 $3$-循环
	\[
	(\alpha\ \beta\ \gamma).
	\]
	
	
	
\end{enumerate}
\backmatter                % 后记与附录
	
	\end{document}
	

	


	
	
